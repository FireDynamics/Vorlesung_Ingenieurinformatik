% Options for packages loaded elsewhere
\PassOptionsToPackage{unicode}{hyperref}
\PassOptionsToPackage{hyphens}{url}
\PassOptionsToPackage{dvipsnames,svgnames,x11names}{xcolor}
%
\documentclass[
  letterpaper,
  DIV=11,
  numbers=noendperiod]{scrreprt}

\usepackage{amsmath,amssymb}
\usepackage{iftex}
\ifPDFTeX
  \usepackage[T1]{fontenc}
  \usepackage[utf8]{inputenc}
  \usepackage{textcomp} % provide euro and other symbols
\else % if luatex or xetex
  \usepackage{unicode-math}
  \defaultfontfeatures{Scale=MatchLowercase}
  \defaultfontfeatures[\rmfamily]{Ligatures=TeX,Scale=1}
\fi
\usepackage{lmodern}
\ifPDFTeX\else  
    % xetex/luatex font selection
\fi
% Use upquote if available, for straight quotes in verbatim environments
\IfFileExists{upquote.sty}{\usepackage{upquote}}{}
\IfFileExists{microtype.sty}{% use microtype if available
  \usepackage[]{microtype}
  \UseMicrotypeSet[protrusion]{basicmath} % disable protrusion for tt fonts
}{}
\makeatletter
\@ifundefined{KOMAClassName}{% if non-KOMA class
  \IfFileExists{parskip.sty}{%
    \usepackage{parskip}
  }{% else
    \setlength{\parindent}{0pt}
    \setlength{\parskip}{6pt plus 2pt minus 1pt}}
}{% if KOMA class
  \KOMAoptions{parskip=half}}
\makeatother
\usepackage{xcolor}
\setlength{\emergencystretch}{3em} % prevent overfull lines
\setcounter{secnumdepth}{5}
% Make \paragraph and \subparagraph free-standing
\makeatletter
\ifx\paragraph\undefined\else
  \let\oldparagraph\paragraph
  \renewcommand{\paragraph}{
    \@ifstar
      \xxxParagraphStar
      \xxxParagraphNoStar
  }
  \newcommand{\xxxParagraphStar}[1]{\oldparagraph*{#1}\mbox{}}
  \newcommand{\xxxParagraphNoStar}[1]{\oldparagraph{#1}\mbox{}}
\fi
\ifx\subparagraph\undefined\else
  \let\oldsubparagraph\subparagraph
  \renewcommand{\subparagraph}{
    \@ifstar
      \xxxSubParagraphStar
      \xxxSubParagraphNoStar
  }
  \newcommand{\xxxSubParagraphStar}[1]{\oldsubparagraph*{#1}\mbox{}}
  \newcommand{\xxxSubParagraphNoStar}[1]{\oldsubparagraph{#1}\mbox{}}
\fi
\makeatother

\usepackage{color}
\usepackage{fancyvrb}
\newcommand{\VerbBar}{|}
\newcommand{\VERB}{\Verb[commandchars=\\\{\}]}
\DefineVerbatimEnvironment{Highlighting}{Verbatim}{commandchars=\\\{\}}
% Add ',fontsize=\small' for more characters per line
\usepackage{framed}
\definecolor{shadecolor}{RGB}{241,243,245}
\newenvironment{Shaded}{\begin{snugshade}}{\end{snugshade}}
\newcommand{\AlertTok}[1]{\textcolor[rgb]{0.68,0.00,0.00}{#1}}
\newcommand{\AnnotationTok}[1]{\textcolor[rgb]{0.37,0.37,0.37}{#1}}
\newcommand{\AttributeTok}[1]{\textcolor[rgb]{0.40,0.45,0.13}{#1}}
\newcommand{\BaseNTok}[1]{\textcolor[rgb]{0.68,0.00,0.00}{#1}}
\newcommand{\BuiltInTok}[1]{\textcolor[rgb]{0.00,0.23,0.31}{#1}}
\newcommand{\CharTok}[1]{\textcolor[rgb]{0.13,0.47,0.30}{#1}}
\newcommand{\CommentTok}[1]{\textcolor[rgb]{0.37,0.37,0.37}{#1}}
\newcommand{\CommentVarTok}[1]{\textcolor[rgb]{0.37,0.37,0.37}{\textit{#1}}}
\newcommand{\ConstantTok}[1]{\textcolor[rgb]{0.56,0.35,0.01}{#1}}
\newcommand{\ControlFlowTok}[1]{\textcolor[rgb]{0.00,0.23,0.31}{\textbf{#1}}}
\newcommand{\DataTypeTok}[1]{\textcolor[rgb]{0.68,0.00,0.00}{#1}}
\newcommand{\DecValTok}[1]{\textcolor[rgb]{0.68,0.00,0.00}{#1}}
\newcommand{\DocumentationTok}[1]{\textcolor[rgb]{0.37,0.37,0.37}{\textit{#1}}}
\newcommand{\ErrorTok}[1]{\textcolor[rgb]{0.68,0.00,0.00}{#1}}
\newcommand{\ExtensionTok}[1]{\textcolor[rgb]{0.00,0.23,0.31}{#1}}
\newcommand{\FloatTok}[1]{\textcolor[rgb]{0.68,0.00,0.00}{#1}}
\newcommand{\FunctionTok}[1]{\textcolor[rgb]{0.28,0.35,0.67}{#1}}
\newcommand{\ImportTok}[1]{\textcolor[rgb]{0.00,0.46,0.62}{#1}}
\newcommand{\InformationTok}[1]{\textcolor[rgb]{0.37,0.37,0.37}{#1}}
\newcommand{\KeywordTok}[1]{\textcolor[rgb]{0.00,0.23,0.31}{\textbf{#1}}}
\newcommand{\NormalTok}[1]{\textcolor[rgb]{0.00,0.23,0.31}{#1}}
\newcommand{\OperatorTok}[1]{\textcolor[rgb]{0.37,0.37,0.37}{#1}}
\newcommand{\OtherTok}[1]{\textcolor[rgb]{0.00,0.23,0.31}{#1}}
\newcommand{\PreprocessorTok}[1]{\textcolor[rgb]{0.68,0.00,0.00}{#1}}
\newcommand{\RegionMarkerTok}[1]{\textcolor[rgb]{0.00,0.23,0.31}{#1}}
\newcommand{\SpecialCharTok}[1]{\textcolor[rgb]{0.37,0.37,0.37}{#1}}
\newcommand{\SpecialStringTok}[1]{\textcolor[rgb]{0.13,0.47,0.30}{#1}}
\newcommand{\StringTok}[1]{\textcolor[rgb]{0.13,0.47,0.30}{#1}}
\newcommand{\VariableTok}[1]{\textcolor[rgb]{0.07,0.07,0.07}{#1}}
\newcommand{\VerbatimStringTok}[1]{\textcolor[rgb]{0.13,0.47,0.30}{#1}}
\newcommand{\WarningTok}[1]{\textcolor[rgb]{0.37,0.37,0.37}{\textit{#1}}}

\providecommand{\tightlist}{%
  \setlength{\itemsep}{0pt}\setlength{\parskip}{0pt}}\usepackage{longtable,booktabs,array}
\usepackage{calc} % for calculating minipage widths
% Correct order of tables after \paragraph or \subparagraph
\usepackage{etoolbox}
\makeatletter
\patchcmd\longtable{\par}{\if@noskipsec\mbox{}\fi\par}{}{}
\makeatother
% Allow footnotes in longtable head/foot
\IfFileExists{footnotehyper.sty}{\usepackage{footnotehyper}}{\usepackage{footnote}}
\makesavenoteenv{longtable}
\usepackage{graphicx}
\makeatletter
\def\maxwidth{\ifdim\Gin@nat@width>\linewidth\linewidth\else\Gin@nat@width\fi}
\def\maxheight{\ifdim\Gin@nat@height>\textheight\textheight\else\Gin@nat@height\fi}
\makeatother
% Scale images if necessary, so that they will not overflow the page
% margins by default, and it is still possible to overwrite the defaults
% using explicit options in \includegraphics[width, height, ...]{}
\setkeys{Gin}{width=\maxwidth,height=\maxheight,keepaspectratio}
% Set default figure placement to htbp
\makeatletter
\def\fps@figure{htbp}
\makeatother

\usepackage{mathtools}
\KOMAoption{captions}{tableheading}
\makeatletter
\@ifpackageloaded{tcolorbox}{}{\usepackage[skins,breakable]{tcolorbox}}
\@ifpackageloaded{fontawesome5}{}{\usepackage{fontawesome5}}
\definecolor{quarto-callout-color}{HTML}{909090}
\definecolor{quarto-callout-note-color}{HTML}{0758E5}
\definecolor{quarto-callout-important-color}{HTML}{CC1914}
\definecolor{quarto-callout-warning-color}{HTML}{EB9113}
\definecolor{quarto-callout-tip-color}{HTML}{00A047}
\definecolor{quarto-callout-caution-color}{HTML}{FC5300}
\definecolor{quarto-callout-color-frame}{HTML}{acacac}
\definecolor{quarto-callout-note-color-frame}{HTML}{4582ec}
\definecolor{quarto-callout-important-color-frame}{HTML}{d9534f}
\definecolor{quarto-callout-warning-color-frame}{HTML}{f0ad4e}
\definecolor{quarto-callout-tip-color-frame}{HTML}{02b875}
\definecolor{quarto-callout-caution-color-frame}{HTML}{fd7e14}
\makeatother
\makeatletter
\@ifpackageloaded{bookmark}{}{\usepackage{bookmark}}
\makeatother
\makeatletter
\@ifpackageloaded{caption}{}{\usepackage{caption}}
\AtBeginDocument{%
\ifdefined\contentsname
  \renewcommand*\contentsname{Table of contents}
\else
  \newcommand\contentsname{Table of contents}
\fi
\ifdefined\listfigurename
  \renewcommand*\listfigurename{List of Figures}
\else
  \newcommand\listfigurename{List of Figures}
\fi
\ifdefined\listtablename
  \renewcommand*\listtablename{List of Tables}
\else
  \newcommand\listtablename{List of Tables}
\fi
\ifdefined\figurename
  \renewcommand*\figurename{Figure}
\else
  \newcommand\figurename{Figure}
\fi
\ifdefined\tablename
  \renewcommand*\tablename{Table}
\else
  \newcommand\tablename{Table}
\fi
}
\@ifpackageloaded{float}{}{\usepackage{float}}
\floatstyle{ruled}
\@ifundefined{c@chapter}{\newfloat{codelisting}{h}{lop}}{\newfloat{codelisting}{h}{lop}[chapter]}
\floatname{codelisting}{Listing}
\newcommand*\listoflistings{\listof{codelisting}{List of Listings}}
\makeatother
\makeatletter
\makeatother
\makeatletter
\@ifpackageloaded{caption}{}{\usepackage{caption}}
\@ifpackageloaded{subcaption}{}{\usepackage{subcaption}}
\makeatother

\ifLuaTeX
  \usepackage{selnolig}  % disable illegal ligatures
\fi
\usepackage{bookmark}

\IfFileExists{xurl.sty}{\usepackage{xurl}}{} % add URL line breaks if available
\urlstyle{same} % disable monospaced font for URLs
\hypersetup{
  pdftitle={Vorlesung Ingenieurinformatik},
  pdfauthor={Lukas Arnold; Simone Arnold; Florian Bagemihl; Matthias Baitsch; Marc Fehr; Maik Poetzsch; Sebastian Seipel},
  colorlinks=true,
  linkcolor={blue},
  filecolor={Maroon},
  citecolor={Blue},
  urlcolor={Blue},
  pdfcreator={LaTeX via pandoc}}


\title{Vorlesung Ingenieurinformatik}
\author{Lukas Arnold \and Simone Arnold \and Florian
Bagemihl \and Matthias Baitsch \and Marc Fehr \and Maik
Poetzsch \and Sebastian Seipel}
\date{2025-07-04}

\begin{document}
\maketitle

\renewcommand*\contentsname{Table of contents}
{
\hypersetup{linkcolor=}
\setcounter{tocdepth}{2}
\tableofcontents
}

\bookmarksetup{startatroot}

\chapter*{Übersicht}\label{uxfcbersicht}
\addcontentsline{toc}{chapter}{Übersicht}

\markboth{Übersicht}{Übersicht}

\section*{Allgemeine Infos}\label{allgemeine-infos}
\addcontentsline{toc}{section}{Allgemeine Infos}

\markright{Allgemeine Infos}

Die Vorlesung \emph{Ingenieurinformatik} an der Bergischen Universität
Wuppertal wurde vom im Jahr 2019 gebildeten Lehrstuhl
\href{https://cce.uni-wuppertal.de/}{Computational Civil Engineering
(CCE)} übernommen. Der CCE-Lehrstuhl beschäftigt sich hauptsächlich mit
der Erforschung und Entwicklung neuer computergestützter Modelle. Im
Zentrum der Anwendung steht die numerische Simulation der Brand- und
Rauchausbreitung in Gebäuden.

Da sich das Skript in der Entwicklung befinden freuen wir uns über
konstruktive Anregungen und Ihr Feedback. So können Sie Ihre Nachfolger
unterstützen.

\textbf{Alle organisatorischen Informationen zum Ablauf finden Sie auf
der
\href{https://cce.uni-wuppertal.de/index.php?id=4178&L=0}{CCE-Webseite
zur Ingenieurinformatik}.}

\section*{Kontakt}\label{kontakt}
\addcontentsline{toc}{section}{Kontakt}

\markright{Kontakt}

So erreichen Sie uns: * Als Teilnehmer der Vorlesung: am besten über den
zugehörigen
\href{https://moodle.uni-wuppertal.de/course/view.php?id=46894}{Moodle-Kurs}
an der Bergischen Universität Wuppertal * Externe Interessenten
benutzten am besten unsere Emailliste * Kontaktmöglichkeiten zu
einzelnen Personen finden Sie auf der
\href{https://cce.uni-wuppertal.de/de/team/}{Mitarbeiterwebseite}

\section*{Abschlussarbeiten}\label{abschlussarbeiten}
\addcontentsline{toc}{section}{Abschlussarbeiten}

\markright{Abschlussarbeiten}

Wir bieten Abschlussarbeiten (BA, MA, PhD) zu vielen verschiedenen
Themen an * eine Themenübersicht und bereits betreuter Arbeiten finden
Sie auf der
\href{https://cce.uni-wuppertal.de/de/abschlussarbeiten/}{Webseite der
Abschlussarbeiten} * Bei der Themenfindung kann auch die
\href{https://www.fz-juelich.de/en/ias/ias-7/profiles/arnold-lukas/journal-articles}{Übersicht
unserer Publikationen} helfen * Bei Interesse kontaktieren Sie bitte
Lukas Arnold

\section*{English Version}\label{english-version}
\addcontentsline{toc}{section}{English Version}

\markright{English Version}

Is an english version planned? Not yet, please contact Lukas Arnold.

\part{Grundlagen Python}

\chapter*{Werkzeugbaustein Python}\label{werkzeugbaustein-python}
\addcontentsline{toc}{chapter}{Werkzeugbaustein Python}

\markboth{Werkzeugbaustein Python}{Werkzeugbaustein Python}

\phantomsection\label{Lizenz}
\begin{figure}

\begin{minipage}{0.20\linewidth}
\includegraphics{index_files/mediabag/books/w-python-minimal/skript/00-bilder/CC-BY.pdf}\end{minipage}%
%
\begin{minipage}{0.80\linewidth}
Bausteine Computergestützter Datenanalyse von Lukas Arnold, Simone
Arnold, Florian Bagemihl, Matthias Baitsch, Marc Fehr, Maik Poetzsch und
Sebastian Seipel. Werkzeugbaustein Python von Maik Poetzsch ist
lizensiert unter
\href{https://creativecommons.org/licenses/by/4.0/deed.de}{CC BY 4.0}.
Das Werk ist abrufbar auf
\href{https://github.com/bausteine-der-datenanalyse/w-python}{GitHub}.
Ausgenommen von der Lizenz sind alle Logos Dritter und anders
gekennzeichneten Inhalte. 2025\end{minipage}%

\end{figure}%

Zitiervorschlag

Arnold, Lukas, Simone Arnold, Matthias Baitsch, Marc Fehr, Maik
Poetzsch, und Sebastian Seipel. 2025. „Bausteine Computergestützter
Datenanalyse. Werkzeugbaustein Python``.
\url{https://github.com/bausteine-der-datenanalyse/w-python}.

BibTeX-Vorlage

\begin{verbatim}
@misc{BCD-w-python-2025,
 title={Bausteine Computergestützter Datenanalyse. Werkzeugbaustein Python},
 author={Arnold, Lukas and Arnold, Simone and Baitsch, Matthias and Fehr, Marc and Poetzsch, Maik and Seipel, Sebastian},
 year={2025},
 url={https://github.com/bausteine-der-datenanalyse/w-python}} 
\end{verbatim}

\section*{Voraussetzungen}\label{voraussetzungen}
\addcontentsline{toc}{section}{Voraussetzungen}

\markright{Voraussetzungen}

Keine Voraussetzungen

\section*{Lernziele}\label{lernziele}
\addcontentsline{toc}{section}{Lernziele}

\markright{Lernziele}

In diesem Bausteine werden die Grundzüge der Programmierung mit Python
vermittelt. In diesem Baustein lernen Sie \ldots{}

\begin{itemize}
\item
  Grundlagen des Programmierens
\item
  Ausgaben in Python, Grundlegende Datentypen, FLusskontrolle
\item
  die Dokumentation zu lesen und zu verwenden
\item
  Module und Pakete laden
\end{itemize}

\chapter{Einführung}\label{einfuxfchrung}

\begin{figure}[H]

{\centering \includegraphics{books/w-python-minimal/skript/00-bilder/python-logo-and-wordmark-cc0-tm.png}

}

\caption{Logo der Programmiersprache Python}

\end{figure}%

Python Logo von Python Software Foundation steht unter der
\href{https://www.gnu.org/licenses/gpl-3.0.html}{GPLv3}. Die
Wort-Bild-Marke ist markenrechtlich geschützt:
\url{https://www.python.org/psf/trademarks/}. Das Werk ist abrufbar auf
\href{https://de.m.wikipedia.org/wiki/Datei:Python_logo_and_wordmark.svg}{wikimedia}.
2008

\chapter{Willkommen bei Python!}\label{willkommen-bei-python}

Python ist eine moderne Programmiersprache, die sich besonders gut für
Einsteigerinnen und Einsteiger eignet. Sie ist leicht verständlich und
wird in vielen Bereichen eingesetzt -- von der Datenanalyse bis hin zur
Webentwicklung.

\begin{quote}
In diesem Kurs lernen Sie Python Schritt für Schritt anhand praktischer
Beispiele.
\end{quote}

\section{Lernziele dieses Kapitels}\label{lernziele-dieses-kapitels}

Am Ende dieses Kapitels können Sie:

\begin{itemize}
\tightlist
\item
  einfache Python-Programme schreiben,
\item
  Text auf dem Bildschirm ausgeben,
\item
  erste Variablen definieren und verwenden.
\end{itemize}

\section{Ihr erstes Programm}\label{ihr-erstes-programm}

Die ersten Schritte in einer neuen Programmiersprache sind immer die
gleichen. WIr lassen uns die Worte `Hello World' ausgeben. Dazu nutzen
wir den print-Befehl \texttt{print()}:

\begin{Shaded}
\begin{Highlighting}[]
\BuiltInTok{print}\NormalTok{(}\StringTok{"Hallo Welt!"}\NormalTok{)}
\end{Highlighting}
\end{Shaded}

\begin{verbatim}
Hallo Welt!
\end{verbatim}

\textbf{Was passiert hier?} - \texttt{print()} ist eine sogenannte
\textbf{Funktion}, die etwas auf dem Bildschirm ausgibt. - Der Text
\texttt{"Hello\ World!"} wird angezeigt. - Texte (auch „Strings``
genannt) stehen immer in Anführungszeichen.

\section{Variablen -- Namen für
Werte}\label{variablen-namen-fuxfcr-werte}

Variablen sind wie beschriftete Schubladen: Sie speichern Informationen
unter einem Namen.

\begin{Shaded}
\begin{Highlighting}[]
\NormalTok{name }\OperatorTok{=} \StringTok{"Frau Müller"}
\NormalTok{alter }\OperatorTok{=} \DecValTok{32}
\end{Highlighting}
\end{Shaded}

Sie können diese Variablen verwenden, um dynamische Ausgaben zu
erzeugen:

\begin{Shaded}
\begin{Highlighting}[]
\BuiltInTok{print}\NormalTok{(name }\OperatorTok{+} \StringTok{" ist "} \OperatorTok{+} \BuiltInTok{str}\NormalTok{(alter) }\OperatorTok{+} \StringTok{" Jahre alt."}\NormalTok{)}
\end{Highlighting}
\end{Shaded}

\begin{verbatim}
Frau Müller ist 32 Jahre alt.
\end{verbatim}

Zu beachten ist hier, dass sie versuchen sowohl eine Zahl, als auch Text
auszugeben. Daher müssen wir mit der Funktion `str()' die Zahl in Text
umwandeln.

\begin{tcolorbox}[enhanced jigsaw, left=2mm, leftrule=.75mm, bottomrule=.15mm, title=\textcolor{quarto-callout-tip-color}{\faLightbulb}\hspace{0.5em}{✏️ Aufgabe: Begrüßung mit Alter}, colback=white, arc=.35mm, breakable, titlerule=0mm, bottomtitle=1mm, colbacktitle=quarto-callout-tip-color!10!white, toprule=.15mm, opacityback=0, coltitle=black, rightrule=.15mm, opacitybacktitle=0.6, toptitle=1mm, colframe=quarto-callout-tip-color-frame]

Schreiben Sie ein Programm, das Sie mit Ihrem Namen begrüßt:

\begin{verbatim}
Hallo Frau Müller!
\end{verbatim}

Tipp: In Python können Sie Texte mit \texttt{+} zusammenfügen. Denken
Sie daran, dass Strings in Anführungszeichen stehen müssen.

\begin{tcolorbox}[enhanced jigsaw, left=2mm, leftrule=.75mm, bottomrule=.15mm, title={Lösung}, colback=white, arc=.35mm, breakable, titlerule=0mm, bottomtitle=1mm, colbacktitle=quarto-callout-caution-color!10!white, toprule=.15mm, opacityback=0, coltitle=black, rightrule=.15mm, opacitybacktitle=0.6, toptitle=1mm, colframe=quarto-callout-caution-color-frame]

\begin{Shaded}
\begin{Highlighting}[]
\NormalTok{mein\_name }\OperatorTok{=} \StringTok{"Ihr Name hier"}
\BuiltInTok{print}\NormalTok{(}\StringTok{"Hallo "} \OperatorTok{+}\NormalTok{ mein\_name }\OperatorTok{+} \StringTok{"!"}\NormalTok{)}
\end{Highlighting}
\end{Shaded}

\begin{verbatim}
Hallo Ihr Name hier!
\end{verbatim}

\end{tcolorbox}

Erweitern Sie Ihr Programm so, dass es eine Begrüßung inklusive Alter
ausgibt:

\begin{verbatim}
Hallo Frau Müller!
Sie sind 32 Jahre alt.
\end{verbatim}

Tipp: Verwenden Sie \texttt{print()} mehrmals oder fügen Sie Texte
zusammen.

\begin{tcolorbox}[enhanced jigsaw, left=2mm, leftrule=.75mm, bottomrule=.15mm, title={Lösung}, colback=white, arc=.35mm, breakable, titlerule=0mm, bottomtitle=1mm, colbacktitle=quarto-callout-caution-color!10!white, toprule=.15mm, opacityback=0, coltitle=black, rightrule=.15mm, opacitybacktitle=0.6, toptitle=1mm, colframe=quarto-callout-caution-color-frame]

\begin{Shaded}
\begin{Highlighting}[]
\NormalTok{name }\OperatorTok{=} \StringTok{"Frau Müller"}
\NormalTok{alter }\OperatorTok{=} \DecValTok{32}

\BuiltInTok{print}\NormalTok{(}\StringTok{"Hallo "} \OperatorTok{+}\NormalTok{ name }\OperatorTok{+} \StringTok{"!"}\NormalTok{)}
\BuiltInTok{print}\NormalTok{(}\StringTok{"Sie sind "} \OperatorTok{+} \BuiltInTok{str}\NormalTok{(alter) }\OperatorTok{+} \StringTok{" Jahre alt."}\NormalTok{)}
\end{Highlighting}
\end{Shaded}

\begin{verbatim}
Hallo Frau Müller!
Sie sind 32 Jahre alt.
\end{verbatim}

\end{tcolorbox}

\end{tcolorbox}

\chapter{Datentypen verstehen}\label{datentypen-verstehen}

\section{Lernziele dieses Kapitels}\label{lernziele-dieses-kapitels-1}

Am Ende dieses Kapitels können Sie:

\begin{itemize}
\tightlist
\item
  die wichtigsten Datentypen unterscheiden,
\item
  mit Zahlen und Texten rechnen bzw. arbeiten,
\item
  einfache Berechnungen und Ausgaben erstellen.
\end{itemize}

\section{Einleitung}\label{einleitung}

In Python gibt es verschiedene \textbf{Datentypen}. Diese beschreiben,
\textbf{welche Art von Daten} Sie in Variablen speichern. Das ist
wichtig, weil viele Operationen -- wie zum Beispiel \texttt{+} -- je
nach Datentyp etwas anderes bedeuten:

\begin{itemize}
\tightlist
\item
  \texttt{+} bei Zahlen bedeutet \textbf{Addition},
\item
  \texttt{+} bei Text bedeutet \textbf{Zusammenfügen} (Konkatenation).
\end{itemize}

Bevor wir also mit komplexeren Programmen arbeiten, sollten wir
verstehen, welche Datentypen es gibt und wie man mit ihnen umgeht.

\section{Die wichtigsten Datentypen}\label{die-wichtigsten-datentypen}

Hier sind die grundlegenden Datentypen in Python:

\begin{longtable}[]{@{}lll@{}}
\toprule\noalign{}
Typ & Beispiel & Bedeutung \\
\midrule\noalign{}
\endhead
\bottomrule\noalign{}
\endlastfoot
\texttt{int} & \texttt{10} & Ganze Zahl \\
\texttt{float} & \texttt{3.14} & Kommazahl \\
\texttt{str} & \texttt{"Hallo"} & Text (String) \\
\texttt{bool} & \texttt{True}, \texttt{False} & Wahrheitswert
(Ja/Nein) \\
\end{longtable}

Sie können den Typ einer Variable mit der Funktion \texttt{type()}
herausfinden:

\begin{Shaded}
\begin{Highlighting}[]
\NormalTok{wert }\OperatorTok{=} \DecValTok{42}
\BuiltInTok{print}\NormalTok{(}\BuiltInTok{type}\NormalTok{(wert))  }\CommentTok{\# Ausgabe: \textless{}class \textquotesingle{}int\textquotesingle{}\textgreater{}}
\end{Highlighting}
\end{Shaded}

\begin{verbatim}
<class 'int'>
\end{verbatim}

\section{\texorpdfstring{Unterschiede zwischen \texttt{int} und
\texttt{float}}{Unterschiede zwischen int und float}}\label{unterschiede-zwischen-int-und-float}

In Python unterscheidet man zwischen \textbf{ganzen Zahlen}
(\texttt{int}) und \textbf{Kommazahlen} (\texttt{float}):

\begin{itemize}
\tightlist
\item
  \texttt{int} steht für „integer`` -- also ganze Zahlen wie \texttt{1},
  \texttt{0}, \texttt{-10}
\item
  \texttt{float} steht für „floating point number`` -- also Zahlen mit
  Dezimalstellen wie \texttt{3.14}, \texttt{0.5}, \texttt{-2.0}
\end{itemize}

\begin{Shaded}
\begin{Highlighting}[]
\NormalTok{a }\OperatorTok{=} \DecValTok{10}       \CommentTok{\# int}
\NormalTok{b }\OperatorTok{=} \FloatTok{2.5}      \CommentTok{\# float}

\BuiltInTok{print}\NormalTok{(}\StringTok{"a:"}\NormalTok{, a, }\StringTok{"| Typ:"}\NormalTok{, }\BuiltInTok{type}\NormalTok{(a))}
\BuiltInTok{print}\NormalTok{(}\StringTok{"b:"}\NormalTok{, b, }\StringTok{"| Typ:"}\NormalTok{, }\BuiltInTok{type}\NormalTok{(b))}
\end{Highlighting}
\end{Shaded}

\begin{verbatim}
a: 10 | Typ: <class 'int'>
b: 2.5 | Typ: <class 'float'>
\end{verbatim}

\begin{tcolorbox}[enhanced jigsaw, left=2mm, leftrule=.75mm, bottomrule=.15mm, title={Important}, colback=white, arc=.35mm, breakable, titlerule=0mm, bottomtitle=1mm, colbacktitle=quarto-callout-important-color!10!white, toprule=.15mm, opacityback=0, coltitle=black, rightrule=.15mm, opacitybacktitle=0.6, toptitle=1mm, colframe=quarto-callout-important-color-frame]

Die Unterscheidung ist wichtig: Manche Rechenoperationen verhalten sich
je nach Datentyp leicht unterschiedlich.

\end{tcolorbox}

\section{\texorpdfstring{Was sind Booleans
(\texttt{bool})?}{Was sind Booleans (bool)?}}\label{was-sind-booleans-bool}

Ein \textbf{Boolean} ist ein Wahrheitswert: Er kann nur zwei Zustände
annehmen:

\begin{itemize}
\tightlist
\item
  \texttt{True} (wahr)
\item
  \texttt{False} (falsch)
\end{itemize}

Solche Werte begegnen uns zum Beispiel bei Fragen wie:

\begin{itemize}
\tightlist
\item
  Ist die Temperatur über 30\,°C?
\item
  Hat die Datei einen bestimmten Namen?
\item
  Ist die Liste leer?
\end{itemize}

\begin{Shaded}
\begin{Highlighting}[]
\NormalTok{ist\_sonnig }\OperatorTok{=} \VariableTok{True}
\NormalTok{hat\_regenschirm }\OperatorTok{=} \VariableTok{False}

\BuiltInTok{print}\NormalTok{(}\StringTok{"Sonnig:"}\NormalTok{, ist\_sonnig)}
\BuiltInTok{print}\NormalTok{(}\StringTok{"Regenschirm dabei?"}\NormalTok{, hat\_regenschirm)}
\BuiltInTok{print}\NormalTok{(}\StringTok{"Typ von \textquotesingle{}ist\_sonnig\textquotesingle{}:"}\NormalTok{, }\BuiltInTok{type}\NormalTok{(ist\_sonnig))}
\end{Highlighting}
\end{Shaded}

\begin{verbatim}
Sonnig: True
Regenschirm dabei? False
Typ von 'ist_sonnig': <class 'bool'>
\end{verbatim}

Booleans werden besonders in \textbf{Bedingungen} und
\textbf{Vergleichen} verwendet, was Sie in Kapitel 4 genauer
kennenlernen.

\section{Rechnen mit Zahlen}\label{rechnen-mit-zahlen}

Python kann wie ein Taschenrechner verwendet werden:

\begin{longtable}[]{@{}cc@{}}
\toprule\noalign{}
Operator & Beschreibung \\
\midrule\noalign{}
\endhead
\bottomrule\noalign{}
\endlastfoot
\texttt{+}, \texttt{-} & Addition / Subtraktion \\
\texttt{*}, \texttt{/} & Multiplikation / Division \\
\texttt{//}, \texttt{\%} & Ganzzahlige Division / Rest \\
\texttt{**} & Potenzieren \\
\end{longtable}

\begin{Shaded}
\begin{Highlighting}[]
\NormalTok{a }\OperatorTok{=} \DecValTok{10}
\NormalTok{b }\OperatorTok{=} \DecValTok{3}

\BuiltInTok{print}\NormalTok{(}\StringTok{"Addition:"}\NormalTok{, a }\OperatorTok{+}\NormalTok{ b)}
\BuiltInTok{print}\NormalTok{(}\StringTok{"Subtraktion:"}\NormalTok{, a }\OperatorTok{{-}}\NormalTok{ b)}
\BuiltInTok{print}\NormalTok{(}\StringTok{"Multiplikation:"}\NormalTok{, a }\OperatorTok{*}\NormalTok{ b)}
\BuiltInTok{print}\NormalTok{(}\StringTok{"Potenzieren"}\NormalTok{, a}\OperatorTok{**}\NormalTok{b)}
\BuiltInTok{print}\NormalTok{(}\StringTok{"Division:"}\NormalTok{, a }\OperatorTok{/}\NormalTok{ b)}
\BuiltInTok{print}\NormalTok{(}\StringTok{"Ganzzahlige Division:"}\NormalTok{, a }\OperatorTok{//}\NormalTok{ b)}
\BuiltInTok{print}\NormalTok{(}\StringTok{"Division mit Rest:"}\NormalTok{, a }\OperatorTok{\%}\NormalTok{ b)}
\end{Highlighting}
\end{Shaded}

\begin{verbatim}
Addition: 13
Subtraktion: 7
Multiplikation: 30
Potenzieren 1000
Division: 3.3333333333333335
Ganzzahlige Division: 3
Division mit Rest: 1
\end{verbatim}

\begin{tcolorbox}[enhanced jigsaw, left=2mm, leftrule=.75mm, bottomrule=.15mm, title=\textcolor{quarto-callout-note-color}{\faInfo}\hspace{0.5em}{Note}, colback=white, arc=.35mm, breakable, titlerule=0mm, bottomtitle=1mm, colbacktitle=quarto-callout-note-color!10!white, toprule=.15mm, opacityback=0, coltitle=black, rightrule=.15mm, opacitybacktitle=0.6, toptitle=1mm, colframe=quarto-callout-note-color-frame]

\texttt{//} bedeutet: Ganzzahldivision, das Ergebnis wird abgerundet.
Alternativ gibt es auch \texttt{\%}. Hier wird eine Ganzzahldivision
durchgeführt und der Rest ausgegeben.

\end{tcolorbox}

\section{Arbeiten mit Text}\label{arbeiten-mit-text}

Texte (Strings) können miteinander kombiniert werden:

\begin{Shaded}
\begin{Highlighting}[]
\NormalTok{vorname }\OperatorTok{=} \StringTok{"Anna"}
\NormalTok{nachname }\OperatorTok{=} \StringTok{"Beispiel"}
\BuiltInTok{print}\NormalTok{(}\StringTok{"Willkommen, "} \OperatorTok{+}\NormalTok{ vorname }\OperatorTok{+} \StringTok{" "} \OperatorTok{+}\NormalTok{ nachname }\OperatorTok{+} \StringTok{"!"}\NormalTok{)}
\end{Highlighting}
\end{Shaded}

\begin{verbatim}
Willkommen, Anna Beispiel!
\end{verbatim}

Wenn Sie Text und Zahlen kombinieren wollen, müssen Sie die Zahl in
einen String umwandeln:

\begin{Shaded}
\begin{Highlighting}[]
\NormalTok{punkte }\OperatorTok{=} \DecValTok{95}
\BuiltInTok{print}\NormalTok{(}\StringTok{"Sie haben "} \OperatorTok{+} \BuiltInTok{str}\NormalTok{(punkte) }\OperatorTok{+} \StringTok{" Punkte erreicht."}\NormalTok{)}
\end{Highlighting}
\end{Shaded}

\begin{verbatim}
Sie haben 95 Punkte erreicht.
\end{verbatim}

\section{Umwandlung von Datentypen
(Typecasting)}\label{umwandlung-von-datentypen-typecasting}

Manchmal müssen Sie einen Wert von einem Datentyp in einen anderen
umwandeln -- z.\,B. eine Zahl in einen Text (String), damit sie
ausgegeben werden kann.

Das nennt man \textbf{Typecasting}. Hier sind die wichtigsten Funktionen
dafür:

\begin{longtable}[]{@{}lll@{}}
\toprule\noalign{}
Funktion & Beschreibung & Beispiel \\
\midrule\noalign{}
\endhead
\bottomrule\noalign{}
\endlastfoot
\texttt{str())} & Zahl → Text & \texttt{str(42)} → \texttt{"42"} \\
\texttt{int()} & Text/Zahl → ganze Zahl & \texttt{int("10")} →
\texttt{10} \\
\texttt{float()} & Text/Zahl → Kommazahl & \texttt{float("3.14")} →
\texttt{3.14} \\
\end{longtable}

\begin{Shaded}
\begin{Highlighting}[]
\CommentTok{\# Beispiel: Zahl als Text anzeigen}
\NormalTok{punkte }\OperatorTok{=} \DecValTok{100}
\BuiltInTok{print}\NormalTok{(}\StringTok{"Sie haben "} \OperatorTok{+} \BuiltInTok{str}\NormalTok{(punkte) }\OperatorTok{+} \StringTok{" Punkte."}\NormalTok{)}

\CommentTok{\# Beispiel: String in Zahl umwandeln und berechnen}
\NormalTok{eingabe }\OperatorTok{=} \StringTok{"3.5"}
\NormalTok{wert }\OperatorTok{=} \BuiltInTok{float}\NormalTok{(eingabe) }\OperatorTok{*} \DecValTok{2}
\BuiltInTok{print}\NormalTok{(}\StringTok{"Doppelt so viel:"}\NormalTok{, wert)}
\end{Highlighting}
\end{Shaded}

\begin{verbatim}
Sie haben 100 Punkte.
Doppelt so viel: 7.0
\end{verbatim}

Achten Sie beim Umwandeln darauf, dass der Inhalt auch wirklich passt --
\texttt{int("abc")} führt zu einem Fehler.

\begin{tcolorbox}[enhanced jigsaw, left=2mm, leftrule=.75mm, bottomrule=.15mm, title=\textcolor{quarto-callout-tip-color}{\faLightbulb}\hspace{0.5em}{Aufgabe: Alter in Tagen}, colback=white, arc=.35mm, breakable, titlerule=0mm, bottomtitle=1mm, colbacktitle=quarto-callout-tip-color!10!white, toprule=.15mm, opacityback=0, coltitle=black, rightrule=.15mm, opacitybacktitle=0.6, toptitle=1mm, colframe=quarto-callout-tip-color-frame]

Berechnen Sie, wie alt eine Person in Tagen ist.

\begin{Shaded}
\begin{Highlighting}[]
\NormalTok{alter\_jahre }\OperatorTok{=} \DecValTok{32}
\NormalTok{tage }\OperatorTok{=}\NormalTok{ alter\_jahre }\OperatorTok{*} \DecValTok{365}
\BuiltInTok{print}\NormalTok{(}\StringTok{"Sie sind ungefähr "} \OperatorTok{+} \BuiltInTok{str}\NormalTok{(tage) }\OperatorTok{+} \StringTok{" Tage alt."}\NormalTok{)}
\end{Highlighting}
\end{Shaded}

\begin{verbatim}
Sie sind ungefähr 11680 Tage alt.
\end{verbatim}

Tipp: Denken Sie an die Umwandlung in einen String, wenn Sie die Zahl
ausgeben möchten.

\begin{tcolorbox}[enhanced jigsaw, left=2mm, leftrule=.75mm, bottomrule=.15mm, title={Lösung}, colback=white, arc=.35mm, breakable, titlerule=0mm, bottomtitle=1mm, colbacktitle=quarto-callout-caution-color!10!white, toprule=.15mm, opacityback=0, coltitle=black, rightrule=.15mm, opacitybacktitle=0.6, toptitle=1mm, colframe=quarto-callout-caution-color-frame]

\begin{Shaded}
\begin{Highlighting}[]
\NormalTok{alter }\OperatorTok{=} \DecValTok{32}
\NormalTok{tage }\OperatorTok{=}\NormalTok{ alter }\OperatorTok{*} \DecValTok{365}
\BuiltInTok{print}\NormalTok{(}\StringTok{"Sie sind ungefähr "} \OperatorTok{+} \BuiltInTok{str}\NormalTok{(tage) }\OperatorTok{+} \StringTok{" Tage alt."}\NormalTok{)}
\end{Highlighting}
\end{Shaded}

\begin{verbatim}
Sie sind ungefähr 11680 Tage alt.
\end{verbatim}

\end{tcolorbox}

\end{tcolorbox}

\chapter{Entscheidungen und
Wiederholungen}\label{entscheidungen-und-wiederholungen}

Programme müssen oft Entscheidungen treffen -- zum Beispiel abhängig von
einer Benutzereingabe oder einem bestimmten Wert. Ebenso müssen
bestimmte Aktionen mehrfach durchgeführt werden.

Dafür gibt es zwei zentrale Elemente in Python:

\begin{itemize}
\tightlist
\item
  \textbf{Kontrollstrukturen}: \texttt{if}, \texttt{elif}, \texttt{else}
\item
  \textbf{Schleifen}: \texttt{while} und \texttt{for}
\end{itemize}

\section{Lernziele dieses Kapitels}\label{lernziele-dieses-kapitels-2}

Am Ende dieses Kapitels können Sie:

\begin{itemize}
\tightlist
\item
  Bedingungen formulieren und mit \texttt{if}, \texttt{elif},
  \texttt{else} nutzen,
\item
  Vergleichsoperatoren verwenden (\texttt{==}, \texttt{\textless{}},
  \texttt{!=}, \ldots),
\item
  Wiederholungen mit \texttt{while} und \texttt{for} umsetzen.
\end{itemize}

\section{\texorpdfstring{Bedingungen mit \texttt{if}, \texttt{elif},
\texttt{else}}{Bedingungen mit if, elif, else}}\label{bedingungen-mit-if-elif-else}

\begin{Shaded}
\begin{Highlighting}[]
\NormalTok{alter }\OperatorTok{=} \DecValTok{17}

\ControlFlowTok{if}\NormalTok{ alter }\OperatorTok{\textgreater{}=} \DecValTok{18}\NormalTok{:}
    \BuiltInTok{print}\NormalTok{(}\StringTok{"Sie sind volljährig."}\NormalTok{)}
\ControlFlowTok{else}\NormalTok{:}
    \BuiltInTok{print}\NormalTok{(}\StringTok{"Sie sind minderjährig."}\NormalTok{)}
\end{Highlighting}
\end{Shaded}

\begin{verbatim}
Sie sind minderjährig.
\end{verbatim}

Mehrere Fälle unterscheiden:

\begin{Shaded}
\begin{Highlighting}[]
\NormalTok{note }\OperatorTok{=} \FloatTok{2.3}

\ControlFlowTok{if}\NormalTok{ note }\OperatorTok{\textless{}=} \FloatTok{1.5}\NormalTok{:}
    \BuiltInTok{print}\NormalTok{(}\StringTok{"Sehr gut"}\NormalTok{)}
\ControlFlowTok{elif}\NormalTok{ note }\OperatorTok{\textless{}=} \FloatTok{2.5}\NormalTok{:}
    \BuiltInTok{print}\NormalTok{(}\StringTok{"Gut"}\NormalTok{)}
\ControlFlowTok{elif}\NormalTok{ note }\OperatorTok{\textless{}=} \FloatTok{3.5}\NormalTok{:}
    \BuiltInTok{print}\NormalTok{(}\StringTok{"Befriedigend"}\NormalTok{)}
\ControlFlowTok{else}\NormalTok{:}
    \BuiltInTok{print}\NormalTok{(}\StringTok{"Ausreichend oder schlechter"}\NormalTok{)}
\end{Highlighting}
\end{Shaded}

\begin{verbatim}
Gut
\end{verbatim}

\section{Vergleichsoperatoren}\label{vergleichsoperatoren}

\begin{longtable}[]{@{}ll@{}}
\toprule\noalign{}
Ausdruck & Bedeutung \\
\midrule\noalign{}
\endhead
\bottomrule\noalign{}
\endlastfoot
\texttt{a\ ==\ b} & gleich \\
\texttt{a\ !=\ b} & ungleich \\
\texttt{a\ \textless{}\ b} & kleiner als \\
\texttt{a\ \textgreater{}\ b} & größer als \\
\texttt{a\ \textless{}=\ b} & kleiner oder gleich \\
\texttt{a\ \textgreater{}=\ b} & größer oder gleich \\
\end{longtable}

\section{\texorpdfstring{Wiederholungen mit
\texttt{while}}{Wiederholungen mit while}}\label{wiederholungen-mit-while}

\begin{Shaded}
\begin{Highlighting}[]
\NormalTok{zähler }\OperatorTok{=} \DecValTok{0}

\ControlFlowTok{while}\NormalTok{ zähler }\OperatorTok{\textless{}} \DecValTok{5}\NormalTok{:}
    \BuiltInTok{print}\NormalTok{(}\StringTok{"Zähler ist:"}\NormalTok{, zähler)}
\NormalTok{    zähler }\OperatorTok{+=} \DecValTok{1}
\end{Highlighting}
\end{Shaded}

\begin{verbatim}
Zähler ist: 0
Zähler ist: 1
Zähler ist: 2
Zähler ist: 3
Zähler ist: 4
\end{verbatim}

\begin{tcolorbox}[enhanced jigsaw, left=2mm, leftrule=.75mm, bottomrule=.15mm, title={Important}, colback=white, arc=.35mm, breakable, titlerule=0mm, bottomtitle=1mm, colbacktitle=quarto-callout-important-color!10!white, toprule=.15mm, opacityback=0, coltitle=black, rightrule=.15mm, opacitybacktitle=0.6, toptitle=1mm, colframe=quarto-callout-important-color-frame]

Achten Sie auf eine Abbruchbedingung -- sonst läuft die Schleife endlos!

\end{tcolorbox}

\section{\texorpdfstring{Schleifen mit \texttt{for} und
\texttt{range()}}{Schleifen mit for und range()}}\label{schleifen-mit-for-und-range}

Wenn Sie eine Schleife \textbf{genau eine bestimmte Anzahl von Malen}
durchlaufen möchten, nutzen Sie \texttt{for} mit \texttt{range()}:

\begin{Shaded}
\begin{Highlighting}[]
\ControlFlowTok{for}\NormalTok{ i }\KeywordTok{in} \BuiltInTok{range}\NormalTok{(}\DecValTok{5}\NormalTok{):}
    \BuiltInTok{print}\NormalTok{(}\StringTok{"Durchlauf:"}\NormalTok{, i)}
\end{Highlighting}
\end{Shaded}

\begin{verbatim}
Durchlauf: 0
Durchlauf: 1
Durchlauf: 2
Durchlauf: 3
Durchlauf: 4
\end{verbatim}

Start- und Endwert festlegen:

\begin{Shaded}
\begin{Highlighting}[]
\ControlFlowTok{for}\NormalTok{ i }\KeywordTok{in} \BuiltInTok{range}\NormalTok{(}\DecValTok{1}\NormalTok{, }\DecValTok{6}\NormalTok{):}
    \BuiltInTok{print}\NormalTok{(i)}
\end{Highlighting}
\end{Shaded}

\begin{verbatim}
1
2
3
4
5
\end{verbatim}

\section{\texorpdfstring{Was macht \texttt{range()}
genau?}{Was macht range() genau?}}\label{was-macht-range-genau}

Die Funktion \texttt{range()} erzeugt eine Abfolge von Zahlen, über die
Sie mit einer \texttt{for}-Schleife iterieren können.

\subsection{Varianten:}\label{varianten}

\begin{Shaded}
\begin{Highlighting}[]
\BuiltInTok{range}\NormalTok{(}\DecValTok{5}\NormalTok{)}
\end{Highlighting}
\end{Shaded}

➡️ ergibt: \texttt{0,\ 1,\ 2,\ 3,\ 4} (startet bei 0, endet \textbf{vor}
5)

\begin{Shaded}
\begin{Highlighting}[]
\BuiltInTok{range}\NormalTok{(}\DecValTok{2}\NormalTok{, }\DecValTok{6}\NormalTok{)}
\end{Highlighting}
\end{Shaded}

➡️ ergibt: \texttt{2,\ 3,\ 4,\ 5} (startet bei 2, endet \textbf{vor} 6)

\begin{Shaded}
\begin{Highlighting}[]
\BuiltInTok{range}\NormalTok{(}\DecValTok{1}\NormalTok{, }\DecValTok{10}\NormalTok{, }\DecValTok{2}\NormalTok{)}
\end{Highlighting}
\end{Shaded}

➡️ ergibt: \texttt{1,\ 3,\ 5,\ 7,\ 9} (Schrittweite = 2)

\texttt{range()} erzeugt keine echte Liste, sondern ein sogenanntes
„range-Objekt``, das wie eine Liste verwendet werden kann.

\begin{tcolorbox}[enhanced jigsaw, left=2mm, leftrule=.75mm, bottomrule=.15mm, title=\textcolor{quarto-callout-tip-color}{\faLightbulb}\hspace{0.5em}{✏️ Aufgabe: Zähle von 1 bis 10}, colback=white, arc=.35mm, breakable, titlerule=0mm, bottomtitle=1mm, colbacktitle=quarto-callout-tip-color!10!white, toprule=.15mm, opacityback=0, coltitle=black, rightrule=.15mm, opacitybacktitle=0.6, toptitle=1mm, colframe=quarto-callout-tip-color-frame]

Nutzen Sie eine \texttt{for}-Schleife, um die Zahlen von 1 bis 10
auszugeben.

\begin{tcolorbox}[enhanced jigsaw, left=2mm, leftrule=.75mm, bottomrule=.15mm, title={Lösung}, colback=white, arc=.35mm, breakable, titlerule=0mm, bottomtitle=1mm, colbacktitle=quarto-callout-caution-color!10!white, toprule=.15mm, opacityback=0, coltitle=black, rightrule=.15mm, opacitybacktitle=0.6, toptitle=1mm, colframe=quarto-callout-caution-color-frame]

\begin{Shaded}
\begin{Highlighting}[]
\ControlFlowTok{for}\NormalTok{ i }\KeywordTok{in} \BuiltInTok{range}\NormalTok{(}\DecValTok{1}\NormalTok{, }\DecValTok{11}\NormalTok{):}
    \BuiltInTok{print}\NormalTok{(i)}
\end{Highlighting}
\end{Shaded}

\begin{verbatim}
1
2
3
4
5
6
7
8
9
10
\end{verbatim}

\end{tcolorbox}

\end{tcolorbox}

\begin{tcolorbox}[enhanced jigsaw, left=2mm, leftrule=.75mm, bottomrule=.15mm, title=\textcolor{quarto-callout-tip-color}{\faLightbulb}\hspace{0.5em}{Aufgabe: Gerade Zahlen ausgeben}, colback=white, arc=.35mm, breakable, titlerule=0mm, bottomtitle=1mm, colbacktitle=quarto-callout-tip-color!10!white, toprule=.15mm, opacityback=0, coltitle=black, rightrule=.15mm, opacitybacktitle=0.6, toptitle=1mm, colframe=quarto-callout-tip-color-frame]

Geben Sie alle geraden Zahlen von 0 bis 20 aus. Tipp: Eine Zahl ist
gerade, wenn \texttt{zahl\ \%\ 2\ ==\ 0}.

\begin{tcolorbox}[enhanced jigsaw, left=2mm, leftrule=.75mm, bottomrule=.15mm, title={Lösung}, colback=white, arc=.35mm, breakable, titlerule=0mm, bottomtitle=1mm, colbacktitle=quarto-callout-caution-color!10!white, toprule=.15mm, opacityback=0, coltitle=black, rightrule=.15mm, opacitybacktitle=0.6, toptitle=1mm, colframe=quarto-callout-caution-color-frame]

\begin{Shaded}
\begin{Highlighting}[]
\ControlFlowTok{for}\NormalTok{ zahl }\KeywordTok{in} \BuiltInTok{range}\NormalTok{(}\DecValTok{0}\NormalTok{, }\DecValTok{21}\NormalTok{):}
    \ControlFlowTok{if}\NormalTok{ zahl }\OperatorTok{\%} \DecValTok{2} \OperatorTok{==} \DecValTok{0}\NormalTok{:}
        \BuiltInTok{print}\NormalTok{(zahl)}
\end{Highlighting}
\end{Shaded}

\begin{verbatim}
0
2
4
6
8
10
12
14
16
18
20
\end{verbatim}

\end{tcolorbox}

\end{tcolorbox}

\chapter{Mehrere Werte speichern}\label{mehrere-werte-speichern}

Bisher haben Sie einzelne Werte in Variablen gespeichert. Doch was, wenn
Sie eine ganze Reihe von Zahlen, Namen oder Werten auf einmal speichern
möchten?

Dafür gibt es in Python \textbf{Listen}. In diesem Kapitel lernen Sie
außerdem, wie man mit \texttt{for}-Schleifen über Listen iteriert.

\section{Was ist eine Liste?}\label{was-ist-eine-liste}

Eine Liste ist eine geordnete Sammlung von Werten eines Datentyps.

\begin{Shaded}
\begin{Highlighting}[]
\NormalTok{namen }\OperatorTok{=}\NormalTok{ [}\StringTok{"Ali"}\NormalTok{, }\StringTok{"Bente"}\NormalTok{, }\StringTok{"Carlos"}\NormalTok{]}
\NormalTok{noten }\OperatorTok{=}\NormalTok{ [}\FloatTok{1.7}\NormalTok{, }\FloatTok{2.3}\NormalTok{, }\FloatTok{1.3}\NormalTok{, }\FloatTok{2.0}\NormalTok{]}
\end{Highlighting}
\end{Shaded}

Auf Elemente greifen Sie mit eckigen Klammern zu:

\begin{Shaded}
\begin{Highlighting}[]
\BuiltInTok{print}\NormalTok{(namen[}\DecValTok{0}\NormalTok{])  }\CommentTok{\# erstes Element}
\BuiltInTok{print}\NormalTok{(noten[}\OperatorTok{{-}}\DecValTok{1}\NormalTok{]) }\CommentTok{\# letztes Element}
\end{Highlighting}
\end{Shaded}

\begin{verbatim}
Ali
2.0
\end{verbatim}

\section{Teile aus Listen ausschneiden --
Slicing}\label{teile-aus-listen-ausschneiden-slicing}

Mit dem sogenannten \textbf{Slicing} können Sie gezielt Ausschnitte aus
einer Liste entnehmen. Dabei geben Sie an, \textbf{wo der Ausschnitt
beginnt und wo er endet} (der Endwert wird \textbf{nicht} mehr
mitgenommen):

\begin{Shaded}
\begin{Highlighting}[]
\NormalTok{zahlen }\OperatorTok{=}\NormalTok{ [}\DecValTok{10}\NormalTok{, }\DecValTok{20}\NormalTok{, }\DecValTok{30}\NormalTok{, }\DecValTok{40}\NormalTok{, }\DecValTok{50}\NormalTok{, }\DecValTok{60}\NormalTok{]}
\BuiltInTok{print}\NormalTok{(zahlen[}\DecValTok{1}\NormalTok{:}\DecValTok{4}\NormalTok{])  }\CommentTok{\# Ausgabe: [20, 30, 40]}
\end{Highlighting}
\end{Shaded}

\begin{verbatim}
[20, 30, 40]
\end{verbatim}

\subsection{\texorpdfstring{Syntax:
\texttt{liste{[}start:stop{]}}}{Syntax: liste{[}start:stop{]}}}\label{syntax-listestartstop}

\begin{itemize}
\tightlist
\item
  \textbf{start}: Index, bei dem das Slicing beginnt (inklusive)
\item
  \textbf{stop}: Index, an dem es endet (exklusive)
\item
  Der Startwert kann auch weggelassen werden: \texttt{{[}:3{]}} → erstes
  bis drittes Element
\item
  Ebenso der Endwert: \texttt{{[}3:{]}} → ab dem vierten Element bis zum
  Ende
\item
  Ganze Kopie: \texttt{{[}:{]}}
\end{itemize}

\begin{Shaded}
\begin{Highlighting}[]
\BuiltInTok{print}\NormalTok{(zahlen[:}\DecValTok{3}\NormalTok{])   }\CommentTok{\# [10, 20, 30]}
\BuiltInTok{print}\NormalTok{(zahlen[}\DecValTok{3}\NormalTok{:])   }\CommentTok{\# [40, 50, 60]}
\BuiltInTok{print}\NormalTok{(zahlen[:])    }\CommentTok{\# vollständige Kopie}
\end{Highlighting}
\end{Shaded}

\begin{verbatim}
[10, 20, 30]
[40, 50, 60]
[10, 20, 30, 40, 50, 60]
\end{verbatim}

\begin{tcolorbox}[enhanced jigsaw, left=2mm, leftrule=.75mm, bottomrule=.15mm, title={Note}, colback=white, arc=.35mm, breakable, titlerule=0mm, bottomtitle=1mm, colbacktitle=quarto-callout-note-color!10!white, toprule=.15mm, opacityback=0, coltitle=black, rightrule=.15mm, opacitybacktitle=0.6, toptitle=1mm, colframe=quarto-callout-note-color-frame]

Sie können auch mit negativen Indizes arbeiten (\texttt{-1} ist das
letzte Element):

\begin{Shaded}
\begin{Highlighting}[]
\BuiltInTok{print}\NormalTok{(zahlen[}\OperatorTok{{-}}\DecValTok{3}\NormalTok{:])  }\CommentTok{\# [40, 50, 60]}
\end{Highlighting}
\end{Shaded}

\begin{verbatim}
[40, 50, 60]
\end{verbatim}

\end{tcolorbox}

\section{Über Listen iterieren}\label{uxfcber-listen-iterieren}

Mit einer \texttt{for}-Schleife können Sie über jedes Element in einer
Liste iterieren:

\begin{Shaded}
\begin{Highlighting}[]
\NormalTok{namen }\OperatorTok{=}\NormalTok{ [}\StringTok{"Ali"}\NormalTok{, }\StringTok{"Bente"}\NormalTok{, }\StringTok{"Carlos"}\NormalTok{]}

\ControlFlowTok{for}\NormalTok{ name }\KeywordTok{in}\NormalTok{ namen:}
    \BuiltInTok{print}\NormalTok{(}\StringTok{"Hallo"}\NormalTok{, name }\OperatorTok{+} \StringTok{"!"}\NormalTok{)}
\end{Highlighting}
\end{Shaded}

\begin{verbatim}
Hallo Ali!
Hallo Bente!
Hallo Carlos!
\end{verbatim}

\section{Erweiterung: Bedingte
Ausgaben}\label{erweiterung-bedingte-ausgaben}

Sie können in der Schleife mit \texttt{if} filtern:

\begin{Shaded}
\begin{Highlighting}[]
\NormalTok{temperaturen }\OperatorTok{=}\NormalTok{ [}\FloatTok{14.2}\NormalTok{, }\FloatTok{17.5}\NormalTok{, }\FloatTok{19.0}\NormalTok{, }\FloatTok{21.3}\NormalTok{, }\FloatTok{18.4}\NormalTok{]}

\ControlFlowTok{for}\NormalTok{ t }\KeywordTok{in}\NormalTok{ temperaturen:}
    \ControlFlowTok{if}\NormalTok{ t }\OperatorTok{\textgreater{}} \DecValTok{18}\NormalTok{:}
        \BuiltInTok{print}\NormalTok{(t, }\StringTok{"ist ein warmer Tag"}\NormalTok{)}
\end{Highlighting}
\end{Shaded}

\begin{verbatim}
19.0 ist ein warmer Tag
21.3 ist ein warmer Tag
18.4 ist ein warmer Tag
\end{verbatim}

\section{Durchschnitt berechnen}\label{durchschnitt-berechnen}

Python stellt nützliche Funktionen bereit, z.\,B. \texttt{sum()} und
\texttt{len()}:

\begin{Shaded}
\begin{Highlighting}[]
\NormalTok{noten }\OperatorTok{=}\NormalTok{ [}\FloatTok{1.7}\NormalTok{, }\FloatTok{2.3}\NormalTok{, }\FloatTok{1.3}\NormalTok{, }\FloatTok{2.0}\NormalTok{]}

\NormalTok{durchschnitt }\OperatorTok{=} \BuiltInTok{sum}\NormalTok{(noten) }\OperatorTok{/} \BuiltInTok{len}\NormalTok{(noten)}
\BuiltInTok{print}\NormalTok{(}\StringTok{"Durchschnittsnote:"}\NormalTok{, }\BuiltInTok{round}\NormalTok{(durchschnitt, }\DecValTok{2}\NormalTok{))}
\end{Highlighting}
\end{Shaded}

\begin{verbatim}
Durchschnittsnote: 1.82
\end{verbatim}

\section{\texorpdfstring{Listen erweitern:
\texttt{.append()}}{Listen erweitern: .append()}}\label{listen-erweitern-.append}

Manchmal kennen Sie die Listenelemente nicht vorher -- dann können Sie
neue Werte \textbf{nachträglich hinzufügen}:

\begin{Shaded}
\begin{Highlighting}[]
\NormalTok{namen }\OperatorTok{=}\NormalTok{ []}

\NormalTok{namen.append(}\StringTok{"Ali"}\NormalTok{)}
\NormalTok{namen.append(}\StringTok{"Bente"}\NormalTok{)}

\BuiltInTok{print}\NormalTok{(namen)}
\end{Highlighting}
\end{Shaded}

\begin{verbatim}
['Ali', 'Bente']
\end{verbatim}

\begin{tcolorbox}[enhanced jigsaw, left=2mm, leftrule=.75mm, bottomrule=.15mm, title={Note}, colback=white, arc=.35mm, breakable, titlerule=0mm, bottomtitle=1mm, colbacktitle=quarto-callout-note-color!10!white, toprule=.15mm, opacityback=0, coltitle=black, rightrule=.15mm, opacitybacktitle=0.6, toptitle=1mm, colframe=quarto-callout-note-color-frame]

Die Methode \texttt{.append())} hängt einen neuen Wert an das Ende der
Liste.

\end{tcolorbox}

\section{Verschachtelte Schleifen}\label{verschachtelte-schleifen}

Wenn Sie mit \textbf{mehrdimensionalen Daten} arbeiten -- z.\,B. eine
Tabelle mit mehreren Zeilen -- können Sie Schleifen \textbf{ineinander
verschachteln}:

\begin{Shaded}
\begin{Highlighting}[]
\NormalTok{wochentage }\OperatorTok{=}\NormalTok{ [}\StringTok{"Mo"}\NormalTok{, }\StringTok{"Di"}\NormalTok{, }\StringTok{"Mi"}\NormalTok{]}
\NormalTok{stunden }\OperatorTok{=}\NormalTok{ [}\DecValTok{1}\NormalTok{, }\DecValTok{2}\NormalTok{, }\DecValTok{3}\NormalTok{]}

\ControlFlowTok{for}\NormalTok{ tag }\KeywordTok{in}\NormalTok{ wochentage:}
    \ControlFlowTok{for}\NormalTok{ stunde }\KeywordTok{in}\NormalTok{ stunden:}
        \BuiltInTok{print}\NormalTok{(}\SpecialStringTok{f"}\SpecialCharTok{\{}\NormalTok{tag}\SpecialCharTok{\}}\SpecialStringTok{, Stunde }\SpecialCharTok{\{}\NormalTok{stunde}\SpecialCharTok{\}}\SpecialStringTok{"}\NormalTok{)}
\end{Highlighting}
\end{Shaded}

\begin{verbatim}
Mo, Stunde 1
Mo, Stunde 2
Mo, Stunde 3
Di, Stunde 1
Di, Stunde 2
Di, Stunde 3
Mi, Stunde 1
Mi, Stunde 2
Mi, Stunde 3
\end{verbatim}

Das ergibt:

\begin{verbatim}
Mo, Stunde 1
Mo, Stunde 2
Mo, Stunde 3
Di, Stunde 1
...
\end{verbatim}

\section{Listen sortieren}\label{listen-sortieren}

Mit \texttt{sorted()} können Sie Listen \textbf{alphabetisch oder
numerisch sortieren}:

\begin{Shaded}
\begin{Highlighting}[]
\NormalTok{namen }\OperatorTok{=}\NormalTok{ [}\StringTok{"Zoe"}\NormalTok{, }\StringTok{"Anna"}\NormalTok{, }\StringTok{"Ben"}\NormalTok{]}
\NormalTok{sortiert }\OperatorTok{=} \BuiltInTok{sorted}\NormalTok{(namen)}

\BuiltInTok{print}\NormalTok{(sortiert)}
\end{Highlighting}
\end{Shaded}

\begin{verbatim}
['Anna', 'Ben', 'Zoe']
\end{verbatim}

\begin{tcolorbox}[enhanced jigsaw, left=2mm, leftrule=.75mm, bottomrule=.15mm, title={Important}, colback=white, arc=.35mm, breakable, titlerule=0mm, bottomtitle=1mm, colbacktitle=quarto-callout-important-color!10!white, toprule=.15mm, opacityback=0, coltitle=black, rightrule=.15mm, opacitybacktitle=0.6, toptitle=1mm, colframe=quarto-callout-important-color-frame]

Die Original-Liste bleibt \textbf{unverändert}.\\
Wenn Sie die Liste direkt verändern möchten, geht das mit:

\begin{Shaded}
\begin{Highlighting}[]
\NormalTok{namen.sort()}
\end{Highlighting}
\end{Shaded}

\end{tcolorbox}

\chapter{Wiederverwendbarer Code mit
Funktionen}\label{wiederverwendbarer-code-mit-funktionen}

Stellen Sie sich vor, Sie müssen eine bestimmte Berechnung mehrfach im
Programm durchführen. Anstatt den Code jedes Mal neu zu schreiben,
können Sie ihn in einer \textbf{Funktion} bündeln.

Funktionen sind ein zentrales Werkzeug, um Code:

\begin{itemize}
\tightlist
\item
  übersichtlich,
\item
  wiederverwendbar und
\item
  testbar zu machen.
\end{itemize}

\section{Lernziele dieses Kapitels}\label{lernziele-dieses-kapitels-3}

Am Ende dieses Kapitels können Sie:

\begin{itemize}
\tightlist
\item
  eigene Funktionen mit \texttt{def} erstellen,
\item
  Parameter übergeben und Rückgabewerte nutzen,
\item
  Funktionen sinnvoll in Programmen einsetzen.
\end{itemize}

\section{Eine Funktion definieren}\label{eine-funktion-definieren}

Eine Funktion besteht aus folgenden Teilen:

\begin{enumerate}
\def\labelenumi{\arabic{enumi}.}
\tightlist
\item
  \textbf{Definition} mit \texttt{def}
\item
  \textbf{Funktionsname}
\item
  \textbf{Parameter in Klammern (optional)}
\item
  \textbf{Einrückung} für den Funktionskörper
\item
  (optional) \textbf{\texttt{return}-Anweisung}
\end{enumerate}

Beispiel:

\begin{Shaded}
\begin{Highlighting}[]
\KeywordTok{def}\NormalTok{ hallo(name}\OperatorTok{=}\StringTok{"Gast"}\NormalTok{):}
\NormalTok{    begruessung }\OperatorTok{=} \StringTok{"Hallo "} \OperatorTok{+}\NormalTok{ name }\OperatorTok{+} \StringTok{"!"}
    \ControlFlowTok{return}\NormalTok{ begruessung}
\end{Highlighting}
\end{Shaded}

Fangen wir mit dem ersten Stichwort an. Funktionen werden mit
\texttt{def} definiert und können beliebig oft aufgerufen werden:

\begin{Shaded}
\begin{Highlighting}[]
\KeywordTok{def}\NormalTok{ begruessung():}
    \BuiltInTok{print}\NormalTok{(}\StringTok{"Hallo und willkommen!"}\NormalTok{)}
\end{Highlighting}
\end{Shaded}

Sie wird erst ausgeführt, wenn Sie sie aufrufen:

\begin{Shaded}
\begin{Highlighting}[]
\NormalTok{begruessung()}
\end{Highlighting}
\end{Shaded}

\begin{verbatim}
Hallo und willkommen!
\end{verbatim}

\section{Parameter übergeben}\label{parameter-uxfcbergeben}

Funktionen können Eingabewerte (Parameter) erhalten:

\begin{Shaded}
\begin{Highlighting}[]
\KeywordTok{def}\NormalTok{ begruessung(name):}
    \BuiltInTok{print}\NormalTok{(}\StringTok{"Hallo"}\NormalTok{, name }\OperatorTok{+} \StringTok{"!"}\NormalTok{)}

\NormalTok{begruessung(}\StringTok{"Alex"}\NormalTok{)}
\end{Highlighting}
\end{Shaded}

\begin{verbatim}
Hallo Alex!
\end{verbatim}

\section{\texorpdfstring{Rückgabewerte mit
\texttt{return}}{Rückgabewerte mit return}}\label{ruxfcckgabewerte-mit-return}

Eine Funktion kann auch einen Wert \textbf{zurückgeben}:

\begin{Shaded}
\begin{Highlighting}[]
\KeywordTok{def}\NormalTok{ quadrat(zahl):}
    \ControlFlowTok{return}\NormalTok{ zahl }\OperatorTok{*}\NormalTok{ zahl}

\NormalTok{ergebnis }\OperatorTok{=}\NormalTok{ quadrat(}\DecValTok{5}\NormalTok{)}
\BuiltInTok{print}\NormalTok{(ergebnis)}
\end{Highlighting}
\end{Shaded}

\begin{verbatim}
25
\end{verbatim}

\section{Beispiel: Umrechnungen}\label{beispiel-umrechnungen}

\subsection{Euro zu US-Dollar}\label{euro-zu-us-dollar}

\begin{Shaded}
\begin{Highlighting}[]
\KeywordTok{def}\NormalTok{ euro\_zu\_usd(betrag\_euro):}
\NormalTok{    wechselkurs }\OperatorTok{=} \FloatTok{1.09}
    \ControlFlowTok{return}\NormalTok{ betrag\_euro }\OperatorTok{*}\NormalTok{ wechselkurs}

\BuiltInTok{print}\NormalTok{(}\StringTok{"20 € sind"}\NormalTok{, euro\_zu\_usd(}\DecValTok{20}\NormalTok{), }\StringTok{"US{-}Dollar."}\NormalTok{)}
\end{Highlighting}
\end{Shaded}

\begin{verbatim}
20 € sind 21.8 US-Dollar.
\end{verbatim}

\begin{tcolorbox}[enhanced jigsaw, left=2mm, leftrule=.75mm, bottomrule=.15mm, title=\textcolor{quarto-callout-tip-color}{\faLightbulb}\hspace{0.5em}{Aufgabe: Begrüßung mit Name}, colback=white, arc=.35mm, breakable, titlerule=0mm, bottomtitle=1mm, colbacktitle=quarto-callout-tip-color!10!white, toprule=.15mm, opacityback=0, coltitle=black, rightrule=.15mm, opacitybacktitle=0.6, toptitle=1mm, colframe=quarto-callout-tip-color-frame]

Erstellen Sie eine Funktion \texttt{begruesse(name)}, die den Namen in
einem Begrüßungstext verwendet:

\begin{Shaded}
\begin{Highlighting}[]
\NormalTok{Hallo Fatima, schön dich zu sehen!}
\end{Highlighting}
\end{Shaded}

\begin{tcolorbox}[enhanced jigsaw, left=2mm, leftrule=.75mm, bottomrule=.15mm, title={Lösung}, colback=white, arc=.35mm, breakable, titlerule=0mm, bottomtitle=1mm, colbacktitle=quarto-callout-caution-color!10!white, toprule=.15mm, opacityback=0, coltitle=black, rightrule=.15mm, opacitybacktitle=0.6, toptitle=1mm, colframe=quarto-callout-caution-color-frame]

\begin{Shaded}
\begin{Highlighting}[]
\KeywordTok{def}\NormalTok{ begruesse(name):}
    \BuiltInTok{print}\NormalTok{(}\StringTok{"Hallo"}\NormalTok{, name }\OperatorTok{+} \StringTok{", schön dich zu sehen!"}\NormalTok{)}

\NormalTok{begruesse(}\StringTok{"Fatima"}\NormalTok{)}
\end{Highlighting}
\end{Shaded}

\begin{verbatim}
Hallo Fatima, schön dich zu sehen!
\end{verbatim}

\end{tcolorbox}

\end{tcolorbox}

\begin{tcolorbox}[enhanced jigsaw, left=2mm, leftrule=.75mm, bottomrule=.15mm, title=\textcolor{quarto-callout-tip-color}{\faLightbulb}\hspace{0.5em}{Aufgabe: Temperaturumrechnung}, colback=white, arc=.35mm, breakable, titlerule=0mm, bottomtitle=1mm, colbacktitle=quarto-callout-tip-color!10!white, toprule=.15mm, opacityback=0, coltitle=black, rightrule=.15mm, opacitybacktitle=0.6, toptitle=1mm, colframe=quarto-callout-tip-color-frame]

Schreiben Sie eine Funktion, die Celsius in Fahrenheit umrechnet:

Formel: {[} F = C \times 1.8 + 32 {]}

\begin{tcolorbox}[enhanced jigsaw, left=2mm, leftrule=.75mm, bottomrule=.15mm, title={Lösung}, colback=white, arc=.35mm, breakable, titlerule=0mm, bottomtitle=1mm, colbacktitle=quarto-callout-caution-color!10!white, toprule=.15mm, opacityback=0, coltitle=black, rightrule=.15mm, opacitybacktitle=0.6, toptitle=1mm, colframe=quarto-callout-caution-color-frame]

\begin{Shaded}
\begin{Highlighting}[]
\KeywordTok{def}\NormalTok{ celsius\_zu\_fahrenheit(c):}
    \ControlFlowTok{return}\NormalTok{ c }\OperatorTok{*} \FloatTok{1.8} \OperatorTok{+} \DecValTok{32}

\BuiltInTok{print}\NormalTok{(celsius\_zu\_fahrenheit(}\DecValTok{20}\NormalTok{))}
\end{Highlighting}
\end{Shaded}

\begin{verbatim}
68.0
\end{verbatim}

\end{tcolorbox}

\end{tcolorbox}

\section{Parameter mit
Standardwerten}\label{parameter-mit-standardwerten}

Sie können Parametern \textbf{Standardwerte} zuweisen. So kann die
Funktion auch ohne Angabe eines Werts aufgerufen werden:

\begin{Shaded}
\begin{Highlighting}[]
\KeywordTok{def}\NormalTok{ begruessung(name}\OperatorTok{=}\StringTok{"Gast"}\NormalTok{):}
    \BuiltInTok{print}\NormalTok{(}\StringTok{"Hallo"}\NormalTok{, name }\OperatorTok{+} \StringTok{"!"}\NormalTok{)}

\NormalTok{begruessung()         }\CommentTok{\# Hallo Gast!}
\NormalTok{begruessung(}\StringTok{"Maria"}\NormalTok{)  }\CommentTok{\# Hallo Maria!}
\end{Highlighting}
\end{Shaded}

\begin{verbatim}
Hallo Gast!
Hallo Maria!
\end{verbatim}

\begin{tcolorbox}[enhanced jigsaw, left=2mm, leftrule=.75mm, bottomrule=.15mm, title=\textcolor{quarto-callout-note-color}{\faInfo}\hspace{0.5em}{\texttt{print()} vs.~\texttt{return}}, colback=white, arc=.35mm, breakable, titlerule=0mm, bottomtitle=1mm, colbacktitle=quarto-callout-note-color!10!white, toprule=.15mm, opacityback=0, coltitle=black, rightrule=.15mm, opacitybacktitle=0.6, toptitle=1mm, colframe=quarto-callout-note-color-frame]

Diese beiden Begriffe werden oft verwechselt:

\begin{longtable}[]{@{}ll@{}}
\toprule\noalign{}
Ausdruck & Bedeutung \\
\midrule\noalign{}
\endhead
\bottomrule\noalign{}
\endlastfoot
\texttt{print()} & zeigt einen Text auf dem Bildschirm \\
\texttt{return} & gibt einen Wert an den Aufrufer zurück \\
\end{longtable}

Beispiel:

\begin{Shaded}
\begin{Highlighting}[]
\KeywordTok{def}\NormalTok{ verdoppeln(x):}
    \ControlFlowTok{return}\NormalTok{ x }\OperatorTok{*} \DecValTok{2}

\CommentTok{\# Ausgabe sichtbar machen}
\BuiltInTok{print}\NormalTok{(verdoppeln(}\DecValTok{5}\NormalTok{))  }\CommentTok{\# Ausgabe: 10}
\end{Highlighting}
\end{Shaded}

\begin{verbatim}
10
\end{verbatim}

\end{tcolorbox}

\chapter{Arbeiten mit Dateien}\label{arbeiten-mit-dateien}

Programme arbeiten oft nicht nur mit Benutzereingaben, sondern auch mit
\textbf{Textdateien} -- zum Beispiel um Daten zu speichern oder zu
laden.

Python bietet einfache Funktionen, um:

\begin{itemize}
\tightlist
\item
  Dateien \textbf{zu öffnen},
\item
  ihren \textbf{Inhalt zu lesen} oder \textbf{hineinzuschreiben},
\item
  und die Datei \textbf{wieder zu schließen}.
\end{itemize}

\section{Lernziele dieses Kapitels}\label{lernziele-dieses-kapitels-4}

Am Ende dieses Kapitels können Sie:

\begin{itemize}
\tightlist
\item
  Dateien mit \texttt{open()} öffnen,
\item
  Inhalte aus Textdateien einlesen,
\item
  Texte in Dateien schreiben,
\item
  mit \texttt{with}-Blöcken sicher und einfach arbeiten.
\end{itemize}

\section{Eine Datei einlesen}\label{eine-datei-einlesen}

\begin{Shaded}
\begin{Highlighting}[]
\CommentTok{\# Beispiel: Datei lesen}
\ControlFlowTok{with} \BuiltInTok{open}\NormalTok{(}\StringTok{"01{-}daten/beispiel.txt"}\NormalTok{, }\StringTok{"r"}\NormalTok{) }\ImportTok{as}\NormalTok{ datei:}
\NormalTok{    inhalt }\OperatorTok{=}\NormalTok{ datei.read()}
    \BuiltInTok{print}\NormalTok{(inhalt)}
\end{Highlighting}
\end{Shaded}

\begin{verbatim}
Dies ist ein Test.
\end{verbatim}

\begin{itemize}
\tightlist
\item
  \texttt{"r"} steht für \textbf{read} (lesen).
\item
  \texttt{with} sorgt dafür, dass die Datei nach dem Lesen automatisch
  geschlossen wird.
\item
  \texttt{read()} liest den \textbf{gesamten Inhalt} der Datei als
  String.
\end{itemize}

\section{Zeilenweise lesen}\label{zeilenweise-lesen}

\begin{Shaded}
\begin{Highlighting}[]
\ControlFlowTok{with} \BuiltInTok{open}\NormalTok{(}\StringTok{"01{-}daten/beispiel.txt"}\NormalTok{, }\StringTok{"r"}\NormalTok{) }\ImportTok{as}\NormalTok{ datei:}
    \ControlFlowTok{for}\NormalTok{ zeile }\KeywordTok{in}\NormalTok{ datei:}
        \BuiltInTok{print}\NormalTok{(}\StringTok{"Zeile:"}\NormalTok{, zeile.strip())}
\end{Highlighting}
\end{Shaded}

\begin{verbatim}
Zeile: Dies ist ein Test.
\end{verbatim}

\begin{tcolorbox}[enhanced jigsaw, left=2mm, leftrule=.75mm, bottomrule=.15mm, title=\textcolor{quarto-callout-note-color}{\faInfo}\hspace{0.5em}{Note}, colback=white, arc=.35mm, breakable, titlerule=0mm, bottomtitle=1mm, colbacktitle=quarto-callout-note-color!10!white, toprule=.15mm, opacityback=0, coltitle=black, rightrule=.15mm, opacitybacktitle=0.6, toptitle=1mm, colframe=quarto-callout-note-color-frame]

\texttt{.strip()} entfernt Leerzeichen und Zeilenumbrüche am Anfang und
Ende.

\end{tcolorbox}

\begin{tcolorbox}[enhanced jigsaw, left=2mm, leftrule=.75mm, bottomrule=.15mm, title=\textcolor{quarto-callout-tip-color}{\faLightbulb}\hspace{0.5em}{Aufgabe: Datei lesen}, colback=white, arc=.35mm, breakable, titlerule=0mm, bottomtitle=1mm, colbacktitle=quarto-callout-tip-color!10!white, toprule=.15mm, opacityback=0, coltitle=black, rightrule=.15mm, opacitybacktitle=0.6, toptitle=1mm, colframe=quarto-callout-tip-color-frame]

Angenommen, es gibt eine Datei \texttt{gruesse.txt} mit folgendem
Inhalt:

\begin{verbatim}
Hallo Anna
Guten Morgen Ben
Willkommen Carla
\end{verbatim}

Schreiben Sie ein Programm, das jede Zeile einzeln einliest und mit
\texttt{print(...)} wiedergibt.

\begin{tcolorbox}[enhanced jigsaw, left=2mm, leftrule=.75mm, bottomrule=.15mm, title={Lösung}, colback=white, arc=.35mm, breakable, titlerule=0mm, bottomtitle=1mm, colbacktitle=quarto-callout-caution-color!10!white, toprule=.15mm, opacityback=0, coltitle=black, rightrule=.15mm, opacitybacktitle=0.6, toptitle=1mm, colframe=quarto-callout-caution-color-frame]

\begin{Shaded}
\begin{Highlighting}[]
\ControlFlowTok{with} \BuiltInTok{open}\NormalTok{(}\StringTok{"01{-}daten/gruesse.txt"}\NormalTok{, }\StringTok{"r"}\NormalTok{) }\ImportTok{as}\NormalTok{ f:}
    \ControlFlowTok{for}\NormalTok{ zeile }\KeywordTok{in}\NormalTok{ f:}
        \BuiltInTok{print}\NormalTok{(zeile.strip())}
\end{Highlighting}
\end{Shaded}

\begin{verbatim}
Hallo Anna
Guten Morgen Ben
Willkommen Carla
\end{verbatim}

\end{tcolorbox}

\end{tcolorbox}

\section{\texorpdfstring{Alle Zeilen auf einmal lesen mit
\texttt{readlines()}}{Alle Zeilen auf einmal lesen mit readlines()}}\label{alle-zeilen-auf-einmal-lesen-mit-readlines}

Statt über eine Datei zu iterieren, können Sie alle Zeilen auf einmal
als Liste einlesen:

\begin{Shaded}
\begin{Highlighting}[]
\ControlFlowTok{with} \BuiltInTok{open}\NormalTok{(}\StringTok{"01{-}daten/beispiel.txt"}\NormalTok{, }\StringTok{"r"}\NormalTok{) }\ImportTok{as}\NormalTok{ f:}
\NormalTok{    zeilen }\OperatorTok{=}\NormalTok{ f.readlines()}
    \BuiltInTok{print}\NormalTok{(zeilen)}
\end{Highlighting}
\end{Shaded}

\begin{verbatim}
['Dies ist ein Test.']
\end{verbatim}

\begin{tcolorbox}[enhanced jigsaw, left=2mm, leftrule=.75mm, bottomrule=.15mm, title=\textcolor{quarto-callout-important-color}{\faExclamation}\hspace{0.5em}{Important}, colback=white, arc=.35mm, breakable, titlerule=0mm, bottomtitle=1mm, colbacktitle=quarto-callout-important-color!10!white, toprule=.15mm, opacityback=0, coltitle=black, rightrule=.15mm, opacitybacktitle=0.6, toptitle=1mm, colframe=quarto-callout-important-color-frame]

Oftmals bestehen die eingelesenen Zeilen aus Werten, die durch ein
spezifisches Trennzeichen z.B. ``,''. Um diese Zeilen dann in die
einzelnen Werte zu trennen, benutzen wir die \texttt{.split()}-Funktion.

\begin{Shaded}
\begin{Highlighting}[]
\NormalTok{zeile }\OperatorTok{=} \StringTok{"319,12,14,190,342"}
\BuiltInTok{print}\NormalTok{(zeile.split(}\StringTok{","}\NormalTok{))}
\end{Highlighting}
\end{Shaded}

\begin{verbatim}
['319', '12', '14', '190', '342']
\end{verbatim}

Jede Zeile endet mit \texttt{\textbackslash{}n}, deshalb kann eine
Nachbearbeitung mit \texttt{.strip()} sinnvoll sein:

\begin{Shaded}
\begin{Highlighting}[]
\ControlFlowTok{for}\NormalTok{ zeile }\KeywordTok{in}\NormalTok{ zeilen:}
    \BuiltInTok{print}\NormalTok{(zeile.strip())}
\end{Highlighting}
\end{Shaded}

\begin{verbatim}
Dies ist ein Test.
\end{verbatim}

\end{tcolorbox}

\section{In eine Datei schreiben}\label{in-eine-datei-schreiben}

\begin{Shaded}
\begin{Highlighting}[]
\ControlFlowTok{with} \BuiltInTok{open}\NormalTok{(}\StringTok{"ausgabe.txt"}\NormalTok{, }\StringTok{"w"}\NormalTok{) }\ImportTok{as}\NormalTok{ datei:}
\NormalTok{    datei.write(}\StringTok{"Das ist eine neue Zeile.}\CharTok{\textbackslash{}n}\StringTok{"}\NormalTok{)}
\NormalTok{    datei.write(}\StringTok{"Und noch eine."}\NormalTok{)}
\end{Highlighting}
\end{Shaded}

\begin{itemize}
\tightlist
\item
  \texttt{"w"} steht für \textbf{write} (schreiben).
\item
  Achtung: Eine vorhandene Datei wird \textbf{überschrieben}!
\end{itemize}

\section{Zeilenweise schreiben mit
Schleife}\label{zeilenweise-schreiben-mit-schleife}

\begin{Shaded}
\begin{Highlighting}[]
\NormalTok{daten }\OperatorTok{=}\NormalTok{ [}\StringTok{"Apfel"}\NormalTok{, }\StringTok{"Banane"}\NormalTok{, }\StringTok{"Kirsche"}\NormalTok{]}

\ControlFlowTok{with} \BuiltInTok{open}\NormalTok{(}\StringTok{"obst.txt"}\NormalTok{, }\StringTok{"w"}\NormalTok{) }\ImportTok{as}\NormalTok{ f:}
    \ControlFlowTok{for}\NormalTok{ eintrag }\KeywordTok{in}\NormalTok{ daten:}
\NormalTok{        f.write(eintrag }\OperatorTok{+} \StringTok{"}\CharTok{\textbackslash{}n}\StringTok{"}\NormalTok{)}
\end{Highlighting}
\end{Shaded}

\begin{tcolorbox}[enhanced jigsaw, left=2mm, leftrule=.75mm, bottomrule=.15mm, title=\textcolor{quarto-callout-important-color}{\faExclamation}\hspace{0.5em}{Important}, colback=white, arc=.35mm, breakable, titlerule=0mm, bottomtitle=1mm, colbacktitle=quarto-callout-important-color!10!white, toprule=.15mm, opacityback=0, coltitle=black, rightrule=.15mm, opacitybacktitle=0.6, toptitle=1mm, colframe=quarto-callout-important-color-frame]

Jede Zeile endet mit \texttt{\textbackslash{}n} für einen Zeilenumbruch.

\end{tcolorbox}

\begin{tcolorbox}[enhanced jigsaw, left=2mm, leftrule=.75mm, bottomrule=.15mm, title=\textcolor{quarto-callout-tip-color}{\faLightbulb}\hspace{0.5em}{Aufgabe: Liste in Datei schreiben}, colback=white, arc=.35mm, breakable, titlerule=0mm, bottomtitle=1mm, colbacktitle=quarto-callout-tip-color!10!white, toprule=.15mm, opacityback=0, coltitle=black, rightrule=.15mm, opacitybacktitle=0.6, toptitle=1mm, colframe=quarto-callout-tip-color-frame]

Gegeben ist eine Liste von Städten:

\begin{Shaded}
\begin{Highlighting}[]
\NormalTok{staedte }\OperatorTok{=}\NormalTok{ [}\StringTok{"Berlin"}\NormalTok{, }\StringTok{"Hamburg"}\NormalTok{, }\StringTok{"München"}\NormalTok{]}
\end{Highlighting}
\end{Shaded}

\begin{itemize}
\tightlist
\item
  Schreiben Sie ein Programm, das jede Stadt in eine neue Zeile einer
  Datei \texttt{staedte.txt} schreibt.
\end{itemize}

\begin{tcolorbox}[enhanced jigsaw, left=2mm, leftrule=.75mm, bottomrule=.15mm, title={Lösung}, colback=white, arc=.35mm, breakable, titlerule=0mm, bottomtitle=1mm, colbacktitle=quarto-callout-caution-color!10!white, toprule=.15mm, opacityback=0, coltitle=black, rightrule=.15mm, opacitybacktitle=0.6, toptitle=1mm, colframe=quarto-callout-caution-color-frame]

\begin{Shaded}
\begin{Highlighting}[]
\NormalTok{staedte }\OperatorTok{=}\NormalTok{ [}\StringTok{"Berlin"}\NormalTok{, }\StringTok{"Hamburg"}\NormalTok{, }\StringTok{"München"}\NormalTok{]}

\ControlFlowTok{with} \BuiltInTok{open}\NormalTok{(}\StringTok{"staedte.txt"}\NormalTok{, }\StringTok{"w"}\NormalTok{) }\ImportTok{as}\NormalTok{ f:}
    \ControlFlowTok{for}\NormalTok{ stadt }\KeywordTok{in}\NormalTok{ staedte:}
\NormalTok{        f.write(stadt }\OperatorTok{+} \StringTok{"}\CharTok{\textbackslash{}n}\StringTok{"}\NormalTok{)}
\end{Highlighting}
\end{Shaded}

\end{tcolorbox}

\end{tcolorbox}

\section{\texorpdfstring{Dateien manuell schließen mit
\texttt{close()}}{Dateien manuell schließen mit close()}}\label{dateien-manuell-schlieuxdfen-mit-close}

Wenn Sie \textbf{keinen \texttt{with}-Block} verwenden, müssen Sie die
Datei selbst schließen -- sonst bleibt sie geöffnet:

\begin{Shaded}
\begin{Highlighting}[]
\NormalTok{datei }\OperatorTok{=} \BuiltInTok{open}\NormalTok{(}\StringTok{"01{-}daten/beispiel.txt"}\NormalTok{, }\StringTok{"w"}\NormalTok{)}
\NormalTok{datei.write(}\StringTok{"Dies ist ein Test."}\NormalTok{)}
\NormalTok{datei.close()}
\end{Highlighting}
\end{Shaded}

\begin{tcolorbox}[enhanced jigsaw, left=2mm, leftrule=.75mm, bottomrule=.15mm, title=\textcolor{quarto-callout-important-color}{\faExclamation}\hspace{0.5em}{Important}, colback=white, arc=.35mm, breakable, titlerule=0mm, bottomtitle=1mm, colbacktitle=quarto-callout-important-color!10!white, toprule=.15mm, opacityback=0, coltitle=black, rightrule=.15mm, opacitybacktitle=0.6, toptitle=1mm, colframe=quarto-callout-important-color-frame]

\texttt{close()} ist wichtig, damit Änderungen gespeichert werden und
die Datei nicht gesperrt bleibt.

\textbf{Empfehlung}: Nutzen Sie immer \texttt{with\ open(...)}, da
Python die Datei dann automatisch schließt -- auch bei Fehlern.

\end{tcolorbox}

\chapter{Lernzielkontrolle}\label{lernzielkontrolle}

\section{Aufgabe 1: Datentypen und
Typecasting}\label{aufgabe-1-datentypen-und-typecasting}

Gegeben sind folgende Werte:

\begin{itemize}
\tightlist
\item
  \texttt{wert1\ =\ "42"}
\item
  \texttt{wert2\ =\ 3.5}
\item
  \texttt{wert3\ =\ True}
\end{itemize}

Gib mit \texttt{type()} den Datentyp jedes Werts aus und wandle jeden in
einen anderen sinnvollen Typ um. Gib die Ergebnisse aus.

\begin{center}\rule{0.5\linewidth}{0.5pt}\end{center}

\section{Aufgabe 2: Altersprüfung mit
Bedingung}\label{aufgabe-2-alterspruxfcfung-mit-bedingung}

Gegeben sei \texttt{alter\ =\ 16}.\\
Gib abhängig vom Alter aus:

\begin{itemize}
\tightlist
\item
  „Volljährig``, wenn das Alter ≥ 18 ist
\item
  „Minderjährig``, wenn das Alter \textless{} 18 ist
\end{itemize}

\begin{center}\rule{0.5\linewidth}{0.5pt}\end{center}

\section{Aufgabe 3: Zahlen filtern und
mitteln}\label{aufgabe-3-zahlen-filtern-und-mitteln}

Gegeben sei die Liste:

\begin{Shaded}
\begin{Highlighting}[]
\NormalTok{zahlen }\OperatorTok{=}\NormalTok{ [}\DecValTok{5}\NormalTok{, }\DecValTok{8}\NormalTok{, }\DecValTok{13}\NormalTok{, }\DecValTok{20}\NormalTok{, }\DecValTok{33}\NormalTok{, }\DecValTok{40}\NormalTok{]}
\end{Highlighting}
\end{Shaded}

\begin{itemize}
\tightlist
\item
  Gib alle Zahlen \textgreater{} 10 aus.
\item
  Berechne und gib den Durchschnitt dieser Zahlen aus.
\end{itemize}

\begin{center}\rule{0.5\linewidth}{0.5pt}\end{center}

\section{Aufgabe 4: Wochentage \&
Slicing}\label{aufgabe-4-wochentage-slicing}

Gegeben sei:

\begin{Shaded}
\begin{Highlighting}[]
\NormalTok{tage }\OperatorTok{=}\NormalTok{ [}\StringTok{"Mo"}\NormalTok{, }\StringTok{"Di"}\NormalTok{, }\StringTok{"Mi"}\NormalTok{, }\StringTok{"Do"}\NormalTok{, }\StringTok{"Fr"}\NormalTok{, }\StringTok{"Sa"}\NormalTok{, }\StringTok{"So"}\NormalTok{]}
\end{Highlighting}
\end{Shaded}

\begin{itemize}
\tightlist
\item
  Gib alle Wochentage aus, die mit „S`` beginnen.
\item
  Gib nur die Arbeitstage (Mo--Fr) mit Slicing aus.
\end{itemize}

\begin{center}\rule{0.5\linewidth}{0.5pt}\end{center}

\section{Aufgabe 5: Funktion mit Parameter \&
Standardwert}\label{aufgabe-5-funktion-mit-parameter-standardwert}

Erstelle eine Funktion \texttt{begruesse(name,\ sprache="de")}, die je
nach Sprache Folgendes ausgibt:

\begin{itemize}
\tightlist
\item
  Deutsch: „Hallo !{}``
\item
  Englisch: „Hello !{}``
\end{itemize}

\begin{center}\rule{0.5\linewidth}{0.5pt}\end{center}

\section{Aufgabe 6: Hobbys in Datei
schreiben}\label{aufgabe-6-hobbys-in-datei-schreiben}

Gegeben sei eine Liste
\texttt{hobbys\ =\ {[}"Lesen",\ "Kochen",\ "Sport"{]}}.

Schreibe jeden Eintrag in eine neue Zeile der Datei \texttt{hobbys.txt}.

\begin{center}\rule{0.5\linewidth}{0.5pt}\end{center}

\section{Aufgabe 7: Datei lesen \& .split()
verwenden}\label{aufgabe-7-datei-lesen-.split-verwenden}

Angenommen, eine Datei enthält die Zeile:

\begin{verbatim}
Ali,Bente,Carlos,Dana
\end{verbatim}

\begin{itemize}
\tightlist
\item
  Zerlege die Zeichenkette mit \texttt{.split(",")}.
\item
  Gib jeden Namen einzeln aus.
\end{itemize}

\begin{center}\rule{0.5\linewidth}{0.5pt}\end{center}

\section{Aufgabe 8: Namen sortieren und
speichern}\label{aufgabe-8-namen-sortieren-und-speichern}

Gegeben sei:

\begin{Shaded}
\begin{Highlighting}[]
\NormalTok{namen }\OperatorTok{=}\NormalTok{ [}\StringTok{"Zoe"}\NormalTok{, }\StringTok{"Anna"}\NormalTok{, }\StringTok{"Lukas"}\NormalTok{, }\StringTok{"Ben"}\NormalTok{]}
\end{Highlighting}
\end{Shaded}

\begin{itemize}
\tightlist
\item
  Sortiere die Liste alphabetisch.
\item
  Speichere die sortierte Liste in eine Datei \texttt{sortiert.txt}, ein
  Name pro Zeile.
\end{itemize}

\begin{center}\rule{0.5\linewidth}{0.5pt}\end{center}

\section{Aufgabe 9: Zahlen durch 3
ausgeben}\label{aufgabe-9-zahlen-durch-3-ausgeben}

Erstelle mit einer Schleife eine Liste aller Zahlen zwischen 1 und 20,
die durch 3 teilbar sind, und gib sie aus.

\begin{center}\rule{0.5\linewidth}{0.5pt}\end{center}

\section{Aufgabe 10: Verschachtelte
Schleifen}\label{aufgabe-10-verschachtelte-schleifen}

Gegeben seien:

\begin{Shaded}
\begin{Highlighting}[]
\NormalTok{personen }\OperatorTok{=}\NormalTok{ [}\StringTok{"Ali"}\NormalTok{, }\StringTok{"Bente"}\NormalTok{]}
\NormalTok{hobbys }\OperatorTok{=}\NormalTok{ [}\StringTok{"Lesen"}\NormalTok{, }\StringTok{"Sport"}\NormalTok{]}
\end{Highlighting}
\end{Shaded}

Erstelle eine Ausgabe wie:

\begin{verbatim}
Ali hat das Hobby: Lesen
Ali hat das Hobby: Sport
Bente hat das Hobby: Lesen
Bente hat das Hobby: Sport
\end{verbatim}

\part{Matplotlib}

\chapter*{Preamble}\label{preamble}
\addcontentsline{toc}{chapter}{Preamble}

\markboth{Preamble}{Preamble}

\phantomsection\label{Lizenz}
\begin{figure}

\begin{minipage}{0.20\linewidth}
\includegraphics{index_files/mediabag/by.png}\end{minipage}%
%
\begin{minipage}{0.80\linewidth}
Bausteine Computergestützter Datenanalyse. ``Werkzeugbaustein Plotting
in Python'' von Lukas Arnold, Simone Arnold, Florian Bagemihl, Matthias
Baitsch, Marc Fehr, Maik Poetzsch und Sebastian Seipel ist lizensiert
unter \href{https://creativecommons.org/licenses/by/4.0/deed.de}{CC BY
4.0}. Das Werk ist abrufbar unter
\url{https://github.com/bausteine-der-datenanalyse/w-python-plotting}.
Ausgenommen von der Lizenz sind alle Logos und anders gekennzeichneten
Inhalte. 2024\end{minipage}%

\end{figure}%

Zitiervorschlag

Arnold, Lukas, Simone Arnold, Matthias Baitsch, Marc Fehr, Maik
Poetzsch, und Sebastian Seipel. 2024. „Bausteine Computergestützter
Datenanalyse. Werkzeugbaustein Plotting in Python``.
\url{https://github.com/bausteine-der-datenanalyse/w-python-plotting}.

BibTeX-Vorlage

\begin{verbatim}
@misc{BCD-Styleguide-2024,
 title={Bausteine Computergestützter Datenanalyse. Werkzeugbaustein Plotting in Python},
 author={Arnold, Lukas and Arnold, Simone and Baitsch, Matthias and Fehr, Marc and Poetzsch, Maik and Seipel, Sebastian},
 year={2024},
 url={https://github.com/bausteine-der-datenanalyse/w-python-plotting}} 
\end{verbatim}

\chapter*{Intro}\label{intro}
\addcontentsline{toc}{chapter}{Intro}

\markboth{Intro}{Intro}

\section*{Voraussetzungen}\label{voraussetzungen-1}
\addcontentsline{toc}{section}{Voraussetzungen}

\markright{Voraussetzungen}

\begin{itemize}
\tightlist
\item
  Grundlagen Python
\item
  Einbinden von zusätzlichen Paketen
\end{itemize}

\section*{Verwendete Pakete und
Datensätze}\label{verwendete-pakete-und-datensuxe4tze}
\addcontentsline{toc}{section}{Verwendete Pakete und Datensätze}

\markright{Verwendete Pakete und Datensätze}

\begin{itemize}
\tightlist
\item
  matplotlib
\end{itemize}

\section*{Bearbeitungszeit}\label{bearbeitungszeit}
\addcontentsline{toc}{section}{Bearbeitungszeit}

\markright{Bearbeitungszeit}

Geschätzte Bearbeitungszeit: 1h

\section*{Lernziele}\label{lernziele-1}
\addcontentsline{toc}{section}{Lernziele}

\markright{Lernziele}

\begin{itemize}
\tightlist
\item
  Einleitung: wie visualisiere ich Daten in Python
\item
  Anpassen von Plots
\item
  Do's \& Dont's für wissenschaftliche Plots
\end{itemize}

\chapter{Einführung in Matplotlib}\label{einfuxfchrung-in-matplotlib}

Matplotlib ist eine der bekanntesten Bibliotheken zur
Datenvisualisierung in Python. Sie ermöglicht das Erstellen statischer,
animierter und interaktiver Diagramme mit hoher Flexibilität.

\section{Warum Matplotlib?}\label{warum-matplotlib}

\begin{itemize}
\tightlist
\item
  \textbf{Breite Unterstützung:} Funktioniert mit NumPy, Pandas und
  SciPy.
\item
  \textbf{Hohe Anpassbarkeit:} Vollständige Kontrolle über Diagramme.
\item
  \textbf{Integration in Jupyter Notebooks:} Ideal für interaktive
  Datenanalyse.
\item
  \textbf{Kompatibilität:} Unterstützt verschiedene Ausgabeformate (PNG,
  SVG, PDF etc.).
\end{itemize}

\section{Alternativen zu Matplotlib}\label{alternativen-zu-matplotlib}

Während Matplotlib leistungsstark ist, gibt es Alternativen, die für
bestimmte Zwecke besser geeignet sein können: - \textbf{Seaborn:}
Basiert auf Matplotlib, erleichtert statistische Visualisierung. -
\textbf{Plotly:} Erzeugt interaktive Plots, gut für Dashboards. -
\textbf{Bokeh:} Ideal für Web-Anwendungen mit interaktiven
Visualisierungen.

\section{Erstes Beispiel: Einfache Linie
plotten}\label{erstes-beispiel-einfache-linie-plotten}

\begin{Shaded}
\begin{Highlighting}[]
\ImportTok{import}\NormalTok{ matplotlib.pyplot }\ImportTok{as}\NormalTok{ plt}
\ImportTok{import}\NormalTok{ numpy }\ImportTok{as}\NormalTok{ np}

\CommentTok{\# Beispiel{-}Daten}
\NormalTok{t }\OperatorTok{=}\NormalTok{ np.linspace(}\DecValTok{0}\NormalTok{, }\DecValTok{10}\NormalTok{, }\DecValTok{100}\NormalTok{)}
\NormalTok{y }\OperatorTok{=}\NormalTok{ np.sin(t)}

\CommentTok{\# Erstellen des Plots}
\NormalTok{plt.plot(t, y, label}\OperatorTok{=}\StringTok{\textquotesingle{}sin(t)\textquotesingle{}}\NormalTok{)}
\NormalTok{plt.xlabel(}\StringTok{\textquotesingle{}Zeit (s)\textquotesingle{}}\NormalTok{)}
\NormalTok{plt.ylabel(}\StringTok{\textquotesingle{}Amplitude\textquotesingle{}}\NormalTok{)}
\NormalTok{plt.title(}\StringTok{\textquotesingle{}Einfaches Linien{-}Diagramm\textquotesingle{}}\NormalTok{)}
\NormalTok{plt.legend()}
\NormalTok{plt.show()}
\end{Highlighting}
\end{Shaded}

Dieses einfache Beispiel zeigt, wie man mit Matplotlib eine
\textbf{Sinuskurve} visualisieren kann.

\section{Nächste Schritte}\label{nuxe4chste-schritte}

Im nächsten Kapitel werden wir uns mit den verschiedenen Diagrammtypen
beschäftigen, die Matplotlib bietet.

\chapter{Diagrammtypen in Matplotlib}\label{diagrammtypen-in-matplotlib}

Matplotlib bietet eine Vielzahl von Diagrammtypen, die für
unterschiedliche Zwecke geeignet sind. In diesem Kapitel werden die
wichtigsten Diagrammtypen vorgestellt und ihre Anwendungsfälle erklärt.

\section{\texorpdfstring{1. Liniendiagramme
(\texttt{plt.plot()})}{1. Liniendiagramme (plt.plot())}}\label{liniendiagramme-plt.plot}

Liniendiagramme eignen sich hervorragend zur Darstellung von Trends über
Zeit.

\begin{Shaded}
\begin{Highlighting}[]
\ImportTok{import}\NormalTok{ matplotlib.pyplot }\ImportTok{as}\NormalTok{ plt}
\ImportTok{import}\NormalTok{ numpy }\ImportTok{as}\NormalTok{ np}

\NormalTok{t }\OperatorTok{=}\NormalTok{ np.linspace(}\DecValTok{0}\NormalTok{, }\DecValTok{10}\NormalTok{, }\DecValTok{100}\NormalTok{)}
\NormalTok{y }\OperatorTok{=}\NormalTok{ np.sin(t)}

\NormalTok{plt.plot(t, y, label}\OperatorTok{=}\StringTok{\textquotesingle{}sin(t)\textquotesingle{}}\NormalTok{, color}\OperatorTok{=}\StringTok{\textquotesingle{}b\textquotesingle{}}\NormalTok{)}
\NormalTok{plt.xlabel(}\StringTok{\textquotesingle{}Zeit (s)\textquotesingle{}}\NormalTok{)}
\NormalTok{plt.ylabel(}\StringTok{\textquotesingle{}Amplitude\textquotesingle{}}\NormalTok{)}
\NormalTok{plt.title(}\StringTok{\textquotesingle{}Liniendiagramm\textquotesingle{}}\NormalTok{)}
\NormalTok{plt.legend()}
\NormalTok{plt.show()}
\end{Highlighting}
\end{Shaded}

\includegraphics{books/w-python-matplotlib/skript/basic_plot_types_files/figure-pdf/cell-2-output-1.png}

\section{\texorpdfstring{2. Streudiagramme
(\texttt{plt.scatter()})}{2. Streudiagramme (plt.scatter())}}\label{streudiagramme-plt.scatter}

Streudiagramme werden verwendet, um Zusammenhänge zwischen zwei
Variablen darzustellen.

\begin{Shaded}
\begin{Highlighting}[]
\NormalTok{x }\OperatorTok{=}\NormalTok{ np.random.rand(}\DecValTok{50}\NormalTok{)}
\NormalTok{y }\OperatorTok{=}\NormalTok{ np.random.rand(}\DecValTok{50}\NormalTok{)}

\NormalTok{plt.scatter(x, y, color}\OperatorTok{=}\StringTok{\textquotesingle{}r\textquotesingle{}}\NormalTok{, alpha}\OperatorTok{=}\FloatTok{0.5}\NormalTok{)}
\NormalTok{plt.xlabel(}\StringTok{\textquotesingle{}Variable X\textquotesingle{}}\NormalTok{)}
\NormalTok{plt.ylabel(}\StringTok{\textquotesingle{}Variable Y\textquotesingle{}}\NormalTok{)}
\NormalTok{plt.title(}\StringTok{\textquotesingle{}Streudiagramm\textquotesingle{}}\NormalTok{)}
\NormalTok{plt.show()}
\end{Highlighting}
\end{Shaded}

\includegraphics{books/w-python-matplotlib/skript/basic_plot_types_files/figure-pdf/cell-3-output-1.png}

\section{\texorpdfstring{3. Balkendiagramme
(\texttt{plt.bar()})}{3. Balkendiagramme (plt.bar())}}\label{balkendiagramme-plt.bar}

Balkendiagramme eignen sich zur Darstellung kategorialer Daten.

\begin{Shaded}
\begin{Highlighting}[]
\NormalTok{kategorien }\OperatorTok{=}\NormalTok{ [}\StringTok{\textquotesingle{}A\textquotesingle{}}\NormalTok{, }\StringTok{\textquotesingle{}B\textquotesingle{}}\NormalTok{, }\StringTok{\textquotesingle{}C\textquotesingle{}}\NormalTok{, }\StringTok{\textquotesingle{}D\textquotesingle{}}\NormalTok{]}
\NormalTok{werte }\OperatorTok{=}\NormalTok{ [}\DecValTok{3}\NormalTok{, }\DecValTok{7}\NormalTok{, }\DecValTok{1}\NormalTok{, }\DecValTok{5}\NormalTok{]}

\NormalTok{plt.bar(kategorien, werte, color}\OperatorTok{=}\StringTok{\textquotesingle{}g\textquotesingle{}}\NormalTok{)}
\NormalTok{plt.xlabel(}\StringTok{\textquotesingle{}Kategorien\textquotesingle{}}\NormalTok{)}
\NormalTok{plt.ylabel(}\StringTok{\textquotesingle{}Wert\textquotesingle{}}\NormalTok{)}
\NormalTok{plt.title(}\StringTok{\textquotesingle{}Balkendiagramm\textquotesingle{}}\NormalTok{)}
\NormalTok{plt.show()}
\end{Highlighting}
\end{Shaded}

\includegraphics{books/w-python-matplotlib/skript/basic_plot_types_files/figure-pdf/cell-4-output-1.png}

\section{\texorpdfstring{4. Histogramme
(\texttt{plt.hist()})}{4. Histogramme (plt.hist())}}\label{histogramme-plt.hist}

Histogramme zeigen die Verteilung numerischer Daten.

\begin{Shaded}
\begin{Highlighting}[]
\NormalTok{daten }\OperatorTok{=}\NormalTok{ np.random.randn(}\DecValTok{1000}\NormalTok{)}
\NormalTok{plt.hist(daten, bins}\OperatorTok{=}\DecValTok{30}\NormalTok{, color}\OperatorTok{=}\StringTok{\textquotesingle{}purple\textquotesingle{}}\NormalTok{, alpha}\OperatorTok{=}\FloatTok{0.7}\NormalTok{)}
\NormalTok{plt.xlabel(}\StringTok{\textquotesingle{}Wert\textquotesingle{}}\NormalTok{)}
\NormalTok{plt.ylabel(}\StringTok{\textquotesingle{}Häufigkeit\textquotesingle{}}\NormalTok{)}
\NormalTok{plt.title(}\StringTok{\textquotesingle{}Histogramm\textquotesingle{}}\NormalTok{)}
\NormalTok{plt.show()}
\end{Highlighting}
\end{Shaded}

\includegraphics{books/w-python-matplotlib/skript/basic_plot_types_files/figure-pdf/cell-5-output-1.png}

\section{\texorpdfstring{5. Boxplots
(\texttt{plt.boxplot()})}{5. Boxplots (plt.boxplot())}}\label{boxplots-plt.boxplot}

Boxplots helfen, Ausreißer und die Verteilung von Daten zu
visualisieren.

\begin{Shaded}
\begin{Highlighting}[]
\NormalTok{daten }\OperatorTok{=}\NormalTok{ [np.random.randn(}\DecValTok{100}\NormalTok{) }\ControlFlowTok{for}\NormalTok{ \_ }\KeywordTok{in} \BuiltInTok{range}\NormalTok{(}\DecValTok{4}\NormalTok{)]}
\NormalTok{plt.boxplot(daten, labels}\OperatorTok{=}\NormalTok{[}\StringTok{\textquotesingle{}A\textquotesingle{}}\NormalTok{, }\StringTok{\textquotesingle{}B\textquotesingle{}}\NormalTok{, }\StringTok{\textquotesingle{}C\textquotesingle{}}\NormalTok{, }\StringTok{\textquotesingle{}D\textquotesingle{}}\NormalTok{])}
\NormalTok{plt.ylabel(}\StringTok{\textquotesingle{}Wert\textquotesingle{}}\NormalTok{)}
\NormalTok{plt.title(}\StringTok{\textquotesingle{}Boxplot\textquotesingle{}}\NormalTok{)}
\NormalTok{plt.show()}
\end{Highlighting}
\end{Shaded}

\begin{verbatim}
/tmp/ipykernel_4865/2728911591.py:2: MatplotlibDeprecationWarning: The 'labels' parameter of boxplot() has been renamed 'tick_labels' since Matplotlib 3.9; support for the old name will be dropped in 3.11.
  plt.boxplot(daten, labels=['A', 'B', 'C', 'D'])
\end{verbatim}

\includegraphics{books/w-python-matplotlib/skript/basic_plot_types_files/figure-pdf/cell-6-output-2.png}

\section{\texorpdfstring{6. Heatmaps
(\texttt{plt.imshow()})}{6. Heatmaps (plt.imshow())}}\label{heatmaps-plt.imshow}

Heatmaps eignen sich zur Darstellung von 2D-Daten.

\begin{Shaded}
\begin{Highlighting}[]
\NormalTok{daten }\OperatorTok{=}\NormalTok{ np.random.rand(}\DecValTok{10}\NormalTok{, }\DecValTok{10}\NormalTok{)}
\NormalTok{plt.imshow(daten, cmap}\OperatorTok{=}\StringTok{\textquotesingle{}coolwarm\textquotesingle{}}\NormalTok{, interpolation}\OperatorTok{=}\StringTok{\textquotesingle{}nearest\textquotesingle{}}\NormalTok{)}
\NormalTok{plt.colorbar()}
\NormalTok{plt.title(}\StringTok{\textquotesingle{}Heatmap\textquotesingle{}}\NormalTok{)}
\NormalTok{plt.show()}
\end{Highlighting}
\end{Shaded}

\includegraphics{books/w-python-matplotlib/skript/basic_plot_types_files/figure-pdf/cell-7-output-1.png}

\section{Fazit}\label{fazit}

Die Wahl des richtigen Diagrammtyps hängt von der Art der Daten und der
gewünschten Darstellung ab. Im nächsten Kapitel werden wir uns mit der
Anpassung und Gestaltung von Plots beschäftigen.

\chapter{Anpassung und Gestaltung von Plots in
Matplotlib}\label{anpassung-und-gestaltung-von-plots-in-matplotlib}

Ein gut gestaltetes Diagramm verbessert die Lesbarkeit und
Verständlichkeit der dargestellten Daten. In diesem Kapitel werden wir
verschiedene Möglichkeiten zur Anpassung und Gestaltung von Plots in
Matplotlib erkunden.

\section{1. Achsentitel und
Diagrammtitel}\label{achsentitel-und-diagrammtitel}

Klare Achsen- und Diagrammtitel sind essenziell für die Verständlichkeit
eines Plots.

\begin{Shaded}
\begin{Highlighting}[]
\ImportTok{import}\NormalTok{ matplotlib.pyplot }\ImportTok{as}\NormalTok{ plt}
\ImportTok{import}\NormalTok{ numpy }\ImportTok{as}\NormalTok{ np}

\NormalTok{t }\OperatorTok{=}\NormalTok{ np.linspace(}\DecValTok{0}\NormalTok{, }\DecValTok{10}\NormalTok{, }\DecValTok{100}\NormalTok{)}
\NormalTok{y }\OperatorTok{=}\NormalTok{ np.sin(t)}

\NormalTok{plt.plot(t, y, label}\OperatorTok{=}\StringTok{\textquotesingle{}sin(t)\textquotesingle{}}\NormalTok{, color}\OperatorTok{=}\StringTok{\textquotesingle{}b\textquotesingle{}}\NormalTok{)}
\NormalTok{plt.xlabel(}\StringTok{\textquotesingle{}Zeit (s)\textquotesingle{}}\NormalTok{, fontsize}\OperatorTok{=}\DecValTok{12}\NormalTok{)}
\NormalTok{plt.ylabel(}\StringTok{\textquotesingle{}Amplitude\textquotesingle{}}\NormalTok{, fontsize}\OperatorTok{=}\DecValTok{12}\NormalTok{)}
\NormalTok{plt.title(}\StringTok{\textquotesingle{}Liniendiagramm mit Beschriftung\textquotesingle{}}\NormalTok{, fontsize}\OperatorTok{=}\DecValTok{14}\NormalTok{)}
\NormalTok{plt.legend()}
\NormalTok{plt.show()}
\end{Highlighting}
\end{Shaded}

\includegraphics{books/w-python-matplotlib/skript/adapting_plots_files/figure-pdf/cell-2-output-1.png}

\section{2. Anpassung der Achsen}\label{anpassung-der-achsen}

Die Skalierung der Achsen sollte sinnvoll gewählt werden, um die Daten
bestmöglich darzustellen.

\begin{Shaded}
\begin{Highlighting}[]
\NormalTok{plt.plot(t, y, label}\OperatorTok{=}\StringTok{\textquotesingle{}sin(t)\textquotesingle{}}\NormalTok{, color}\OperatorTok{=}\StringTok{\textquotesingle{}b\textquotesingle{}}\NormalTok{)}
\NormalTok{plt.xlabel(}\StringTok{\textquotesingle{}Zeit (s)\textquotesingle{}}\NormalTok{)}
\NormalTok{plt.ylabel(}\StringTok{\textquotesingle{}Amplitude\textquotesingle{}}\NormalTok{)}
\NormalTok{plt.xlim(}\DecValTok{0}\NormalTok{, }\DecValTok{10}\NormalTok{)}
\NormalTok{plt.ylim(}\OperatorTok{{-}}\FloatTok{1.2}\NormalTok{, }\FloatTok{1.2}\NormalTok{)}
\NormalTok{plt.grid(}\VariableTok{True}\NormalTok{, linestyle}\OperatorTok{=}\StringTok{\textquotesingle{}{-}{-}\textquotesingle{}}\NormalTok{, alpha}\OperatorTok{=}\FloatTok{0.7}\NormalTok{)}
\NormalTok{plt.title(}\StringTok{\textquotesingle{}Liniendiagramm mit angepassten Achsen\textquotesingle{}}\NormalTok{)}
\NormalTok{plt.legend()}
\NormalTok{plt.show()}
\end{Highlighting}
\end{Shaded}

\includegraphics{books/w-python-matplotlib/skript/adapting_plots_files/figure-pdf/cell-3-output-1.png}

\section{3. Farben und Linienstile}\label{farben-und-linienstile}

Farben und Linienstile helfen dabei, wichtige Informationen im Plot
hervorzuheben.

\subsection{Wichtige Farben (Standardfarben in
Matplotlib)}\label{wichtige-farben-standardfarben-in-matplotlib}

\begin{longtable}[]{@{}lll@{}}
\toprule\noalign{}
Farbe & Kürzel & Beschreibung \\
\midrule\noalign{}
\endhead
\bottomrule\noalign{}
\endlastfoot
Blau & `b' & blue \\
Grün & `g' & green \\
Rot & `r' & red \\
Cyan & `c' & cyan \\
Magenta & `m' & magenta \\
Gelb & `y' & yellow \\
Schwarz & `k' & black \\
Weiß & `w' & white \\
\end{longtable}

\subsection{Wichtige Linienstile}\label{wichtige-linienstile}

\begin{longtable}[]{@{}lll@{}}
\toprule\noalign{}
Linienstil & Kürzel & Beschreibung \\
\midrule\noalign{}
\endhead
\bottomrule\noalign{}
\endlastfoot
Durchgezogen & `-' & Standardlinie \\
Gestrichelt & `--' & lange Striche \\
Gepunktet & `:' & nur Punkte \\
Strich-Punkt & `-.' & abwechselnd Strich-Punkt \\
\end{longtable}

\begin{Shaded}
\begin{Highlighting}[]
\NormalTok{plt.plot(t, np.sin(t), linestyle}\OperatorTok{=}\StringTok{\textquotesingle{}{-}\textquotesingle{}}\NormalTok{, color}\OperatorTok{=}\StringTok{\textquotesingle{}r\textquotesingle{}}\NormalTok{, label}\OperatorTok{=}\StringTok{\textquotesingle{}sin(t)\textquotesingle{}}\NormalTok{)}
\NormalTok{plt.plot(t, np.cos(t), linestyle}\OperatorTok{=}\StringTok{\textquotesingle{}{-}{-}\textquotesingle{}}\NormalTok{, color}\OperatorTok{=}\StringTok{\textquotesingle{}g\textquotesingle{}}\NormalTok{, label}\OperatorTok{=}\StringTok{\textquotesingle{}cos(t)\textquotesingle{}}\NormalTok{)}
\NormalTok{plt.xlabel(}\StringTok{\textquotesingle{}Zeit (s)\textquotesingle{}}\NormalTok{)}
\NormalTok{plt.ylabel(}\StringTok{\textquotesingle{}Amplitude\textquotesingle{}}\NormalTok{)}
\NormalTok{plt.title(}\StringTok{\textquotesingle{}Anpassung von Farben und Linienstilen\textquotesingle{}}\NormalTok{)}
\NormalTok{plt.legend()}
\NormalTok{plt.show()}
\end{Highlighting}
\end{Shaded}

\includegraphics{books/w-python-matplotlib/skript/adapting_plots_files/figure-pdf/cell-4-output-1.png}

\section{4. Mehrere Plots mit
Subplots}\label{mehrere-plots-mit-subplots}

Manchmal ist es sinnvoll, mehrere Diagramme in einer Abbildung
darzustellen.

\begin{Shaded}
\begin{Highlighting}[]
\NormalTok{fig, axs }\OperatorTok{=}\NormalTok{ plt.subplots(}\DecValTok{2}\NormalTok{, }\DecValTok{1}\NormalTok{, figsize}\OperatorTok{=}\NormalTok{(}\DecValTok{6}\NormalTok{, }\DecValTok{6}\NormalTok{))}
\NormalTok{axs[}\DecValTok{0}\NormalTok{].plot(t, np.sin(t), color}\OperatorTok{=}\StringTok{\textquotesingle{}b\textquotesingle{}}\NormalTok{)}
\NormalTok{axs[}\DecValTok{0}\NormalTok{].set\_title(}\StringTok{\textquotesingle{}Sinusfunktion\textquotesingle{}}\NormalTok{)}
\NormalTok{axs[}\DecValTok{1}\NormalTok{].plot(t, np.cos(t), color}\OperatorTok{=}\StringTok{\textquotesingle{}r\textquotesingle{}}\NormalTok{)}
\NormalTok{axs[}\DecValTok{1}\NormalTok{].set\_title(}\StringTok{\textquotesingle{}Kosinusfunktion\textquotesingle{}}\NormalTok{)}
\NormalTok{plt.tight\_layout()}
\NormalTok{plt.show()}
\end{Highlighting}
\end{Shaded}

\includegraphics{books/w-python-matplotlib/skript/adapting_plots_files/figure-pdf/cell-5-output-1.png}

\section{5. Speichern von Plots}\label{speichern-von-plots}

Man kann Diagramme in verschiedenen Formaten speichern.

\begin{Shaded}
\begin{Highlighting}[]
\NormalTok{plt.plot(t, y, label}\OperatorTok{=}\StringTok{\textquotesingle{}sin(t)\textquotesingle{}}\NormalTok{, color}\OperatorTok{=}\StringTok{\textquotesingle{}b\textquotesingle{}}\NormalTok{)}
\NormalTok{plt.xlabel(}\StringTok{\textquotesingle{}Zeit (s)\textquotesingle{}}\NormalTok{)}
\NormalTok{plt.ylabel(}\StringTok{\textquotesingle{}Amplitude\textquotesingle{}}\NormalTok{)}
\NormalTok{plt.title(}\StringTok{\textquotesingle{}Speicherung eines Plots\textquotesingle{}}\NormalTok{)}
\NormalTok{plt.legend()}
\NormalTok{plt.savefig(}\StringTok{\textquotesingle{}mein\_plot.png\textquotesingle{}}\NormalTok{, dpi}\OperatorTok{=}\DecValTok{300}\NormalTok{)}
\NormalTok{plt.show()}
\end{Highlighting}
\end{Shaded}

\includegraphics{books/w-python-matplotlib/skript/adapting_plots_files/figure-pdf/cell-6-output-1.png}

\section{Fazit}\label{fazit-1}

Durch geschickte Anpassungen lassen sich wissenschaftliche Plots
deutlich verbessern. Im nächsten Kapitel werden wir uns mit erweiterten
Techniken wie logarithmischen Skalen und Annotationen beschäftigen.

\chapter{Erweiterte Techniken in
Matplotlib}\label{erweiterte-techniken-in-matplotlib}

In diesem Kapitel betrachten wir einige fortgeschrittene Funktionen von
Matplotlib, die für die wissenschaftliche Datenvisualisierung besonders
nützlich sind.

\section{1. Logarithmische Skalen}\label{logarithmische-skalen}

Logarithmische Skalen werden oft verwendet, wenn Werte große
Größenordnungen umfassen.

\begin{Shaded}
\begin{Highlighting}[]
\ImportTok{import}\NormalTok{ matplotlib.pyplot }\ImportTok{as}\NormalTok{ plt}
\ImportTok{import}\NormalTok{ numpy }\ImportTok{as}\NormalTok{ np}

\NormalTok{x }\OperatorTok{=}\NormalTok{ np.logspace(}\FloatTok{0.1}\NormalTok{, }\DecValTok{2}\NormalTok{, }\DecValTok{100}\NormalTok{)}
\NormalTok{y }\OperatorTok{=}\NormalTok{ np.log10(x)}

\NormalTok{plt.plot(x, y, label}\OperatorTok{=}\StringTok{\textquotesingle{}log10(x)\textquotesingle{}}\NormalTok{, color}\OperatorTok{=}\StringTok{\textquotesingle{}b\textquotesingle{}}\NormalTok{)}
\NormalTok{plt.xscale(}\StringTok{\textquotesingle{}log\textquotesingle{}}\NormalTok{)}
\NormalTok{plt.xlabel(}\StringTok{\textquotesingle{}X{-}Wert (log{-}Skala)\textquotesingle{}}\NormalTok{)}
\NormalTok{plt.ylabel(}\StringTok{\textquotesingle{}Y{-}Wert\textquotesingle{}}\NormalTok{)}
\NormalTok{plt.title(}\StringTok{\textquotesingle{}Logarithmische Skalierung\textquotesingle{}}\NormalTok{)}
\NormalTok{plt.legend()}
\NormalTok{plt.grid(}\VariableTok{True}\NormalTok{, which}\OperatorTok{=}\StringTok{\textquotesingle{}both\textquotesingle{}}\NormalTok{, linestyle}\OperatorTok{=}\StringTok{\textquotesingle{}{-}{-}\textquotesingle{}}\NormalTok{, alpha}\OperatorTok{=}\FloatTok{0.7}\NormalTok{)}
\NormalTok{plt.show()}
\end{Highlighting}
\end{Shaded}

\includegraphics{books/w-python-matplotlib/skript/advanced_techniques_files/figure-pdf/cell-2-output-1.png}

\section{2. Twin-Achsen für verschiedene
Skalierungen}\label{twin-achsen-fuxfcr-verschiedene-skalierungen}

Manchmal möchte man zwei verschiedene y-Achsen in einem Plot darstellen.

\begin{Shaded}
\begin{Highlighting}[]
\NormalTok{x }\OperatorTok{=}\NormalTok{ np.linspace(}\DecValTok{0}\NormalTok{, }\DecValTok{10}\NormalTok{, }\DecValTok{100}\NormalTok{)}
\NormalTok{y1 }\OperatorTok{=}\NormalTok{ np.sin(x)}
\NormalTok{y2 }\OperatorTok{=}\NormalTok{ np.exp(x }\OperatorTok{/} \DecValTok{3}\NormalTok{)}

\NormalTok{fig, ax1 }\OperatorTok{=}\NormalTok{ plt.subplots()}
\NormalTok{ax2 }\OperatorTok{=}\NormalTok{ ax1.twinx()}
\NormalTok{ax1.plot(x, y1, }\StringTok{\textquotesingle{}g{-}\textquotesingle{}}\NormalTok{, label}\OperatorTok{=}\StringTok{\textquotesingle{}sin(x)\textquotesingle{}}\NormalTok{)}
\NormalTok{ax2.plot(x, y2, }\StringTok{\textquotesingle{}b{-}{-}\textquotesingle{}}\NormalTok{, label}\OperatorTok{=}\StringTok{\textquotesingle{}exp(x/3)\textquotesingle{}}\NormalTok{)}

\NormalTok{ax1.set\_xlabel(}\StringTok{\textquotesingle{}X{-}Wert\textquotesingle{}}\NormalTok{)}
\NormalTok{ax1.set\_ylabel(}\StringTok{\textquotesingle{}Sinus\textquotesingle{}}\NormalTok{, color}\OperatorTok{=}\StringTok{\textquotesingle{}g\textquotesingle{}}\NormalTok{)}
\NormalTok{ax2.set\_ylabel(}\StringTok{\textquotesingle{}Exponentiell\textquotesingle{}}\NormalTok{, color}\OperatorTok{=}\StringTok{\textquotesingle{}b\textquotesingle{}}\NormalTok{)}
\NormalTok{ax1.set\_title(}\StringTok{\textquotesingle{}Twin{-}Achsen für unterschiedliche Skalierungen\textquotesingle{}}\NormalTok{)}
\NormalTok{plt.show()}
\end{Highlighting}
\end{Shaded}

\includegraphics{books/w-python-matplotlib/skript/advanced_techniques_files/figure-pdf/cell-3-output-1.png}

\section{3. Annotationen in
Diagrammen}\label{annotationen-in-diagrammen}

Wichtige Punkte oder Werte in einem Diagramm können mit Annotationen
hervorgehoben werden.

\begin{Shaded}
\begin{Highlighting}[]
\NormalTok{x }\OperatorTok{=}\NormalTok{ np.linspace(}\DecValTok{0}\NormalTok{, }\DecValTok{10}\NormalTok{, }\DecValTok{100}\NormalTok{)}
\NormalTok{y }\OperatorTok{=}\NormalTok{ np.sin(x)}

\NormalTok{plt.plot(x, y, label}\OperatorTok{=}\StringTok{\textquotesingle{}sin(x)\textquotesingle{}}\NormalTok{)}
\NormalTok{plt.xlabel(}\StringTok{\textquotesingle{}X{-}Wert\textquotesingle{}}\NormalTok{)}
\NormalTok{plt.ylabel(}\StringTok{\textquotesingle{}Amplitude\textquotesingle{}}\NormalTok{)}
\NormalTok{plt.title(}\StringTok{\textquotesingle{}Annotationen in Matplotlib\textquotesingle{}}\NormalTok{)}
\NormalTok{plt.annotate(}\StringTok{\textquotesingle{}Maximalwert\textquotesingle{}}\NormalTok{, xy}\OperatorTok{=}\NormalTok{(np.pi}\OperatorTok{/}\DecValTok{2}\NormalTok{, }\DecValTok{1}\NormalTok{), xytext}\OperatorTok{=}\NormalTok{(}\DecValTok{2}\NormalTok{, }\FloatTok{1.2}\NormalTok{),}
\NormalTok{             arrowprops}\OperatorTok{=}\BuiltInTok{dict}\NormalTok{(facecolor}\OperatorTok{=}\StringTok{\textquotesingle{}red\textquotesingle{}}\NormalTok{, shrink}\OperatorTok{=}\FloatTok{0.05}\NormalTok{))}
\NormalTok{plt.legend()}
\NormalTok{plt.show()}
\end{Highlighting}
\end{Shaded}

\includegraphics{books/w-python-matplotlib/skript/advanced_techniques_files/figure-pdf/cell-4-output-1.png}

\section{Fazit}\label{fazit-2}

Diese erweiterten Funktionen helfen dabei, wissenschaftliche Plots noch
informativer zu gestalten. Im nächsten Kapitel werden wir Best Practices
und typische Fehler in der wissenschaftlichen Visualisierung betrachten.

\chapter{Best Practices in Matplotlib: Fehler und
Verbesserungen}\label{best-practices-in-matplotlib-fehler-und-verbesserungen}

In diesem Kapitel zeigen wir für häufige Problemstellungen jeweils ein
schlechtes und ein verbessertes Beispiel.

\section{1. Fehlende Beschriftungen}\label{fehlende-beschriftungen}

\subsection{❌ Schlechtes Beispiel}\label{schlechtes-beispiel}

\begin{Shaded}
\begin{Highlighting}[]
\ImportTok{import}\NormalTok{ matplotlib.pyplot }\ImportTok{as}\NormalTok{ plt}
\ImportTok{import}\NormalTok{ numpy }\ImportTok{as}\NormalTok{ np}

\NormalTok{x }\OperatorTok{=}\NormalTok{ np.linspace(}\DecValTok{0}\NormalTok{, }\DecValTok{10}\NormalTok{, }\DecValTok{100}\NormalTok{)}
\NormalTok{y }\OperatorTok{=}\NormalTok{ np.sin(x)}

\NormalTok{plt.plot(x, y)}
\NormalTok{plt.show()}
\end{Highlighting}
\end{Shaded}

\includegraphics{books/w-python-matplotlib/skript/scientific_plotting_files/figure-pdf/cell-2-output-1.png}

\subsection{✅ Besseres Beispiel}\label{besseres-beispiel}

\begin{Shaded}
\begin{Highlighting}[]
\NormalTok{plt.plot(x, y, label}\OperatorTok{=}\StringTok{\textquotesingle{}sin(x)\textquotesingle{}}\NormalTok{, color}\OperatorTok{=}\StringTok{\textquotesingle{}b\textquotesingle{}}\NormalTok{)}
\NormalTok{plt.xlabel(}\StringTok{\textquotesingle{}Zeit (s)\textquotesingle{}}\NormalTok{)}
\NormalTok{plt.ylabel(}\StringTok{\textquotesingle{}Amplitude\textquotesingle{}}\NormalTok{)}
\NormalTok{plt.title(}\StringTok{\textquotesingle{}Sinuskurve\textquotesingle{}}\NormalTok{)}
\NormalTok{plt.legend()}
\NormalTok{plt.show()}
\end{Highlighting}
\end{Shaded}

\includegraphics{books/w-python-matplotlib/skript/scientific_plotting_files/figure-pdf/cell-3-output-1.png}

\section{2. Ungünstige Farbwahl}\label{unguxfcnstige-farbwahl}

\subsection{❌ Schlechtes Beispiel}\label{schlechtes-beispiel-1}

\begin{Shaded}
\begin{Highlighting}[]
\NormalTok{plt.plot(x, y, color}\OperatorTok{=}\StringTok{\textquotesingle{}yellow\textquotesingle{}}\NormalTok{)}
\NormalTok{plt.show()}
\end{Highlighting}
\end{Shaded}

\includegraphics{books/w-python-matplotlib/skript/scientific_plotting_files/figure-pdf/cell-4-output-1.png}

\subsection{✅ Besseres Beispiel}\label{besseres-beispiel-1}

\begin{Shaded}
\begin{Highlighting}[]
\NormalTok{plt.plot(x, y, color}\OperatorTok{=}\StringTok{\textquotesingle{}darkblue\textquotesingle{}}\NormalTok{)}
\NormalTok{plt.grid(}\VariableTok{True}\NormalTok{, linestyle}\OperatorTok{=}\StringTok{\textquotesingle{}{-}{-}\textquotesingle{}}\NormalTok{, alpha}\OperatorTok{=}\FloatTok{0.7}\NormalTok{)}
\NormalTok{plt.title(}\StringTok{\textquotesingle{}Gute Kontraste für bessere Lesbarkeit\textquotesingle{}}\NormalTok{)}
\NormalTok{plt.show()}
\end{Highlighting}
\end{Shaded}

\includegraphics{books/w-python-matplotlib/skript/scientific_plotting_files/figure-pdf/cell-5-output-1.png}

\section{3. Keine sinnvolle
Achsenskalierung}\label{keine-sinnvolle-achsenskalierung}

\subsection{❌ Schlechtes Beispiel}\label{schlechtes-beispiel-2}

\begin{Shaded}
\begin{Highlighting}[]
\NormalTok{plt.plot(x, y)}
\NormalTok{plt.ylim(}\FloatTok{0.5}\NormalTok{, }\DecValTok{1}\NormalTok{)}
\NormalTok{plt.show()}
\end{Highlighting}
\end{Shaded}

\includegraphics{books/w-python-matplotlib/skript/scientific_plotting_files/figure-pdf/cell-6-output-1.png}

\subsection{✅ Besseres Beispiel}\label{besseres-beispiel-2}

\begin{Shaded}
\begin{Highlighting}[]
\NormalTok{plt.plot(x, y)}
\NormalTok{plt.ylim(}\OperatorTok{{-}}\FloatTok{1.2}\NormalTok{, }\FloatTok{1.2}\NormalTok{)}
\NormalTok{plt.xlim(}\DecValTok{0}\NormalTok{, }\DecValTok{10}\NormalTok{)}
\NormalTok{plt.grid(}\VariableTok{True}\NormalTok{)}
\NormalTok{plt.title(}\StringTok{\textquotesingle{}Sinnvolle Achsenskalierung\textquotesingle{}}\NormalTok{)}
\NormalTok{plt.show()}
\end{Highlighting}
\end{Shaded}

\includegraphics{books/w-python-matplotlib/skript/scientific_plotting_files/figure-pdf/cell-7-output-1.png}

\section{4. Überladung durch zu viele
Linien}\label{uxfcberladung-durch-zu-viele-linien}

\subsection{❌ Schlechtes Beispiel}\label{schlechtes-beispiel-3}

\begin{Shaded}
\begin{Highlighting}[]
\ControlFlowTok{for}\NormalTok{ i }\KeywordTok{in} \BuiltInTok{range}\NormalTok{(}\DecValTok{10}\NormalTok{):}
\NormalTok{    plt.plot(x, np.sin(x }\OperatorTok{+}\NormalTok{ i }\OperatorTok{*} \FloatTok{0.2}\NormalTok{))}
\NormalTok{plt.show()}
\end{Highlighting}
\end{Shaded}

\includegraphics{books/w-python-matplotlib/skript/scientific_plotting_files/figure-pdf/cell-8-output-1.png}

\subsection{✅ Besseres Beispiel}\label{besseres-beispiel-3}

\begin{Shaded}
\begin{Highlighting}[]
\NormalTok{plt.plot(x, np.sin(x), label}\OperatorTok{=}\StringTok{\textquotesingle{}sin(x)\textquotesingle{}}\NormalTok{)}
\NormalTok{plt.plot(x, np.cos(x), label}\OperatorTok{=}\StringTok{\textquotesingle{}cos(x)\textquotesingle{}}\NormalTok{)}
\NormalTok{plt.legend()}
\NormalTok{plt.title(}\StringTok{\textquotesingle{}Weniger ist mehr: Reduzierte Informationsdichte\textquotesingle{}}\NormalTok{)}
\NormalTok{plt.grid(}\VariableTok{True}\NormalTok{)}
\NormalTok{plt.show()}
\end{Highlighting}
\end{Shaded}

\includegraphics{books/w-python-matplotlib/skript/scientific_plotting_files/figure-pdf/cell-9-output-1.png}

\section{Fazit}\label{fazit-3}

Gute Plots zeichnen sich durch klare Beschriftungen, gute Lesbarkeit und
eine sinnvolle Informationsdichte aus.

\part{NumPy}

\chapter*{Preamble}\label{preamble-1}
\addcontentsline{toc}{chapter}{Preamble}

\markboth{Preamble}{Preamble}

\phantomsection\label{Lizenz}
\begin{figure}

\begin{minipage}{0.20\linewidth}
\includegraphics{index_files/mediabag/by.png}\end{minipage}%
%
\begin{minipage}{0.80\linewidth}
Bausteine Computergestützter Datenanalyse. ``Numpy Grundlagen'' von
Lukas Arnold, Simone Arnold, Florian Bagemihl, Matthias Baitsch, Marc
Fehr, Maik Poetzsch und Sebastian Seipel ist lizensiert unter
\href{https://creativecommons.org/licenses/by/4.0/deed.de}{CC BY 4.0}.
Das Werk ist abrufbar unter
\url{https://github.com/bausteine-der-datenanalyse/w-python-numpy-grundlagen}.
Ausgenommen von der Lizenz sind alle Logos und anders gekennzeichneten
Inhalte. 2024\end{minipage}%

\end{figure}%

Zitiervorschlag

Arnold, Lukas, Simone Arnold, Matthias Baitsch, Marc Fehr, Maik
Poetzsch, und Sebastian Seipel. 2024. „Bausteine Computergestützter
Datenanalyse. Werkzeugbaustein NumPy``.
\url{https://github.com/bausteine-der-datenanalyse/w-python-numpy-grundlagen}.

BibTeX-Vorlage

\begin{verbatim}
@misc{BCD-Styleguide-2024,
 title={Bausteine Computergestützter Datenanalyse. Werkzeugbaustein NumPy},
 author={Arnold, Lukas and Arnold, Simone and Baitsch, Matthias and Fehr, Marc and Poetzsch, Maik and Seipel, Sebastian},
 year={2024},
 url={https://github.com/bausteine-der-datenanalyse/w-python-numpy-grundlagen}} 
\end{verbatim}

\chapter*{Intro}\label{intro-1}
\addcontentsline{toc}{chapter}{Intro}

\markboth{Intro}{Intro}

\section*{Voraussetzungen}\label{voraussetzungen-2}
\addcontentsline{toc}{section}{Voraussetzungen}

\markright{Voraussetzungen}

\begin{itemize}
\tightlist
\item
  Grundlagen Python
\item
  Einbinden von zusätzlichen Paketen
\item
  Plotten mit Matplotlib
\end{itemize}

\section*{Verwendete Pakete und
Datensätze}\label{verwendete-pakete-und-datensuxe4tze-1}
\addcontentsline{toc}{section}{Verwendete Pakete und Datensätze}

\markright{Verwendete Pakete und Datensätze}

\subsection*{Pakete}\label{pakete}
\addcontentsline{toc}{subsection}{Pakete}

\begin{itemize}
\tightlist
\item
  \href{https://numpy.org/}{NumPy}
\item
  \href{https://matplotlib.org/}{Matplotlib}
\end{itemize}

\subsection*{Datensätze}\label{datensuxe4tze}
\addcontentsline{toc}{subsection}{Datensätze}

\begin{itemize}
\tightlist
\item
  \href{https://firedynamics.github.io/LectureComputerScience/_downloads/0d1a3bfbc82fa134e08585d6151e9f46/TC01.csv}{TC01.csv}
\item
  \href{https://upload.wikimedia.org/wikipedia/commons/thumb/6/6a/Mona_Lisa.jpg/677px-Mona_Lisa.jpg}{Bild:
  Mona Lisa}
\item
  \href{https://firedynamics.github.io/LectureComputerScience/_downloads/592f1fc843fc7c01bdcad17bf85ec15c/campus_haspel.jpeg}{Bild:
  Campus}
\end{itemize}

\section*{Bearbeitungszeit}\label{bearbeitungszeit-1}
\addcontentsline{toc}{section}{Bearbeitungszeit}

\markright{Bearbeitungszeit}

Geschätzte Bearbeitungszeit: 2h

\section*{Lernziele}\label{lernziele-2}
\addcontentsline{toc}{section}{Lernziele}

\markright{Lernziele}

\begin{itemize}
\tightlist
\item
  Einleitung: was ist NumPy, Vor- und Nachteile
\item
  Nutzen des NumPy-Moduls
\item
  Erstellen von NumPy-Arrays
\item
  Slicing
\item
  Lesen und schreiben von Dateien
\item
  Arbeiten mit Bildern
\end{itemize}

\chapter{Einführung NumPy}\label{einfuxfchrung-numpy}

NumPy ist eine leistungsstarke Bibliothek für Python, die für
numerisches Rechnen und Datenanalyse verwendet wird. Daher auch der Name
NumPy, ein Akronym für ``Numerisches Python'' (englisch: ``Numeric
Python'' oder ``Numerical Python''). NumPy selbst ist hauptsächlich in
der Programmiersprache C geschrieben, weshalb NumPy generell sehr
schnell ist.

NumPy bietet ein effizientes Arbeiten mit kleinen und großen Vektoren
und Matrizen, die so ansonsten nur umständlich in nativem Python
implementiert werden würden. Dabei bietet NumPy auch die Möglichkeit,
einfach mit Vektoren und Matrizen zu rechnen, und das auch für sehr
große Datenmengen.

Diese Einführung wird Ihnen dabei helfen, die Grundlagen von NumPy zu
verstehen und zu nutzen.

\section{Vorteile \& Nachteile}\label{vorteile-nachteile}

Fast immer sind Operationen mit Numpy Datenstrukturen schneller. Im
Gegensatz zu nativen Python Listen kann man dort aber nur einen Datentyp
pro Liste speichern.

\begin{tcolorbox}[enhanced jigsaw, left=2mm, leftrule=.75mm, bottomrule=.15mm, title=\textcolor{quarto-callout-note-color}{\faInfo}\hspace{0.5em}{Warum ist numpy oftmals schneller?}, colback=white, arc=.35mm, breakable, titlerule=0mm, bottomtitle=1mm, colbacktitle=quarto-callout-note-color!10!white, toprule=.15mm, opacityback=0, coltitle=black, rightrule=.15mm, opacitybacktitle=0.6, toptitle=1mm, colframe=quarto-callout-note-color-frame]

NumPy implementiert eine effizientere Speicherung von Listen im
Speicher. Nativ speichert Python Listeninhalte aufgeteilt, wo gerade
Platz ist.

\begin{figure}[H]

\centering{

\includegraphics{index_files/mediabag/books/w-python-numpy-grundlagen/skript/../skript/00-bilder/data_memory_list.pdf}

}

\caption{\label{fig-python_memory}Speicherung von Daten in nativem
Python}

\end{figure}%

Dagegen werden NumPy Arrays und Matritzen zusammenhängend gespeichert,
was einen effizienteren Datenaufruf ermöglicht.

\begin{figure}[H]

\centering{

\includegraphics{index_files/mediabag/books/w-python-numpy-grundlagen/skript/../skript/00-bilder/data_memory_numpy.pdf}

}

\caption{\label{fig-numpy_memory}Speicherung von Daten bei Numpy}

\end{figure}%

Dies bedeutet aber auch, dass es eine Erweiterung der Liste deutlich
schneller ist als eine Erweiterung von Arrays oder Matrizen. Bei Listen
kann jeder freie Platz genutzt werden, während Arrays und Matrizen an
einen neuen Ort im Speicher kopiert werden müssen.

\end{tcolorbox}

\section{Einbinden des Pakets}\label{einbinden-des-pakets}

NumPy wird über folgende Zeile eingebunden. Dabei hat sich global der
Standard entwickelt, als Alias \texttt{np} zu verwenden.

\begin{Shaded}
\begin{Highlighting}[]
\ImportTok{import}\NormalTok{ numpy }\ImportTok{as}\NormalTok{ np}
\end{Highlighting}
\end{Shaded}

\section{Referenzen}\label{referenzen}

Sämtliche hier vorgestellten Funktionen lassen sich in der (englischen)
NumPy-Dokumentation nachschlagen:
\href{https://numpy.org/doc/}{Dokumentation}

\chapter{Erstellen von NumPy arrays}\label{erstellen-von-numpy-arrays}

Typischerweise werden in Python Vektoren durch Listen und Matrizen durch
geschachtelte Listen ausgedrückt. Beispielsweise würde man den Vektor

\begin{figure}

\begin{minipage}{0.33\linewidth}
\[
(1, 2, 3, 4, 5, 6) 
\]\end{minipage}%
%
\begin{minipage}{0.33\linewidth}
und die Matrix\end{minipage}%
%
\begin{minipage}{0.33\linewidth}
\[
\begin{pmatrix}
1 & 2 & 3\\
4 & 5 & 6
\end{pmatrix}
\]\end{minipage}%

\end{figure}%

nativ in Python so erstellen:

\begin{Shaded}
\begin{Highlighting}[]
\NormalTok{liste }\OperatorTok{=}\NormalTok{ [}\DecValTok{1}\NormalTok{, }\DecValTok{2}\NormalTok{, }\DecValTok{3}\NormalTok{, }\DecValTok{4}\NormalTok{, }\DecValTok{5}\NormalTok{, }\DecValTok{6}\NormalTok{]}

\NormalTok{matrix }\OperatorTok{=}\NormalTok{ [[}\DecValTok{1}\NormalTok{, }\DecValTok{2}\NormalTok{, }\DecValTok{3}\NormalTok{], [}\DecValTok{4}\NormalTok{, }\DecValTok{5}\NormalTok{, }\DecValTok{6}\NormalTok{]]}

\BuiltInTok{print}\NormalTok{(liste)}
\BuiltInTok{print}\NormalTok{(matrix)}
\end{Highlighting}
\end{Shaded}

\begin{verbatim}
[1, 2, 3, 4, 5, 6]
[[1, 2, 3], [4, 5, 6]]
\end{verbatim}

Möchte man jetzt NumPy Arrays verwenden benutzt man den Befehl
\texttt{np.array()}.

\begin{Shaded}
\begin{Highlighting}[]
\NormalTok{liste }\OperatorTok{=}\NormalTok{ np.array([}\DecValTok{1}\NormalTok{, }\DecValTok{2}\NormalTok{, }\DecValTok{3}\NormalTok{, }\DecValTok{4}\NormalTok{, }\DecValTok{5}\NormalTok{, }\DecValTok{6}\NormalTok{])}

\NormalTok{matrix }\OperatorTok{=}\NormalTok{ np.array([[}\DecValTok{1}\NormalTok{, }\DecValTok{2}\NormalTok{, }\DecValTok{3}\NormalTok{], [}\DecValTok{4}\NormalTok{, }\DecValTok{5}\NormalTok{, }\DecValTok{6}\NormalTok{]])}

\BuiltInTok{print}\NormalTok{(liste)}
\BuiltInTok{print}\NormalTok{(matrix)}
\end{Highlighting}
\end{Shaded}

\begin{verbatim}
[1 2 3 4 5 6]
[[1 2 3]
 [4 5 6]]
\end{verbatim}

Betrachtet man die Ausgaben der \texttt{print()} Befehle fallen zwei
Sachen auf. Zum einen fallen die Kommata weg und zum anderen wird die
Matrix passend ausgegeben.

Es gibt auch die Möglichkeit, höherdimensionale Arrays zu erstellen.
Dabei wird eine neue Ebene der Verschachtelung benutzt. Im folgenden
Beispiel wird eine drei-dimensionale Matrix erstellt.

\begin{Shaded}
\begin{Highlighting}[]
\NormalTok{matrix\_3d }\OperatorTok{=}\NormalTok{ np.array([[[}\DecValTok{1}\NormalTok{, }\DecValTok{2}\NormalTok{, }\DecValTok{3}\NormalTok{], [}\DecValTok{4}\NormalTok{, }\DecValTok{5}\NormalTok{, }\DecValTok{6}\NormalTok{]], [[}\DecValTok{7}\NormalTok{, }\DecValTok{8}\NormalTok{, }\DecValTok{9}\NormalTok{], [}\DecValTok{10}\NormalTok{, }\DecValTok{11}\NormalTok{, }\DecValTok{12}\NormalTok{]]])}
\end{Highlighting}
\end{Shaded}

Es gilt als ``good practice'' Arrays immer zu initialisieren. Dafür
bietet NumPy drei Funktionen um vorinitialisierte Arrays zu erzeugen.
Alternativ können Arrays auch mit festgesetzten Werten initialisiert
werden. Dafür kann entweder die Funktion \texttt{np.zeros()}verwendet
werden die alle Werte auf 0 setzt, oder aber \texttt{np.ones()}welche
alle Werte mit 1 initialisiert. Der Funktion wird die Form im Format
\texttt{{[}Reihen,Spalten{]}} übergeben. Möchte man alle Einträge auf
einen spezifischen Wert setzen, kann man den Befehl \texttt{np.full()}
benutzen.

\begin{Shaded}
\begin{Highlighting}[]
\NormalTok{np.zeros([}\DecValTok{2}\NormalTok{,}\DecValTok{3}\NormalTok{])}
\end{Highlighting}
\end{Shaded}

\begin{verbatim}
array([[0., 0., 0.],
       [0., 0., 0.]])
\end{verbatim}

\begin{Shaded}
\begin{Highlighting}[]
\NormalTok{np.ones([}\DecValTok{2}\NormalTok{,}\DecValTok{3}\NormalTok{])}
\end{Highlighting}
\end{Shaded}

\begin{verbatim}
array([[1., 1., 1.],
       [1., 1., 1.]])
\end{verbatim}

\begin{Shaded}
\begin{Highlighting}[]
\NormalTok{np.full([}\DecValTok{2}\NormalTok{,}\DecValTok{3}\NormalTok{],}\DecValTok{7}\NormalTok{)}
\end{Highlighting}
\end{Shaded}

\begin{verbatim}
array([[7, 7, 7],
       [7, 7, 7]])
\end{verbatim}

\begin{tcolorbox}[enhanced jigsaw, left=2mm, leftrule=.75mm, bottomrule=.15mm, title=\textcolor{quarto-callout-tip-color}{\faLightbulb}\hspace{0.5em}{Wie könnte man auch Arrays die mit einer Zahl x gefühlt sind erstellen?}, colback=white, arc=.35mm, breakable, titlerule=0mm, bottomtitle=1mm, colbacktitle=quarto-callout-tip-color!10!white, toprule=.15mm, opacityback=0, coltitle=black, rightrule=.15mm, opacitybacktitle=0.6, toptitle=1mm, colframe=quarto-callout-tip-color-frame]

Der Trick beseht hierbei ein Array mit \texttt{np.ones()} zu
initialisiere und dieses Array dann mit der Zahl x zu multiplizieren. Im
folgenden Beispiel ist \texttt{x\ =\ 5}

\begin{Shaded}
\begin{Highlighting}[]
\NormalTok{np.ones([}\DecValTok{2}\NormalTok{,}\DecValTok{3}\NormalTok{]) }\OperatorTok{*} \DecValTok{5}
\end{Highlighting}
\end{Shaded}

\begin{verbatim}
array([[5., 5., 5.],
       [5., 5., 5.]])
\end{verbatim}

\end{tcolorbox}

Möchte man zum Beispiel für eine Achse in einem Plot einen Vektor mit
gleichmäßig verteilten Werten erstellen, bieten sich in NumPy zwei
Möglichkeiten. Mit den Befehlen
\texttt{np.linspace(Start,Stop,\#Anzahl\ Werte)} und
\texttt{np.arrange(Start,Stop,Abstand\ zwischen\ Werten)} können solche
Arrays erstellt werden.

\begin{Shaded}
\begin{Highlighting}[]
\NormalTok{np.linspace(}\DecValTok{0}\NormalTok{,}\DecValTok{1}\NormalTok{,}\DecValTok{11}\NormalTok{)}
\end{Highlighting}
\end{Shaded}

\begin{verbatim}
array([0. , 0.1, 0.2, 0.3, 0.4, 0.5, 0.6, 0.7, 0.8, 0.9, 1. ])
\end{verbatim}

\begin{Shaded}
\begin{Highlighting}[]
\NormalTok{np.arange(}\DecValTok{0}\NormalTok{,}\DecValTok{10}\NormalTok{,}\DecValTok{2}\NormalTok{)}
\end{Highlighting}
\end{Shaded}

\begin{verbatim}
array([0, 2, 4, 6, 8])
\end{verbatim}

\begin{tcolorbox}[enhanced jigsaw, left=2mm, leftrule=.75mm, bottomrule=.15mm, title=\textcolor{quarto-callout-tip-color}{\faLightbulb}\hspace{0.5em}{Zwischenübung: Array Erstellung}, colback=white, arc=.35mm, breakable, titlerule=0mm, bottomtitle=1mm, colbacktitle=quarto-callout-tip-color!10!white, toprule=.15mm, opacityback=0, coltitle=black, rightrule=.15mm, opacitybacktitle=0.6, toptitle=1mm, colframe=quarto-callout-tip-color-frame]

Erstellen Sie jeweils ein NumPy-Array, mit dem folgenden Inhalt:

\begin{enumerate}
\def\labelenumi{\arabic{enumi}.}
\tightlist
\item
  mit den Werten 1, 7, 42, 99
\item
  zehn mal die Zahl 5
\item
  mit den Zahlen von 35 \textbf{bis einschließlich} 50
\item
  mit allen geraden Zahlen von 20 \textbf{bis einschließlich} 40
\item
  eine Matrix mit 5 Spalten und 4 Reihen mit dem Wert 4 an jeder Stelle
\item
  mit 10 Werten die gleichmäßig zwischen 22 und einschlieslich 40
  verteilt sind
\end{enumerate}

\begin{tcolorbox}[enhanced jigsaw, left=2mm, leftrule=.75mm, bottomrule=.15mm, title={Lösung}, colback=white, arc=.35mm, breakable, titlerule=0mm, bottomtitle=1mm, colbacktitle=quarto-callout-caution-color!10!white, toprule=.15mm, opacityback=0, coltitle=black, rightrule=.15mm, opacitybacktitle=0.6, toptitle=1mm, colframe=quarto-callout-caution-color-frame]

\begin{Shaded}
\begin{Highlighting}[]
\CommentTok{\# 1. }
\BuiltInTok{print}\NormalTok{(np.array([}\DecValTok{1}\NormalTok{, }\DecValTok{7}\NormalTok{, }\DecValTok{42}\NormalTok{, }\DecValTok{99}\NormalTok{]))}
\end{Highlighting}
\end{Shaded}

\begin{verbatim}
[ 1  7 42 99]
\end{verbatim}

\begin{Shaded}
\begin{Highlighting}[]
\CommentTok{\# 2. }
\BuiltInTok{print}\NormalTok{(np.full(}\DecValTok{10}\NormalTok{,}\DecValTok{5}\NormalTok{))}
\end{Highlighting}
\end{Shaded}

\begin{verbatim}
[5 5 5 5 5 5 5 5 5 5]
\end{verbatim}

\begin{Shaded}
\begin{Highlighting}[]
\CommentTok{\# 3. }
\BuiltInTok{print}\NormalTok{(np.arange(}\DecValTok{35}\NormalTok{, }\DecValTok{51}\NormalTok{))}
\end{Highlighting}
\end{Shaded}

\begin{verbatim}
[35 36 37 38 39 40 41 42 43 44 45 46 47 48 49 50]
\end{verbatim}

\begin{Shaded}
\begin{Highlighting}[]
\CommentTok{\# 4. }
\BuiltInTok{print}\NormalTok{(np.arange(}\DecValTok{20}\NormalTok{, }\DecValTok{41}\NormalTok{, }\DecValTok{2}\NormalTok{))}
\end{Highlighting}
\end{Shaded}

\begin{verbatim}
[20 22 24 26 28 30 32 34 36 38 40]
\end{verbatim}

\begin{Shaded}
\begin{Highlighting}[]
\CommentTok{\# 5. }
\BuiltInTok{print}\NormalTok{(np.full([}\DecValTok{4}\NormalTok{,}\DecValTok{5}\NormalTok{],}\DecValTok{4}\NormalTok{))}
\end{Highlighting}
\end{Shaded}

\begin{verbatim}
[[4 4 4 4 4]
 [4 4 4 4 4]
 [4 4 4 4 4]
 [4 4 4 4 4]]
\end{verbatim}

\begin{Shaded}
\begin{Highlighting}[]
\CommentTok{\# 6. }
\BuiltInTok{print}\NormalTok{(np.linspace(}\DecValTok{22}\NormalTok{, }\DecValTok{40}\NormalTok{, }\DecValTok{10}\NormalTok{))}
\end{Highlighting}
\end{Shaded}

\begin{verbatim}
[22. 24. 26. 28. 30. 32. 34. 36. 38. 40.]
\end{verbatim}

\end{tcolorbox}

\end{tcolorbox}

\chapter{Größe, Struktur und Typ}\label{gruxf6uxdfe-struktur-und-typ}

Wenn man sich nicht mehr sicher ist, welche Struktur oder Form ein Array
hat oder oder diese Größen zum Beispiel für Schleifen nutzen möchte,
bietet NumPy folgende Funktionen für das Auslesen dieser Größen an.

\begin{Shaded}
\begin{Highlighting}[]
\NormalTok{matrix }\OperatorTok{=}\NormalTok{ np.array([[}\DecValTok{1}\NormalTok{, }\DecValTok{2}\NormalTok{, }\DecValTok{3}\NormalTok{], [}\DecValTok{4}\NormalTok{, }\DecValTok{5}\NormalTok{, }\DecValTok{6}\NormalTok{]])}
\end{Highlighting}
\end{Shaded}

\texttt{np.shape()} gibt die Längen der einzelnen Dimension in Form
einer Liste zurück.

\begin{Shaded}
\begin{Highlighting}[]
\NormalTok{np.shape(matrix)}
\end{Highlighting}
\end{Shaded}

\begin{verbatim}
(2, 3)
\end{verbatim}

Die native Python Funktion \texttt{len()} gibt dagegen nur die Länge der
ersten Dimension, also die Anzahl der Elemente in den äußeren Klammern
wieder. Im obrigen Beispiel würde \texttt{len()} also die beiden Listen
\texttt{{[}1,\ 2,\ 3{]}} und \texttt{{[}4,\ 5,\ 6{]}} sehen.

\begin{Shaded}
\begin{Highlighting}[]
\BuiltInTok{len}\NormalTok{(matrix)}
\end{Highlighting}
\end{Shaded}

\begin{verbatim}
2
\end{verbatim}

Die Funktion \texttt{np.ndym()} gibt im Gegensatz zu \texttt{np.shape()}
nur die Anzahl der Dimensionen zurück.

\begin{Shaded}
\begin{Highlighting}[]
\NormalTok{np.ndim(matrix)}
\end{Highlighting}
\end{Shaded}

\begin{verbatim}
2
\end{verbatim}

\begin{tcolorbox}[enhanced jigsaw, left=2mm, leftrule=.75mm, bottomrule=.15mm, title=\textcolor{quarto-callout-tip-color}{\faLightbulb}\hspace{0.5em}{Die Ausgabe von \texttt{np.ndim()} kann mit \texttt{np.shape()}und einer
nativen Python Funktion erreicht werden. Wie?}, colback=white, arc=.35mm, breakable, titlerule=0mm, bottomtitle=1mm, colbacktitle=quarto-callout-tip-color!10!white, toprule=.15mm, opacityback=0, coltitle=black, rightrule=.15mm, opacitybacktitle=0.6, toptitle=1mm, colframe=quarto-callout-tip-color-frame]

\texttt{np.ndim()} gibt die Länge der Liste von \texttt{np.shape()} aus

\begin{Shaded}
\begin{Highlighting}[]
\BuiltInTok{len}\NormalTok{(np.shape(matrix))}
\end{Highlighting}
\end{Shaded}

\begin{verbatim}
2
\end{verbatim}

\end{tcolorbox}

Möchte man die Anzahl aller Elemente in einem Array ausgeben kann man
die Funktion \texttt{np.size()} benutzen.

\begin{Shaded}
\begin{Highlighting}[]
\NormalTok{np.size(matrix)}
\end{Highlighting}
\end{Shaded}

\begin{verbatim}
6
\end{verbatim}

NumPy Arrays können verschiedene Datentypen beinhalten. Im folgenden
haben wir drei verschiedene Arrays mit einem jeweils anderen Datentyp.

\begin{Shaded}
\begin{Highlighting}[]
\NormalTok{typ\_a }\OperatorTok{=}\NormalTok{ np.array([}\DecValTok{1}\NormalTok{, }\DecValTok{2}\NormalTok{, }\DecValTok{3}\NormalTok{, }\DecValTok{4}\NormalTok{, }\DecValTok{5}\NormalTok{])}
\NormalTok{typ\_b }\OperatorTok{=}\NormalTok{ np.array([}\FloatTok{0.1}\NormalTok{, }\FloatTok{0.2}\NormalTok{, }\FloatTok{0.3}\NormalTok{, }\FloatTok{0.4}\NormalTok{, }\FloatTok{0.5}\NormalTok{])}
\NormalTok{typ\_c }\OperatorTok{=}\NormalTok{ np.array([}\StringTok{"Montag"}\NormalTok{, }\StringTok{"Dienstag"}\NormalTok{, }\StringTok{"Mittwoch"}\NormalTok{])}
\end{Highlighting}
\end{Shaded}

Mit der Methode \texttt{np.dtype} können wir den Datentyp von Arrays
ausgeben lassen. Meist wird dabei der Typ plus eine Zahl ausgegeben,
welche die zum Speichern benötigte Bytezahl angibt. Das Array
\emph{typ\_a} beinhaltet den Datentyp int64, also ganze Zahlen.

\begin{Shaded}
\begin{Highlighting}[]
\BuiltInTok{print}\NormalTok{(typ\_a.dtype)}
\end{Highlighting}
\end{Shaded}

\begin{verbatim}
int64
\end{verbatim}

Das Array \emph{typ\_b} beinhaltet den Datentyp float64, wobei float für
Gleitkommazahlen steht.

\begin{Shaded}
\begin{Highlighting}[]
\BuiltInTok{print}\NormalTok{(typ\_b.dtype)}
\end{Highlighting}
\end{Shaded}

\begin{verbatim}
float64
\end{verbatim}

Das Array \emph{typ\_c} beinhaltet den Datentyp U8, wobei das U für
Unicode steht. Hier wird als Unicodetext gespeichert.

\begin{Shaded}
\begin{Highlighting}[]
\BuiltInTok{print}\NormalTok{(typ\_c.dtype)}
\end{Highlighting}
\end{Shaded}

\begin{verbatim}
<U8
\end{verbatim}

Im folgenden finden Sie eine Tabelle mit den typischen Datentypen, die
sie häufig antreffen.

\begin{longtable}[]{@{}lll@{}}
\caption{Typische Datentypen in
NumPy}\label{tbl-datatypes}\tabularnewline
\toprule\noalign{}
Datentyp & Numpy Name & Beispiele \\
\midrule\noalign{}
\endfirsthead
\toprule\noalign{}
Datentyp & Numpy Name & Beispiele \\
\midrule\noalign{}
\endhead
\bottomrule\noalign{}
\endlastfoot
Wahrheitswert & \texttt{bool} & {[}True, False, True{]} \\
Ganze Zahl & \texttt{int} & {[}-2, 5, -6, 7, 3{]} \\
positive Ganze Zahlen & \texttt{uint} & {[}1, 2, 3, 4, 5{]} \\
Kommazahlen & \texttt{float} & {[}1.3, 7.4, 3.5, 5.5{]} \\
komplexe zahlen & \texttt{complex} & {[}-1 + 9j, 2-77j, 72 + 11j{]} \\
Textzeichen & \texttt{U} & {[}``montag'', ``dienstag''{]} \\
\end{longtable}

\begin{tcolorbox}[enhanced jigsaw, left=2mm, leftrule=.75mm, bottomrule=.15mm, title=\textcolor{quarto-callout-tip-color}{\faLightbulb}\hspace{0.5em}{Zwischenübung: Arrayinformationen auslesen}, colback=white, arc=.35mm, breakable, titlerule=0mm, bottomtitle=1mm, colbacktitle=quarto-callout-tip-color!10!white, toprule=.15mm, opacityback=0, coltitle=black, rightrule=.15mm, opacitybacktitle=0.6, toptitle=1mm, colframe=quarto-callout-tip-color-frame]

Gegeben sei folgende Matrix:

\begin{Shaded}
\begin{Highlighting}[]
\NormalTok{matrix }\OperatorTok{=}\NormalTok{ np.array([[[ }\DecValTok{0}\NormalTok{,  }\DecValTok{1}\NormalTok{,  }\DecValTok{2}\NormalTok{,  }\DecValTok{3}\NormalTok{],}
\NormalTok{                 [ }\DecValTok{4}\NormalTok{,  }\DecValTok{5}\NormalTok{,  }\DecValTok{6}\NormalTok{,  }\DecValTok{7}\NormalTok{],}
\NormalTok{                 [ }\DecValTok{8}\NormalTok{,  }\DecValTok{9}\NormalTok{, }\DecValTok{10}\NormalTok{, }\DecValTok{11}\NormalTok{]],}

\NormalTok{                [[}\DecValTok{12}\NormalTok{, }\DecValTok{13}\NormalTok{, }\DecValTok{14}\NormalTok{, }\DecValTok{15}\NormalTok{],}
\NormalTok{                 [}\DecValTok{16}\NormalTok{, }\DecValTok{17}\NormalTok{, }\DecValTok{18}\NormalTok{, }\DecValTok{19}\NormalTok{],}
\NormalTok{                 [}\DecValTok{20}\NormalTok{, }\DecValTok{21}\NormalTok{, }\DecValTok{22}\NormalTok{, }\DecValTok{23}\NormalTok{]],}

\NormalTok{                [[}\DecValTok{24}\NormalTok{, }\DecValTok{25}\NormalTok{, }\DecValTok{26}\NormalTok{, }\DecValTok{27}\NormalTok{],}
\NormalTok{                 [}\DecValTok{28}\NormalTok{, }\DecValTok{29}\NormalTok{, }\DecValTok{30}\NormalTok{, }\DecValTok{31}\NormalTok{],}
\NormalTok{                 [}\DecValTok{32}\NormalTok{, }\DecValTok{33}\NormalTok{, }\DecValTok{34}\NormalTok{, }\DecValTok{35}\NormalTok{]]])}
\end{Highlighting}
\end{Shaded}

Bestimmen Sie durch anschauen die Anzahl an Dimensionen und die Länge
jeder Dimension. Von welchem Typ ist der Inhalt dieser Matrix?

Überprüfen Sie daraufhin Ihre Ergebnisse in dem Sie die passenden
NumPy-Funktionen anwenden.

\begin{tcolorbox}[enhanced jigsaw, left=2mm, leftrule=.75mm, bottomrule=.15mm, title={Lösung}, colback=white, arc=.35mm, breakable, titlerule=0mm, bottomtitle=1mm, colbacktitle=quarto-callout-caution-color!10!white, toprule=.15mm, opacityback=0, coltitle=black, rightrule=.15mm, opacitybacktitle=0.6, toptitle=1mm, colframe=quarto-callout-caution-color-frame]

\begin{Shaded}
\begin{Highlighting}[]
\NormalTok{matrix }\OperatorTok{=}\NormalTok{ np.array([[[ }\DecValTok{0}\NormalTok{,  }\DecValTok{1}\NormalTok{,  }\DecValTok{2}\NormalTok{,  }\DecValTok{3}\NormalTok{],}
\NormalTok{                 [ }\DecValTok{4}\NormalTok{,  }\DecValTok{5}\NormalTok{,  }\DecValTok{6}\NormalTok{,  }\DecValTok{7}\NormalTok{],}
\NormalTok{                 [ }\DecValTok{8}\NormalTok{,  }\DecValTok{9}\NormalTok{, }\DecValTok{10}\NormalTok{, }\DecValTok{11}\NormalTok{]],}

\NormalTok{                [[}\DecValTok{12}\NormalTok{, }\DecValTok{13}\NormalTok{, }\DecValTok{14}\NormalTok{, }\DecValTok{15}\NormalTok{],}
\NormalTok{                 [}\DecValTok{16}\NormalTok{, }\DecValTok{17}\NormalTok{, }\DecValTok{18}\NormalTok{, }\DecValTok{19}\NormalTok{],}
\NormalTok{                 [}\DecValTok{20}\NormalTok{, }\DecValTok{21}\NormalTok{, }\DecValTok{22}\NormalTok{, }\DecValTok{23}\NormalTok{]],}

\NormalTok{                [[}\DecValTok{24}\NormalTok{, }\DecValTok{25}\NormalTok{, }\DecValTok{26}\NormalTok{, }\DecValTok{27}\NormalTok{],}
\NormalTok{                 [}\DecValTok{28}\NormalTok{, }\DecValTok{29}\NormalTok{, }\DecValTok{30}\NormalTok{, }\DecValTok{31}\NormalTok{],}
\NormalTok{                 [}\DecValTok{32}\NormalTok{, }\DecValTok{33}\NormalTok{, }\DecValTok{34}\NormalTok{, }\DecValTok{35}\NormalTok{]]])}

\NormalTok{anzahl\_dimensionen }\OperatorTok{=}\NormalTok{ np.ndim(matrix)}

\BuiltInTok{print}\NormalTok{(}\StringTok{"Anzahl unterschiedlicher Dimensionen: "}\NormalTok{, anzahl\_dimensionen)}

\NormalTok{laenge\_dimensionen }\OperatorTok{=}\NormalTok{ np.shape(matrix)}

\BuiltInTok{print}\NormalTok{(}\StringTok{"Länge der einzelnen DImensionen: "}\NormalTok{, laenge\_dimensionen)}

\BuiltInTok{print}\NormalTok{(matrix.dtype)}
\end{Highlighting}
\end{Shaded}

\begin{verbatim}
Anzahl unterschiedlicher Dimensionen:  3
Länge der einzelnen DImensionen:  (3, 3, 4)
int64
\end{verbatim}

\end{tcolorbox}

\end{tcolorbox}

\chapter{Rechnen mit Arrays}\label{rechnen-mit-arrays}

\section{Arithmetische Funktionen}\label{arithmetische-funktionen}

Ein großer Vorteil an NumPy ist das Rechnen mit Arrays. Ohne NumPy
müsste man entweder eine \texttt{Schleife} oder aber
\texttt{List\ comprehension} benutzen, um mit sämtlichen Werten in der
Liste zu rechnen. In NumPy fällt diese Unannehmlichkeit weg.

\begin{Shaded}
\begin{Highlighting}[]
\NormalTok{a }\OperatorTok{=}\NormalTok{ np.array([}\DecValTok{1}\NormalTok{, }\DecValTok{2}\NormalTok{, }\DecValTok{3}\NormalTok{, }\DecValTok{4}\NormalTok{, }\DecValTok{5}\NormalTok{])}

\NormalTok{b }\OperatorTok{=}\NormalTok{ np.array([}\DecValTok{9}\NormalTok{, }\DecValTok{8}\NormalTok{, }\DecValTok{7}\NormalTok{, }\DecValTok{6}\NormalTok{, }\DecValTok{5}\NormalTok{])}
\end{Highlighting}
\end{Shaded}

Normale mathematische Operationen, wie die Addition, lassen sich auf
zwei Arten ausdrücken. Entweder über die \texttt{np.add()} Funktion oder
aber simpel über das \texttt{+} Zeichen.

\begin{Shaded}
\begin{Highlighting}[]
\NormalTok{np.add(a,b)}
\end{Highlighting}
\end{Shaded}

\begin{verbatim}
array([10, 10, 10, 10, 10])
\end{verbatim}

\begin{Shaded}
\begin{Highlighting}[]
\NormalTok{a }\OperatorTok{+}\NormalTok{ b}
\end{Highlighting}
\end{Shaded}

\begin{verbatim}
array([10, 10, 10, 10, 10])
\end{verbatim}

Ohne NumPy würde die Operation folgendermaßen aussehen:

\begin{Shaded}
\begin{Highlighting}[]
\NormalTok{ergebnis }\OperatorTok{=}\NormalTok{ np.ones(}\DecValTok{5}\NormalTok{)}
\ControlFlowTok{for}\NormalTok{ i }\KeywordTok{in} \BuiltInTok{range}\NormalTok{(}\BuiltInTok{len}\NormalTok{(a)):}
\NormalTok{    ergebnis[i] }\OperatorTok{=}\NormalTok{ a[i] }\OperatorTok{+}\NormalTok{ b[i]}

\BuiltInTok{print}\NormalTok{(ergebnis)}
\end{Highlighting}
\end{Shaded}

\begin{verbatim}
[10. 10. 10. 10. 10.]
\end{verbatim}

Für die anderen Rechenarten existieren auch Funktionen:
\texttt{np.subtract()}, \texttt{np.multiply()} und \texttt{np.divide()}.

Auch für die anderen höheren Rechenoperationen gibt es ebenfalls
Funktionen:

\begin{itemize}
\tightlist
\item
  \texttt{np.exp(a)}
\item
  \texttt{np.sqrt(a)}
\item
  \texttt{np.power(a,\ 3)}
\item
  \texttt{np.sin(a)}
\item
  \texttt{np.cos(a)}
\item
  \texttt{np.tan(a)}
\item
  \texttt{np.log(a)}
\item
  \texttt{a.dot(b)}
\end{itemize}

\begin{tcolorbox}[enhanced jigsaw, left=2mm, leftrule=.75mm, bottomrule=.15mm, title=\textcolor{quarto-callout-warning-color}{\faExclamationTriangle}\hspace{0.5em}{Arbeiten mit Winkelfunktionen}, colback=white, arc=.35mm, breakable, titlerule=0mm, bottomtitle=1mm, colbacktitle=quarto-callout-warning-color!10!white, toprule=.15mm, opacityback=0, coltitle=black, rightrule=.15mm, opacitybacktitle=0.6, toptitle=1mm, colframe=quarto-callout-warning-color-frame]

Wie auch am Taschenrechner birgt das Arbeiten mit den Winkelfunktionen
(sin, cos, \ldots) die Fehlerquelle, dass man nicht mit Radian-Werten,
sondern mit Grad-Werten arbeitet. Die Winkelfunktionen in numpy erwarten
jedoch Radian-Werte.

Für eine einfache Umrechnung bietet NumPy die Funktionen
\texttt{np.grad2rad()}und \texttt{np.rad2grad()}.

\end{tcolorbox}

\section{Vergleiche}\label{vergleiche}

NumPy-Arrays lassen sich auch miteinander vergleichen. Betrachten wir
die folgenden zwei Arrays:

\begin{Shaded}
\begin{Highlighting}[]
\NormalTok{a }\OperatorTok{=}\NormalTok{ np.array([}\DecValTok{1}\NormalTok{, }\DecValTok{2}\NormalTok{, }\DecValTok{3}\NormalTok{, }\DecValTok{4}\NormalTok{, }\DecValTok{5}\NormalTok{])}

\NormalTok{b }\OperatorTok{=}\NormalTok{ np.array([}\DecValTok{9}\NormalTok{, }\DecValTok{2}\NormalTok{, }\DecValTok{7}\NormalTok{, }\DecValTok{4}\NormalTok{, }\DecValTok{5}\NormalTok{])}
\end{Highlighting}
\end{Shaded}

Möchten wir feststellen, ob diese zwei Arrays identisch sind, können wir
den \texttt{==}-Komparator benutzen. Dieser vergleicht die Arrays
elementweise.

\begin{Shaded}
\begin{Highlighting}[]
\NormalTok{a }\OperatorTok{==}\NormalTok{ b}
\end{Highlighting}
\end{Shaded}

\begin{verbatim}
array([False,  True, False,  True,  True])
\end{verbatim}

Es ist außerdem möglich Arrays mit den \texttt{\textgreater{}}- und
\texttt{\textless{}}-Operatoren zu vergleichen:

\begin{Shaded}
\begin{Highlighting}[]
\NormalTok{a }\OperatorTok{\textless{}}\NormalTok{ b}
\end{Highlighting}
\end{Shaded}

\begin{verbatim}
array([ True, False,  True, False, False])
\end{verbatim}

Möchte man Arrays mit Gleitkommazahlen vergleichen, ist es oftmals
nötig, eine gewisse Toleranz zu benutzen, da bei Rechenoperationen
minimale Rundungsfehler entstehen können.

\begin{Shaded}
\begin{Highlighting}[]
\NormalTok{a }\OperatorTok{=}\NormalTok{ np.array(}\FloatTok{0.1} \OperatorTok{+} \FloatTok{0.2}\NormalTok{)}
\NormalTok{b }\OperatorTok{=}\NormalTok{ np.array(}\FloatTok{0.3}\NormalTok{)}
\NormalTok{a }\OperatorTok{==}\NormalTok{ b}
\end{Highlighting}
\end{Shaded}

\begin{verbatim}
np.False_
\end{verbatim}

Für diesen Fall gibt es eine Vergleichsfunktion
\texttt{np.isclose(a,b,atol)}, wobei \texttt{atol} für die absolute
Toleranz steht. Im folgenden Beispiel wird eine absolute Toleranz von
0,001 verwendet.

\begin{Shaded}
\begin{Highlighting}[]
\NormalTok{a }\OperatorTok{=}\NormalTok{ np.array(}\FloatTok{0.1} \OperatorTok{+} \FloatTok{0.2}\NormalTok{)}
\NormalTok{b }\OperatorTok{=}\NormalTok{ np.array(}\FloatTok{0.3}\NormalTok{)}
\BuiltInTok{print}\NormalTok{(np.isclose(a, b, atol}\OperatorTok{=}\FloatTok{0.001}\NormalTok{))}
\end{Highlighting}
\end{Shaded}

\begin{verbatim}
True
\end{verbatim}

\begin{tcolorbox}[enhanced jigsaw, left=2mm, leftrule=.75mm, bottomrule=.15mm, title=\textcolor{quarto-callout-note-color}{\faInfo}\hspace{0.5em}{Warum ist 0.1 + 0.2 nicht gleich 0.3?}, colback=white, arc=.35mm, breakable, titlerule=0mm, bottomtitle=1mm, colbacktitle=quarto-callout-note-color!10!white, toprule=.15mm, opacityback=0, coltitle=black, rightrule=.15mm, opacitybacktitle=0.6, toptitle=1mm, colframe=quarto-callout-note-color-frame]

Zahlen werden intern als Binärzahlen dargestellt. So wie 1/3 nicht mit
einer endlichen Anzahl an Ziffern korrekt dargestellt werden kann müssen
Zahlen ggf. gerundet werden, um im Binärsystem dargestellt zu werden.

\begin{Shaded}
\begin{Highlighting}[]
\NormalTok{a }\OperatorTok{=} \FloatTok{0.1}
\NormalTok{b }\OperatorTok{=} \FloatTok{0.2}
\BuiltInTok{print}\NormalTok{(a }\OperatorTok{+}\NormalTok{ b)}
\end{Highlighting}
\end{Shaded}

\begin{verbatim}
0.30000000000000004
\end{verbatim}

\end{tcolorbox}

\section{Aggregatfunktionen}\label{aggregatfunktionen}

Für verschiedene Auswertungen benötigen wir Funktionen, wie etwa die
Summen oder die Mittelwert-Funktion. Starten wir mit einem Beispiel
Array a:

\begin{Shaded}
\begin{Highlighting}[]
\NormalTok{a }\OperatorTok{=}\NormalTok{ np.array([}\DecValTok{1}\NormalTok{, }\DecValTok{2}\NormalTok{, }\DecValTok{3}\NormalTok{, }\DecValTok{4}\NormalTok{, }\DecValTok{8}\NormalTok{])}
\end{Highlighting}
\end{Shaded}

Die Summer wird über die Funktion \texttt{np.sum()} berechnet.

\begin{Shaded}
\begin{Highlighting}[]
\NormalTok{np.}\BuiltInTok{sum}\NormalTok{(a)}
\end{Highlighting}
\end{Shaded}

\begin{verbatim}
np.int64(18)
\end{verbatim}

Natürlich lassen sich auch der Minimalwert und der Maximalwert eines
Arrays ermitteln. Die beiden Funktionen lauten \texttt{np.min()}und
\texttt{np.max()}.

\begin{Shaded}
\begin{Highlighting}[]
\NormalTok{np.}\BuiltInTok{min}\NormalTok{(a)}
\end{Highlighting}
\end{Shaded}

\begin{verbatim}
np.int64(1)
\end{verbatim}

Möchte man nicht das Maximum selbst, sondern die Position des Maximums
bestimmen, wird statt \texttt{np.max} die Funktion
\texttt{np.argmax}verwendet.

Für statistische Auswertungen werden häufig die Funktion für den
Mittelwert \texttt{np.mean()}, die Funktion für den Median
\texttt{np.median()}und die Funktion für die Standardabweichung
\texttt{np.std()}verwendet.

\begin{Shaded}
\begin{Highlighting}[]
\NormalTok{np.mean(a)}
\end{Highlighting}
\end{Shaded}

\begin{verbatim}
np.float64(3.6)
\end{verbatim}

\begin{Shaded}
\begin{Highlighting}[]
\NormalTok{np.median(a)}
\end{Highlighting}
\end{Shaded}

\begin{verbatim}
np.float64(3.0)
\end{verbatim}

\begin{Shaded}
\begin{Highlighting}[]
\NormalTok{np.std(a)}
\end{Highlighting}
\end{Shaded}

\begin{verbatim}
np.float64(2.4166091947189146)
\end{verbatim}

\begin{tcolorbox}[enhanced jigsaw, left=2mm, leftrule=.75mm, bottomrule=.15mm, title=\textcolor{quarto-callout-tip-color}{\faLightbulb}\hspace{0.5em}{Zwischenübung: Rechnen mit Arrays}, colback=white, arc=.35mm, breakable, titlerule=0mm, bottomtitle=1mm, colbacktitle=quarto-callout-tip-color!10!white, toprule=.15mm, opacityback=0, coltitle=black, rightrule=.15mm, opacitybacktitle=0.6, toptitle=1mm, colframe=quarto-callout-tip-color-frame]

Gegeben sind zwei eindimensionale Arrays a und b:

a = np.array({[}10, 20, 30, 40, 50, 60, 70, 80, 90, 100{]}) und b =
np.array({[}5, 15, 25, 35, 45, 55, 65, 75, 85, 95{]})

\begin{enumerate}
\def\labelenumi{\arabic{enumi}.}
\tightlist
\item
  Erstellen Sie ein neues Array, das die Sinuswerte der addierten Arrays
  a und b enthält.
\item
  Berechnen Sie die Summe, den Mittelwert und die Standardabweichung der
  Elemente in a.
\item
  Finden Sie den größten und den kleinsten Wert in a und b.
\end{enumerate}

\begin{tcolorbox}[enhanced jigsaw, left=2mm, leftrule=.75mm, bottomrule=.15mm, title={Lösung}, colback=white, arc=.35mm, breakable, titlerule=0mm, bottomtitle=1mm, colbacktitle=quarto-callout-caution-color!10!white, toprule=.15mm, opacityback=0, coltitle=black, rightrule=.15mm, opacitybacktitle=0.6, toptitle=1mm, colframe=quarto-callout-caution-color-frame]

\begin{Shaded}
\begin{Highlighting}[]
\NormalTok{a }\OperatorTok{=}\NormalTok{ np.array([}\DecValTok{10}\NormalTok{, }\DecValTok{20}\NormalTok{, }\DecValTok{30}\NormalTok{, }\DecValTok{40}\NormalTok{, }\DecValTok{50}\NormalTok{, }\DecValTok{60}\NormalTok{, }\DecValTok{70}\NormalTok{, }\DecValTok{80}\NormalTok{, }\DecValTok{90}\NormalTok{, }\DecValTok{100}\NormalTok{])}
\NormalTok{b }\OperatorTok{=}\NormalTok{ np.array([}\DecValTok{5}\NormalTok{, }\DecValTok{15}\NormalTok{, }\DecValTok{25}\NormalTok{, }\DecValTok{35}\NormalTok{, }\DecValTok{45}\NormalTok{, }\DecValTok{55}\NormalTok{, }\DecValTok{65}\NormalTok{, }\DecValTok{75}\NormalTok{, }\DecValTok{85}\NormalTok{, }\DecValTok{95}\NormalTok{])}

\CommentTok{\# 1.}
\NormalTok{sin\_ab }\OperatorTok{=}\NormalTok{ np.sin(a }\OperatorTok{+}\NormalTok{ b)}

\CommentTok{\# 2.}
\NormalTok{sum\_a }\OperatorTok{=}\NormalTok{ np.}\BuiltInTok{sum}\NormalTok{(a)}
\NormalTok{mean\_a }\OperatorTok{=}\NormalTok{ np.mean(a)}
\NormalTok{std\_a }\OperatorTok{=}\NormalTok{ np.std(a)}

\CommentTok{\# 3.}
\NormalTok{max\_a }\OperatorTok{=}\NormalTok{ np.}\BuiltInTok{max}\NormalTok{(a)}
\NormalTok{min\_a }\OperatorTok{=}\NormalTok{ np.}\BuiltInTok{min}\NormalTok{(a)}
\NormalTok{max\_b }\OperatorTok{=}\NormalTok{ np.}\BuiltInTok{max}\NormalTok{(b)}
\NormalTok{min\_b }\OperatorTok{=}\NormalTok{ np.}\BuiltInTok{min}\NormalTok{(b)}
\end{Highlighting}
\end{Shaded}

\end{tcolorbox}

\end{tcolorbox}

\chapter{Slicing}\label{slicing}

\section{Normales Slicing mit
Zahlenwerten}\label{normales-slicing-mit-zahlenwerten}

\begin{figure}

\centering{

\includegraphics{books/w-python-numpy-grundlagen/skript/../skript/00-bilder/slicing.png}

}

\caption{\label{fig-slicing}Ansprechen der einzelnen Achsen für den
ein-, zwei- und dreidimensionallen Fall inkl. jeweiligem Beispiel}

\end{figure}%

Möchte man jetzt Daten innerhalb eines Arrays auswählen so geschieht das
in der Form:

\begin{enumerate}
\def\labelenumi{\arabic{enumi}.}
\tightlist
\item
  {[}a{]} wobei ein einzelner Wert an Position a ausgegeben wird
\item
  {[}a:b{]} wobei alle Werte von Position a bis Position b-1 ausgegeben
  werden
\item
  {[}a:b:c{]} wobei die Werte von Position a bis Position b-1 mit einer
  Schrittweite von c ausgegeben werden
\end{enumerate}

\begin{Shaded}
\begin{Highlighting}[]
\NormalTok{liste }\OperatorTok{=}\NormalTok{ np.array([}\DecValTok{1}\NormalTok{, }\DecValTok{2}\NormalTok{, }\DecValTok{3}\NormalTok{, }\DecValTok{4}\NormalTok{, }\DecValTok{5}\NormalTok{, }\DecValTok{6}\NormalTok{])}
\end{Highlighting}
\end{Shaded}

\begin{Shaded}
\begin{Highlighting}[]
\CommentTok{\# Auswählen des ersten Elements}
\NormalTok{liste[}\DecValTok{0}\NormalTok{]}
\end{Highlighting}
\end{Shaded}

\begin{verbatim}
np.int64(1)
\end{verbatim}

\begin{Shaded}
\begin{Highlighting}[]
\CommentTok{\# Auswählen des letzen Elements}
\NormalTok{liste[}\OperatorTok{{-}}\DecValTok{1}\NormalTok{]}
\end{Highlighting}
\end{Shaded}

\begin{verbatim}
np.int64(6)
\end{verbatim}

\begin{Shaded}
\begin{Highlighting}[]
\CommentTok{\# Auswählen einer Reihe von Elementen}
\NormalTok{liste[}\DecValTok{1}\NormalTok{:}\DecValTok{4}\NormalTok{]}
\end{Highlighting}
\end{Shaded}

\begin{verbatim}
array([2, 3, 4])
\end{verbatim}

Für zwei-dimensionale Arrays wählt man getrennt durch ein Komma mit
einer zweiten Zahl die zweite Dimension aus.

\begin{Shaded}
\begin{Highlighting}[]
\NormalTok{matrix }\OperatorTok{=}\NormalTok{ np.array([[}\DecValTok{1}\NormalTok{, }\DecValTok{2}\NormalTok{, }\DecValTok{3}\NormalTok{], [}\DecValTok{4}\NormalTok{, }\DecValTok{5}\NormalTok{, }\DecValTok{6}\NormalTok{]])}
\end{Highlighting}
\end{Shaded}

\begin{Shaded}
\begin{Highlighting}[]
\CommentTok{\# Auswählen einer Elements}
\NormalTok{matrix[}\DecValTok{1}\NormalTok{,}\DecValTok{1}\NormalTok{]}
\end{Highlighting}
\end{Shaded}

\begin{verbatim}
np.int64(5)
\end{verbatim}

Für drei-dimensionale Arrays wählt man getrennt durch ein Komma mit
einer weiteren Zahl die dritte Dimension aus. Dabei wird dieses jedoch
an die erste Stelle gesetzt.

\begin{Shaded}
\begin{Highlighting}[]
\NormalTok{matrix\_3d }\OperatorTok{=}\NormalTok{ np.array([[[}\DecValTok{1}\NormalTok{, }\DecValTok{2}\NormalTok{, }\DecValTok{3}\NormalTok{], [}\DecValTok{4}\NormalTok{, }\DecValTok{5}\NormalTok{, }\DecValTok{6}\NormalTok{]], [[}\DecValTok{7}\NormalTok{, }\DecValTok{8}\NormalTok{, }\DecValTok{9}\NormalTok{], [}\DecValTok{10}\NormalTok{, }\DecValTok{11}\NormalTok{, }\DecValTok{12}\NormalTok{]]])}
\BuiltInTok{print}\NormalTok{(matrix\_3d)}
\end{Highlighting}
\end{Shaded}

\begin{verbatim}
[[[ 1  2  3]
  [ 4  5  6]]

 [[ 7  8  9]
  [10 11 12]]]
\end{verbatim}

\begin{Shaded}
\begin{Highlighting}[]
\CommentTok{\# Auswählen eines Elements}
\NormalTok{matrix\_3d[}\DecValTok{1}\NormalTok{,}\DecValTok{0}\NormalTok{,}\DecValTok{2}\NormalTok{]}
\end{Highlighting}
\end{Shaded}

\begin{verbatim}
np.int64(9)
\end{verbatim}

\section{Slicing mit logischen Werten (Boolesche
Masken)}\label{slicing-mit-logischen-werten-boolesche-masken}

Beim logischen Slicing wird eine boolesche Maske verwendet, um bestimmte
Elemente eines Arrays auszuwählen. Die Maske ist ein Array gleicher
Länge wie das Original, das aus \texttt{True} oder \texttt{False} Werten
besteht.

\begin{Shaded}
\begin{Highlighting}[]
\CommentTok{\# Erstellen wir ein Beispiel Array}
\NormalTok{a }\OperatorTok{=}\NormalTok{ np.array([}\DecValTok{1}\NormalTok{, }\DecValTok{2}\NormalTok{, }\DecValTok{3}\NormalTok{, }\DecValTok{4}\NormalTok{, }\DecValTok{5}\NormalTok{, }\DecValTok{6}\NormalTok{])}

\CommentTok{\# Erstellen der Maske}
\NormalTok{maske }\OperatorTok{=}\NormalTok{ a }\OperatorTok{\textgreater{}} \DecValTok{3}

\BuiltInTok{print}\NormalTok{(maske)}
\end{Highlighting}
\end{Shaded}

\begin{verbatim}
[False False False  True  True  True]
\end{verbatim}

Wir erhalten also ein Array mit boolschen Werten. Verwenden wir diese
Maske nun zum slicen, erhalten wir alle Werte an den Stellen, an denen
die Maske den Wert \texttt{True} besitzt.

\begin{Shaded}
\begin{Highlighting}[]
\CommentTok{\# Anwenden der Maske}
\BuiltInTok{print}\NormalTok{(a[maske])}
\end{Highlighting}
\end{Shaded}

\begin{verbatim}
[4 5 6]
\end{verbatim}

\begin{tcolorbox}[enhanced jigsaw, left=2mm, leftrule=.75mm, bottomrule=.15mm, title=\textcolor{quarto-callout-warning-color}{\faExclamationTriangle}\hspace{0.5em}{Warning}, colback=white, arc=.35mm, breakable, titlerule=0mm, bottomtitle=1mm, colbacktitle=quarto-callout-warning-color!10!white, toprule=.15mm, opacityback=0, coltitle=black, rightrule=.15mm, opacitybacktitle=0.6, toptitle=1mm, colframe=quarto-callout-warning-color-frame]

Das Verwenden von booleschen Arrays ist nur im numpy-Modul möglich. Es
ist nicht Möglich dieses Vorgehen auf native Python Listen anzuwenden.
Hier muss durch die Liste iterriert werden.

\begin{Shaded}
\begin{Highlighting}[]
\NormalTok{a }\OperatorTok{=}\NormalTok{ [}\DecValTok{1}\NormalTok{, }\DecValTok{2}\NormalTok{, }\DecValTok{3}\NormalTok{, }\DecValTok{4}\NormalTok{, }\DecValTok{5}\NormalTok{, }\DecValTok{6}\NormalTok{]}
\NormalTok{ergebniss }\OperatorTok{=}\NormalTok{ [x }\ControlFlowTok{for}\NormalTok{ x }\KeywordTok{in}\NormalTok{ a }\ControlFlowTok{if}\NormalTok{ x }\OperatorTok{\textgreater{}} \DecValTok{3}\NormalTok{]}
\BuiltInTok{print}\NormalTok{(ergebniss) }
\end{Highlighting}
\end{Shaded}

\begin{verbatim}
[4, 5, 6]
\end{verbatim}

\end{tcolorbox}

\begin{tcolorbox}[enhanced jigsaw, left=2mm, leftrule=.75mm, bottomrule=.15mm, title=\textcolor{quarto-callout-tip-color}{\faLightbulb}\hspace{0.5em}{Zwischenübung: Array-Slicing}, colback=white, arc=.35mm, breakable, titlerule=0mm, bottomtitle=1mm, colbacktitle=quarto-callout-tip-color!10!white, toprule=.15mm, opacityback=0, coltitle=black, rightrule=.15mm, opacitybacktitle=0.6, toptitle=1mm, colframe=quarto-callout-tip-color-frame]

Wählen Sie die farblich markierten Bereiche aus dem Array ``matrix'' mit
den eben gelernten Möglichkeiten des Array-Slicing aus.

\includegraphics{index_files/mediabag/books/w-python-numpy-grundlagen/skript/../skript/00-bilder/exercise_slicing.pdf}

\begin{Shaded}
\begin{Highlighting}[]
\NormalTok{matrix }\OperatorTok{=}\NormalTok{ np.array([}
\NormalTok{    [}\DecValTok{2}\NormalTok{, }\DecValTok{11}\NormalTok{, }\DecValTok{18}\NormalTok{, }\DecValTok{47}\NormalTok{, }\DecValTok{33}\NormalTok{, }\DecValTok{48}\NormalTok{, }\DecValTok{9}\NormalTok{, }\DecValTok{31}\NormalTok{, }\DecValTok{8}\NormalTok{, }\DecValTok{41}\NormalTok{],}
\NormalTok{    [}\DecValTok{55}\NormalTok{, }\DecValTok{1}\NormalTok{, }\DecValTok{8}\NormalTok{, }\DecValTok{3}\NormalTok{, }\DecValTok{91}\NormalTok{, }\DecValTok{56}\NormalTok{, }\DecValTok{17}\NormalTok{, }\DecValTok{54}\NormalTok{, }\DecValTok{23}\NormalTok{, }\DecValTok{12}\NormalTok{],}
\NormalTok{    [}\DecValTok{19}\NormalTok{, }\DecValTok{99}\NormalTok{, }\DecValTok{56}\NormalTok{, }\DecValTok{72}\NormalTok{, }\DecValTok{6}\NormalTok{, }\DecValTok{13}\NormalTok{, }\DecValTok{34}\NormalTok{, }\DecValTok{16}\NormalTok{, }\DecValTok{77}\NormalTok{, }\DecValTok{56}\NormalTok{],}
\NormalTok{    [}\DecValTok{37}\NormalTok{, }\DecValTok{75}\NormalTok{, }\DecValTok{67}\NormalTok{, }\DecValTok{5}\NormalTok{, }\DecValTok{46}\NormalTok{, }\DecValTok{98}\NormalTok{, }\DecValTok{57}\NormalTok{, }\DecValTok{19}\NormalTok{, }\DecValTok{14}\NormalTok{, }\DecValTok{7}\NormalTok{],}
\NormalTok{    [}\DecValTok{4}\NormalTok{, }\DecValTok{57}\NormalTok{, }\DecValTok{32}\NormalTok{, }\DecValTok{78}\NormalTok{, }\DecValTok{56}\NormalTok{, }\DecValTok{12}\NormalTok{, }\DecValTok{43}\NormalTok{, }\DecValTok{61}\NormalTok{, }\DecValTok{3}\NormalTok{, }\DecValTok{88}\NormalTok{],}
\NormalTok{    [}\DecValTok{96}\NormalTok{, }\DecValTok{16}\NormalTok{, }\DecValTok{92}\NormalTok{, }\DecValTok{18}\NormalTok{, }\DecValTok{50}\NormalTok{, }\DecValTok{90}\NormalTok{, }\DecValTok{35}\NormalTok{, }\DecValTok{15}\NormalTok{, }\DecValTok{36}\NormalTok{, }\DecValTok{97}\NormalTok{],}
\NormalTok{    [}\DecValTok{75}\NormalTok{, }\DecValTok{4}\NormalTok{, }\DecValTok{38}\NormalTok{, }\DecValTok{53}\NormalTok{, }\DecValTok{1}\NormalTok{, }\DecValTok{79}\NormalTok{, }\DecValTok{56}\NormalTok{, }\DecValTok{73}\NormalTok{, }\DecValTok{45}\NormalTok{, }\DecValTok{56}\NormalTok{],}
\NormalTok{    [}\DecValTok{15}\NormalTok{, }\DecValTok{76}\NormalTok{, }\DecValTok{11}\NormalTok{, }\DecValTok{93}\NormalTok{, }\DecValTok{87}\NormalTok{, }\DecValTok{8}\NormalTok{, }\DecValTok{2}\NormalTok{, }\DecValTok{58}\NormalTok{, }\DecValTok{86}\NormalTok{, }\DecValTok{94}\NormalTok{],}
\NormalTok{    [}\DecValTok{51}\NormalTok{, }\DecValTok{14}\NormalTok{, }\DecValTok{60}\NormalTok{, }\DecValTok{57}\NormalTok{, }\DecValTok{74}\NormalTok{, }\DecValTok{42}\NormalTok{, }\DecValTok{59}\NormalTok{, }\DecValTok{71}\NormalTok{, }\DecValTok{88}\NormalTok{, }\DecValTok{52}\NormalTok{],}
\NormalTok{    [}\DecValTok{49}\NormalTok{, }\DecValTok{6}\NormalTok{, }\DecValTok{43}\NormalTok{, }\DecValTok{39}\NormalTok{, }\DecValTok{17}\NormalTok{, }\DecValTok{18}\NormalTok{, }\DecValTok{95}\NormalTok{, }\DecValTok{6}\NormalTok{, }\DecValTok{44}\NormalTok{, }\DecValTok{75}\NormalTok{]}
\NormalTok{])}
\end{Highlighting}
\end{Shaded}

\begin{tcolorbox}[enhanced jigsaw, left=2mm, leftrule=.75mm, bottomrule=.15mm, title={Lösung}, colback=white, arc=.35mm, breakable, titlerule=0mm, bottomtitle=1mm, colbacktitle=quarto-callout-caution-color!10!white, toprule=.15mm, opacityback=0, coltitle=black, rightrule=.15mm, opacitybacktitle=0.6, toptitle=1mm, colframe=quarto-callout-caution-color-frame]

\begin{itemize}
\tightlist
\item
  Rot: matrix{[}1,3{]}
\item
  Grün: matrix{[}4:6,2:6{]}
\item
  Pink: matrix{[}:,7{]}
\item
  Orange: matrix{[}7,:5{]}
\item
  Blau: matrix{[}-1,-1{]}
\end{itemize}

\end{tcolorbox}

\end{tcolorbox}

\chapter{Array Manipulation}\label{array-manipulation}

\section{Ändern der Form}\label{uxe4ndern-der-form}

Durch verschiedene Funktionen lassen sich die Form und die Einträge der
Arrays verändern.

Eine der wichtigsten Array Operationen ist das Transponieren. Dabei
werden Reihen in Spalten und Spalten in Reihe umgewandelt.

\begin{Shaded}
\begin{Highlighting}[]
\NormalTok{matrix }\OperatorTok{=}\NormalTok{ np.array([[}\DecValTok{1}\NormalTok{, }\DecValTok{2}\NormalTok{, }\DecValTok{3}\NormalTok{], [}\DecValTok{4}\NormalTok{, }\DecValTok{5}\NormalTok{, }\DecValTok{6}\NormalTok{]])}
\BuiltInTok{print}\NormalTok{(matrix)}
\end{Highlighting}
\end{Shaded}

\begin{verbatim}
[[1 2 3]
 [4 5 6]]
\end{verbatim}

Transponieren wir dieses Array nun erhalten wir:

\begin{Shaded}
\begin{Highlighting}[]
\BuiltInTok{print}\NormalTok{(np.transpose(matrix))}
\end{Highlighting}
\end{Shaded}

\begin{verbatim}
[[1 4]
 [2 5]
 [3 6]]
\end{verbatim}

Haben wir ein nun diese Matrix und wollen daraus einen Vektor erstellen
so können wir die Funktion \texttt{np.flatten()} benutzen:

\begin{Shaded}
\begin{Highlighting}[]
\NormalTok{vector }\OperatorTok{=}\NormalTok{ matrix.flatten()}
\BuiltInTok{print}\NormalTok{(vector)}
\end{Highlighting}
\end{Shaded}

\begin{verbatim}
[1 2 3 4 5 6]
\end{verbatim}

Um wieder eine zweidimensionale Datenstruktur zu erhalten, benutzen wir
die Funktion \texttt{np.reshape(Ziel,\ Form)}

\begin{Shaded}
\begin{Highlighting}[]
\BuiltInTok{print}\NormalTok{(np.reshape(matrix, [}\DecValTok{3}\NormalTok{, }\DecValTok{2}\NormalTok{]))}
\end{Highlighting}
\end{Shaded}

\begin{verbatim}
[[1 2]
 [3 4]
 [5 6]]
\end{verbatim}

Möchten wir den Inhalt eines bereits bestehenden Arrays erweitern,
verkleinern oder ändern bietet NumPy ebenfalls die passenden Funktionen.

Haben wir ein leeres Array oder wollen wir ein schon volles Array
erweitern benutzen wir die Funktion \texttt{np.append()}. Dabei hängen
wir einen Wert an das bereits bestehende Array an.

\begin{Shaded}
\begin{Highlighting}[]
\NormalTok{liste }\OperatorTok{=}\NormalTok{ np.array([}\DecValTok{1}\NormalTok{, }\DecValTok{2}\NormalTok{, }\DecValTok{3}\NormalTok{, }\DecValTok{4}\NormalTok{, }\DecValTok{5}\NormalTok{, }\DecValTok{6}\NormalTok{])}

\NormalTok{neue\_liste }\OperatorTok{=}\NormalTok{ np.append(liste, }\DecValTok{7}\NormalTok{)}
\BuiltInTok{print}\NormalTok{(neue\_liste)}
\end{Highlighting}
\end{Shaded}

\begin{verbatim}
[1 2 3 4 5 6 7]
\end{verbatim}

Gegebenenfalls ist es nötig einen Wert nicht am Ende, sondern an einer
beliebigen Position im Array einzufügen. Das passende Werkzeug ist hier
die Funktion \texttt{np.insert(Array,\ Position,\ Einschub)}. Im
folgenden Beispiel wird an der dritten Stelle die Zahl 7 eingesetzt.

\begin{Shaded}
\begin{Highlighting}[]
\NormalTok{liste }\OperatorTok{=}\NormalTok{ np.array([}\DecValTok{1}\NormalTok{, }\DecValTok{2}\NormalTok{, }\DecValTok{3}\NormalTok{, }\DecValTok{4}\NormalTok{, }\DecValTok{5}\NormalTok{, }\DecValTok{6}\NormalTok{])}

\NormalTok{neue\_liste }\OperatorTok{=}\NormalTok{ np.insert(liste, }\DecValTok{3}\NormalTok{, }\DecValTok{7}\NormalTok{)}
\BuiltInTok{print}\NormalTok{(neue\_liste)}
\end{Highlighting}
\end{Shaded}

\begin{verbatim}
[1 2 3 7 4 5 6]
\end{verbatim}

Wenn sich neue Elemente einfügen lassen, können natürlich auch Elemente
gelöscht werden. Hierfür wird die Funktion
\texttt{np.delete(Array\ ,\ Position)} benutzt, die ein Array und die
Position der zu löschenden Funktion übergeben bekommt.

\begin{Shaded}
\begin{Highlighting}[]
\NormalTok{liste }\OperatorTok{=}\NormalTok{ np.array([}\DecValTok{1}\NormalTok{, }\DecValTok{2}\NormalTok{, }\DecValTok{3}\NormalTok{, }\DecValTok{4}\NormalTok{, }\DecValTok{5}\NormalTok{, }\DecValTok{6}\NormalTok{])}

\NormalTok{neue\_liste }\OperatorTok{=}\NormalTok{ np.delete(liste, }\DecValTok{3}\NormalTok{)}
\BuiltInTok{print}\NormalTok{(neue\_liste)}
\end{Highlighting}
\end{Shaded}

\begin{verbatim}
[1 2 3 5 6]
\end{verbatim}

Zuletzt wollen wir uns noch die Verbindung zweier Arrays anschauen. Im
folgenden Beispiel wird dabei das Array \texttt{b} an das Array
\texttt{a} mithilfe der Funktion
\texttt{np.concatenate((Array\ a,\ Array\ b))}angehängt.

\begin{Shaded}
\begin{Highlighting}[]
\NormalTok{a }\OperatorTok{=}\NormalTok{ np.array([}\DecValTok{1}\NormalTok{, }\DecValTok{2}\NormalTok{, }\DecValTok{3}\NormalTok{, }\DecValTok{4}\NormalTok{, }\DecValTok{5}\NormalTok{, }\DecValTok{6}\NormalTok{])}
\NormalTok{b }\OperatorTok{=}\NormalTok{ np.array([}\DecValTok{7}\NormalTok{, }\DecValTok{8}\NormalTok{, }\DecValTok{9}\NormalTok{, }\DecValTok{10}\NormalTok{])}

\NormalTok{neue\_liste }\OperatorTok{=}\NormalTok{ np.concatenate((a, b))}
\BuiltInTok{print}\NormalTok{(neue\_liste)}
\end{Highlighting}
\end{Shaded}

\begin{verbatim}
[ 1  2  3  4  5  6  7  8  9 10]
\end{verbatim}

\section{Sortieren von Arrays}\label{sortieren-von-arrays}

NumPy bietet auch die Möglichkeit, Arrays zu sortieren. Im folgenden
Beispiel starten wir mit einem unsortierten Array. Mit der Funktion
\texttt{np.sort()} erhalten wir ein sortiertes Array.

\begin{Shaded}
\begin{Highlighting}[]
\ImportTok{import}\NormalTok{ numpy }\ImportTok{as}\NormalTok{ np}
\NormalTok{unsortiert }\OperatorTok{=}\NormalTok{ np.array([}\DecValTok{4}\NormalTok{, }\DecValTok{2}\NormalTok{, }\DecValTok{1}\NormalTok{, }\DecValTok{6}\NormalTok{, }\DecValTok{3}\NormalTok{, }\DecValTok{5}\NormalTok{])}

\NormalTok{sortiert }\OperatorTok{=}\NormalTok{ np.sort(unsortiert)}

\BuiltInTok{print}\NormalTok{(sortiert)}
\end{Highlighting}
\end{Shaded}

\begin{verbatim}
[1 2 3 4 5 6]
\end{verbatim}

\section{Unterlisten mit einzigartigen
Werten}\label{unterlisten-mit-einzigartigen-werten}

Arbeitet man mit Daten bei denen zum Beispiel Projekte Personalnummern
zugeordnet werden hat man Daten mit einer endlichen Anzahl an
Personalnummern, die jedoch mehrfach vorkommen können wenn diese an mehr
als einem Projekt gleichzeitig arbeiten.

Möchte man nun eine Liste die jede Nummer nur einmal enthält, kann die
Funtkion \texttt{np.unique} verwendet werden.

\begin{Shaded}
\begin{Highlighting}[]
\ImportTok{import}\NormalTok{ numpy }\ImportTok{as}\NormalTok{ np}
\NormalTok{liste\_mit\_dopplungen }\OperatorTok{=}\NormalTok{ np.array([}\DecValTok{4}\NormalTok{, }\DecValTok{1}\NormalTok{, }\DecValTok{1}\NormalTok{, }\DecValTok{6}\NormalTok{, }\DecValTok{3}\NormalTok{, }\DecValTok{4}\NormalTok{, }\DecValTok{7}\NormalTok{, }\DecValTok{3}\NormalTok{, }\DecValTok{3}\NormalTok{])}

\NormalTok{einzigartige\_werte }\OperatorTok{=}\NormalTok{ np.unique(liste\_mit\_dopplungen)}

\BuiltInTok{print}\NormalTok{(einzigartige\_werte)}
\end{Highlighting}
\end{Shaded}

\begin{verbatim}
[1 3 4 6 7]
\end{verbatim}

Setzt man dann noch die Option \texttt{return\_counts=True} kann in
einer zweiten Variable gespeichert werden, wie oft jeder Wert vorkommt.

\begin{Shaded}
\begin{Highlighting}[]
\ImportTok{import}\NormalTok{ numpy }\ImportTok{as}\NormalTok{ np}
\NormalTok{liste\_mit\_dopplungen }\OperatorTok{=}\NormalTok{ np.array([}\DecValTok{4}\NormalTok{, }\DecValTok{1}\NormalTok{, }\DecValTok{1}\NormalTok{, }\DecValTok{6}\NormalTok{, }\DecValTok{3}\NormalTok{, }\DecValTok{4}\NormalTok{, }\DecValTok{7}\NormalTok{, }\DecValTok{3}\NormalTok{, }\DecValTok{3}\NormalTok{])}

\NormalTok{einzigartige\_werte, anzahl }\OperatorTok{=}\NormalTok{ np.unique(liste\_mit\_dopplungen, return\_counts}\OperatorTok{=}\VariableTok{True}\NormalTok{)}

\BuiltInTok{print}\NormalTok{(anzahl)}
\end{Highlighting}
\end{Shaded}

\begin{verbatim}
[2 3 2 1 1]
\end{verbatim}

\begin{tcolorbox}[enhanced jigsaw, left=2mm, leftrule=.75mm, bottomrule=.15mm, title=\textcolor{quarto-callout-tip-color}{\faLightbulb}\hspace{0.5em}{Zwischenübung: Arraymanipulation}, colback=white, arc=.35mm, breakable, titlerule=0mm, bottomtitle=1mm, colbacktitle=quarto-callout-tip-color!10!white, toprule=.15mm, opacityback=0, coltitle=black, rightrule=.15mm, opacitybacktitle=0.6, toptitle=1mm, colframe=quarto-callout-tip-color-frame]

Gegeben ist das folgende zweidimensionale Array matrix:

\begin{Shaded}
\begin{Highlighting}[]
\NormalTok{matrix }\OperatorTok{=}\NormalTok{ np.array([}
\NormalTok{    [}\DecValTok{4}\NormalTok{, }\DecValTok{7}\NormalTok{, }\DecValTok{2}\NormalTok{, }\DecValTok{8}\NormalTok{],}
\NormalTok{    [}\DecValTok{1}\NormalTok{, }\DecValTok{5}\NormalTok{, }\DecValTok{3}\NormalTok{, }\DecValTok{6}\NormalTok{],}
\NormalTok{    [}\DecValTok{9}\NormalTok{, }\DecValTok{2}\NormalTok{, }\DecValTok{4}\NormalTok{, }\DecValTok{7}\NormalTok{]}
\NormalTok{])}
\end{Highlighting}
\end{Shaded}

\begin{enumerate}
\def\labelenumi{\arabic{enumi}.}
\tightlist
\item
  Ändern Sie die Form des Arrays matrix in ein eindimensionales Array.
\item
  Sortieren Sie das eindimensionale Array in aufsteigender Reihenfolge.
\item
  Ändern Sie die Form des sortierten Arrays in ein zweidimensionales
  Array mit 2 Zeilen und 6 Spalten.
\item
  Bestimmen Sie die eindeutigen Elemente im ursprünglichen Array matrix
  und geben Sie diese aus.
\end{enumerate}

\begin{tcolorbox}[enhanced jigsaw, left=2mm, leftrule=.75mm, bottomrule=.15mm, title={Lösung}, colback=white, arc=.35mm, breakable, titlerule=0mm, bottomtitle=1mm, colbacktitle=quarto-callout-caution-color!10!white, toprule=.15mm, opacityback=0, coltitle=black, rightrule=.15mm, opacitybacktitle=0.6, toptitle=1mm, colframe=quarto-callout-caution-color-frame]

\begin{Shaded}
\begin{Highlighting}[]
\NormalTok{matrix }\OperatorTok{=}\NormalTok{ np.array([}
\NormalTok{    [}\DecValTok{4}\NormalTok{, }\DecValTok{7}\NormalTok{, }\DecValTok{2}\NormalTok{, }\DecValTok{8}\NormalTok{],}
\NormalTok{    [}\DecValTok{1}\NormalTok{, }\DecValTok{5}\NormalTok{, }\DecValTok{3}\NormalTok{, }\DecValTok{6}\NormalTok{],}
\NormalTok{    [}\DecValTok{9}\NormalTok{, }\DecValTok{2}\NormalTok{, }\DecValTok{4}\NormalTok{, }\DecValTok{7}\NormalTok{]}
\NormalTok{])}

\CommentTok{\# 1. Ändern der Form in ein eindimensionales Array}
\NormalTok{flat\_array }\OperatorTok{=}\NormalTok{ matrix.flatten()}

\CommentTok{\# 2. Sortieren des eindimensionalen Arrays in aufsteigender Reihenfolge}
\NormalTok{sorted\_array }\OperatorTok{=}\NormalTok{ np.sort(flat\_array)}

\CommentTok{\# 3. Ändern der Form des sortierten Arrays in ein 2x6{-}Array}
\NormalTok{reshaped\_array }\OperatorTok{=}\NormalTok{ sorted\_array.reshape(}\DecValTok{2}\NormalTok{, }\DecValTok{6}\NormalTok{)}

\CommentTok{\# 4. Bestimmen der eindeutigen Elemente im ursprünglichen Array}
\NormalTok{unique\_elements\_original }\OperatorTok{=}\NormalTok{ np.unique(matrix)}
\end{Highlighting}
\end{Shaded}

\end{tcolorbox}

\end{tcolorbox}

\chapter{Lesen und Schreiben von
Dateien}\label{lesen-und-schreiben-von-dateien}

Das Modul numpy stellt Funktionen zum Lesen und Schreiben von
strukturierten Textdateien bereit.

\section{Lesen von Dateien}\label{lesen-von-dateien}

Zum Lesen von strukturierten Textdateien, z.B. im CSV-Format (comma
separated values), kann die \texttt{np.loadtxt()}-Funktion verwendet
werden. Diese bekommt als Argumente den einzulesenden Dateinamen und
weitere Optionen zur Definition der Struktur der Daten. Der Rückgabewert
ist ein (mehrdimensionales) Array.

Im folgenden Beispiel wird die Datei
\href{https://firedynamics.github.io/LectureComputerScience/_downloads/0d1a3bfbc82fa134e08585d6151e9f46/TC01.csv}{TC01.csv}
eingelesen und deren Inhalt graphisch dargestellt. Die erste Zeile der
Datei wird dabei ignoriert, da sie als Kommentar -- eingeleitet durch
das \#-Zeichen -- interpretiert wird.

\begin{Shaded}
\begin{Highlighting}[]
\NormalTok{dateiname }\OperatorTok{=} \StringTok{\textquotesingle{}01{-}daten/TC01.csv\textquotesingle{}}
\NormalTok{daten }\OperatorTok{=}\NormalTok{ np.loadtxt(dateiname)}
\end{Highlighting}
\end{Shaded}

\begin{Shaded}
\begin{Highlighting}[]
\BuiltInTok{print}\NormalTok{(}\StringTok{"Daten:"}\NormalTok{, daten)}
\BuiltInTok{print}\NormalTok{(}\StringTok{"Form:"}\NormalTok{, daten.shape)}
\end{Highlighting}
\end{Shaded}

\begin{verbatim}
Daten: [20.1 20.1 20.1 ... 24.3 24.2 24.2]
Form: (1513,)
\end{verbatim}

\begin{Shaded}
\begin{Highlighting}[]
\NormalTok{plt.plot(daten)}
\NormalTok{plt.xlabel(}\StringTok{\textquotesingle{}Datenindex\textquotesingle{}}\NormalTok{)}
\NormalTok{plt.ylabel(}\StringTok{\textquotesingle{}Temperatur in °C\textquotesingle{}}\NormalTok{)}\OperatorTok{;}
\end{Highlighting}
\end{Shaded}

\includegraphics{books/w-python-numpy-grundlagen/skript/array_read_n_write_files/figure-pdf/cell-5-output-1.pdf}

Standardmäßig erwartet die \texttt{np.loadtxt()}-Funktion Komma
separierte Werte. Werden die Daten durch ein anderes Trennzeichen
getrennt, kann mit der Option \texttt{delimiter\ =\ ""} ein anderes
Trenzeichen ausgewählt werden. Beispielsweise würde der Funktionsaufruf
bei einem Semikolon folgendermaßen aussehen:
\texttt{np.loadtxt(data.txt,\ delimiter\ =\ ";")}

Beginnt die Datei mit den Daten mit Zeilen bezüglich zusätzlichen
Informationen wie Einheiten oder Experimentdaten, können diese mit der
Option \texttt{skiprows=\ \#Reihen}übersprungen werden.

\section{Schreiben von Dateien}\label{schreiben-von-dateien}

Zum Schreiben von Arrays in Dateien, kann die in numpy verfügbare
Funktion \texttt{np.savetxt()} verwendet werden. Dieser müssen
mindestens die zu schreibenden Arrays als auch ein Dateiname übergeben
werden. Darüber hinaus sind zahlreiche Formatierungs- bzw.
Strukturierungsoptionen möglich.

Folgendes Beispiel skaliert die oben eingelesenen Daten und schreib
jeden zehnten Wert in eine Datei. Dabei wird auch ein Kommentar
(\texttt{header}-Argument) am Anfang der Datei erzeugt. Das
Ausgabeformat der Zahlen kann mit dem \texttt{fmt}-Argument angegeben
werden. Das Format ähnelt der Darstellungsweise, welche bei den
formatierten Zeichenketten vorgestellt wurde.

\begin{Shaded}
\begin{Highlighting}[]
\NormalTok{wertebereich }\OperatorTok{=}\NormalTok{ np.}\BuiltInTok{max}\NormalTok{(daten) }\OperatorTok{{-}}\NormalTok{ np.}\BuiltInTok{min}\NormalTok{(daten)}
\NormalTok{daten\_skaliert }\OperatorTok{=}\NormalTok{ ( daten }\OperatorTok{{-}}\NormalTok{ np.}\BuiltInTok{min}\NormalTok{(daten) ) }\OperatorTok{/}\NormalTok{ wertebereich}
\NormalTok{daten\_skaliert }\OperatorTok{=}\NormalTok{ daten\_skaliert[::}\DecValTok{10}\NormalTok{]}
\end{Highlighting}
\end{Shaded}

\begin{Shaded}
\begin{Highlighting}[]
\NormalTok{plt.plot(daten\_skaliert)}
\NormalTok{plt.xlabel(}\StringTok{\textquotesingle{}Datenindex\textquotesingle{}}\NormalTok{)}
\NormalTok{plt.ylabel(}\StringTok{\textquotesingle{}Skalierte Temperatur\textquotesingle{}}\NormalTok{)}\OperatorTok{;}
\end{Highlighting}
\end{Shaded}

\includegraphics{books/w-python-numpy-grundlagen/skript/array_read_n_write_files/figure-pdf/cell-7-output-1.pdf}

Beim Schreiben der Datei wird ein mehrzeiliger Kommentar mithilfe des
Zeilenumbruchzeichens \texttt{\textbackslash{}n} definiert. Die Ausgabe
der Gleitkommazahlen wird mit \texttt{\%5.2f} formatiert, was 5 Stellen
insgesamt und zwei Nachkommastellen entspricht.

\begin{Shaded}
\begin{Highlighting}[]
\CommentTok{\# Zuweisung ist auf mehrere Zeilen aufgeteilt, aufgrund der }
\CommentTok{\# schmalen Darstellung im Skript}
\NormalTok{kommentar }\OperatorTok{=} \SpecialStringTok{f\textquotesingle{}Daten aus }\SpecialCharTok{\{}\NormalTok{dateiname}\SpecialCharTok{\}}\SpecialStringTok{ skaliert auf den Bereich \textquotesingle{}} \OperatorTok{+} \OperatorTok{\textbackslash{}}
             \StringTok{\textquotesingle{}0 bis 1 }\CharTok{\textbackslash{}n}\StringTok{originales Min / Max:\textquotesingle{}} \OperatorTok{+} \OperatorTok{\textbackslash{}}
            \SpecialStringTok{f\textquotesingle{}}\SpecialCharTok{\{}\NormalTok{np}\SpecialCharTok{.}\BuiltInTok{min}\NormalTok{(daten)}\SpecialCharTok{\}}\SpecialStringTok{/}\SpecialCharTok{\{}\NormalTok{np}\SpecialCharTok{.}\BuiltInTok{max}\NormalTok{(daten)}\SpecialCharTok{\}}\SpecialStringTok{\textquotesingle{}}
\NormalTok{neu\_dateiname }\OperatorTok{=} \StringTok{\textquotesingle{}01{-}daten/TC01\_skaliert.csv\textquotesingle{}}

\NormalTok{np.savetxt(neu\_dateiname, daten\_skaliert, }
\NormalTok{           header}\OperatorTok{=}\NormalTok{kommentar, fmt}\OperatorTok{=}\StringTok{\textquotesingle{}}\SpecialCharTok{\%5.2f}\StringTok{\textquotesingle{}}\NormalTok{)}
\end{Highlighting}
\end{Shaded}

Zum Veranschaulichen werden die ersten Zeilen der neuen Datei
ausgegeben.

\begin{Shaded}
\begin{Highlighting}[]
\CommentTok{\# Einlesen der ersten Zeilen der neu erstellten Datei}
\NormalTok{datei }\OperatorTok{=} \BuiltInTok{open}\NormalTok{(neu\_dateiname, }\StringTok{\textquotesingle{}r\textquotesingle{}}\NormalTok{)}
\ControlFlowTok{for}\NormalTok{ i }\KeywordTok{in} \BuiltInTok{range}\NormalTok{(}\DecValTok{10}\NormalTok{):}
    \BuiltInTok{print}\NormalTok{( datei.readline() , end}\OperatorTok{=}\StringTok{\textquotesingle{}\textquotesingle{}}\NormalTok{)}
\NormalTok{datei.close()}
\end{Highlighting}
\end{Shaded}

\begin{verbatim}
# Daten aus 01-daten/TC01.csv skaliert auf den Bereich 0 bis 1 
# originales Min / Max:20.1/31.1
 0.00
 0.00
 0.00
 0.01
 0.01
 0.01
 0.01
 0.01
\end{verbatim}

\chapter{Arbeiten mit Bildern}\label{arbeiten-mit-bildern}

Bilder werden digital als Matrizen gespeichert. Dabei werden pro Pixel
drei Farbwerte (rot, grün, blau) gespeichert. Aus diesen drei Farbwerten
(Wert 0-255) werden dann alle gewünschten Farben zusammengestellt.

\begin{figure}

\centering{

\includegraphics{index_files/mediabag/books/w-python-numpy-grundlagen/skript/../skript/00-bilder/pixel_mona_lisa_split.pdf}

}

\caption{\label{fig-pixel_colors}Ein hochaufgelöstes Bild besteht aus
sehr vielen Pixeln. Jedes Pixel enthät 3 Farbwerte, einen für die Fabre
Grün, einen für Blau und einen für Rot.}

\end{figure}%

Aufgrund der digitalen Darstellung von Bildern lassen sich diese mit den
Werkzeugen von NumPy leicht bearbeiten. Wir verwenden für folgendes
Beispiel als Bild die Monas Lisa. Das Bild ist unter folgendem
\href{https://upload.wikimedia.org/wikipedia/commons/thumb/6/6a/Mona_Lisa.jpg/677px-Mona_Lisa.jpg}{Link}
zu finden.

Importieren wir dieses Bild nun mit der Funktion \texttt{imread()}aus
dem matplotlib-package, sehen wir das es um ein dreidimensionales numpy
Array handelt.

\begin{Shaded}
\begin{Highlighting}[]
\ImportTok{import}\NormalTok{ matplotlib.pyplot }\ImportTok{as}\NormalTok{ plt}

\NormalTok{data }\OperatorTok{=}\NormalTok{ plt.imread(}\StringTok{"00{-}bilder/mona\_lisa.jpg"}\NormalTok{)}
\BuiltInTok{print}\NormalTok{(}\StringTok{"Form:"}\NormalTok{, data.shape)}
\end{Highlighting}
\end{Shaded}

\begin{verbatim}
Form: (1024, 677, 3)
\end{verbatim}

Schauen wir uns einmal mit der \texttt{print()}-Funktion einen
Ausschnitt dieser Daten an.

\begin{Shaded}
\begin{Highlighting}[]
\BuiltInTok{print}\NormalTok{(data)}
\end{Highlighting}
\end{Shaded}

\begin{verbatim}
[[[ 68  62  38]
  [ 88  82  56]
  [ 92  87  55]
  ...
  [ 54  97  44]
  [ 68 110  60]
  [ 69 111  63]]

 [[ 65  59  33]
  [ 68  63  34]
  [ 83  78  46]
  ...
  [ 66 103  51]
  [ 66 103  52]
  [ 66 102  56]]

 [[ 97  90  62]
  [ 87  80  51]
  [ 78  72  38]
  ...
  [ 79 106  53]
  [ 62  89  38]
  [ 62  88  41]]

 ...

 [[ 25  14  18]
  [ 21  10  14]
  [ 20   9  13]
  ...
  [ 11   5   9]
  [ 11   5   9]
  [ 10   4   8]]

 [[ 23  12  16]
  [ 23  12  16]
  [ 21  10  14]
  ...
  [ 11   5   9]
  [ 11   5   9]
  [ 10   4   8]]

 [[ 22  11  15]
  [ 26  15  19]
  [ 24  13  17]
  ...
  [ 11   5   9]
  [ 10   4   8]
  [  9   3   7]]]
\end{verbatim}

Mit der Funktion \texttt{plt.imshow} kann das Bild in Echtfarben
dargestellt werden. Dies funktioniert, da die Funktion die einzelnen
Ebenen, hier der letzte Index, des Datensatzes als Farbinformationen
(rot, grün, blau) interpretiert. Wäre noch eine vierte Ebene dabei,
würde sie als individueller Transparenzwert verwendet worden.

\begin{Shaded}
\begin{Highlighting}[]
\NormalTok{plt.imshow(data)}
\end{Highlighting}
\end{Shaded}

\includegraphics{books/w-python-numpy-grundlagen/skript/array_images_files/figure-pdf/cell-5-output-1.pdf}

Natürlich können auch die einzelnen Farbebenen individuell betrachtet
werden. Dazu wird der letzte Index festgehalten. Hier betrachten wir nur
den reten Anteil des Bildes. Stellen wir ein einfaches Array dar, werden
die Daten in schwarz-weiß ausgegeben. Mit Hilfe der Option
\texttt{cmap=\textquotesingle{}Reds\textquotesingle{}} können wir die
Farbskala anpassen.

\begin{Shaded}
\begin{Highlighting}[]
\CommentTok{\# Als Farbskale wird die Rotskala }
\CommentTok{\# verwendet \textquotesingle{}Reds\textquotesingle{}}
\NormalTok{plt.imshow( data[:,:,}\DecValTok{0}\NormalTok{], cmap}\OperatorTok{=}\StringTok{\textquotesingle{}Reds\textquotesingle{}}\NormalTok{ )}
\NormalTok{plt.colorbar()}
\NormalTok{plt.show()}
\end{Highlighting}
\end{Shaded}

\includegraphics{books/w-python-numpy-grundlagen/skript/array_images_files/figure-pdf/cell-6-output-1.pdf}

Da die Bilddaten als Arrays gespeichert sind, sind viele der möglichen
Optionen, z.B. zur Teilauswahl oder Operationen, verfügbar. Das untere
Beispiel zeigt einen Ausschnitt im Rotkanal des Bildes.

\begin{Shaded}
\begin{Highlighting}[]
\NormalTok{bereich }\OperatorTok{=}\NormalTok{ np.array(data[}\DecValTok{450}\NormalTok{:}\DecValTok{500}\NormalTok{, }\DecValTok{550}\NormalTok{:}\DecValTok{600}\NormalTok{,}\DecValTok{0}\NormalTok{], dtype}\OperatorTok{=}\BuiltInTok{float}\NormalTok{)}
\NormalTok{plt.imshow( bereich, cmap}\OperatorTok{=}\StringTok{"Greys"}\NormalTok{ )}
\NormalTok{plt.colorbar()}
\end{Highlighting}
\end{Shaded}

\includegraphics{books/w-python-numpy-grundlagen/skript/array_images_files/figure-pdf/cell-7-output-1.pdf}

Betrachten wir nun eine komplexere Operation an Bilddaten, den
\href{https://de.wikipedia.org/wiki/Laplace-Operator}{Laplace-Operator}.
Er kann genutzt werden um Ränder von Objekten zu identifizieren. Dazu
wird für jeden Bildpunkt \(B_{i,j}\) -- außer an den Rändern
--~folgender Wert \(\phi_{i, j}\) berechnet:

\[ \phi_{i, j} = \left|B_{i-1, j} + B_{i, j-1} - 4\cdot B_{i, j} + B_{i+1, j} + B_{i, j+1}\right| \]

Folgende Funktion implementiert diese Operation. Darüber hinaus werden
alle Werte von \(\phi\) unterhalb eines Schwellwerts auf Null und
oberhalb auf 255 gesetzt.

\begin{Shaded}
\begin{Highlighting}[]
\KeywordTok{def}\NormalTok{ img\_lap(data, schwellwert}\OperatorTok{=}\DecValTok{25}\NormalTok{):}
    
    \CommentTok{\# Erstellung einer Kopie der Daten, nun jedoch als}
    \CommentTok{\# Array mit Gleitkommazahlen}
\NormalTok{    bereich }\OperatorTok{=}\NormalTok{ np.array(data, dtype}\OperatorTok{=}\BuiltInTok{float}\NormalTok{)}
    
    \CommentTok{\# Aufteilung der obigen Gleichung in zwei Teile}
\NormalTok{    lapx }\OperatorTok{=}\NormalTok{ bereich[}\DecValTok{2}\NormalTok{:, :] }\OperatorTok{{-}} \DecValTok{2}\OperatorTok{*}\NormalTok{bereich[}\DecValTok{1}\NormalTok{:}\OperatorTok{{-}}\DecValTok{1}\NormalTok{, :] }\OperatorTok{+}\NormalTok{ bereich[:}\OperatorTok{{-}}\DecValTok{2}\NormalTok{, :]}
\NormalTok{    lapy }\OperatorTok{=}\NormalTok{ bereich[:, }\DecValTok{2}\NormalTok{:] }\OperatorTok{{-}} \DecValTok{2}\OperatorTok{*}\NormalTok{bereich[:, }\DecValTok{1}\NormalTok{:}\OperatorTok{{-}}\DecValTok{1}\NormalTok{] }\OperatorTok{+}\NormalTok{ bereich[:, :}\OperatorTok{{-}}\DecValTok{2}\NormalTok{]}
    
    \CommentTok{\# Zusammenführung der Teile und Bildung des Betrags}
\NormalTok{    lap }\OperatorTok{=}\NormalTok{ np.}\BuiltInTok{abs}\NormalTok{(lapx[:,}\DecValTok{1}\NormalTok{:}\OperatorTok{{-}}\DecValTok{1}\NormalTok{] }\OperatorTok{+}\NormalTok{ lapy[}\DecValTok{1}\NormalTok{:}\OperatorTok{{-}}\DecValTok{1}\NormalTok{, :])}
    
    \CommentTok{\# Schwellwertanalyse}
\NormalTok{    lap[lap }\OperatorTok{\textgreater{}}\NormalTok{ schwellwert] }\OperatorTok{=} \DecValTok{255}
\NormalTok{    lap[lap }\OperatorTok{\textless{}}\NormalTok{ schwellwert] }\OperatorTok{=} \DecValTok{0}
    
    \ControlFlowTok{return}\NormalTok{ lap}
\end{Highlighting}
\end{Shaded}

Betrachten wir ein Bild vom Haspel Campus in Wuppertal ein:
\href{https://firedynamics.github.io/LectureComputerScience/_downloads/592f1fc843fc7c01bdcad17bf85ec15c/campus_haspel.jpeg}{Bild}.
Die Anwendung des Laplace-Operators auf den oberen Bildausschnitt ergibt
folgende Ausgabe:

\begin{Shaded}
\begin{Highlighting}[]
\NormalTok{data }\OperatorTok{=}\NormalTok{ plt.imread(}\StringTok{\textquotesingle{}01{-}daten/campus\_haspel.jpeg\textquotesingle{}}\NormalTok{)}
\NormalTok{bereich }\OperatorTok{=}\NormalTok{ np.array(data[}\DecValTok{1320}\NormalTok{:}\DecValTok{1620}\NormalTok{, }\DecValTok{400}\NormalTok{:}\DecValTok{700}\NormalTok{, }\DecValTok{1}\NormalTok{], dtype}\OperatorTok{=}\BuiltInTok{float}\NormalTok{)}

\NormalTok{lap }\OperatorTok{=}\NormalTok{ img\_lap(bereich)}

\NormalTok{plt.figure(figsize}\OperatorTok{=}\NormalTok{(}\DecValTok{9}\NormalTok{, }\DecValTok{3}\NormalTok{))}

\NormalTok{ax }\OperatorTok{=}\NormalTok{ plt.subplot(}\DecValTok{1}\NormalTok{, }\DecValTok{3}\NormalTok{, }\DecValTok{1}\NormalTok{)}
\NormalTok{ax.imshow(data, cmap}\OperatorTok{=}\StringTok{"Greys\_r"}\NormalTok{)}

\NormalTok{ax }\OperatorTok{=}\NormalTok{ plt.subplot(}\DecValTok{1}\NormalTok{, }\DecValTok{3}\NormalTok{, }\DecValTok{2}\NormalTok{)}
\NormalTok{ax.imshow(bereich, cmap}\OperatorTok{=}\StringTok{"Greys\_r"}\NormalTok{)}\OperatorTok{;}

\NormalTok{ax }\OperatorTok{=}\NormalTok{ plt.subplot(}\DecValTok{1}\NormalTok{, }\DecValTok{3}\NormalTok{, }\DecValTok{3}\NormalTok{)}
\NormalTok{ax.imshow(lap, cmap}\OperatorTok{=}\StringTok{"Greys"}\NormalTok{)}\OperatorTok{;}
\end{Highlighting}
\end{Shaded}

\includegraphics{books/w-python-numpy-grundlagen/skript/array_images_files/figure-pdf/cell-9-output-1.pdf}

Wir können damit ganz klar die Formen des Fensters erkennen.

Wollen wir zum Beispiel eine Farbkomponente bearbeiten und dann das Bild
wieder zusammensetzen, benötigen wir die Funktion
\texttt{np.dstack((rot,\ grün,\ blau)).astype(\textquotesingle{}uint8\textquotesingle{})},
wobei \texttt{rot}, \texttt{grün}und \texttt{blau} die jeweiligen
2D-Arrays sind. Versuchen wir nun die grüne Farbe aus dem Baum links zu
entfernen.

Wichtig ist, dass die Daten nach dem Zusammensetzen im Format
\texttt{uint8} vorliegen, deswegen die Methode
\texttt{.astype(\textquotesingle{}uint8\textquotesingle{})}.

\begin{Shaded}
\begin{Highlighting}[]
\NormalTok{data }\OperatorTok{=}\NormalTok{ plt.imread(}\StringTok{\textquotesingle{}01{-}daten/campus\_haspel.jpeg\textquotesingle{}}\NormalTok{)}

\CommentTok{\# Speichern der einzelnen Farben in Arrays}
\NormalTok{rot }\OperatorTok{=}\NormalTok{ np.array(data[:, :, }\DecValTok{0}\NormalTok{], dtype}\OperatorTok{=}\BuiltInTok{float}\NormalTok{)}
\NormalTok{gruen }\OperatorTok{=}\NormalTok{ np.array(data[:, :, }\DecValTok{1}\NormalTok{], dtype}\OperatorTok{=}\BuiltInTok{float}\NormalTok{)}
\NormalTok{blau }\OperatorTok{=}\NormalTok{ np.array(data[:, :, }\DecValTok{2}\NormalTok{], dtype}\OperatorTok{=}\BuiltInTok{float}\NormalTok{)}

\CommentTok{\# Setzen wir den Bereich des linken Baumes im Array auf 0}
\NormalTok{gruen\_neu }\OperatorTok{=}\NormalTok{ gruen.copy()}
\NormalTok{gruen\_neu[}\DecValTok{800}\NormalTok{:}\DecValTok{2000}\NormalTok{, }\DecValTok{700}\NormalTok{:}\DecValTok{1700}\NormalTok{] }\OperatorTok{=} \DecValTok{0}

\NormalTok{zusammengesetzt }\OperatorTok{=}\NormalTok{ np.dstack((rot, gruen\_neu, blau)).astype(}\StringTok{\textquotesingle{}uint8\textquotesingle{}}\NormalTok{)}

\NormalTok{plt.figure(figsize}\OperatorTok{=}\NormalTok{(}\DecValTok{8}\NormalTok{, }\DecValTok{5}\NormalTok{))}

\NormalTok{ax }\OperatorTok{=}\NormalTok{ plt.subplot(}\DecValTok{1}\NormalTok{, }\DecValTok{2}\NormalTok{, }\DecValTok{1}\NormalTok{)}
\NormalTok{ax.imshow(data, cmap}\OperatorTok{=}\StringTok{"Greys\_r"}\NormalTok{)}

\NormalTok{ax }\OperatorTok{=}\NormalTok{ plt.subplot(}\DecValTok{1}\NormalTok{, }\DecValTok{2}\NormalTok{, }\DecValTok{2}\NormalTok{)}
\NormalTok{ax.imshow(zusammengesetzt)}
\end{Highlighting}
\end{Shaded}

\includegraphics{books/w-python-numpy-grundlagen/skript/array_images_files/figure-pdf/cell-10-output-1.pdf}

\begin{tcolorbox}[enhanced jigsaw, left=2mm, leftrule=.75mm, bottomrule=.15mm, title=\textcolor{quarto-callout-tip-color}{\faLightbulb}\hspace{0.5em}{Zwischenübung: Bilder bearbeiten}, colback=white, arc=.35mm, breakable, titlerule=0mm, bottomtitle=1mm, colbacktitle=quarto-callout-tip-color!10!white, toprule=.15mm, opacityback=0, coltitle=black, rightrule=.15mm, opacitybacktitle=0.6, toptitle=1mm, colframe=quarto-callout-tip-color-frame]

Lesen Sie folgendes Bild vom Haspel Campus in Wuppertal ein:
\href{https://firedynamics.github.io/LectureComputerScience/_downloads/592f1fc843fc7c01bdcad17bf85ec15c/campus_haspel.jpeg}{Bild}

Extrahieren Sie den blauen Anteil und lassen Sie sich die Zeile in der
Mitte des Bildes ausgeben, so wie einen beliebigen Bildauschnitt.

\begin{tcolorbox}[enhanced jigsaw, left=2mm, leftrule=.75mm, bottomrule=.15mm, title={Lösung}, colback=white, arc=.35mm, breakable, titlerule=0mm, bottomtitle=1mm, colbacktitle=quarto-callout-caution-color!10!white, toprule=.15mm, opacityback=0, coltitle=black, rightrule=.15mm, opacitybacktitle=0.6, toptitle=1mm, colframe=quarto-callout-caution-color-frame]

\begin{Shaded}
\begin{Highlighting}[]
\ImportTok{import}\NormalTok{ numpy }\ImportTok{as}\NormalTok{ np}
\ImportTok{import}\NormalTok{ matplotlib.pyplot }\ImportTok{as}\NormalTok{ plt}

\NormalTok{data }\OperatorTok{=}\NormalTok{ plt.imread(}\StringTok{\textquotesingle{}01{-}daten/campus\_haspel.jpeg\textquotesingle{}}\NormalTok{)}

\NormalTok{form }\OperatorTok{=}\NormalTok{  data.shape}
\BuiltInTok{print}\NormalTok{( }\StringTok{"Form:"}\NormalTok{, data.shape )}

\NormalTok{blau }\OperatorTok{=}\NormalTok{  data[:,:,}\DecValTok{2}\NormalTok{]}
\NormalTok{plt.imshow(blau, cmap}\OperatorTok{=}\StringTok{\textquotesingle{}Blues\textquotesingle{}}\NormalTok{)}

\NormalTok{zeile }\OperatorTok{=}\NormalTok{  data[}\BuiltInTok{int}\NormalTok{(form[}\DecValTok{0}\NormalTok{]}\OperatorTok{/}\DecValTok{2}\NormalTok{),:,}\DecValTok{2}\NormalTok{]}
\BuiltInTok{print}\NormalTok{(zeile)}

\NormalTok{ausschnitt }\OperatorTok{=}\NormalTok{  data[}\DecValTok{10}\NormalTok{:}\DecValTok{50}\NormalTok{,}\DecValTok{10}\NormalTok{:}\DecValTok{50}\NormalTok{,:]}
\NormalTok{plt.imshow(ausschnitt)}
\end{Highlighting}
\end{Shaded}

\begin{verbatim}
Form: (3024, 4032, 3)
[221 220 220 ...  28  28  28]
\end{verbatim}

\includegraphics{books/w-python-numpy-grundlagen/skript/array_images_files/figure-pdf/cell-11-output-2.pdf}

\end{tcolorbox}

\end{tcolorbox}

\chapter{Lernzielkontrolle}\label{lernzielkontrolle-1}

Herzlich willkommen zur Lernzielkontrolle!

Diese Selbstlernkontrolle dient dazu, Ihr Verständnis der bisher
behandelten Themen zu überprüfen und Ihnen die Möglichkeit zu geben,
Ihren Lernfortschritt eigenständig zu bewerten. Sie ist so konzipiert,
dass Sie Ihre Stärken und Schwächen erkennen und gezielt an den
Bereichen arbeiten können, die noch verbessert werden müssen.

Es stehen hier zwei Möglichkeiten zur Verfügung ihr Wissen zu prüfen.
Sie können das Quiz benutzen, welches Sie automatisch durch die
verschiedenen Themen führt. ALternativ finden Sie darunter normale Frage
wie Sie bisher im Skript verwendet wurden.

Bitte nehmen Sie sich ausreichend Zeit für die Bearbeitung der Fragen
und gehen Sie diese in Ruhe durch. Seien Sie ehrlich zu sich selbst und
versuchen Sie, die Aufgaben ohne Hilfsmittel zu lösen, um ein
realistisches Bild Ihres aktuellen Wissensstands zu erhalten. Sollten
Sie bei einer Frage Schwierigkeiten haben, ist dies ein Hinweis darauf,
dass Sie in diesem Bereich noch weiter üben sollten.

Viel Erfolg bei der Bearbeitung und beim weiteren Lernen!

\section*{Aufgabe 1}\label{aufgabe-1}
\addcontentsline{toc}{section}{Aufgabe 1}

\markright{Aufgabe 1}

Wie wird das NumPy-Paket typischerweise eingebunden?

\section*{Aufgabe 2}\label{aufgabe-2}
\addcontentsline{toc}{section}{Aufgabe 2}

\markright{Aufgabe 2}

Erstellen Sie mit Hilfe von NumPy die folgenden Arrays:

\begin{enumerate}
\def\labelenumi{\arabic{enumi}.}
\tightlist
\item
  Erstellen sie aus der Liste {[}1, 2, 3{]} ein numPy Array
\item
  Ein eindimensionales Array, das die Zahlen von 0 bis 9 enthält.
\item
  Ein zweidimensionales Array der Form 3×33×3, das nur aus Einsen
  besteht.
\item
  Ein eindimensionales Array, das die Zahlen von 10 bis 50
  (einschließlich) in Schritten von 5 enthält.
\end{enumerate}

\section*{Aufgabe 3}\label{aufgabe-3}
\addcontentsline{toc}{section}{Aufgabe 3}

\markright{Aufgabe 3}

Was ist der Unterschied zwischenden den Funktionen \texttt{np.ndim},
\texttt{np.shape} und \texttt{np.size}

\section*{Aufgabe 4}\label{aufgabe-4}
\addcontentsline{toc}{section}{Aufgabe 4}

\markright{Aufgabe 4}

Welchen Datentyp besitzt folgendes Array? Mit welcher Funktion kann ich
den Datentypen eines Arrays auslesen?

\begin{Shaded}
\begin{Highlighting}[]
\NormalTok{vector }\OperatorTok{=}\NormalTok{ np.array([ }\FloatTok{4.8}\NormalTok{,  }\FloatTok{8.2}\NormalTok{, }\FloatTok{15.6}\NormalTok{, }\FloatTok{16.6}\NormalTok{, }\FloatTok{23.2}\NormalTok{, }\FloatTok{42.8}\NormalTok{ ])}
\end{Highlighting}
\end{Shaded}

\section*{Aufgabe 5}\label{aufgabe-5}
\addcontentsline{toc}{section}{Aufgabe 5}

\markright{Aufgabe 5}

Führen Sie mit den folgenden zwei Arrays diese mathematischen
Operationen durch:

a = {[}5, 1, 3, 6, 4{]} und b = {[}6, 5, 2, 6, 9{]}

\begin{enumerate}
\def\labelenumi{\arabic{enumi}.}
\tightlist
\item
  Addieren Sie beide Arrays
\item
  Berechnen Sie das elementweise Produkt von a und b
\item
  Addieren Sie zu jedem Eintrag von a 3 dazu
\end{enumerate}

\section*{Aufgabe 6}\label{aufgabe-6}
\addcontentsline{toc}{section}{Aufgabe 6}

\markright{Aufgabe 6}

a = {[}9, 2, 3, 1, 3{]}

\begin{enumerate}
\def\labelenumi{\arabic{enumi}.}
\tightlist
\item
  Bestimmen Sie Mittelwert und Standardabweichung für das Array a
\item
  Bestimmen Sie Minimum und Maximum der Liste
\end{enumerate}

\section*{Aufgabe 7}\label{aufgabe-7}
\addcontentsline{toc}{section}{Aufgabe 7}

\markright{Aufgabe 7}

\begin{Shaded}
\begin{Highlighting}[]
\NormalTok{matrix }\OperatorTok{=}\NormalTok{ np.array([}
\NormalTok{    [ }\DecValTok{1}\NormalTok{,  }\DecValTok{2}\NormalTok{,  }\DecValTok{3}\NormalTok{,  }\DecValTok{4}\NormalTok{,  }\DecValTok{5}\NormalTok{],}
\NormalTok{    [ }\DecValTok{6}\NormalTok{,  }\DecValTok{7}\NormalTok{,  }\DecValTok{8}\NormalTok{,  }\DecValTok{9}\NormalTok{, }\DecValTok{10}\NormalTok{],}
\NormalTok{    [}\DecValTok{11}\NormalTok{, }\DecValTok{12}\NormalTok{, }\DecValTok{13}\NormalTok{, }\DecValTok{14}\NormalTok{, }\DecValTok{15}\NormalTok{],}
\NormalTok{    [}\DecValTok{16}\NormalTok{, }\DecValTok{17}\NormalTok{, }\DecValTok{18}\NormalTok{, }\DecValTok{19}\NormalTok{, }\DecValTok{20}\NormalTok{],}
\NormalTok{    [}\DecValTok{21}\NormalTok{, }\DecValTok{22}\NormalTok{, }\DecValTok{23}\NormalTok{, }\DecValTok{24}\NormalTok{, }\DecValTok{25}\NormalTok{]}
\NormalTok{])}
\end{Highlighting}
\end{Shaded}

\begin{enumerate}
\def\labelenumi{\arabic{enumi}.}
\tightlist
\item
  Extrahieren Sie die erste Zeile.
\item
  Extrahieren Sie die letzte Spalte.
\item
  Extrahieren Sie die Untermatrix, die aus den Zeilen 2 bis 4 und den
  Spalten 1 bis 3 besteht.
\end{enumerate}

\section*{Aufgabe 8}\label{aufgabe-8}
\addcontentsline{toc}{section}{Aufgabe 8}

\markright{Aufgabe 8}

\begin{Shaded}
\begin{Highlighting}[]
\NormalTok{array }\OperatorTok{=}\NormalTok{ np.arange(}\DecValTok{1}\NormalTok{, }\DecValTok{21}\NormalTok{)}
\end{Highlighting}
\end{Shaded}

\begin{enumerate}
\def\labelenumi{\arabic{enumi}.}
\tightlist
\item
  Ändern Sie die Form des Arrays in eine zweidimensionale Matrix der
  Form 4×5.
\item
  Ändern Sie die Form des Arrays in eine zweidimensionale Matrix der
  Form 5×4.
\item
  Ändern Sie die Form des Arrays in eine dreidimensionale Matrix der
  Form 2×2×5.
\item
  Flachen Sie das dreidimensionale Array aus Aufgabe 3 wieder zu einem
  eindimensionalen Array ab.
\item
  Transponieren Sie die 4×54×5-Matrix aus Aufgabe 1.
\end{enumerate}

\section*{Aufgabe 9}\label{aufgabe-9}
\addcontentsline{toc}{section}{Aufgabe 9}

\markright{Aufgabe 9}

Mit welchen zwei Funktionen können Daten aus einer Datei gelesen und in
einer Datei gespeichert werden?

\section*{Aufgabe 10}\label{aufgabe-10}
\addcontentsline{toc}{section}{Aufgabe 10}

\markright{Aufgabe 10}

Sie möchten aus einem Bild die Bilddaten einer Farkomponente isolieren.
Was müssen Sie dafür tun?

\begin{tcolorbox}[enhanced jigsaw, left=2mm, leftrule=.75mm, bottomrule=.15mm, title={Lösung}, colback=white, arc=.35mm, breakable, titlerule=0mm, bottomtitle=1mm, colbacktitle=quarto-callout-caution-color!10!white, toprule=.15mm, opacityback=0, coltitle=black, rightrule=.15mm, opacitybacktitle=0.6, toptitle=1mm, colframe=quarto-callout-caution-color-frame]

\subsection*{Aufgabe 1}\label{aufgabe-1-1}
\addcontentsline{toc}{subsection}{Aufgabe 1}

\begin{Shaded}
\begin{Highlighting}[]
\ImportTok{import}\NormalTok{ numpy }\ImportTok{as}\NormalTok{ np}
\end{Highlighting}
\end{Shaded}

\subsection*{Aufgabe 2}\label{aufgabe-2-1}
\addcontentsline{toc}{subsection}{Aufgabe 2}

\begin{Shaded}
\begin{Highlighting}[]
\CommentTok{\# 1.}
\NormalTok{np.array([}\DecValTok{1}\NormalTok{, }\DecValTok{2}\NormalTok{, }\DecValTok{3}\NormalTok{])}

\CommentTok{\# 2. }
\BuiltInTok{print}\NormalTok{(np.arange(}\DecValTok{10}\NormalTok{))}

\CommentTok{\# 3. }
\BuiltInTok{print}\NormalTok{(np.ones((}\DecValTok{3}\NormalTok{, }\DecValTok{3}\NormalTok{)))}

\CommentTok{\# 4. }
\BuiltInTok{print}\NormalTok{(np.arange(}\DecValTok{10}\NormalTok{, }\DecValTok{51}\NormalTok{, }\DecValTok{5}\NormalTok{))}
\end{Highlighting}
\end{Shaded}

\begin{verbatim}
[0 1 2 3 4 5 6 7 8 9]
[[1. 1. 1.]
 [1. 1. 1.]
 [1. 1. 1.]]
[10 15 20 25 30 35 40 45 50]
\end{verbatim}

\subsection*{Aufgabe 3}\label{aufgabe-3-1}
\addcontentsline{toc}{subsection}{Aufgabe 3}

\texttt{np.ndim}: Gibt die Anzahl der Dimensionen zurück
\texttt{np.shape}: Gibt die Längen der einzelnen Dimensionen wieder
\texttt{np.size}: Gibt die Anzahl aller Elemente aus

\subsection*{Aufgabe 4}\label{aufgabe-4-1}
\addcontentsline{toc}{subsection}{Aufgabe 4}

Da es sich hier um Gleitkommazahlen handelt, ist der Datentyp
\texttt{float.64}.

\begin{Shaded}
\begin{Highlighting}[]
\NormalTok{vector }\OperatorTok{=}\NormalTok{ np.array([ }\FloatTok{4.8}\NormalTok{,  }\FloatTok{8.2}\NormalTok{, }\FloatTok{15.6}\NormalTok{, }\FloatTok{16.6}\NormalTok{, }\FloatTok{23.2}\NormalTok{, }\FloatTok{42.8}\NormalTok{ ])}
\BuiltInTok{print}\NormalTok{(vector.dtype)}
\end{Highlighting}
\end{Shaded}

\begin{verbatim}
float64
\end{verbatim}

\subsection*{Aufgabe 5}\label{aufgabe-5-1}
\addcontentsline{toc}{subsection}{Aufgabe 5}

\begin{Shaded}
\begin{Highlighting}[]
\NormalTok{a }\OperatorTok{=}\NormalTok{ np.array([}\DecValTok{5}\NormalTok{, }\DecValTok{1}\NormalTok{, }\DecValTok{3}\NormalTok{, }\DecValTok{6}\NormalTok{, }\DecValTok{4}\NormalTok{])}
\NormalTok{b }\OperatorTok{=}\NormalTok{ np.array([}\DecValTok{6}\NormalTok{, }\DecValTok{5}\NormalTok{, }\DecValTok{2}\NormalTok{, }\DecValTok{6}\NormalTok{, }\DecValTok{9}\NormalTok{])}

\CommentTok{\# 1.}
\NormalTok{ergebnis }\OperatorTok{=}\NormalTok{ a }\OperatorTok{+}\NormalTok{ b}
\BuiltInTok{print}\NormalTok{(}\StringTok{"Die Summe beider Vektoren ergibt: "}\NormalTok{, ergebnis) }

\CommentTok{\# 2.}
\NormalTok{ergebnis }\OperatorTok{=}\NormalTok{ a }\OperatorTok{*}\NormalTok{ b}
\BuiltInTok{print}\NormalTok{(}\StringTok{"Das Produkt beider Vektoren ergibt: "}\NormalTok{, ergebnis) }

\CommentTok{\# 3.}
\NormalTok{ergebnis }\OperatorTok{=}\NormalTok{ a }\OperatorTok{+} \DecValTok{3}
\BuiltInTok{print}\NormalTok{(}\StringTok{"Die Summe von a und 3 ergibt: "}\NormalTok{, ergebnis) }
\end{Highlighting}
\end{Shaded}

\begin{verbatim}
Die Summe beider Vektoren ergibt:  [11  6  5 12 13]
Das Produkt beider Vektoren ergibt:  [30  5  6 36 36]
Die Summe von a und 3 ergibt:  [8 4 6 9 7]
\end{verbatim}

\subsection*{Aufgabe 6}\label{aufgabe-6-1}
\addcontentsline{toc}{subsection}{Aufgabe 6}

\begin{Shaded}
\begin{Highlighting}[]
\NormalTok{a }\OperatorTok{=}\NormalTok{ np.array([}\DecValTok{9}\NormalTok{, }\DecValTok{2}\NormalTok{, }\DecValTok{3}\NormalTok{, }\DecValTok{1}\NormalTok{, }\DecValTok{3}\NormalTok{])}

\CommentTok{\# 1.}
\NormalTok{mittelwert }\OperatorTok{=}\NormalTok{ np.mean(a)}
\BuiltInTok{print}\NormalTok{(}\StringTok{"Der Mittelwert ist: "}\NormalTok{, mittelwert)}

\NormalTok{standardabweichung }\OperatorTok{=}\NormalTok{ np.std(a)}
\BuiltInTok{print}\NormalTok{(}\StringTok{"Die Standardabweichung von a beträgt: "}\NormalTok{, standardabweichung) }

\CommentTok{\# 2.}
\NormalTok{minimum }\OperatorTok{=}\NormalTok{ np.}\BuiltInTok{min}\NormalTok{(a)}
\BuiltInTok{print}\NormalTok{(}\StringTok{"Das Minimum beträgt: "}\NormalTok{, minimum)}

\NormalTok{maximum }\OperatorTok{=}\NormalTok{ np.}\BuiltInTok{max}\NormalTok{(a)}
\BuiltInTok{print}\NormalTok{(}\StringTok{"Das Maximum beträgt: "}\NormalTok{, maximum)}
\end{Highlighting}
\end{Shaded}

\begin{verbatim}
Der Mittelwert ist:  3.6
Die Standardabweichung von a beträgt:  2.8000000000000003
Das Minimum beträgt:  1
Das Maximum beträgt:  9
\end{verbatim}

\subsection*{Aufgabe 7}\label{aufgabe-7-1}
\addcontentsline{toc}{subsection}{Aufgabe 7}

\begin{Shaded}
\begin{Highlighting}[]
\NormalTok{matrix }\OperatorTok{=}\NormalTok{ np.array([}
\NormalTok{    [ }\DecValTok{1}\NormalTok{,  }\DecValTok{2}\NormalTok{,  }\DecValTok{3}\NormalTok{,  }\DecValTok{4}\NormalTok{,  }\DecValTok{5}\NormalTok{],}
\NormalTok{    [ }\DecValTok{6}\NormalTok{,  }\DecValTok{7}\NormalTok{,  }\DecValTok{8}\NormalTok{,  }\DecValTok{9}\NormalTok{, }\DecValTok{10}\NormalTok{],}
\NormalTok{    [}\DecValTok{11}\NormalTok{, }\DecValTok{12}\NormalTok{, }\DecValTok{13}\NormalTok{, }\DecValTok{14}\NormalTok{, }\DecValTok{15}\NormalTok{],}
\NormalTok{    [}\DecValTok{16}\NormalTok{, }\DecValTok{17}\NormalTok{, }\DecValTok{18}\NormalTok{, }\DecValTok{19}\NormalTok{, }\DecValTok{20}\NormalTok{],}
\NormalTok{    [}\DecValTok{21}\NormalTok{, }\DecValTok{22}\NormalTok{, }\DecValTok{23}\NormalTok{, }\DecValTok{24}\NormalTok{, }\DecValTok{25}\NormalTok{]}
\NormalTok{])}

\CommentTok{\# 1. Erste Zeile}
\BuiltInTok{print}\NormalTok{(matrix[}\DecValTok{0}\NormalTok{,:])}

\CommentTok{\# 2.}
\BuiltInTok{print}\NormalTok{(matrix[:,}\OperatorTok{{-}}\DecValTok{1}\NormalTok{])}

\CommentTok{\# 3.}
\BuiltInTok{print}\NormalTok{(matrix[}\DecValTok{1}\NormalTok{:}\DecValTok{4}\NormalTok{,}\DecValTok{0}\NormalTok{:}\DecValTok{3}\NormalTok{])}
\end{Highlighting}
\end{Shaded}

\begin{verbatim}
[1 2 3 4 5]
[ 5 10 15 20 25]
[[ 6  7  8]
 [11 12 13]
 [16 17 18]]
\end{verbatim}

\subsection*{Aufgabe 8}\label{aufgabe-8-1}
\addcontentsline{toc}{subsection}{Aufgabe 8}

\begin{Shaded}
\begin{Highlighting}[]
\NormalTok{array }\OperatorTok{=}\NormalTok{ np.arange(}\DecValTok{1}\NormalTok{, }\DecValTok{21}\NormalTok{)}

\CommentTok{\# 1. Ändern der Form in eine 4x5{-}Matrix}
\NormalTok{matrix\_4x5 }\OperatorTok{=}\NormalTok{ array.reshape(}\DecValTok{4}\NormalTok{, }\DecValTok{5}\NormalTok{)}

\CommentTok{\# 2. Ändern der Form in eine 5x4{-}Matrix}
\NormalTok{matrix\_5x4 }\OperatorTok{=}\NormalTok{ array.reshape(}\DecValTok{5}\NormalTok{, }\DecValTok{4}\NormalTok{)}

\CommentTok{\# 3. Ändern der Form in eine 2x2x5{-}Matrix}
\NormalTok{matrix\_2x2x5 }\OperatorTok{=}\NormalTok{ array.reshape(}\DecValTok{2}\NormalTok{, }\DecValTok{2}\NormalTok{, }\DecValTok{5}\NormalTok{)}

\CommentTok{\# 4. Abflachen der 2x2x5{-}Matrix zu einem eindimensionalen Array}
\NormalTok{flattened\_array }\OperatorTok{=}\NormalTok{ matrix\_2x2x5.flatten()}

\CommentTok{\# 5. Transponieren der 4x5{-}Matrix}
\NormalTok{transposed\_matrix }\OperatorTok{=}\NormalTok{ matrix\_4x5.T}

\CommentTok{\# Ausgabe der Ergebnisse (optional)}
\BuiltInTok{print}\NormalTok{(}\StringTok{"Originales Array:"}\NormalTok{, array)}
\BuiltInTok{print}\NormalTok{(}\StringTok{"4x5{-}Matrix:}\CharTok{\textbackslash{}n}\StringTok{"}\NormalTok{, matrix\_4x5)}
\BuiltInTok{print}\NormalTok{(}\StringTok{"5x4{-}Matrix:}\CharTok{\textbackslash{}n}\StringTok{"}\NormalTok{, matrix\_5x4)}
\BuiltInTok{print}\NormalTok{(}\StringTok{"2x2x5{-}Matrix:}\CharTok{\textbackslash{}n}\StringTok{"}\NormalTok{, matrix\_2x2x5)}
\BuiltInTok{print}\NormalTok{(}\StringTok{"Abgeflachtes Array:"}\NormalTok{, flattened\_array)}
\BuiltInTok{print}\NormalTok{(}\StringTok{"Transponierte 4x5{-}Matrix:}\CharTok{\textbackslash{}n}\StringTok{"}\NormalTok{, transposed\_matrix)}
\end{Highlighting}
\end{Shaded}

\begin{verbatim}
Originales Array: [ 1  2  3  4  5  6  7  8  9 10 11 12 13 14 15 16 17 18 19 20]
4x5-Matrix:
 [[ 1  2  3  4  5]
 [ 6  7  8  9 10]
 [11 12 13 14 15]
 [16 17 18 19 20]]
5x4-Matrix:
 [[ 1  2  3  4]
 [ 5  6  7  8]
 [ 9 10 11 12]
 [13 14 15 16]
 [17 18 19 20]]
2x2x5-Matrix:
 [[[ 1  2  3  4  5]
  [ 6  7  8  9 10]]

 [[11 12 13 14 15]
  [16 17 18 19 20]]]
Abgeflachtes Array: [ 1  2  3  4  5  6  7  8  9 10 11 12 13 14 15 16 17 18 19 20]
Transponierte 4x5-Matrix:
 [[ 1  6 11 16]
 [ 2  7 12 17]
 [ 3  8 13 18]
 [ 4  9 14 19]
 [ 5 10 15 20]]
\end{verbatim}

\subsection*{Aufgabe 9}\label{aufgabe-9-1}
\addcontentsline{toc}{subsection}{Aufgabe 9}

Die passenden Funktionen sind \texttt{np.loadtxt()} und
\texttt{np.savetxt()}.

\subsection*{Aufgabe 10}\label{aufgabe-10-1}
\addcontentsline{toc}{subsection}{Aufgabe 10}

Typischerweise sind Bilddaten große Matrizen wobei die Farben in drei
unterschieldichen Matrizen gespeichert werden. Dabei ist die
Farbreihenfolge oft ``Rot'', ``Grün'' und ``Blau''. Dementsprechen
müssen wir wenn wie Daten in der Matrix \texttt{data} gespeichert sind
mit Slicing eine Dimension auswählen: \texttt{data{[}:,:,0{]}}, wobei
die Zahl 0-2 für die jeweilige Farbe steht.

\end{tcolorbox}

\chapter{Übung}\label{uxfcbung}

\section{Aufgabe 1 Filmdatenbank}\label{aufgabe-1-filmdatenbank}

In der ersten Aufgabe wollen wir fiktive Daten für Filmbewertungen
untersuchen. Das Datenset ist dabei vereinfacht und beinhaltet folgende
Spalten:

\begin{enumerate}
\def\labelenumi{\arabic{enumi}.}
\tightlist
\item
  Film ID
\item
  Benutzer ID
\item
  Bewertung
\end{enumerate}

Hier ist das Datenset:

\begin{Shaded}
\begin{Highlighting}[]
\ImportTok{import}\NormalTok{ numpy }\ImportTok{as}\NormalTok{ np}

\NormalTok{bewertungen }\OperatorTok{=}\NormalTok{ np.array([}
\NormalTok{    [}\DecValTok{1}\NormalTok{, }\DecValTok{101}\NormalTok{, }\FloatTok{4.5}\NormalTok{],}
\NormalTok{    [}\DecValTok{1}\NormalTok{, }\DecValTok{102}\NormalTok{, }\FloatTok{3.0}\NormalTok{],}
\NormalTok{    [}\DecValTok{2}\NormalTok{, }\DecValTok{101}\NormalTok{, }\FloatTok{2.5}\NormalTok{],}
\NormalTok{    [}\DecValTok{2}\NormalTok{, }\DecValTok{103}\NormalTok{, }\FloatTok{4.0}\NormalTok{],}
\NormalTok{    [}\DecValTok{3}\NormalTok{, }\DecValTok{101}\NormalTok{, }\FloatTok{5.0}\NormalTok{],}
\NormalTok{    [}\DecValTok{3}\NormalTok{, }\DecValTok{104}\NormalTok{, }\FloatTok{3.5}\NormalTok{],}
\NormalTok{    [}\DecValTok{3}\NormalTok{, }\DecValTok{105}\NormalTok{, }\FloatTok{4.0}\NormalTok{]}
\NormalTok{])}
\end{Highlighting}
\end{Shaded}

\begin{tcolorbox}[enhanced jigsaw, left=2mm, leftrule=.75mm, bottomrule=.15mm, title=\textcolor{quarto-callout-tip-color}{\faLightbulb}\hspace{0.5em}{a) Bestimmen Sie die jemals niedrigste und höchste Bewertung, die je
gegeben wurde}, colback=white, arc=.35mm, breakable, titlerule=0mm, bottomtitle=1mm, colbacktitle=quarto-callout-tip-color!10!white, toprule=.15mm, opacityback=0, coltitle=black, rightrule=.15mm, opacitybacktitle=0.6, toptitle=1mm, colframe=quarto-callout-tip-color-frame]

\begin{tcolorbox}[enhanced jigsaw, left=2mm, leftrule=.75mm, bottomrule=.15mm, title={Lösung}, colback=white, arc=.35mm, breakable, titlerule=0mm, bottomtitle=1mm, colbacktitle=quarto-callout-caution-color!10!white, toprule=.15mm, opacityback=0, coltitle=black, rightrule=.15mm, opacitybacktitle=0.6, toptitle=1mm, colframe=quarto-callout-caution-color-frame]

\begin{Shaded}
\begin{Highlighting}[]
\NormalTok{niedrigste\_bewertung }\OperatorTok{=}\NormalTok{ np.}\BuiltInTok{min}\NormalTok{(bewertungen[:,}\DecValTok{2}\NormalTok{])}

\BuiltInTok{print}\NormalTok{(}\StringTok{"Die niedrigste jemals gegebene Bertung ist:"}\NormalTok{, niedrigste\_bewertung)}

\NormalTok{hoechste\_bewertung }\OperatorTok{=}\NormalTok{ np.}\BuiltInTok{max}\NormalTok{(bewertungen[:,}\DecValTok{2}\NormalTok{])}

\BuiltInTok{print}\NormalTok{(}\StringTok{"Die hoechste jemals gegebene Bertung ist:"}\NormalTok{, hoechste\_bewertung)}
\end{Highlighting}
\end{Shaded}

\begin{verbatim}
Die niedrigste jemals gegebene Bertung ist: 2.5
Die hoechste jemals gegebene Bertung ist: 5.0
\end{verbatim}

\end{tcolorbox}

\end{tcolorbox}

\begin{tcolorbox}[enhanced jigsaw, left=2mm, leftrule=.75mm, bottomrule=.15mm, title=\textcolor{quarto-callout-tip-color}{\faLightbulb}\hspace{0.5em}{b) Nennen Sie alle Bewertungen für Film 1}, colback=white, arc=.35mm, breakable, titlerule=0mm, bottomtitle=1mm, colbacktitle=quarto-callout-tip-color!10!white, toprule=.15mm, opacityback=0, coltitle=black, rightrule=.15mm, opacitybacktitle=0.6, toptitle=1mm, colframe=quarto-callout-tip-color-frame]

\begin{tcolorbox}[enhanced jigsaw, left=2mm, leftrule=.75mm, bottomrule=.15mm, title={Lösung}, colback=white, arc=.35mm, breakable, titlerule=0mm, bottomtitle=1mm, colbacktitle=quarto-callout-caution-color!10!white, toprule=.15mm, opacityback=0, coltitle=black, rightrule=.15mm, opacitybacktitle=0.6, toptitle=1mm, colframe=quarto-callout-caution-color-frame]

\begin{Shaded}
\begin{Highlighting}[]
\NormalTok{bewertungen\_film\_1 }\OperatorTok{=}\NormalTok{ bewertungen[np.where(bewertungen[:,}\DecValTok{0}\NormalTok{]}\OperatorTok{==}\DecValTok{1}\NormalTok{)]}

\BuiltInTok{print}\NormalTok{(}\StringTok{"Bewertungen für Film 1:}\CharTok{\textbackslash{}n}\StringTok{"}\NormalTok{, bewertungen\_film\_1)}
\end{Highlighting}
\end{Shaded}

\begin{verbatim}
Bewertungen für Film 1:
 [[  1.  101.    4.5]
 [  1.  102.    3. ]]
\end{verbatim}

\end{tcolorbox}

\end{tcolorbox}

\begin{tcolorbox}[enhanced jigsaw, left=2mm, leftrule=.75mm, bottomrule=.15mm, title=\textcolor{quarto-callout-tip-color}{\faLightbulb}\hspace{0.5em}{c) Nennen Sie alle Bewertungen von Person 101}, colback=white, arc=.35mm, breakable, titlerule=0mm, bottomtitle=1mm, colbacktitle=quarto-callout-tip-color!10!white, toprule=.15mm, opacityback=0, coltitle=black, rightrule=.15mm, opacitybacktitle=0.6, toptitle=1mm, colframe=quarto-callout-tip-color-frame]

\begin{tcolorbox}[enhanced jigsaw, left=2mm, leftrule=.75mm, bottomrule=.15mm, title={Lösung}, colback=white, arc=.35mm, breakable, titlerule=0mm, bottomtitle=1mm, colbacktitle=quarto-callout-caution-color!10!white, toprule=.15mm, opacityback=0, coltitle=black, rightrule=.15mm, opacitybacktitle=0.6, toptitle=1mm, colframe=quarto-callout-caution-color-frame]

\begin{Shaded}
\begin{Highlighting}[]
\NormalTok{bewertungen\_101 }\OperatorTok{=}\NormalTok{ bewertungen[np.where(bewertungen[:,}\DecValTok{1}\NormalTok{]}\OperatorTok{==}\DecValTok{101}\NormalTok{)]}

\BuiltInTok{print}\NormalTok{(}\StringTok{"Bewertungen von Person 101:}\CharTok{\textbackslash{}n}\StringTok{"}\NormalTok{, bewertungen\_101)}
\end{Highlighting}
\end{Shaded}

\begin{verbatim}
Bewertungen von Person 101:
 [[  1.  101.    4.5]
 [  2.  101.    2.5]
 [  3.  101.    5. ]]
\end{verbatim}

\end{tcolorbox}

\end{tcolorbox}

\begin{tcolorbox}[enhanced jigsaw, left=2mm, leftrule=.75mm, bottomrule=.15mm, title=\textcolor{quarto-callout-tip-color}{\faLightbulb}\hspace{0.5em}{d) Berechnen Sie die mittlere Bewertung für jeden Film und geben Sie
diese nacheinander aus}, colback=white, arc=.35mm, breakable, titlerule=0mm, bottomtitle=1mm, colbacktitle=quarto-callout-tip-color!10!white, toprule=.15mm, opacityback=0, coltitle=black, rightrule=.15mm, opacitybacktitle=0.6, toptitle=1mm, colframe=quarto-callout-tip-color-frame]

\begin{tcolorbox}[enhanced jigsaw, left=2mm, leftrule=.75mm, bottomrule=.15mm, title={Lösung}, colback=white, arc=.35mm, breakable, titlerule=0mm, bottomtitle=1mm, colbacktitle=quarto-callout-caution-color!10!white, toprule=.15mm, opacityback=0, coltitle=black, rightrule=.15mm, opacitybacktitle=0.6, toptitle=1mm, colframe=quarto-callout-caution-color-frame]

\begin{Shaded}
\begin{Highlighting}[]
\ControlFlowTok{for}\NormalTok{ ID }\KeywordTok{in}\NormalTok{ [}\DecValTok{1}\NormalTok{, }\DecValTok{2}\NormalTok{, }\DecValTok{3}\NormalTok{]:}

\NormalTok{    mittelwert }\OperatorTok{=}\NormalTok{ np.mean(bewertungen[np.where(bewertungen[:,}\DecValTok{0}\NormalTok{]}\OperatorTok{==}\NormalTok{ID),}\DecValTok{2}\NormalTok{])}

    \BuiltInTok{print}\NormalTok{(}\StringTok{"Die Mittlere Bewertung für Film"}\NormalTok{, ID, }\StringTok{"beträgt:"}\NormalTok{, mittelwert) }
\end{Highlighting}
\end{Shaded}

\begin{verbatim}
Die Mittlere Bewertung für Film 1 beträgt: 3.75
Die Mittlere Bewertung für Film 2 beträgt: 3.25
Die Mittlere Bewertung für Film 3 beträgt: 4.166666666666667
\end{verbatim}

\end{tcolorbox}

\end{tcolorbox}

\begin{tcolorbox}[enhanced jigsaw, left=2mm, leftrule=.75mm, bottomrule=.15mm, title=\textcolor{quarto-callout-tip-color}{\faLightbulb}\hspace{0.5em}{e) Finden SIe den Film mit der höchsten Bewertung}, colback=white, arc=.35mm, breakable, titlerule=0mm, bottomtitle=1mm, colbacktitle=quarto-callout-tip-color!10!white, toprule=.15mm, opacityback=0, coltitle=black, rightrule=.15mm, opacitybacktitle=0.6, toptitle=1mm, colframe=quarto-callout-tip-color-frame]

\begin{tcolorbox}[enhanced jigsaw, left=2mm, leftrule=.75mm, bottomrule=.15mm, title={Lösung}, colback=white, arc=.35mm, breakable, titlerule=0mm, bottomtitle=1mm, colbacktitle=quarto-callout-caution-color!10!white, toprule=.15mm, opacityback=0, coltitle=black, rightrule=.15mm, opacitybacktitle=0.6, toptitle=1mm, colframe=quarto-callout-caution-color-frame]

\begin{Shaded}
\begin{Highlighting}[]
\NormalTok{index\_hoechste\_bewertung }\OperatorTok{=}\NormalTok{ np.argmax(bewertungen[:,}\DecValTok{2}\NormalTok{])}

\BuiltInTok{print}\NormalTok{(bewertungen[index\_hoechste\_bewertung,:])}
\end{Highlighting}
\end{Shaded}

\begin{verbatim}
[  3. 101.   5.]
\end{verbatim}

\end{tcolorbox}

\end{tcolorbox}

\begin{tcolorbox}[enhanced jigsaw, left=2mm, leftrule=.75mm, bottomrule=.15mm, title=\textcolor{quarto-callout-tip-color}{\faLightbulb}\hspace{0.5em}{f) Finden Sie die Person mit den meisten Bewertungen}, colback=white, arc=.35mm, breakable, titlerule=0mm, bottomtitle=1mm, colbacktitle=quarto-callout-tip-color!10!white, toprule=.15mm, opacityback=0, coltitle=black, rightrule=.15mm, opacitybacktitle=0.6, toptitle=1mm, colframe=quarto-callout-tip-color-frame]

\begin{tcolorbox}[enhanced jigsaw, left=2mm, leftrule=.75mm, bottomrule=.15mm, title={Lösung}, colback=white, arc=.35mm, breakable, titlerule=0mm, bottomtitle=1mm, colbacktitle=quarto-callout-caution-color!10!white, toprule=.15mm, opacityback=0, coltitle=black, rightrule=.15mm, opacitybacktitle=0.6, toptitle=1mm, colframe=quarto-callout-caution-color-frame]

\begin{Shaded}
\begin{Highlighting}[]
\NormalTok{einzigartige\_person, anzahl }\OperatorTok{=}\NormalTok{ np.unique(bewertungen[:, }\DecValTok{1}\NormalTok{],return\_counts}\OperatorTok{=}\VariableTok{True}\NormalTok{)}

\NormalTok{meist\_aktiver\_person }\OperatorTok{=}\NormalTok{ einzigartige\_person[np.argmax(anzahl)]}

\BuiltInTok{print}\NormalTok{(}\StringTok{"Personen mit den meisten Bewertungen:"}\NormalTok{, meist\_aktiver\_person)}
\end{Highlighting}
\end{Shaded}

\begin{verbatim}
Personen mit den meisten Bewertungen: 101.0
\end{verbatim}

\end{tcolorbox}

\end{tcolorbox}

\begin{tcolorbox}[enhanced jigsaw, left=2mm, leftrule=.75mm, bottomrule=.15mm, title=\textcolor{quarto-callout-tip-color}{\faLightbulb}\hspace{0.5em}{g) Nennen Sie alle Filme mit einer Wertung von 4 oder besser.}, colback=white, arc=.35mm, breakable, titlerule=0mm, bottomtitle=1mm, colbacktitle=quarto-callout-tip-color!10!white, toprule=.15mm, opacityback=0, coltitle=black, rightrule=.15mm, opacitybacktitle=0.6, toptitle=1mm, colframe=quarto-callout-tip-color-frame]

\begin{tcolorbox}[enhanced jigsaw, left=2mm, leftrule=.75mm, bottomrule=.15mm, title={Lösung}, colback=white, arc=.35mm, breakable, titlerule=0mm, bottomtitle=1mm, colbacktitle=quarto-callout-caution-color!10!white, toprule=.15mm, opacityback=0, coltitle=black, rightrule=.15mm, opacitybacktitle=0.6, toptitle=1mm, colframe=quarto-callout-caution-color-frame]

\begin{Shaded}
\begin{Highlighting}[]
\NormalTok{index\_bewertung\_besser\_vier }\OperatorTok{=}\NormalTok{ bewertungen[:,}\DecValTok{2}\NormalTok{] }\OperatorTok{\textgreater{}=} \DecValTok{4}

\BuiltInTok{print}\NormalTok{(}\StringTok{"Filme mit einer Wertung von 4 oder besser:"}\NormalTok{)}

\BuiltInTok{print}\NormalTok{(bewertungen[index\_bewertung\_besser\_vier,:])}
\end{Highlighting}
\end{Shaded}

\begin{verbatim}
Filme mit einer Wertung von 4 oder besser:
[[  1.  101.    4.5]
 [  2.  103.    4. ]
 [  3.  101.    5. ]
 [  3.  105.    4. ]]
\end{verbatim}

\end{tcolorbox}

\end{tcolorbox}

\begin{tcolorbox}[enhanced jigsaw, left=2mm, leftrule=.75mm, bottomrule=.15mm, title=\textcolor{quarto-callout-tip-color}{\faLightbulb}\hspace{0.5em}{h) Film Nr. 4 ist erschienen. Der Film wurde von Person 102 mit einer
Note von 3.5 bewertet. Fügen Sie diesen zur Datenbank hinzu.}, colback=white, arc=.35mm, breakable, titlerule=0mm, bottomtitle=1mm, colbacktitle=quarto-callout-tip-color!10!white, toprule=.15mm, opacityback=0, coltitle=black, rightrule=.15mm, opacitybacktitle=0.6, toptitle=1mm, colframe=quarto-callout-tip-color-frame]

\begin{tcolorbox}[enhanced jigsaw, left=2mm, leftrule=.75mm, bottomrule=.15mm, title={Lösung}, colback=white, arc=.35mm, breakable, titlerule=0mm, bottomtitle=1mm, colbacktitle=quarto-callout-caution-color!10!white, toprule=.15mm, opacityback=0, coltitle=black, rightrule=.15mm, opacitybacktitle=0.6, toptitle=1mm, colframe=quarto-callout-caution-color-frame]

\begin{Shaded}
\begin{Highlighting}[]
\NormalTok{neue\_bewertung }\OperatorTok{=}\NormalTok{ np.array([}\DecValTok{4}\NormalTok{, }\DecValTok{102}\NormalTok{, }\FloatTok{3.5}\NormalTok{])}

\NormalTok{bewertungen }\OperatorTok{=}\NormalTok{ np.append(bewertungen, [neue\_bewertung], axis}\OperatorTok{=}\DecValTok{0}\NormalTok{)}

\BuiltInTok{print}\NormalTok{(bewertungen)}
\end{Highlighting}
\end{Shaded}

\begin{verbatim}
[[  1.  101.    4.5]
 [  1.  102.    3. ]
 [  2.  101.    2.5]
 [  2.  103.    4. ]
 [  3.  101.    5. ]
 [  3.  104.    3.5]
 [  3.  105.    4. ]
 [  4.  102.    3.5]]
\end{verbatim}

\end{tcolorbox}

\end{tcolorbox}

\begin{tcolorbox}[enhanced jigsaw, left=2mm, leftrule=.75mm, bottomrule=.15mm, title=\textcolor{quarto-callout-tip-color}{\faLightbulb}\hspace{0.5em}{i) Person 102 hat sich Film Nr. 1 nochmal angesehen und hat das Ende
jetzt doch verstanden. Dementsprechend soll die Berwertung jetzt auf 5.0
geändert werden.}, colback=white, arc=.35mm, breakable, titlerule=0mm, bottomtitle=1mm, colbacktitle=quarto-callout-tip-color!10!white, toprule=.15mm, opacityback=0, coltitle=black, rightrule=.15mm, opacitybacktitle=0.6, toptitle=1mm, colframe=quarto-callout-tip-color-frame]

\begin{tcolorbox}[enhanced jigsaw, left=2mm, leftrule=.75mm, bottomrule=.15mm, title={Lösung}, colback=white, arc=.35mm, breakable, titlerule=0mm, bottomtitle=1mm, colbacktitle=quarto-callout-caution-color!10!white, toprule=.15mm, opacityback=0, coltitle=black, rightrule=.15mm, opacitybacktitle=0.6, toptitle=1mm, colframe=quarto-callout-caution-color-frame]

\begin{Shaded}
\begin{Highlighting}[]
\NormalTok{bewertungen[(bewertungen[:, }\DecValTok{0}\NormalTok{] }\OperatorTok{==} \DecValTok{1}\NormalTok{) }\OperatorTok{\&} 
\NormalTok{            (bewertungen[:, }\DecValTok{1}\NormalTok{] }\OperatorTok{==} \DecValTok{102}\NormalTok{), }\DecValTok{2}\NormalTok{] }\OperatorTok{=} \FloatTok{5.0}

\BuiltInTok{print}\NormalTok{(}\StringTok{"Aktualisieren der Bewertung:}\CharTok{\textbackslash{}n}\StringTok{"}\NormalTok{, bewertungen)}
\end{Highlighting}
\end{Shaded}

\begin{verbatim}
Aktualisieren der Bewertung:
 [[  1.  101.    4.5]
 [  1.  102.    5. ]
 [  2.  101.    2.5]
 [  2.  103.    4. ]
 [  3.  101.    5. ]
 [  3.  104.    3.5]
 [  3.  105.    4. ]
 [  4.  102.    3.5]]
\end{verbatim}

\end{tcolorbox}

\end{tcolorbox}

\section{Aufgabe 2 - Kryptographie -
Caesar-Chiffre}\label{aufgabe-2---kryptographie---caesar-chiffre}

In dieser Aufgabe wollen wir Text sowohl ver- als auch entschlüsseln.

Jedes Zeichen hat über die sogenannte ASCII-Tabelle einen Zahlenwert
zugeordnet.

\begin{longtable}[]{@{}llll@{}}
\caption{Ascii-Tabelle}\label{tbl-ascii}\tabularnewline
\toprule\noalign{}
Buchstabe & ASCII Code & Buchstabe & ASCII Code \\
\midrule\noalign{}
\endfirsthead
\toprule\noalign{}
Buchstabe & ASCII Code & Buchstabe & ASCII Code \\
\midrule\noalign{}
\endhead
\bottomrule\noalign{}
\endlastfoot
a & 97 & n & 110 \\
b & 98 & o & 111 \\
c & 99 & p & 112 \\
d & 100 & q & 113 \\
e & 101 & r & 114 \\
f & 102 & s & 115 \\
g & 103 & t & 116 \\
h & 104 & u & 117 \\
i & 105 & v & 118 \\
j & 106 & w & 119 \\
k & 107 & x & 120 \\
l & 108 & y & 121 \\
m & 109 & z & 122 \\
\end{longtable}

Der Einfachheit halber ist im Folgenden schon der Code zur Umwandlung
von Buchstaben in Zahlenwerten und wieder zurück aufgeführt. Außerdem
beschränken wir uns auf Texte mit kleinen Buchstaben.

Ihre Aufgabe ist nun die Zahlenwerte zu verändern.

Zunächste wollen wir eine einfache Caesar-Chiffre anwenden. Dabei werden
alle Buchstaben um eine gewisse Anzahl verschoben. Ist Beispielsweise
der der Verschlüsselungswert ``1'' wird aus einem A ein B, einem M, ein
N. Ist der Wert ``4'' wird aus einem A ein E und aus einem M ein Q. Die
Verschiebung findet zyklisch statt, das heißt bei einer Verschiebung von
1 wird aus einem Z ein A.

\begin{Shaded}
\begin{Highlighting}[]
\ImportTok{import}\NormalTok{ numpy }\ImportTok{as}\NormalTok{ np}

\CommentTok{\# Funktion, die einen Buchstaben in ihren ASCII{-}Wert umwandelt}
\KeywordTok{def}\NormalTok{ buchstabe\_zu\_ascii(c):}
    \ControlFlowTok{return}\NormalTok{ np.array([}\BuiltInTok{ord}\NormalTok{(c)])}

\CommentTok{\# Funktion, die einen ASCII{-}Wert in den passenden Buchstaben umwandelt}
\KeywordTok{def}\NormalTok{ ascii\_zu\_buchstabe(a):}
    \ControlFlowTok{return} \BuiltInTok{chr}\NormalTok{(a)}
\end{Highlighting}
\end{Shaded}

\begin{tcolorbox}[enhanced jigsaw, left=2mm, leftrule=.75mm, bottomrule=.15mm, title=\textcolor{quarto-callout-tip-color}{\faLightbulb}\hspace{0.5em}{1. Überlegen Sie sich zunächst wie man diese zyklische Verschiebung
mathematisch ausdrücken könnte (Hinweis: Modulo Rechnung)}, colback=white, arc=.35mm, breakable, titlerule=0mm, bottomtitle=1mm, colbacktitle=quarto-callout-tip-color!10!white, toprule=.15mm, opacityback=0, coltitle=black, rightrule=.15mm, opacitybacktitle=0.6, toptitle=1mm, colframe=quarto-callout-tip-color-frame]

\begin{tcolorbox}[enhanced jigsaw, left=2mm, leftrule=.75mm, bottomrule=.15mm, title={Lösung}, colback=white, arc=.35mm, breakable, titlerule=0mm, bottomtitle=1mm, colbacktitle=quarto-callout-caution-color!10!white, toprule=.15mm, opacityback=0, coltitle=black, rightrule=.15mm, opacitybacktitle=0.6, toptitle=1mm, colframe=quarto-callout-caution-color-frame]

\[ \textrm{ASCII}_{\textrm{verschoben}} = (\textrm{ASCII} - 97 + \textrm{Versatz}) \textrm{ mod } 26 + 97\]

\end{tcolorbox}

\end{tcolorbox}

\begin{tcolorbox}[enhanced jigsaw, left=2mm, leftrule=.75mm, bottomrule=.15mm, title=\textcolor{quarto-callout-tip-color}{\faLightbulb}\hspace{0.5em}{2. Schreiben Sie Code der mit einer Schleife alle Zeichen umwandelt.}, colback=white, arc=.35mm, breakable, titlerule=0mm, bottomtitle=1mm, colbacktitle=quarto-callout-tip-color!10!white, toprule=.15mm, opacityback=0, coltitle=black, rightrule=.15mm, opacitybacktitle=0.6, toptitle=1mm, colframe=quarto-callout-tip-color-frame]

Zunächst sollen alle Zeichen in Ascii Code umgewandelt werden. Dann wird
die Formel auf die Zahlenwerte angewendet und schlussendlich in einer
dritten schleife wieder alle Werte in Buchstaben übersetzt.

\begin{tcolorbox}[enhanced jigsaw, left=2mm, leftrule=.75mm, bottomrule=.15mm, title={Lösung}, colback=white, arc=.35mm, breakable, titlerule=0mm, bottomtitle=1mm, colbacktitle=quarto-callout-caution-color!10!white, toprule=.15mm, opacityback=0, coltitle=black, rightrule=.15mm, opacitybacktitle=0.6, toptitle=1mm, colframe=quarto-callout-caution-color-frame]

\begin{Shaded}
\begin{Highlighting}[]
\ImportTok{import}\NormalTok{ numpy }\ImportTok{as}\NormalTok{ np}

\CommentTok{\# Funktion, die einen Buchstaben in ihren ASCII{-}Wert umwandelt}
\KeywordTok{def}\NormalTok{ buchstabe\_zu\_ascii(c):}
    \ControlFlowTok{return} \BuiltInTok{ord}\NormalTok{(c)}

\CommentTok{\# Funktion, die einen ASCII{-}Wert in den passenden Buchstaben umwandelt}
\KeywordTok{def}\NormalTok{ ascii\_zu\_buchstabe(a):}
    \ControlFlowTok{return} \BuiltInTok{chr}\NormalTok{(a)}

\NormalTok{klartext }\OperatorTok{=} \StringTok{"abrakadabra"}
\NormalTok{versatz }\OperatorTok{=} \DecValTok{3}

\NormalTok{umgewandelter\_text }\OperatorTok{=}\NormalTok{ []}
\NormalTok{verschluesselte\_zahl }\OperatorTok{=}\NormalTok{ []}
\NormalTok{verschluesselter\_text}\OperatorTok{=}\NormalTok{ []}



\ControlFlowTok{for}\NormalTok{ buchstabe }\KeywordTok{in}\NormalTok{ klartext:}
\NormalTok{    umgewandelter\_text.append(buchstabe\_zu\_ascii(buchstabe))}
\BuiltInTok{print}\NormalTok{(umgewandelter\_text)}


\ControlFlowTok{for}\NormalTok{ zahl }\KeywordTok{in}\NormalTok{ umgewandelter\_text:    }
\NormalTok{    verschluesselt }\OperatorTok{=}\NormalTok{ (zahl }\OperatorTok{{-}} \DecValTok{97} \OperatorTok{+}\NormalTok{ versatz) }\OperatorTok{\%} \DecValTok{26} \OperatorTok{+} \DecValTok{97}
\NormalTok{    verschluesselte\_zahl.append(verschluesselt)}
\BuiltInTok{print}\NormalTok{(verschluesselte\_zahl)}


\ControlFlowTok{for}\NormalTok{ zahl }\KeywordTok{in}\NormalTok{ verschluesselte\_zahl:    }
\NormalTok{    verschluesselter\_text.append(ascii\_zu\_buchstabe(zahl))}
\BuiltInTok{print}\NormalTok{(verschluesselter\_text)}
\end{Highlighting}
\end{Shaded}

\begin{verbatim}
[97, 98, 114, 97, 107, 97, 100, 97, 98, 114, 97]
[100, 101, 117, 100, 110, 100, 103, 100, 101, 117, 100]
['d', 'e', 'u', 'd', 'n', 'd', 'g', 'd', 'e', 'u', 'd']
\end{verbatim}

\end{tcolorbox}

\end{tcolorbox}

\begin{tcolorbox}[enhanced jigsaw, left=2mm, leftrule=.75mm, bottomrule=.15mm, title=\textcolor{quarto-callout-tip-color}{\faLightbulb}\hspace{0.5em}{3. Ersetzen Sie die Schleife, indem Sie die Rechenoperation mit einem
NumPy-Array durchführen}, colback=white, arc=.35mm, breakable, titlerule=0mm, bottomtitle=1mm, colbacktitle=quarto-callout-tip-color!10!white, toprule=.15mm, opacityback=0, coltitle=black, rightrule=.15mm, opacitybacktitle=0.6, toptitle=1mm, colframe=quarto-callout-tip-color-frame]

\begin{tcolorbox}[enhanced jigsaw, left=2mm, leftrule=.75mm, bottomrule=.15mm, title={Lösung}, colback=white, arc=.35mm, breakable, titlerule=0mm, bottomtitle=1mm, colbacktitle=quarto-callout-caution-color!10!white, toprule=.15mm, opacityback=0, coltitle=black, rightrule=.15mm, opacitybacktitle=0.6, toptitle=1mm, colframe=quarto-callout-caution-color-frame]

\begin{Shaded}
\begin{Highlighting}[]
\ImportTok{import}\NormalTok{ numpy }\ImportTok{as}\NormalTok{ np}

\CommentTok{\# Funktion, die einen Buchstaben in ihren ASCII{-}Wert umwandelt}
\KeywordTok{def}\NormalTok{ buchstabe\_zu\_ascii(c):}
    \ControlFlowTok{return} \BuiltInTok{ord}\NormalTok{(c)}

\CommentTok{\# Funktion, die einen ASCII{-}Wert in den passenden Buchstaben umwandelt}
\KeywordTok{def}\NormalTok{ ascii\_zu\_buchstabe(a):}
    \ControlFlowTok{return} \BuiltInTok{chr}\NormalTok{(a)}

\NormalTok{klartext }\OperatorTok{=} \StringTok{"abrakadabra"}
\NormalTok{versatz }\OperatorTok{=} \DecValTok{3}

\NormalTok{umgewandelter\_text }\OperatorTok{=}\NormalTok{ []}
\NormalTok{verschluesselte\_zahl }\OperatorTok{=}\NormalTok{ []}
\NormalTok{verschluesselter\_text}\OperatorTok{=}\NormalTok{ []}



\ControlFlowTok{for}\NormalTok{ buchstabe }\KeywordTok{in}\NormalTok{ klartext:}
\NormalTok{    umgewandelter\_text.append(buchstabe\_zu\_ascii(buchstabe))}
\BuiltInTok{print}\NormalTok{(umgewandelter\_text)}

\NormalTok{umgewandelter\_text }\OperatorTok{=}\NormalTok{ np.array(umgewandelter\_text)}
\NormalTok{verschluesselte\_zahl }\OperatorTok{=}\NormalTok{ (umgewandelter\_text }\OperatorTok{{-}} \DecValTok{97} \OperatorTok{+}\NormalTok{ versatz) }\OperatorTok{\%} \DecValTok{26} \OperatorTok{+} \DecValTok{97}
\BuiltInTok{print}\NormalTok{(verschluesselte\_zahl)}

\ControlFlowTok{for}\NormalTok{ zahl }\KeywordTok{in}\NormalTok{ verschluesselte\_zahl:    }
\NormalTok{    verschluesselter\_text.append(ascii\_zu\_buchstabe(zahl))}
\BuiltInTok{print}\NormalTok{(verschluesselter\_text)}
\end{Highlighting}
\end{Shaded}

\begin{verbatim}
[97, 98, 114, 97, 107, 97, 100, 97, 98, 114, 97]
[100 101 117 100 110 100 103 100 101 117 100]
['d', 'e', 'u', 'd', 'n', 'd', 'g', 'd', 'e', 'u', 'd']
\end{verbatim}

\end{tcolorbox}

\end{tcolorbox}

\begin{tcolorbox}[enhanced jigsaw, left=2mm, leftrule=.75mm, bottomrule=.15mm, title=\textcolor{quarto-callout-tip-color}{\faLightbulb}\hspace{0.5em}{4. Schreiben sie den Code so um, dass der verschlüsselte Text
entschlüsselt wird.}, colback=white, arc=.35mm, breakable, titlerule=0mm, bottomtitle=1mm, colbacktitle=quarto-callout-tip-color!10!white, toprule=.15mm, opacityback=0, coltitle=black, rightrule=.15mm, opacitybacktitle=0.6, toptitle=1mm, colframe=quarto-callout-tip-color-frame]

\begin{tcolorbox}[enhanced jigsaw, left=2mm, leftrule=.75mm, bottomrule=.15mm, title={Lösung}, colback=white, arc=.35mm, breakable, titlerule=0mm, bottomtitle=1mm, colbacktitle=quarto-callout-caution-color!10!white, toprule=.15mm, opacityback=0, coltitle=black, rightrule=.15mm, opacitybacktitle=0.6, toptitle=1mm, colframe=quarto-callout-caution-color-frame]

\begin{Shaded}
\begin{Highlighting}[]
\ImportTok{import}\NormalTok{ numpy }\ImportTok{as}\NormalTok{ np}

\CommentTok{\# Funktion, die einen Buchstaben in ihren ASCII{-}Wert umwandelt}
\KeywordTok{def}\NormalTok{ buchstabe\_zu\_ascii(c):}
    \ControlFlowTok{return} \BuiltInTok{ord}\NormalTok{(c)}

\CommentTok{\# Funktion, die einen ASCII{-}Wert in den passenden Buchstaben umwandelt}
\KeywordTok{def}\NormalTok{ ascii\_zu\_buchstabe(a):}
    \ControlFlowTok{return} \BuiltInTok{chr}\NormalTok{(a)}


\NormalTok{versatz }\OperatorTok{=} \DecValTok{3}

\NormalTok{umgewandelter\_text }\OperatorTok{=}\NormalTok{ []}
\NormalTok{verschluesselte\_zahl }\OperatorTok{=}\NormalTok{ []}
\NormalTok{entschluesselter\_text}\OperatorTok{=}\NormalTok{ []}



\ControlFlowTok{for}\NormalTok{ buchstabe }\KeywordTok{in}\NormalTok{ verschluesselter\_text:}
\NormalTok{    umgewandelter\_text.append(buchstabe\_zu\_ascii(buchstabe))}
\BuiltInTok{print}\NormalTok{(umgewandelter\_text)}

\NormalTok{umgewandelter\_text }\OperatorTok{=}\NormalTok{ np.array(umgewandelter\_text)}
\NormalTok{verschluesselte\_zahl }\OperatorTok{=}\NormalTok{ (umgewandelter\_text }\OperatorTok{{-}} \DecValTok{97} \OperatorTok{{-}}\NormalTok{ versatz) }\OperatorTok{\%} \DecValTok{26} \OperatorTok{+} \DecValTok{97}
\BuiltInTok{print}\NormalTok{(verschluesselte\_zahl)}

\ControlFlowTok{for}\NormalTok{ zahl }\KeywordTok{in}\NormalTok{ verschluesselte\_zahl:    }
\NormalTok{    entschluesselter\_text.append(ascii\_zu\_buchstabe(zahl))}
\BuiltInTok{print}\NormalTok{(entschluesselter\_text)}
\end{Highlighting}
\end{Shaded}

\begin{verbatim}
[100, 101, 117, 100, 110, 100, 103, 100, 101, 117, 100]
[ 97  98 114  97 107  97 100  97  98 114  97]
['a', 'b', 'r', 'a', 'k', 'a', 'd', 'a', 'b', 'r', 'a']
\end{verbatim}

\end{tcolorbox}

\end{tcolorbox}

\part{Arbeiten mit Daten}

\chapter{Datenanalyse und
Modellierung}\label{datenanalyse-und-modellierung}

\section{Einleitung}\label{einleitung-1}

In dieser Einheit lernen Sie, wie man reale Messdaten -- etwa aus
Experimenten oder Ingenieurprojekten -- mit NumPy und Matplotlib
verarbeitet und analysiert. Der Fokus liegt dabei auf einem praxisnahen
Umgang mit Daten im CSV-Format.

\section{Lernziele dieses Kapitels}\label{lernziele-dieses-kapitels-5}

Sie lernen in dieser Einheit:

\begin{itemize}
\tightlist
\item
  wie Sie strukturierte CSV-Daten in NumPy-Arrays überführen,
\item
  wie Sie fehlende Werte erkennen und ersetzen,
\item
  wie Sie typische statistische Kennzahlen berechnen,
\item
  wie Sie Daten mit Matplotlib visualisieren,
\item
  wie Sie Daten interpolieren und Trends glätten,
\item
  und wie Sie reale Anwendungen -- z.\,B. aus dem Bauwesen --
  analysieren.
\end{itemize}

\section{CSV-Dateien: Ein typisches Format für
Messdaten}\label{csv-dateien-ein-typisches-format-fuxfcr-messdaten}

CSV-Dateien („Comma Separated Values``) sind weit verbreitet -- etwa
für:

\begin{itemize}
\tightlist
\item
  Temperaturverläufe,
\item
  Messreihen aus Experimenten,
\item
  Logdaten von Sensoren.
\end{itemize}

Zu Beginn wird ein Beispiel betrachtet: Temperatur, Luftfeuchtigkeit und
CO₂-Werte. Diese Datei enthält auch einige \textbf{fehlende Werte}, wie
sie in realen Daten oft vorkommen.

\begin{Shaded}
\begin{Highlighting}[]
\ImportTok{import}\NormalTok{ numpy }\ImportTok{as}\NormalTok{ np}
\NormalTok{data }\OperatorTok{=}\NormalTok{ np.genfromtxt(}\StringTok{"beispiel.csv"}\NormalTok{, delimiter}\OperatorTok{=}\StringTok{","}\NormalTok{, skip\_header}\OperatorTok{=}\DecValTok{1}\NormalTok{)}
\BuiltInTok{print}\NormalTok{(data)}
\end{Highlighting}
\end{Shaded}

\begin{verbatim}
[[ 21.1  45.  400. ]
 [ 22.5   nan 420. ]
 [  nan  50.  410. ]
 [ 20.   48.    nan]
 [ 23.3  47.  430. ]]
\end{verbatim}

\section{Fehlende Werte erkennen und
bereinigen}\label{fehlende-werte-erkennen-und-bereinigen}

Fehlende Werte werden beim Einlesen als \texttt{np.nan} (Not a Number)
codiert. Zunächst wird gezählt, wie viele Werte fehlen:

\begin{Shaded}
\begin{Highlighting}[]
\BuiltInTok{print}\NormalTok{(np.isnan(data).}\BuiltInTok{sum}\NormalTok{(axis}\OperatorTok{=}\DecValTok{0}\NormalTok{))}
\end{Highlighting}
\end{Shaded}

\begin{verbatim}
[1 1 1]
\end{verbatim}

Um die Analyse nicht zu verfälschen, werden sie ersetzt -- z.\,B. durch
den Mittelwert der Spalte:

\begin{Shaded}
\begin{Highlighting}[]
\ControlFlowTok{for}\NormalTok{ i }\KeywordTok{in} \BuiltInTok{range}\NormalTok{(data.shape[}\DecValTok{1}\NormalTok{]):}
\NormalTok{    mean }\OperatorTok{=}\NormalTok{ np.nanmean(data[:, i])}
\NormalTok{    data[:, i] }\OperatorTok{=}\NormalTok{ np.where(np.isnan(data[:, i]), mean, data[:, i])}
\end{Highlighting}
\end{Shaded}

\section{Statistische Kennzahlen
berechnen}\label{statistische-kennzahlen-berechnen}

Typische Kennwerte zur Beschreibung von Daten:

\begin{itemize}
\tightlist
\item
  \textbf{Mittelwert}: Durchschnitt
\item
  \textbf{Standardabweichung}: Streuung
\item
  \textbf{Minimum und Maximum}
\end{itemize}

\begin{Shaded}
\begin{Highlighting}[]
\BuiltInTok{print}\NormalTok{(}\StringTok{"Mittelwerte:"}\NormalTok{, np.mean(data, axis}\OperatorTok{=}\DecValTok{0}\NormalTok{))}
\BuiltInTok{print}\NormalTok{(}\StringTok{"Standardabweichung:"}\NormalTok{, np.std(data, axis}\OperatorTok{=}\DecValTok{0}\NormalTok{))}
\end{Highlighting}
\end{Shaded}

\begin{verbatim}
Mittelwerte: [ 21.725  47.5   415.   ]
Standardabweichung: [ 1.13556154  1.61245155 10.        ]
\end{verbatim}

\section{Daten visualisieren}\label{daten-visualisieren}

Mit Matplotlib lassen sich Daten übersichtlich darstellen. Es werden
z.\,B. Linien- und Histogrammplots genutzt.

\begin{Shaded}
\begin{Highlighting}[]
\ImportTok{import}\NormalTok{ matplotlib.pyplot }\ImportTok{as}\NormalTok{ plt}

\NormalTok{labels }\OperatorTok{=}\NormalTok{ [}\StringTok{"Temperatur (°C)"}\NormalTok{, }\StringTok{"Luftfeuchtigkeit (\%)"}\NormalTok{, }\StringTok{"CO₂ (ppm)"}\NormalTok{]}

\ControlFlowTok{for}\NormalTok{ i }\KeywordTok{in} \BuiltInTok{range}\NormalTok{(data.shape[}\DecValTok{1}\NormalTok{]):}
\NormalTok{    plt.plot(data[:, i], label}\OperatorTok{=}\NormalTok{labels[i])}
\NormalTok{plt.legend()}
\NormalTok{plt.title(}\StringTok{"Messwerte im Verlauf"}\NormalTok{)}
\NormalTok{plt.grid(}\VariableTok{True}\NormalTok{)}
\NormalTok{plt.show()}
\end{Highlighting}
\end{Shaded}

\includegraphics{books/dataanalysis/introduction_files/figure-pdf/cell-6-output-1.pdf}

\chapter{Interpolation -- Lücken
schließen}\label{interpolation-luxfccken-schlieuxdfen}

Was tun, wenn Werte fehlen? In vielen Datensätzen gibt es Lücken -- zum
Beispiel, weil Messungen nur an bestimmten Punkten vorgenommen wurden.
Interpolation ist eine Methode, mit der wir Zwischenwerte schätzen
können, also Werte innerhalb eines bekannten Wertebereichs.

Im Gegensatz dazu versucht Extrapolation, Werte außerhalb des bekannten
Bereichs vorherzusagen -- was in der Regel mit größerer Unsicherheit
verbunden ist.

Bei der Interpolation wird eine Modellfunktion gesucht, welche die
Messdaten exakt abbildet.

\begin{tcolorbox}[enhanced jigsaw, left=2mm, leftrule=.75mm, bottomrule=.15mm, title=\textcolor{quarto-callout-note-color}{\faInfo}\hspace{0.5em}{Theorie - Modellierung}, colback=white, arc=.35mm, breakable, titlerule=0mm, bottomtitle=1mm, colbacktitle=quarto-callout-note-color!10!white, toprule=.15mm, opacityback=0, coltitle=black, rightrule=.15mm, opacitybacktitle=0.6, toptitle=1mm, colframe=quarto-callout-note-color-frame]

Die Modellierung von Daten hat das Ziel eine Menge von Daten durch einen
funktionalen Zusammenhang abzubilden. Beispielhaft können Daten aus
Experimenten oder Simulationen stark verrauscht und so für eine
Weiterverarbeitung nicht geeignet sein. Eine mittelnde Funktion kann den
Datensatz stark vereinfachen. Oder es existieren nur wenige Datenpunkte
und die Zwischenstellen müssen durch eine Funktion bestimmt werden.

Generell kann die Modellierung von Daten auf folgendes Problem
verallgemeinert werden:

\begin{enumerate}
\def\labelenumi{\arabic{enumi}.}
\tightlist
\item
  Gegeben sind \(n\) Messpunktpaare \((x_i, y_i)\) mit
  \(x_i, y_i \in \mathbb{R}\)
\item
  Gesucht ist eine Modellfunktion \(y(x)\), welche die Messpunktpaare
  approximiert
\end{enumerate}

Ein möglicher Ansatz ist die Darstellung der Modellfunktion als Summe
von \(m\) Basisfunktionen \(\phi_i(x)\) mit den Koeffizienten
\(\beta_i\).

\[  y(x) = \sum_{i=1}^{m}\beta_i \cdot \phi_i(x) = \beta_1\cdot \phi_1(x) + \cdots + \beta_m\cdot \phi_m(x) \]

Die Koeffizienten \(\beta_i\) müssen dabei so bestimmt werden, dass
\(y(x)\) so gut wie möglich -- oder gar exakt -- die Messpunkte
approximieren.

Als Abstandmaß zwischen einer Modellfunktion und den Messpunkten kann
die \href{https://de.wikipedia.org/wiki/Folgenraum\#lp}{L2-Norm}
verwendet werden. Diese ist definiert als

\[  || y(x) - (x_i, y_i) ||_2 = \sum_{i=1}^n \left(y(x_i) - y_i\right)^2 \quad .\]

Eine solche Norm gibt ein Maß für die Qualität einer Approximation: je
kleiner der Abstand, desto besser die Qualität. Dies ermöglicht das
Finden optimaler Koeffizienten und wird beispielsweise in der Methode
der kleinsten Quadrate genutzt, in der ein Satz an Koeffizienten gesucht
wird, der die L2-Norm minimiert.

\end{tcolorbox}

\section{Übersicht}\label{uxfcbersicht-1}

In vielen praktischen Anwendungen werden Polynome als Basisfunktionen
der Modellfunktion angenommen. Vorteile von Polynomen:

\begin{itemize}
\tightlist
\item
  Polynome sind leicht zu differenzieren und integrieren
\item
  Annäherung von beliebigen Funktionen durch Polynome möglich, siehe
  \href{https://de.wikipedia.org/wiki/Taylorreihe}{Taylor-Entwicklung}
\item
  Auswertung ist sehr einfach und dadurch schnell, d.h. sie benötigt nur
  wenige schnelle arithmetische Operationen (Addition und
  Multiplikation)
\end{itemize}

Ein Beispiel für eine Basis aus Polynomen:

\[ \phi_1(x)=1,\quad \phi_2(x)=x,\quad \phi_3(x)=x^2,\quad \cdots,\quad \phi_m =x^{m-1} \]

\section{Polynome}\label{polynome}

Polynome \(P(x)\) sind Funktionen in Form einer Summe von
Potenzfunktionen mit natürlichzahligen Exponenten
\(( x^i, i \in \mathbb{N})\) mit den entsprechenden Koeffzienten
\(a_i\):

\[ P(x) = \sum_{i=0}^n a_i x^i = a_n x^n + a_{n-1} x^{n-1} + \cdots + a_2 x^2 + a_1 x + a_0, \quad i, n \in \mathbb{N}, a_i \in \mathbb{R} \]

\begin{itemize}
\tightlist
\item
  Als Grad eines Polynoms wird der Term mit dem höchsten Exponenten und
  nichtverschwindenden Koeffizienten (der sogenannte Leitkoeffizient)
  bezeichnet.
\item
  Ein Polynom mit Grad \(n\) hat \(n\), teilweise
  \href{https://de.wikipedia.org/wiki/Komplexe_Zahl}{komplexe},
  Nullstellen.
\end{itemize}

In Python, d.h. im numpy-Modul, werden Polynome durch ihre Koeffizienten
representiert. Im Allgemeinen wird ein Polynom mit dem Grad \(n\) durch
folgendes Array dargestellt

\begin{Shaded}
\begin{Highlighting}[]
\NormalTok{[an, ..., a2, a1, a0]}
\end{Highlighting}
\end{Shaded}

So z.B. für \(P(x) = x^3 + 5x^2 - 2x + 3\):

\begin{Shaded}
\begin{Highlighting}[]
\NormalTok{P }\OperatorTok{=}\NormalTok{ np.array([}\DecValTok{1}\NormalTok{, }\DecValTok{5}\NormalTok{, }\OperatorTok{{-}}\DecValTok{2}\NormalTok{, }\DecValTok{3}\NormalTok{])}
\BuiltInTok{print}\NormalTok{(P)}
\end{Highlighting}
\end{Shaded}

\begin{verbatim}
[ 1  5 -2  3]
\end{verbatim}

Die Auswertung des Polynoms an einem Punkt oder einem Array erfolgt mit
der \texttt{np.polyval}-Funktion.

\begin{Shaded}
\begin{Highlighting}[]
\NormalTok{x }\OperatorTok{=} \DecValTok{1}
\NormalTok{y }\OperatorTok{=}\NormalTok{ np.polyval(P, x)}
\BuiltInTok{print}\NormalTok{(}\SpecialStringTok{f"P(x=}\SpecialCharTok{\{}\NormalTok{x}\SpecialCharTok{\}}\SpecialStringTok{) = }\SpecialCharTok{\{}\NormalTok{y}\SpecialCharTok{\}}\SpecialStringTok{"}\NormalTok{)}
\end{Highlighting}
\end{Shaded}

\begin{verbatim}
P(x=1) = 7
\end{verbatim}

\begin{Shaded}
\begin{Highlighting}[]
\NormalTok{x }\OperatorTok{=}\NormalTok{ np.array([}\OperatorTok{{-}}\DecValTok{1}\NormalTok{, }\DecValTok{0}\NormalTok{, }\DecValTok{1}\NormalTok{])}
\NormalTok{y }\OperatorTok{=}\NormalTok{ np.polyval(P, x)}
\BuiltInTok{print}\NormalTok{(}\SpecialStringTok{f"P(x=}\SpecialCharTok{\{}\NormalTok{x}\SpecialCharTok{\}}\SpecialStringTok{) = }\SpecialCharTok{\{}\NormalTok{y}\SpecialCharTok{\}}\SpecialStringTok{"}\NormalTok{)}
\end{Highlighting}
\end{Shaded}

\begin{verbatim}
P(x=[-1  0  1]) = [9 3 7]
\end{verbatim}

Für die graphische Darstellung im Bereich \(x \in [-6, 2]\) können die
bekannten numpy und matplotlib Funktionen verwendet werden.

\begin{Shaded}
\begin{Highlighting}[]
\NormalTok{x }\OperatorTok{=}\NormalTok{ np.linspace(}\OperatorTok{{-}}\DecValTok{6}\NormalTok{, }\DecValTok{2}\NormalTok{, }\DecValTok{50}\NormalTok{)}
\NormalTok{y }\OperatorTok{=}\NormalTok{ np.polyval(P, x)}

\NormalTok{plt.plot(x, y)}
\NormalTok{plt.xlabel(}\StringTok{\textquotesingle{}x\textquotesingle{}}\NormalTok{)}
\NormalTok{plt.ylabel(}\StringTok{\textquotesingle{}y(x)\textquotesingle{}}\NormalTok{)}
\NormalTok{plt.grid()}
\end{Highlighting}
\end{Shaded}

\includegraphics{books/dataanalysis/interpolation_files/figure-pdf/cell-6-output-1.pdf}

Um die Nullstellen eines Polynoms zu finden, kann die numpy-Funktion
\texttt{np.roots} genutzt werden. Für das obige Polynom können folgende
Nullstellen bestimmt werden.

\begin{Shaded}
\begin{Highlighting}[]
\NormalTok{nstellen }\OperatorTok{=}\NormalTok{ np.roots(P)}

\CommentTok{\# direkte Ausgabe des Arrays}
\BuiltInTok{print}\NormalTok{(}\StringTok{"Nullstellen: "}\NormalTok{)}
\BuiltInTok{print}\NormalTok{(nstellen)}
\end{Highlighting}
\end{Shaded}

\begin{verbatim}
Nullstellen: 
[-5.46628038+0.j        0.23314019+0.703182j  0.23314019-0.703182j]
\end{verbatim}

\begin{Shaded}
\begin{Highlighting}[]
\BuiltInTok{print}\NormalTok{(}\StringTok{"Nullstellen: "}\NormalTok{)}
\CommentTok{\# schönere Ausgabe des Arrays}
\ControlFlowTok{for}\NormalTok{ i, z }\KeywordTok{in} \BuiltInTok{enumerate}\NormalTok{(nstellen):}
    \ControlFlowTok{if}\NormalTok{ z.imag }\OperatorTok{==} \DecValTok{0}\NormalTok{:}
        \BuiltInTok{print}\NormalTok{(}\SpecialStringTok{f"  x\_}\SpecialCharTok{\{}\NormalTok{i}\OperatorTok{+}\DecValTok{1}\SpecialCharTok{\}}\SpecialStringTok{ = }\SpecialCharTok{\{}\NormalTok{z}\SpecialCharTok{.}\NormalTok{real}\SpecialCharTok{:.2\}}\SpecialStringTok{"}\NormalTok{)}
    \ControlFlowTok{else}\NormalTok{:}
        \BuiltInTok{print}\NormalTok{(}\SpecialStringTok{f"  x\_}\SpecialCharTok{\{}\NormalTok{i}\OperatorTok{+}\DecValTok{1}\SpecialCharTok{\}}\SpecialStringTok{ = }\SpecialCharTok{\{}\NormalTok{z}\SpecialCharTok{.}\NormalTok{real}\SpecialCharTok{:.2\}}\SpecialStringTok{ }\SpecialCharTok{\{}\NormalTok{z}\SpecialCharTok{.}\NormalTok{imag}\SpecialCharTok{:+.2\}}\SpecialStringTok{i"}\NormalTok{)}
\end{Highlighting}
\end{Shaded}

\begin{verbatim}
Nullstellen: 
  x_1 = -5.5
  x_2 = 0.23 +0.7i
  x_3 = 0.23 -0.7i
\end{verbatim}

In diesem Beispiel sind zwei der Nullstellen komplex. Eine komplexe Zahl
\(z\) wird in Python als Summe des Realteils (\$
Re\() und Imaginärteils (\) Im\$). Letzterer wird durch ein
nachfolgendes \texttt{j}, die imaginäre Einheit, gekennzeichnet.

\[ z = Re(z) + Im(z)j\]

Die Nullstellen können auch zur alternativen Darstellung des Polynoms
verwendet werden. Sind \(x_i\) die \(n\) Nullstellen, so ist das Polynom
\(n\)-ten Grades durch folgendes Produkt beschrieben:

\[ P(x) = \prod_{i=1}^n \left(x - x_i\right) = (x - x_1)\cdot (x - x_2) \cdot \cdots \cdot (x - x_n) \]

Seien beispielsweise 1 und 2 die Nullstellen eines Polynoms, so lautet
dieses:

\[  P(x) = (x - 1)(x - 2) = x^2 - 3x +2 \]

Die numpy-Funktion \texttt{np.poly} kann aus den Nullstellen die
Polynomkoeffizienten bestimmen. Anhand des obigen Beispiels lautet der
Funktionsaufruf:

\begin{Shaded}
\begin{Highlighting}[]
\NormalTok{nstellen }\OperatorTok{=}\NormalTok{ [}\DecValTok{1}\NormalTok{, }\DecValTok{2}\NormalTok{]}
\NormalTok{koeffizienten }\OperatorTok{=}\NormalTok{ np.poly(nstellen)}

\BuiltInTok{print}\NormalTok{(}\StringTok{"Nullstellen:"}\NormalTok{, nstellen)}
\BuiltInTok{print}\NormalTok{(}\StringTok{"Koeffizienten:"}\NormalTok{, koeffizienten)}
\end{Highlighting}
\end{Shaded}

\begin{verbatim}
Nullstellen: [1, 2]
Koeffizienten: [ 1. -3.  2.]
\end{verbatim}

Das Modul numpy stellt viele praktische Funktionen zum Umgang mit
Polynomen zur Verfügung. So existieren Funktionen um Polynome
auszuwerten, die Nullstellen zu finden, zu addieren, zu multiplizieren,
abzuleiten oder zu integrieren. Eine Übersicht ist in der
\href{https://numpy.org/doc/stable/reference/routines.polynomials.poly1d.html}{numpy-Dokumentation}
gegeben.

\section{Interpolation}\label{interpolation}

Interpolation ist eine Methode, um Datenpunkte zwischen gegebenen
Messpunkten zu konstruieren. Dazu wird eine Funktion gesucht, die alle
Messpunkte exakt abbildet, was gleichbedeutend damit ist, dass die
L2-Norm zwischen Funktion und Punkten Null ist.

Zwei Punkte können z.B. mit einer Geraden interpoliert werden. D.h. für
zwei Messpunktpaare \((x_1, y_1)\) und \((x_2, y_2)\) mit
\(x_1\neq x_2\) existiert ein Koeffizientensatz, sodass die L2-Norm
zwischen den Messpunkten und der Modellfunktion

\[y(x) = \beta_1 x + \beta_0\]

verschwindet.

\begin{Shaded}
\begin{Highlighting}[]
\CommentTok{\# Beispieldaten aus y(x) = {-}x + 2}

\NormalTok{N }\OperatorTok{=} \DecValTok{50}
\NormalTok{dx }\OperatorTok{=} \FloatTok{0.25}

\KeywordTok{def}\NormalTok{ fnk(x):}
    \ControlFlowTok{return} \OperatorTok{{-}}\NormalTok{x }\OperatorTok{+} \DecValTok{2}

\NormalTok{x }\OperatorTok{=}\NormalTok{ np.array([}\DecValTok{1}\NormalTok{, }\DecValTok{2}\NormalTok{])}
\NormalTok{y }\OperatorTok{=}\NormalTok{ fnk(x)}

\NormalTok{plt.scatter(x, y, color}\OperatorTok{=}\StringTok{\textquotesingle{}C1\textquotesingle{}}\NormalTok{, label}\OperatorTok{=}\StringTok{"Messpunkte"}\NormalTok{, zorder}\OperatorTok{=}\DecValTok{3}\NormalTok{)}

\NormalTok{x\_modell }\OperatorTok{=}\NormalTok{ np.linspace(np.}\BuiltInTok{min}\NormalTok{(x), np.}\BuiltInTok{max}\NormalTok{(x), N)}
\NormalTok{plt.plot(x\_modell, fnk(x\_modell), color}\OperatorTok{=}\StringTok{\textquotesingle{}C0\textquotesingle{}}\NormalTok{, label}\OperatorTok{=}\StringTok{"Modellfunktion"}\NormalTok{)}

\NormalTok{x\_linie }\OperatorTok{=}\NormalTok{ np.linspace(np.}\BuiltInTok{min}\NormalTok{(x)}\OperatorTok{{-}}\NormalTok{dx, np.}\BuiltInTok{max}\NormalTok{(x)}\OperatorTok{+}\NormalTok{dx, N)}
\NormalTok{plt.plot(x\_linie, fnk(x\_linie), }\StringTok{\textquotesingle{}{-}{-}\textquotesingle{}}\NormalTok{, alpha}\OperatorTok{=}\FloatTok{0.3}\NormalTok{, color}\OperatorTok{=}\StringTok{\textquotesingle{}C0\textquotesingle{}}\NormalTok{)}

\NormalTok{plt.xlabel(}\StringTok{"x"}\NormalTok{)}
\NormalTok{plt.ylabel(}\StringTok{"y"}\NormalTok{)}
\NormalTok{plt.legend()}
\NormalTok{plt.grid()}
\end{Highlighting}
\end{Shaded}

\includegraphics{books/dataanalysis/interpolation_files/figure-pdf/cell-10-output-1.pdf}

Für drei Messpunkte muss ein Polynom zweiten Grades verwendet werden, um
die Punkte exakt zu erfassen.

\[y(x) = \beta_2 x^2 + \beta_1 x + \beta_0\]

\begin{Shaded}
\begin{Highlighting}[]
\CommentTok{\# Beispieldaten aus y(x) = 3x\^{}2 {-}4x {-} 1}

\NormalTok{N }\OperatorTok{=} \DecValTok{50}
\NormalTok{dx }\OperatorTok{=} \FloatTok{0.25}

\KeywordTok{def}\NormalTok{ fnk(x):}
    \ControlFlowTok{return} \DecValTok{3}\OperatorTok{*}\NormalTok{x}\OperatorTok{**}\DecValTok{2}\OperatorTok{{-}}\DecValTok{4}\OperatorTok{*}\NormalTok{x }\OperatorTok{{-}} \DecValTok{1}

\NormalTok{x }\OperatorTok{=}\NormalTok{ np.array([}\OperatorTok{{-}}\DecValTok{1}\NormalTok{, }\DecValTok{2}\NormalTok{, }\DecValTok{3}\NormalTok{])}
\NormalTok{y }\OperatorTok{=}\NormalTok{ fnk(x)}

\NormalTok{plt.scatter(x, y, color}\OperatorTok{=}\StringTok{\textquotesingle{}C1\textquotesingle{}}\NormalTok{, label}\OperatorTok{=}\StringTok{"Messpunkte"}\NormalTok{, zorder}\OperatorTok{=}\DecValTok{3}\NormalTok{)}

\NormalTok{x\_modell }\OperatorTok{=}\NormalTok{ np.linspace(np.}\BuiltInTok{min}\NormalTok{(x), np.}\BuiltInTok{max}\NormalTok{(x), N)}
\NormalTok{plt.plot(x\_modell, fnk(x\_modell), color}\OperatorTok{=}\StringTok{\textquotesingle{}C0\textquotesingle{}}\NormalTok{, label}\OperatorTok{=}\StringTok{"Modellfunktion"}\NormalTok{)}

\NormalTok{x\_linie }\OperatorTok{=}\NormalTok{ np.linspace(np.}\BuiltInTok{min}\NormalTok{(x)}\OperatorTok{{-}}\NormalTok{dx, np.}\BuiltInTok{max}\NormalTok{(x)}\OperatorTok{+}\NormalTok{dx, N)}
\NormalTok{plt.plot(x\_linie, fnk(x\_linie), }\StringTok{\textquotesingle{}{-}{-}\textquotesingle{}}\NormalTok{, alpha}\OperatorTok{=}\FloatTok{0.3}\NormalTok{, color}\OperatorTok{=}\StringTok{\textquotesingle{}C0\textquotesingle{}}\NormalTok{)}

\NormalTok{plt.xlabel(}\StringTok{"x"}\NormalTok{)}
\NormalTok{plt.ylabel(}\StringTok{"y"}\NormalTok{)}
\NormalTok{plt.legend()}
\NormalTok{plt.grid()}
\end{Highlighting}
\end{Shaded}

\includegraphics{books/dataanalysis/interpolation_files/figure-pdf/cell-11-output-1.pdf}

Dies kann verallgemeinert werden: \(n\) Messpunkte können exakt mit
einem Polynom (\$ n-1\$)-ten Grades abgebildet werden. Die Suche nach
den passenden Koeffizienten ist das Lagrangesche Interpolationsproblem.
Für das gesuchte Polynom \(P(x)\) gilt:

\[  P(x_i) = y_i \quad i \in 1, \dots, n \]

Die Existenz und Eindeutigkeit eines solchen Polynoms kann gezeigt
werden. Das gesuchte Polynom lautet:

\[  P(x) = \sum_{i=1}^n y_i I_i(x) \]
\[  \text{mit}\quad I_i(x) = \prod_{j=1, i\neq j}^n \frac{x-x_j}{x_i - x_j} \]

Alternativ kann auch ein Gleichungssystem, welches durch die
Polynomialbasis \(\phi_i(x)\) gegeben ist, gelöst werden. Für die \(n\)
Punktepaare gilt jeweils:

\[  y(x_i) = \sum_{i=1}^{m}\beta_i \cdot \phi_i(x_i) = \beta_1\cdot \phi_1(x_i) + \cdots + \beta_m\cdot \phi_m(x_i) = y_i \]

Das allgemeine Geleichungssystem lautet

\[  \left( \begin{matrix} \phi_1(x_1) & \phi_2(x_1) & \cdots & \phi_m(x_1) \\ 
             \phi_1(x_2) & \phi_2(x_2) & \cdots & \phi_m(x_2) \\ 
             \vdots &\vdots & \ddots & \vdots \\ 
             \phi_1(x_n) & \phi_2(x_n) & \cdots & \phi_m(x_n) \\ 
             \end{matrix}\right). 
\left(  \begin{matrix} \beta_1 \\  
      \beta_2 \\   \vdots \\   \beta_m \\ 
      \end{matrix} \right) 
= \left(  \begin{matrix} y_1 \\  y_2 \\   \vdots \\   y_n \\ \end{matrix} \right) \]

und mit der Polynomialbasis

\[ \underbrace{\left( \begin{matrix} 1 & x_1 & \cdots & x_1^{n-1} \\ 1 & x_2 & \cdots & x_2^{n-1} \\ \vdots &\vdots & \ddots & \vdots \\ 1 & x_n & \cdots & x_n^{n-1} \\ \end{matrix}\right)}_{V}. \left(  \begin{matrix} \beta_1 \\  \beta_2 \\   \vdots \\   \beta_m \\ \end{matrix} \right) = \left(  \begin{matrix} y_1 \\  y_2 \\   \vdots \\   y_n \\ \end{matrix} \right)\]

Die Matrix \(V\) heisst
\href{https://de.wikipedia.org/wiki/Vandermonde-Matrix}{Vandermonde-Matrix}
und kann exakt gelöst werden, für \(m=n\) und wenn für alle
\(i, j, i\neq j\) gilt \(x_i \neq x_j\).

In Python kann das Interpolationsproblem mit der
\href{https://numpy.org/doc/stable/reference/generated/numpy.polyfit.html}{Funktion
\texttt{np.polyfit}} gelöst werden. Das folgende Beispiel demonstriert
deren Anwendung.

Die Messtellen folgen in dem Beispiel der Funktion \(f(x)\), welche nur
zur Generierung der Datenpunkte verwendet wird.

\[ f(x) = \frac{1}{2} + \frac{1}{1+x^2}\]

Zunächst werden die Messpunkte generiert.

\begin{Shaded}
\begin{Highlighting}[]
\KeywordTok{def}\NormalTok{ fnk(x):}
    \ControlFlowTok{return} \FloatTok{0.5} \OperatorTok{+} \DecValTok{1}\OperatorTok{/}\NormalTok{(}\DecValTok{1}\OperatorTok{+}\NormalTok{x}\OperatorTok{**}\DecValTok{2}\NormalTok{)}
\end{Highlighting}
\end{Shaded}

\begin{Shaded}
\begin{Highlighting}[]
\NormalTok{xmin }\OperatorTok{=} \OperatorTok{{-}}\DecValTok{5}
\NormalTok{xmax }\OperatorTok{=}  \DecValTok{5}
\NormalTok{x }\OperatorTok{=}\NormalTok{ np.linspace(xmin, xmax, }\DecValTok{100}\NormalTok{)}
\NormalTok{y }\OperatorTok{=}\NormalTok{ fnk(x)}
\end{Highlighting}
\end{Shaded}

\begin{Shaded}
\begin{Highlighting}[]
\NormalTok{n }\OperatorTok{=} \DecValTok{5}
\NormalTok{xi }\OperatorTok{=}\NormalTok{ np.linspace(xmin, xmax, n)}
\NormalTok{yi }\OperatorTok{=}\NormalTok{ fnk(xi)}
\end{Highlighting}
\end{Shaded}

Nun folgt die Interpolation für 5 und 15 Messpunkte.

\begin{Shaded}
\begin{Highlighting}[]
\NormalTok{P }\OperatorTok{=}\NormalTok{ np.polyfit(xi, yi, n}\OperatorTok{{-}}\DecValTok{1}\NormalTok{)}
\BuiltInTok{print}\NormalTok{(}\StringTok{"Interpolationskoeffizienten:"}\NormalTok{)}
\BuiltInTok{print}\NormalTok{(P)}
\end{Highlighting}
\end{Shaded}

\begin{verbatim}
Interpolationskoeffizienten:
[ 5.30503979e-03  4.23767299e-17 -1.71087533e-01  7.45353051e-16
  1.50000000e+00]
\end{verbatim}

\begin{Shaded}
\begin{Highlighting}[]
\NormalTok{plt.plot(x, y, color}\OperatorTok{=}\StringTok{\textquotesingle{}C0\textquotesingle{}}\NormalTok{, alpha}\OperatorTok{=}\FloatTok{0.5}\NormalTok{, label}\OperatorTok{=}\StringTok{\textquotesingle{}generierende Funktion\textquotesingle{}}\NormalTok{)}
\NormalTok{plt.plot(x, np.polyval(P, x), color}\OperatorTok{=}\StringTok{\textquotesingle{}C2\textquotesingle{}}\NormalTok{, label}\OperatorTok{=}\StringTok{\textquotesingle{}Interpolation\textquotesingle{}}\NormalTok{)}
\NormalTok{plt.scatter(xi, yi, color}\OperatorTok{=}\StringTok{\textquotesingle{}C1\textquotesingle{}}\NormalTok{, label}\OperatorTok{=}\StringTok{\textquotesingle{}Messpunkte\textquotesingle{}}\NormalTok{, zorder}\OperatorTok{=}\DecValTok{3}\NormalTok{)}
\NormalTok{plt.legend()}
\NormalTok{plt.grid()}
\end{Highlighting}
\end{Shaded}

\includegraphics{books/dataanalysis/interpolation_files/figure-pdf/cell-16-output-1.pdf}

\begin{Shaded}
\begin{Highlighting}[]
\NormalTok{n }\OperatorTok{=} \DecValTok{15}
\NormalTok{xi }\OperatorTok{=}\NormalTok{ np.linspace(xmin, xmax, n)}
\NormalTok{yi }\OperatorTok{=}\NormalTok{ fnk(xi)}

\NormalTok{P }\OperatorTok{=}\NormalTok{ np.polyfit(xi, yi, n}\OperatorTok{{-}}\DecValTok{1}\NormalTok{)}

\NormalTok{plt.plot(x, y, color}\OperatorTok{=}\StringTok{\textquotesingle{}C0\textquotesingle{}}\NormalTok{, alpha}\OperatorTok{=}\FloatTok{0.5}\NormalTok{, label}\OperatorTok{=}\StringTok{\textquotesingle{}generierende Funktion\textquotesingle{}}\NormalTok{)}
\NormalTok{plt.plot(x, np.polyval(P, x), color}\OperatorTok{=}\StringTok{\textquotesingle{}C2\textquotesingle{}}\NormalTok{, label}\OperatorTok{=}\StringTok{\textquotesingle{}Interpolation\textquotesingle{}}\NormalTok{)}
\NormalTok{plt.scatter(xi, yi, color}\OperatorTok{=}\StringTok{\textquotesingle{}C1\textquotesingle{}}\NormalTok{, label}\OperatorTok{=}\StringTok{\textquotesingle{}Messpunkte\textquotesingle{}}\NormalTok{, zorder}\OperatorTok{=}\DecValTok{3}\NormalTok{)}
\NormalTok{plt.legend()}
\NormalTok{plt.grid()}
\end{Highlighting}
\end{Shaded}

\includegraphics{books/dataanalysis/interpolation_files/figure-pdf/cell-17-output-1.pdf}

Die Interpolation erfüllt immer die geforderte Bedingung
\(y(x_i) = y_i\). Jedoch führen Polynome mit einem hohen Grad oft zu
nicht sinnvollen Ergebnissen. Es entstehen starke Überschwinger, welche
mit zunehmendem Grad immer stärker werden. Eine alternative
Interpolationsmethode stellen Splines dar, welche mehrere, niedrige
Polynome zur Interpolation vieler Punkte verwenden.

\chapter{Fitting}\label{fitting}

Beim Fitting wird eine Modellfunktion gesucht, weche die Messdaten nicht
unbedingt exakt abbildet. Wird ein Polynom verwendet, so hat es eine
Grad, welcher deutlich kleiner ist, als die Anzahl der Messpunkte.
Lineare Regression ist ein Beispiel für ein Fitting durch ein Polynom
mit dem Grad Eins.

Zum Fitten durch ein Polynom kann die Funktion \texttt{np.polyfit}
verwendet werden, genauso wie bei der Polynominterpolation. Diesmal
jedoch mit einem kleineren Polynomgrad.

Im folgenden Beispiel werden zunächst Modelldaten generiert und dann mit
entsprechenden Polynomen gefittet.

\begin{Shaded}
\begin{Highlighting}[]
\NormalTok{xmin }\OperatorTok{=} \DecValTok{0}
\NormalTok{xmax }\OperatorTok{=} \DecValTok{5}
\NormalTok{x }\OperatorTok{=}\NormalTok{ np.linspace(xmin, xmax, }\DecValTok{100}\NormalTok{)}

\NormalTok{ni }\OperatorTok{=} \DecValTok{25}

\CommentTok{\# x{-}Werte mit leichtem Rauschen}
\NormalTok{xi }\OperatorTok{=}\NormalTok{ np.linspace(xmin, xmax, ni) }\OperatorTok{+} \FloatTok{0.2}\OperatorTok{*}\NormalTok{(}\DecValTok{2} \OperatorTok{*}\NormalTok{ np.random.random(ni) }\OperatorTok{{-}}\DecValTok{1}\NormalTok{)}

\CommentTok{\# y(x) = 2x+0.5 mit leichtem Rauschen}
\NormalTok{yi }\OperatorTok{=} \DecValTok{2}\OperatorTok{*}\NormalTok{xi }\OperatorTok{+} \FloatTok{0.5} \OperatorTok{+} \DecValTok{2}\OperatorTok{*}\NormalTok{(}\DecValTok{2} \OperatorTok{*}\NormalTok{ np.random.random(ni) }\OperatorTok{{-}}\DecValTok{1}\NormalTok{)}

\NormalTok{plt.scatter(xi, yi, color}\OperatorTok{=}\StringTok{\textquotesingle{}C1\textquotesingle{}}\NormalTok{)}
\NormalTok{plt.grid()}
\end{Highlighting}
\end{Shaded}

\includegraphics{books/dataanalysis/fitting_files/figure-pdf/cell-3-output-1.pdf}

\begin{Shaded}
\begin{Highlighting}[]
\NormalTok{P1 }\OperatorTok{=}\NormalTok{ np.polyfit(xi, yi, }\DecValTok{1}\NormalTok{)}

\NormalTok{plt.scatter(xi, yi, color}\OperatorTok{=}\StringTok{\textquotesingle{}C1\textquotesingle{}}\NormalTok{, zorder}\OperatorTok{=}\DecValTok{3}\NormalTok{, label}\OperatorTok{=}\StringTok{\textquotesingle{}Messpunkte\textquotesingle{}}\NormalTok{)}
\NormalTok{plt.plot(x, np.polyval(P1, x), color}\OperatorTok{=}\StringTok{\textquotesingle{}C0\textquotesingle{}}\NormalTok{, label}\OperatorTok{=}\StringTok{"Modellfunktion"}\NormalTok{)}
\NormalTok{plt.grid()}
\NormalTok{plt.legend()}
\end{Highlighting}
\end{Shaded}

\includegraphics{books/dataanalysis/fitting_files/figure-pdf/cell-4-output-1.pdf}

\begin{Shaded}
\begin{Highlighting}[]
\CommentTok{\# x{-}Werte mit leichtem Rauschen}
\NormalTok{xi }\OperatorTok{=}\NormalTok{ np.linspace(xmin, xmax, ni) }\OperatorTok{+} \FloatTok{0.2}\OperatorTok{*}\NormalTok{(}\DecValTok{2} \OperatorTok{*}\NormalTok{ np.random.random(ni) }\OperatorTok{{-}}\DecValTok{1}\NormalTok{)}

\CommentTok{\# y(x) = 2x+0.5 mit leichtem Rauschen}
\NormalTok{yi }\OperatorTok{=}\NormalTok{ (xi }\OperatorTok{{-}} \DecValTok{2}\NormalTok{)}\OperatorTok{**}\DecValTok{2} \OperatorTok{{-}}\DecValTok{2}\OperatorTok{*}\NormalTok{xi }\OperatorTok{+} \FloatTok{2.5} \OperatorTok{+} \DecValTok{2}\OperatorTok{*}\NormalTok{(}\DecValTok{2} \OperatorTok{*}\NormalTok{ np.random.random(ni) }\OperatorTok{{-}}\DecValTok{1}\NormalTok{)}
\end{Highlighting}
\end{Shaded}

\begin{Shaded}
\begin{Highlighting}[]
\NormalTok{P1 }\OperatorTok{=}\NormalTok{ np.polyfit(xi, yi, }\DecValTok{1}\NormalTok{)}
\NormalTok{P2 }\OperatorTok{=}\NormalTok{ np.polyfit(xi, yi, }\DecValTok{2}\NormalTok{)}

\NormalTok{plt.scatter(xi, yi, color}\OperatorTok{=}\StringTok{\textquotesingle{}C1\textquotesingle{}}\NormalTok{, zorder}\OperatorTok{=}\DecValTok{3}\NormalTok{, label}\OperatorTok{=}\StringTok{\textquotesingle{}Messpunkte\textquotesingle{}}\NormalTok{)}
\NormalTok{plt.plot(x, np.polyval(P1, x), color}\OperatorTok{=}\StringTok{\textquotesingle{}C0\textquotesingle{}}\NormalTok{, label}\OperatorTok{=}\StringTok{"Modellfunktion, Grad 1"}\NormalTok{)}
\NormalTok{plt.plot(x, np.polyval(P2, x), color}\OperatorTok{=}\StringTok{\textquotesingle{}C2\textquotesingle{}}\NormalTok{, label}\OperatorTok{=}\StringTok{"Modellfunktion, Grad 2"}\NormalTok{)}
\NormalTok{plt.grid()}
\NormalTok{plt.legend()}
\NormalTok{plt.show()}
\end{Highlighting}
\end{Shaded}

\includegraphics{books/dataanalysis/fitting_files/figure-pdf/cell-6-output-1.pdf}

\chapter{Splines}\label{splines}

Polynominterpolation versucht eine globale Modellfunktion zu finden.
Jedoch eignen sich Polynome mit hohen Graden im Allgemeinen nicht für
eine Interpolation vieler Punkte. Einen anderen Ansatz verfolgen
Splines. Diese sind Polynomzüge, welche die einzelnen Messpunkte
verbinden und deren Grad klein -- typischerweise zwischen eins und drei
-- ist.

\section{Definition}\label{definition}

Für \(n+1\) Messpunkte \((x_i, y_i)\) kann eine Splinefunktion \(s_k\),
hier ein Polynomspline, wie folgt definiert werden.

\begin{itemize}
\tightlist
\item
  Vorausgesetzt ist, dass die Messpunkte sortiert sind, d.h.
  \(x_0 < x_1 < \cdots < x_n\)
\item
  für jedes \(i = 0\dots n − 1\) ist \(s_k\) ein Polynom vom Grad \(k\)
  auf dem Intervall \(\left[x_i , x_{i+1}\right]\)
\item
  \(s_k\) ist auf \(\left[x_0 , x_n \right]\) \((k − 1)\)-mal stetig
  differenzierbar
\end{itemize}

Beispiele: * \(k = 1\): Polygonzug * \(k = 3\): kubische Polynomsplines
(B-Splines)

\section{Kubische Splines}\label{kubische-splines}

Die in der Praxis häufig eingesetzten kubischen Polynomsplines \(s_3\)
(\$ k= 3\$) haben folgende Eigenschaften: *
\(s_3|\left[x_i,x_{i+1}\right] = \beta_0 +\beta_1 x + \beta_2 x^2 + \beta_3 x^3\)
* \(s_3\) ist zweimal stetig differenzierbar auf
\(\left[x_0,x_n\right]\), also insbesondere an den Stützpunkten \(x_i\)
der Messpunkte

Die Koeffizienten \(\beta_i\) werden wie folgt bestimmt * aus den
\(n + 1\) Messpunkten ergeben sich \(n\) Intervalle, d.h. mit jeweils
vier Koeffizienten sind es insgesamt \(4n\) Koeffizienten * Exakte
Darstellung der Messpunkte (\$ n + 1\$ Gleichungen), d.h.:
\(s_3(x_i) = y_i\) * Glattheitsbedingungen an den inneren Messpunkten
(\$ i=1\dots n-1\(), mit jeweils (\) n − 1\$ Gleichungen):
\[ s_3'(x_i)_- =s_3'(x_i)_+\] \[ s_3''(x_i)_- =s_3''(x_i)_+\]
\[ s_3'''(x_i)_- =s_3'''(x_i)_+\]

\begin{itemize}
\tightlist
\item
  Damit sind es \(4n − 2\) Gleichungen für \(4n\) Koeffizienten
\end{itemize}

Um die beiden fehlenden Gleichungen zu finden bzw. zu bestimmen werden
Randbedingungen oder Abschlussbedingungen benötigt. Die gängigsten
Bedingungen sind: * natürliche Splines: die Krümung am Rand
verschwindet, d.h.: \[ s_3''(x_0) = s_3''(x_n) = 0 \] * periodische
Splines: die Steigung und Krümung ist an beiden Rändern gleich
\[ s_3'(x_0) = s_3'(x_n)\] \[ s_3''(x_0) = s_3''(x_n)\] * Hermite
Splines: die Steigungen am Rand werden explizit vorgegeben (hier durch
\(u\) und \(v\)) \[ s_3'(x_0) = u \] \[ s_3'(x_n) = v \]

\section{Anwendung}\label{anwendung}

Im Folgenden werden zwei Beispiele, \(s_1\) und \(s_3\), für die
Erstellung von Splines mit Python vorgestellt.

\begin{Shaded}
\begin{Highlighting}[]
\CommentTok{\# Erzeugung von Messpunkten}
\NormalTok{n }\OperatorTok{=} \DecValTok{7}
\NormalTok{xi }\OperatorTok{=}\NormalTok{ np.linspace(}\DecValTok{0}\NormalTok{, np.pi, n)}
\NormalTok{yi }\OperatorTok{=}\NormalTok{ np.sin(xi)}
\end{Highlighting}
\end{Shaded}

Für die \(s_1\) Splines, kann die Funktion \texttt{np.interp} verwendet
werden. Sie führt eine lineare Interpolation zwischen gegebenen
Wertepaaren durch.

\begin{Shaded}
\begin{Highlighting}[]
\CommentTok{\# Wertebereich für die Visualisierung der Interpolation}
\NormalTok{x }\OperatorTok{=}\NormalTok{ np.linspace(}\DecValTok{0}\NormalTok{, np.pi, n}\OperatorTok{*}\DecValTok{6}\NormalTok{)}
\NormalTok{y }\OperatorTok{=}\NormalTok{ np.sin(x)}
\end{Highlighting}
\end{Shaded}

\begin{Shaded}
\begin{Highlighting}[]
\CommentTok{\# Interpolation}
\NormalTok{y\_s1 }\OperatorTok{=}\NormalTok{ np.interp(x, xi, yi)}
\end{Highlighting}
\end{Shaded}

\begin{Shaded}
\begin{Highlighting}[]
\NormalTok{plt.plot(x,y, alpha}\OperatorTok{=}\FloatTok{0.3}\NormalTok{, color}\OperatorTok{=}\StringTok{\textquotesingle{}C2\textquotesingle{}}\NormalTok{, lw}\OperatorTok{=}\DecValTok{5}\NormalTok{, }
\NormalTok{         label}\OperatorTok{=}\StringTok{\textquotesingle{}Generierende Funktion\textquotesingle{}}\NormalTok{)}
\NormalTok{plt.plot(x, y\_s1, color}\OperatorTok{=}\StringTok{\textquotesingle{}C0\textquotesingle{}}\NormalTok{, label}\OperatorTok{=}\StringTok{\textquotesingle{}Interpolation\textquotesingle{}}\NormalTok{)}
\NormalTok{plt.scatter(x, y\_s1, s}\OperatorTok{=}\DecValTok{3}\NormalTok{, zorder}\OperatorTok{=}\DecValTok{3}\NormalTok{, color}\OperatorTok{=}\StringTok{\textquotesingle{}C0\textquotesingle{}}\NormalTok{)}
\NormalTok{plt.scatter(xi, yi, color}\OperatorTok{=}\StringTok{\textquotesingle{}C1\textquotesingle{}}\NormalTok{, label}\OperatorTok{=}\StringTok{\textquotesingle{}Messpunkte\textquotesingle{}}\NormalTok{)}

\NormalTok{plt.xlabel(}\StringTok{\textquotesingle{}x\textquotesingle{}}\NormalTok{)}
\NormalTok{plt.ylabel(}\StringTok{\textquotesingle{}y\textquotesingle{}}\NormalTok{)}
\NormalTok{plt.legend()}\OperatorTok{;}
\end{Highlighting}
\end{Shaded}

\includegraphics{books/dataanalysis/splines_files/figure-pdf/cell-6-output-1.pdf}

Die \(s_3\) Splines können mit Funktionen aus dem scipy-Modul berechnet
werden. Dazu werden zunächst die Koeffizienten bestimmt
(\texttt{scipy.interpolate.splrep}) und diese ermöglichen die gewünschte
Auswertung, welche mit der Funktion \texttt{scipy.interpolate.splev}
vorgenommen werden kann.

\begin{Shaded}
\begin{Highlighting}[]
\ImportTok{import}\NormalTok{ scipy.interpolate }\ImportTok{as}\NormalTok{ si}
\end{Highlighting}
\end{Shaded}

\begin{Shaded}
\begin{Highlighting}[]
\NormalTok{s3 }\OperatorTok{=}\NormalTok{ si.splrep(xi, yi)}
\NormalTok{y\_s3 }\OperatorTok{=}\NormalTok{ si.splev(x, s3)}
\end{Highlighting}
\end{Shaded}

\begin{Shaded}
\begin{Highlighting}[]
\NormalTok{plt.plot(x,y, alpha}\OperatorTok{=}\FloatTok{0.3}\NormalTok{, color}\OperatorTok{=}\StringTok{\textquotesingle{}C2\textquotesingle{}}\NormalTok{, lw}\OperatorTok{=}\DecValTok{5}\NormalTok{, }
\NormalTok{         label}\OperatorTok{=}\StringTok{\textquotesingle{}Generierende Funktion\textquotesingle{}}\NormalTok{)}
\NormalTok{plt.plot(x, y\_s3, color}\OperatorTok{=}\StringTok{\textquotesingle{}C0\textquotesingle{}}\NormalTok{, label}\OperatorTok{=}\StringTok{\textquotesingle{}Interpolation\textquotesingle{}}\NormalTok{)}
\NormalTok{plt.scatter(x, y\_s3, s}\OperatorTok{=}\DecValTok{3}\NormalTok{, zorder}\OperatorTok{=}\DecValTok{3}\NormalTok{, color}\OperatorTok{=}\StringTok{\textquotesingle{}C0\textquotesingle{}}\NormalTok{)}
\NormalTok{plt.scatter(xi, yi, color}\OperatorTok{=}\StringTok{\textquotesingle{}C1\textquotesingle{}}\NormalTok{, label}\OperatorTok{=}\StringTok{\textquotesingle{}Messpunkte\textquotesingle{}}\NormalTok{)}

\NormalTok{plt.xlabel(}\StringTok{\textquotesingle{}x\textquotesingle{}}\NormalTok{)}
\NormalTok{plt.ylabel(}\StringTok{\textquotesingle{}y\textquotesingle{}}\NormalTok{)}
\NormalTok{plt.legend()}\OperatorTok{;}
\end{Highlighting}
\end{Shaded}

\includegraphics{books/dataanalysis/splines_files/figure-pdf/cell-9-output-1.pdf}

\chapter{Trendglättung -- Rauschen
reduzieren}\label{trendgluxe4ttung-rauschen-reduzieren}

Verrauschte Daten? Ein \textbf{gleitender Mittelwert} glättet Kurven:

\begin{Shaded}
\begin{Highlighting}[]
\NormalTok{data }\OperatorTok{=}\NormalTok{ np.genfromtxt(}\StringTok{"trenddaten\_mit\_rauschen.csv"}\NormalTok{, delimiter}\OperatorTok{=}\StringTok{","}\NormalTok{, skip\_header}\OperatorTok{=}\DecValTok{1}\NormalTok{)}
\NormalTok{x }\OperatorTok{=}\NormalTok{ data[:, }\DecValTok{0}\NormalTok{]}
\NormalTok{y }\OperatorTok{=}\NormalTok{ data[:, }\DecValTok{1}\NormalTok{]}

\NormalTok{window }\OperatorTok{=} \DecValTok{5}
\NormalTok{weights }\OperatorTok{=}\NormalTok{ np.ones(window) }\OperatorTok{/}\NormalTok{ window}
\NormalTok{y\_smooth }\OperatorTok{=}\NormalTok{ np.convolve(y, weights, mode}\OperatorTok{=}\StringTok{\textquotesingle{}valid\textquotesingle{}}\NormalTok{)}

\NormalTok{plt.plot(x, y, label}\OperatorTok{=}\StringTok{"Original"}\NormalTok{, alpha}\OperatorTok{=}\FloatTok{0.5}\NormalTok{)}
\NormalTok{plt.plot(x[(window}\OperatorTok{{-}}\DecValTok{1}\NormalTok{)}\OperatorTok{//}\DecValTok{2}\NormalTok{:}\OperatorTok{{-}}\NormalTok{(window}\OperatorTok{//}\DecValTok{2}\NormalTok{)], y\_smooth, label}\OperatorTok{=}\StringTok{"Geglättet"}\NormalTok{, color}\OperatorTok{=}\StringTok{\textquotesingle{}red\textquotesingle{}}\NormalTok{)}
\NormalTok{plt.legend()}
\NormalTok{plt.grid(}\VariableTok{True}\NormalTok{)}
\NormalTok{plt.title(}\StringTok{"Trendglättung"}\NormalTok{)}
\NormalTok{plt.show()}
\end{Highlighting}
\end{Shaded}

\includegraphics{books/dataanalysis/glaetten_files/figure-pdf/cell-3-output-1.pdf}

\chapter{Anwendungsbeispiele}\label{anwendungsbeispiele}

\section{Anwendung: Ballonfahrt-Daten
analysieren}\label{anwendung-ballonfahrt-daten-analysieren}

\begin{Shaded}
\begin{Highlighting}[]
\NormalTok{ballon }\OperatorTok{=}\NormalTok{ np.genfromtxt(}\StringTok{"messdaten\_ballonfahrt.txt"}\NormalTok{, delimiter}\OperatorTok{=}\StringTok{","}\NormalTok{, skip\_header}\OperatorTok{=}\DecValTok{1}\NormalTok{)}
\NormalTok{zeit }\OperatorTok{=}\NormalTok{ ballon[:, }\DecValTok{0}\NormalTok{]}
\NormalTok{hoehe }\OperatorTok{=}\NormalTok{ ballon[:, }\DecValTok{1}\NormalTok{]}

\NormalTok{plt.plot(zeit, hoehe)}
\NormalTok{plt.title(}\StringTok{"Ballonfahrt – Höhe über Zeit"}\NormalTok{)}
\NormalTok{plt.xlabel(}\StringTok{"Zeit (s)"}\NormalTok{)}
\NormalTok{plt.ylabel(}\StringTok{"Höhe (m)"}\NormalTok{)}
\NormalTok{plt.grid(}\VariableTok{True}\NormalTok{)}
\NormalTok{plt.show()}

\NormalTok{geschwindigkeit }\OperatorTok{=}\NormalTok{ np.gradient(hoehe, zeit)}
\NormalTok{plt.plot(zeit, geschwindigkeit, color}\OperatorTok{=}\StringTok{"orange"}\NormalTok{)}
\NormalTok{plt.title(}\StringTok{"Geschwindigkeit"}\NormalTok{)}
\NormalTok{plt.grid(}\VariableTok{True}\NormalTok{)}
\NormalTok{plt.show()}
\end{Highlighting}
\end{Shaded}

\includegraphics{books/dataanalysis/anwendungsbeispiel_files/figure-pdf/cell-3-output-1.pdf}

\includegraphics{books/dataanalysis/anwendungsbeispiel_files/figure-pdf/cell-3-output-2.pdf}

\section{Anwendung: Balkenverformung im
Bauingenieurwesen}\label{anwendung-balkenverformung-im-bauingenieurwesen}

Ein Träger wird in der Mitte belastet. Die Durchbiegung wird an 50
Punkten gemessen:

\begin{Shaded}
\begin{Highlighting}[]
\NormalTok{balken }\OperatorTok{=}\NormalTok{ np.genfromtxt(}\StringTok{"balken\_durchbiegung.csv"}\NormalTok{, delimiter}\OperatorTok{=}\StringTok{","}\NormalTok{, skip\_header}\OperatorTok{=}\DecValTok{1}\NormalTok{)}
\NormalTok{x }\OperatorTok{=}\NormalTok{ balken[:, }\DecValTok{0}\NormalTok{]}
\NormalTok{y }\OperatorTok{=}\NormalTok{ balken[:, }\DecValTok{1}\NormalTok{]}

\NormalTok{window }\OperatorTok{=} \DecValTok{7}
\NormalTok{weights }\OperatorTok{=}\NormalTok{ np.ones(window) }\OperatorTok{/}\NormalTok{ window}
\NormalTok{y\_smooth }\OperatorTok{=}\NormalTok{ np.convolve(y, weights, mode}\OperatorTok{=}\StringTok{\textquotesingle{}valid\textquotesingle{}}\NormalTok{)}

\NormalTok{plt.plot(x, y, label}\OperatorTok{=}\StringTok{"Messung"}\NormalTok{, alpha}\OperatorTok{=}\FloatTok{0.5}\NormalTok{)}
\NormalTok{plt.plot(x[(window}\OperatorTok{{-}}\DecValTok{1}\NormalTok{)}\OperatorTok{//}\DecValTok{2}\NormalTok{:}\OperatorTok{{-}}\NormalTok{(window}\OperatorTok{//}\DecValTok{2}\NormalTok{)], y\_smooth, label}\OperatorTok{=}\StringTok{"Geglättet"}\NormalTok{, color}\OperatorTok{=}\StringTok{\textquotesingle{}red\textquotesingle{}}\NormalTok{)}
\NormalTok{plt.title(}\StringTok{"Durchbiegung eines Trägers"}\NormalTok{)}
\NormalTok{plt.xlabel(}\StringTok{"Position (m)"}\NormalTok{)}
\NormalTok{plt.ylabel(}\StringTok{"Durchbiegung (mm)"}\NormalTok{)}
\NormalTok{plt.legend()}
\NormalTok{plt.grid(}\VariableTok{True}\NormalTok{)}
\NormalTok{plt.show()}
\end{Highlighting}
\end{Shaded}

\includegraphics{books/dataanalysis/anwendungsbeispiel_files/figure-pdf/cell-4-output-1.pdf}

\section{Zusammenfassung}\label{zusammenfassung}

In dieser Einheit haben Sie gelernt:

\begin{itemize}
\tightlist
\item
  wie Daten eingelesen und bereinigt werden,
\item
  wie man sie analysiert und visualisiert,
\item
  wie Interpolation und Glättung funktionieren,
\item
  wie reale Datensätze aus Technik und Naturwissenschaft ausgewertet
  werden können.
\end{itemize}

Diese Fähigkeiten sind grundlegend für jede datengetriebene Analyse im
Ingenieurbereich.

\part{Algorithmen}

\chapter{Einführung}\label{einfuxfchrung-1}

\section{Definition}\label{definition-1}

Ein Algorithmus ist eine formale Vorschrift darüber, wie die Lösung
einer Fragestellung gefunden werden kann. Dabei handelt es sich meist um
eine Folge von einfachen Anweisungen, welche zur Lösung komplexer
Probleme führen kann.

\begin{figure}[H]

{\centering \includegraphics{index_files/mediabag/books/a-algorithmen/skript/00-bilder/algorithmus.pdf}

}

\caption{Algorithmus}

\end{figure}%

Algorithmen sollten so formuliert sein, dass sie nicht nur für einzelne
explizite Fragestellungen, sondern auch im Allgemeinen anwendbar sind.
Das wird am Beispiel des schriftliche Dividierens oder anhand eines
Kuchenrezepts deutlich. Beide bestehen aus einfachen Anweisungen und
lösen ein komplexeres Problem. Allerdings kann die Rechenvorschrift für
beliebige Divisionsaufgaben eingesetzt werden, während das Kuchenrezept
nur zur Herstellung eines speziellen Kuchens führt.

Die obigen Beispiele für einen Algorithmus sind zwei von vielen, welche
von Menschen eingesetzt werden (können):

\begin{itemize}
\tightlist
\item
  Schriftliches Rechnen
\item
  Lösen von linearen Gleichungssystemen
\item
  Bestimmung des Durchschnitts
\item
  Lösen eines Zauberwürfels
\end{itemize}

Viele Algorithmen aus unserem Alltag sind aus sehr elementaren
Anweisungen aufgebaut. Trotz der Einfachheit der Anweisungen, können sie
von Menschen nicht eingesetzt werden, da die erforderliche Anzahl von
Operationen sehr hoch sein kann. An dieser Stelle kommen Computer zum
Einsatz. Wie in diesem Kapitel gezeigt wird, können mit den
Grundrechenarten und Logischen Verknüpfungen komplexe Probleme gelöst
werden.

\section{Beispiele}\label{beispiele}

Beispiele für Algorithmen aus dem Alltag bzw. Ingenieurwesen, welche auf
Computer zurückgreifen:

\begin{itemize}
\tightlist
\item
  Numerische Lösung von Differentialgleichungen (z.B. Strukturmechanik,
  Wärmetransport)
\item
  Suchmaschinen im Internet
\item
  Vorschläge beim online Einkaufen oder Medienkonsum
\item
  Autonavigation
\end{itemize}

\chapter{Von der Idee zum Code}\label{von-der-idee-zum-code}

Ein einfacher Algorithmus zur Bestimmung des maximalen Werts einer
beliebig großen Menge von Zahlen ist wie folgt definiert:

\begin{enumerate}
\def\labelenumi{\arabic{enumi}.}
\tightlist
\item
  Eingabe: Menge \(\mathsf A\) von \(\mathsf n\) Zahlen, hier
  durchnumerierte Werteliste \(\mathsf A=A_0, \dots, A_{n-1}\).
\item
  Setzte Hilfswert (Variable) \(\mathsf m\) auf das erste Element der
  Liste, d.h. \(\mathsf m = A_0\).
\item
  Gehe alle Elemente von \(\mathsf A\) durch, wobei das aktuelle Element
  als \(\mathsf a\) bezeichnet wird:
\item
  Falls das aktuelle Element \(\mathsf a\) größer ist als \(\mathsf m\):
  * setzte \(\mathsf m = a\) *, dann mache weiter mit dem nächsten
  Element in Schritt 3
\item
  Falls nicht: mache weiter mit dem nächsten Element in Schritt 3
\item
  Nachdem alle Elemente aus \(\mathsf A\) in Schritt 3 durchlaufen
  wurden, enthält \(\mathsf m\) den maximalen Wert der Liste
  \(\mathsf A\).
\end{enumerate}

Dieser Algorithmus mag Ihnen auf den ersten Blick kompliziert. Gehen wir
nochmal einen Schritt zurück und schauen uns einen Algorithmus an den
wir bereits kennengelernt haben: Prüfen ob eine Zahl gerade ist.

\begin{enumerate}
\def\labelenumi{\arabic{enumi}.}
\tightlist
\item
  Eingabe: Eine Zahl X
\item
  Berechne \(X\text{ mod }2\)
\item
  Prüfe ob das Ergebnis gleich Null ist
\item
  Falls ja ist die Zahl gerade und wir geben den Text ``die Zahl ist
  gerade'' aus.
\item
  Falls nein ist die Zahl ungerade und wir geben den Text ``die Zahl ist
  ungerade'' aus.
\end{enumerate}

Diesen Auflistung können wir uns besser visualisieren. Dazu benutzen wir
sogenannte Flussdiagramme.

\begin{tcolorbox}[enhanced jigsaw, left=2mm, leftrule=.75mm, bottomrule=.15mm, title=\textcolor{quarto-callout-note-color}{\faInfo}\hspace{0.5em}{Flussdiagramme - Visuelle Darstellung von Abläufen}, colback=white, arc=.35mm, breakable, titlerule=0mm, bottomtitle=1mm, colbacktitle=quarto-callout-note-color!10!white, toprule=.15mm, opacityback=0, coltitle=black, rightrule=.15mm, opacitybacktitle=0.6, toptitle=1mm, colframe=quarto-callout-note-color-frame]

Ein \textbf{Flussdiagramm} (engl. \emph{flowchart}) ist eine grafische
Methode zur Darstellung von Algorithmen. Es zeigt den Ablauf eines
Programms oder Prozesses durch standardisierte Symbole und Pfeile. So
lassen sich komplexe Abläufe leicht nachvollziehen und logisch
überprüfen.

\textbf{Typische Symbole:}

\begin{itemize}
\tightlist
\item
  \textbf{Prozess (Anweisung)}: Rechteck -- z.\,B. „Berechne Fläche``
\item
  \textbf{Entscheidung}: Raute -- z.\,B. ``Ist x \textgreater{} 0?''
\item
  \textbf{Start/Ende}: Ellipse -- z.\,B. ``für alle a in A''
\item
  \textbf{Pfeile}: Zeigen den Ablauf von einem Schritt zum nächsten
\end{itemize}

\textbf{Beispiel:} Der Algorithmus zur Bestimmung, ob eine Zahl gerade
ist:

\begin{figure}[H]

{\centering \includegraphics{index_files/mediabag/books/a-algorithmen/skript/00-bilder/Zahl_gerade.pdf}

}

\caption{Flussdiagramm}

\end{figure}%

\end{tcolorbox}

Flussdiagramme sind nicht der einzige Weg, um komplexere Algorithmen
verständlicher aufzuschreiben. Eine weitere Möglichkeit bieter hier
sogenannter Pseudocode:

\begin{tcolorbox}[enhanced jigsaw, left=2mm, leftrule=.75mm, bottomrule=.15mm, title=\textcolor{quarto-callout-note-color}{\faInfo}\hspace{0.5em}{Pseudocode -- Vom Gedanken zum Programm}, colback=white, arc=.35mm, breakable, titlerule=0mm, bottomtitle=1mm, colbacktitle=quarto-callout-note-color!10!white, toprule=.15mm, opacityback=0, coltitle=black, rightrule=.15mm, opacitybacktitle=0.6, toptitle=1mm, colframe=quarto-callout-note-color-frame]

\textbf{Pseudocode} ist eine formalisierte Beschreibung eines
Algorithmus in einfacher, strukturierter Sprache -- eine Mischung aus
natürlicher Sprache und Programmierlogik. Er ist
\textbf{sprachunabhängig} und dient zur Planung, nicht zur direkten
Ausführung.

\textbf{Typische Merkmale:} - Klare
\textbf{Schritt-für-Schritt-Struktur} - Verwendung von
\textbf{Kontrollstrukturen} wie \texttt{wenn}, \texttt{solange},
\texttt{wiederhole} - Keine konkrete Syntax einer Programmiersprache

\textbf{Beispiel:}

\begin{Shaded}
\begin{Highlighting}[]
\NormalTok{BEGIN}
\NormalTok{  Lese Zahl x ein}
\NormalTok{  WENN x mod 2 = 0 DANN}
\NormalTok{    Gib "x ist gerade" aus}
\NormalTok{  SONST}
\NormalTok{    Gib "x ist ungerade" aus}
\NormalTok{ENDE}
\end{Highlighting}
\end{Shaded}

\end{tcolorbox}

Kommen wir nun zu dem ``komplexeren Algorithmus zurück, der das Maximum
in einer Liste von Zahlen finden soll:

\begin{enumerate}
\def\labelenumi{\arabic{enumi}.}
\tightlist
\item
  Eingabe: Menge \(\mathsf A\) von \(\mathsf n\) Zahlen, hier
  durchnumerierte Werteliste \(\mathsf A=A_0, \dots, A_{n-1}\).
\item
  Setzte Hilfswert (Variable) \(\mathsf m\) auf das erste Element der
  Liste, d.h. \(\mathsf m = A_0\).
\item
  Gehe alle Elemente von \(\mathsf A\) durch, wobei das aktuelle Element
  als \(\mathsf a\) bezeichnet wird:
\item
  Falls das aktuelle Element \(\mathsf a\) größer ist als \(\mathsf m\):
  * setzte \(\mathsf m = a\) *, dann mache weiter mit dem nächsten
  Element in Schritt 3
\item
  Falls nicht: mache weiter mit dem nächsten Element in Schritt 3
\item
  Nachdem alle Elemente aus \(\mathsf A\) in Schritt 3 durchlaufen
  wurden, enthält \(\mathsf m\) den maximalen Wert der Liste
  \(\mathsf A\).
\end{enumerate}

Testen Sie einmal selbst, ob Sie das passende Flussdiagramm erstellen
können. In der folgenden Box finden Sie die Musterlösung. Sie brauchen
hierfür auch die Verzweigung für Schleifen: \textbf{Verzweigung} (blau):
Abfrage einer Bedingung, welche entscheidet welche folgenden Elemente
ausgeführt werden, hier wird geprüft ob \(\mathsf a > m\)

\begin{tcolorbox}[enhanced jigsaw, left=2mm, leftrule=.75mm, bottomrule=.15mm, title=\textcolor{quarto-callout-caution-color}{\faFire}\hspace{0.5em}{Flussdiagramm: Maximumsuche}, colback=white, arc=.35mm, breakable, titlerule=0mm, bottomtitle=1mm, colbacktitle=quarto-callout-caution-color!10!white, toprule=.15mm, opacityback=0, coltitle=black, rightrule=.15mm, opacitybacktitle=0.6, toptitle=1mm, colframe=quarto-callout-caution-color-frame]

\begin{figure}[H]

{\centering \includegraphics{index_files/mediabag/books/a-algorithmen/skript/00-bilder/alg_findmax.pdf}

}

\caption{Algorithmus zur Bestimmung des Maximums einer Zahlenliste}

\end{figure}%

\end{tcolorbox}

Wie oben beschrieben, können wir aber nicht nur Flussdiagramme zur
Darstellung von Algorithmen nutzen, sondern auch sogenannten Pseudocode.
Versuchen Sie einmal selbst den ALgorithmus zur Maximumsuche in
Pseudocode zu formulieren.

\begin{tcolorbox}[enhanced jigsaw, left=2mm, leftrule=.75mm, bottomrule=.15mm, title=\textcolor{quarto-callout-caution-color}{\faFire}\hspace{0.5em}{Pseudocode: Maximumsuche}, colback=white, arc=.35mm, breakable, titlerule=0mm, bottomtitle=1mm, colbacktitle=quarto-callout-caution-color!10!white, toprule=.15mm, opacityback=0, coltitle=black, rightrule=.15mm, opacitybacktitle=0.6, toptitle=1mm, colframe=quarto-callout-caution-color-frame]

\begin{Shaded}
\begin{Highlighting}[]
\NormalTok{Eingabe: Liste von Zahlen L (z. B. L = [5, 8, 2, 9, 3])}

\NormalTok{1. Setze max\_wert := L[0]        // Initialisiere Hilfswert mit dem ersten Element der Liste}

\NormalTok{2. Für jedes Element x in L:}
\NormalTok{    a. Falls x \textgreater{} max\_wert:}
\NormalTok{        i. Setze max\_wert := x}
\NormalTok{    b. Andernfalls:}
\NormalTok{        i. Tue nichts (fortfahren)}

\NormalTok{3. Ausgabe: max\_wert             // max\_wert enthält nun den größten Wert der Liste}
\end{Highlighting}
\end{Shaded}

\end{tcolorbox}

Ein Beispiel für den Ablauf des Algoritmus für eine Liste von 20 Zahlen
ist:

\begin{Shaded}
\begin{Highlighting}[]
\ImportTok{import}\NormalTok{ numpy }\ImportTok{as}\NormalTok{ np}
\NormalTok{np.set\_printoptions(linewidth}\OperatorTok{=}\DecValTok{50}\NormalTok{)}
\CommentTok{\# A = np.random.randint(0, high=1000, size=20)}
\NormalTok{A }\OperatorTok{=}\NormalTok{ np.array([}\DecValTok{203}\NormalTok{, }\DecValTok{433}\NormalTok{, }\DecValTok{504}\NormalTok{, }\DecValTok{602}\NormalTok{, }\DecValTok{567}\NormalTok{, }\DecValTok{762}\NormalTok{, }\DecValTok{183}\NormalTok{, }\DecValTok{482}\NormalTok{, }\DecValTok{471}\NormalTok{, }\DecValTok{741}\NormalTok{, }\DecValTok{854}\NormalTok{, }\DecValTok{486}\NormalTok{, }\DecValTok{350}\NormalTok{, }\DecValTok{550}\NormalTok{, }\DecValTok{885}\NormalTok{, }\DecValTok{395}\NormalTok{, }\DecValTok{203}\NormalTok{, }\DecValTok{288}\NormalTok{, }\DecValTok{909}\NormalTok{, }\DecValTok{644}\NormalTok{])}

\BuiltInTok{print}\NormalTok{(}\StringTok{\textquotesingle{}Schritt 1:\textquotesingle{}}\NormalTok{)}
\BuiltInTok{print}\NormalTok{(}\StringTok{\textquotesingle{}==========\textquotesingle{}}\NormalTok{)}
\BuiltInTok{print}\NormalTok{(}\StringTok{\textquotesingle{}A =\textquotesingle{}}\NormalTok{, np.array2string(A, separator}\OperatorTok{=}\StringTok{\textquotesingle{}, \textquotesingle{}}\NormalTok{))}

\BuiltInTok{print}\NormalTok{()}
\BuiltInTok{print}\NormalTok{(}\StringTok{\textquotesingle{}Schritt 2:\textquotesingle{}}\NormalTok{)}
\BuiltInTok{print}\NormalTok{(}\StringTok{\textquotesingle{}==========\textquotesingle{}}\NormalTok{)}
\NormalTok{m }\OperatorTok{=}\NormalTok{ A[}\DecValTok{0}\NormalTok{]}
\BuiltInTok{print}\NormalTok{(}\StringTok{\textquotesingle{}m = A[0] =\textquotesingle{}}\NormalTok{, m)}

\BuiltInTok{print}\NormalTok{()}
\BuiltInTok{print}\NormalTok{(}\StringTok{\textquotesingle{}Schritt 3:\textquotesingle{}}\NormalTok{)}
\BuiltInTok{print}\NormalTok{(}\StringTok{\textquotesingle{}==========\textquotesingle{}}\NormalTok{)}
\ControlFlowTok{for}\NormalTok{ a }\KeywordTok{in}\NormalTok{ A:}
    \BuiltInTok{print}\NormalTok{(}\StringTok{\textquotesingle{}a = }\SpecialCharTok{\{:3d\}}\StringTok{, m = }\SpecialCharTok{\{:3d\}}\StringTok{\textquotesingle{}}\NormalTok{.}\BuiltInTok{format}\NormalTok{(a, m), end}\OperatorTok{=}\StringTok{\textquotesingle{}\textquotesingle{}}\NormalTok{)}
    \ControlFlowTok{if}\NormalTok{ a }\OperatorTok{\textgreater{}}\NormalTok{ m:}
\NormalTok{        m }\OperatorTok{=}\NormalTok{ a}
        \BuiltInTok{print}\NormalTok{(}\StringTok{\textquotesingle{}, da a \textgreater{} m ist, setzte m auf m=\textquotesingle{}}\NormalTok{, m)}
    \ControlFlowTok{else}\NormalTok{:}
        \BuiltInTok{print}\NormalTok{()}

\BuiltInTok{print}\NormalTok{()}
\BuiltInTok{print}\NormalTok{(}\StringTok{\textquotesingle{}Schritt 4:\textquotesingle{}}\NormalTok{)}
\BuiltInTok{print}\NormalTok{(}\StringTok{\textquotesingle{}==========\textquotesingle{}}\NormalTok{)}
\BuiltInTok{print}\NormalTok{(}\StringTok{\textquotesingle{}Maximaler Wert in A: m =\textquotesingle{}}\NormalTok{, m)}
\end{Highlighting}
\end{Shaded}

\begin{verbatim}
Schritt 1:
==========
A = [203, 433, 504, 602, 567, 762, 183, 482, 471, 741,
 854, 486, 350, 550, 885, 395, 203, 288, 909, 644]

Schritt 2:
==========
m = A[0] = 203

Schritt 3:
==========
a = 203, m = 203
a = 433, m = 203, da a > m ist, setzte m auf m= 433
a = 504, m = 433, da a > m ist, setzte m auf m= 504
a = 602, m = 504, da a > m ist, setzte m auf m= 602
a = 567, m = 602
a = 762, m = 602, da a > m ist, setzte m auf m= 762
a = 183, m = 762
a = 482, m = 762
a = 471, m = 762
a = 741, m = 762
a = 854, m = 762, da a > m ist, setzte m auf m= 854
a = 486, m = 854
a = 350, m = 854
a = 550, m = 854
a = 885, m = 854, da a > m ist, setzte m auf m= 885
a = 395, m = 885
a = 203, m = 885
a = 288, m = 885
a = 909, m = 885, da a > m ist, setzte m auf m= 909
a = 644, m = 909

Schritt 4:
==========
Maximaler Wert in A: m = 909
\end{verbatim}

\section{Kapitelübersicht}\label{kapiteluxfcbersicht}

In diesem Kapitel werden folgende Themen behandelt:

\begin{itemize}
\tightlist
\item
  Sortieralgorithmen
\item
  Eigenschaften von Algorithmen
\item
  Numerische Algorithmen
\end{itemize}

\chapter{Sortieralgorithmen}\label{sortieralgorithmen}

Sortieralgoritmen werden genutzt um Listen von Werten der Größe nach zu
sortieren. Anwendung finden diese Algorithmen bei Datenbanken oder
Suchvorgängen. Insbesondere bei langen Listen mit Millionen oder
Milliarden Einträgen ist es wichtig, dass der Algorithmus mit möglichst
wenigen Operationen pro Element auskommt. Diese, als Komplexität
bezeichnete Eigenschaft, wird im nächsten Kapitel genauer erläutert.

Zunächst werden zwei einfache Sortieralgorithmen

\begin{itemize}
\tightlist
\item
  \href{https://de.wikipedia.org/wiki/Selectionsort}{Selectionsort}
\item
  \href{https://de.wikipedia.org/wiki/Bubblesort}{Bubblesort}
\end{itemize}

vorgestellt. Diese werden in der Praxis kaum noch eingesetzt, da es eine
Vielzahl anderer
\href{https://de.wikipedia.org/wiki/Sortierverfahren}{Sortierverfahren}
gibt, welche effektiver arbeiten. Jedoch eignen sich diese beiden
besonders gut, um die Grundideen zu verdeutlichen.

\section{Selectionsort}\label{selectionsort}

Folgende Grundidee liegt dem Selectionsort zugrunde: Es wird der
minimale Wert der Liste gesucht, dann der zweit-kleinste und so weiter
bis die ganze Liste sortiert ist. Dies kann als Abfolge dieser Schritte
für eine Liste mit \(\mathsf n\) Elementen formalisiert werden.

\begin{enumerate}
\def\labelenumi{\arabic{enumi}.}
\tightlist
\item
  Wiederhole die Schritte 2 bis 4 \(\mathsf n\) Mal. Setzte die
  Hilfsvariable \(i\) initial auf Null.
\item
  Suche den minimalen Wert der Liste ab dem \(\mathsf i\)-ten Element.
\item
  Tausche dieses Element mit dem \(\mathsf i\)-ten Element.
\item
  Erhöhe den Wert von \(\mathsf i\) um Eins.
\item
  Die Vertauschungen der Elemente haben zu einer sortierten Liste
  geführt.
\end{enumerate}

Der Selectionsort kann auch als folgendes Flussdiagramm dargestellt
werden.

\begin{figure}[H]

{\centering \includegraphics{index_files/mediabag/books/a-algorithmen/skript/00-bilder/alg_selsort.pdf}

}

\caption{Flussdiagramm des Selectionsort}

\end{figure}%

Als Zahlenbeispiel wird die Liste mit den Elementen 42, 6, 22, 11, 54,
12, 31 mit dem Selectionsort sortiert.

\begin{Shaded}
\begin{Highlighting}[]
\NormalTok{A }\OperatorTok{=}\NormalTok{ [}\DecValTok{42}\NormalTok{, }\DecValTok{6}\NormalTok{, }\DecValTok{22}\NormalTok{, }\DecValTok{11}\NormalTok{, }\DecValTok{54}\NormalTok{, }\DecValTok{12}\NormalTok{, }\DecValTok{31}\NormalTok{]}

\BuiltInTok{print}\NormalTok{(}\StringTok{\textquotesingle{}Zu sortierende Werteliste \textquotesingle{}}\NormalTok{, A)}
\BuiltInTok{print}\NormalTok{()}

\NormalTok{n }\OperatorTok{=} \BuiltInTok{len}\NormalTok{(A)}
\ControlFlowTok{for}\NormalTok{ i }\KeywordTok{in} \BuiltInTok{range}\NormalTok{(n):}
\NormalTok{    mv }\OperatorTok{=}\NormalTok{ A[i]}
\NormalTok{    mi }\OperatorTok{=}\NormalTok{ i}
    \ControlFlowTok{for}\NormalTok{ j }\KeywordTok{in} \BuiltInTok{range}\NormalTok{(i, n):}
        \ControlFlowTok{if}\NormalTok{ A[j] }\OperatorTok{\textless{}}\NormalTok{ mv:}
\NormalTok{            mv }\OperatorTok{=}\NormalTok{ A[j]}
\NormalTok{            mi }\OperatorTok{=}\NormalTok{ j}

    \BuiltInTok{print}\NormalTok{(}\StringTok{"Iteration }\SpecialCharTok{\{:2d\}}\StringTok{: "}\NormalTok{.}\BuiltInTok{format}\NormalTok{(i}\OperatorTok{+}\DecValTok{1}\NormalTok{))}
    \BuiltInTok{print}\NormalTok{(}\StringTok{"   Minimum von "}\NormalTok{, A[i:n], }\StringTok{"ist"}\NormalTok{, mv)}
\NormalTok{    A[mi] }\OperatorTok{=}\NormalTok{ A[i]}
\NormalTok{    A[i] }\OperatorTok{=}\NormalTok{ mv}
    \BuiltInTok{print}\NormalTok{(}\StringTok{"   Sortiert / Unsortiert: "}\NormalTok{, A[:i}\OperatorTok{+}\DecValTok{1}\NormalTok{], }\StringTok{"/"}\NormalTok{, A[i}\OperatorTok{+}\DecValTok{1}\NormalTok{:])}
    \BuiltInTok{print}\NormalTok{()}
\end{Highlighting}
\end{Shaded}

\begin{verbatim}
Zu sortierende Werteliste  [42, 6, 22, 11, 54, 12, 31]

Iteration  1: 
   Minimum von  [42, 6, 22, 11, 54, 12, 31] ist 6
   Sortiert / Unsortiert:  [6] / [42, 22, 11, 54, 12, 31]

Iteration  2: 
   Minimum von  [42, 22, 11, 54, 12, 31] ist 11
   Sortiert / Unsortiert:  [6, 11] / [22, 42, 54, 12, 31]

Iteration  3: 
   Minimum von  [22, 42, 54, 12, 31] ist 12
   Sortiert / Unsortiert:  [6, 11, 12] / [42, 54, 22, 31]

Iteration  4: 
   Minimum von  [42, 54, 22, 31] ist 22
   Sortiert / Unsortiert:  [6, 11, 12, 22] / [54, 42, 31]

Iteration  5: 
   Minimum von  [54, 42, 31] ist 31
   Sortiert / Unsortiert:  [6, 11, 12, 22, 31] / [42, 54]

Iteration  6: 
   Minimum von  [42, 54] ist 42
   Sortiert / Unsortiert:  [6, 11, 12, 22, 31, 42] / [54]

Iteration  7: 
   Minimum von  [54] ist 54
   Sortiert / Unsortiert:  [6, 11, 12, 22, 31, 42, 54] / []
\end{verbatim}

\section{Bubblesort}\label{bubblesort}

Im Gegensatz zum Selectionsort beruht die Idee des Bubblesort auf rein
lokalen Operationen. D.h. hier wird nicht nach den maximalen Werten
gesucht, sondern durch Vertauschungen eine Sortierung erzielt. Das
Verfahren für eine Liste mit \(\mathsf n\) Elementen ist durch folgende
Vorschrift gegeben.

\begin{enumerate}
\def\labelenumi{\arabic{enumi}.}
\tightlist
\item
  Die Schritte 2 bis 4 werden \(\mathsf n\) Mal durchgeführt. Die
  Hilfsvariable \(\mathsf i\) wird initial auf Null gesetzt.
\item
  Starte beim \(\mathsf i\)-ten Element und iteriere bis zum Ende der
  Liste. Falls das aktuell betrachtete Element größer ist als das
  Folgende, tausche beide Elemente.
\item
  Falls in Schritt 2 keine Vertauschungen durchgeführt wurden, gehe zu
  Schritt 5.
\item
  Addiere Eins zum Wert der Variable \(\mathsf i\).
\item
  Die Liste ist sortiert und der Algorithmus ist fertig.
\end{enumerate}

Das nachfolgende Flussdiagramm verdeutlicht den Ablauf des Bubblesort
Algorithmus. Bevor Sie sich das Diagram anschauen, versuchen Sie es
einmal selbst zu erstellen.

\begin{tcolorbox}[enhanced jigsaw, left=2mm, leftrule=.75mm, bottomrule=.15mm, title=\textcolor{quarto-callout-caution-color}{\faFire}\hspace{0.5em}{Flussdiagramm: Bubblesort}, colback=white, arc=.35mm, breakable, titlerule=0mm, bottomtitle=1mm, colbacktitle=quarto-callout-caution-color!10!white, toprule=.15mm, opacityback=0, coltitle=black, rightrule=.15mm, opacitybacktitle=0.6, toptitle=1mm, colframe=quarto-callout-caution-color-frame]

\begin{figure}[H]

{\centering \includegraphics{index_files/mediabag/books/a-algorithmen/skript/00-bilder/alg_bubblesort.pdf}

}

\caption{Flussdiagramm des Bubblesort}

\end{figure}%

\end{tcolorbox}

Der Ablauf des Bubblesort wird beispielhaft für die Liste 42, 6, 22, 11,
54, 12, 31 (gleich der im obigen Beispiel) vorgeführt.

\begin{Shaded}
\begin{Highlighting}[]
\NormalTok{A }\OperatorTok{=}\NormalTok{ [}\DecValTok{42}\NormalTok{, }\DecValTok{6}\NormalTok{, }\DecValTok{22}\NormalTok{, }\DecValTok{11}\NormalTok{, }\DecValTok{54}\NormalTok{, }\DecValTok{12}\NormalTok{, }\DecValTok{31}\NormalTok{]}

\BuiltInTok{print}\NormalTok{(}\StringTok{\textquotesingle{}Zu sortierende Werteliste \textquotesingle{}}\NormalTok{, A)}
\BuiltInTok{print}\NormalTok{()}

\NormalTok{n }\OperatorTok{=} \BuiltInTok{len}\NormalTok{(A)}
\NormalTok{swapped }\OperatorTok{=} \VariableTok{True}
\NormalTok{i }\OperatorTok{=} \DecValTok{0}
\ControlFlowTok{while}\NormalTok{ swapped:}
\NormalTok{    swapped }\OperatorTok{=} \VariableTok{False}
    \BuiltInTok{print}\NormalTok{(}\StringTok{"Iteration }\SpecialCharTok{\{:2d\}}\StringTok{: "}\NormalTok{.}\BuiltInTok{format}\NormalTok{(}\BuiltInTok{len}\NormalTok{(A) }\OperatorTok{{-}}\NormalTok{ n }\OperatorTok{+} \DecValTok{1}\NormalTok{))}
    \BuiltInTok{print}\NormalTok{(}\StringTok{"   Liste zu Beginn der Iteration: "}\NormalTok{, A)}
    \ControlFlowTok{for}\NormalTok{ j }\KeywordTok{in} \BuiltInTok{range}\NormalTok{(n}\OperatorTok{{-}}\DecValTok{1}\NormalTok{):}
        \ControlFlowTok{if}\NormalTok{ A[j}\OperatorTok{+}\DecValTok{1}\NormalTok{] }\OperatorTok{\textless{}}\NormalTok{ A[j]:}
            \BuiltInTok{print}\NormalTok{(}\StringTok{"   Tausche: "}\NormalTok{, A[j], }\StringTok{"und"}\NormalTok{, A[j}\OperatorTok{+}\DecValTok{1}\NormalTok{])}
\NormalTok{            mv }\OperatorTok{=}\NormalTok{ A[j]}
\NormalTok{            A[j] }\OperatorTok{=}\NormalTok{ A[j}\OperatorTok{+}\DecValTok{1}\NormalTok{]}
\NormalTok{            A[j}\OperatorTok{+}\DecValTok{1}\NormalTok{] }\OperatorTok{=}\NormalTok{ mv}
            \BuiltInTok{print}\NormalTok{(}\StringTok{"   Liste nach Tausch: "}\NormalTok{, A)}
\NormalTok{            swapped }\OperatorTok{=} \VariableTok{True}
        \ControlFlowTok{else}\NormalTok{:}
            \BuiltInTok{print}\NormalTok{(}\StringTok{"   Elemente "}\NormalTok{, A[j], }\StringTok{"und"}\NormalTok{, A[j}\OperatorTok{+}\DecValTok{1}\NormalTok{], }\StringTok{" müssen nicht getauscht werden"}\NormalTok{)}
\NormalTok{    n }\OperatorTok{{-}=} \DecValTok{1}
    \ControlFlowTok{if} \KeywordTok{not}\NormalTok{ swapped:}
        \BuiltInTok{print}\NormalTok{(}\StringTok{"   kein Tausch mehr notwendig, Liste ist nun sortiert"}\NormalTok{)}
    \BuiltInTok{print}\NormalTok{()}
\end{Highlighting}
\end{Shaded}

\begin{verbatim}
Zu sortierende Werteliste  [42, 6, 22, 11, 54, 12, 31]

Iteration  1: 
   Liste zu Beginn der Iteration:  [42, 6, 22, 11, 54, 12, 31]
   Tausche:  42 und 6
   Liste nach Tausch:  [6, 42, 22, 11, 54, 12, 31]
   Tausche:  42 und 22
   Liste nach Tausch:  [6, 22, 42, 11, 54, 12, 31]
   Tausche:  42 und 11
   Liste nach Tausch:  [6, 22, 11, 42, 54, 12, 31]
   Elemente  42 und 54  müssen nicht getauscht werden
   Tausche:  54 und 12
   Liste nach Tausch:  [6, 22, 11, 42, 12, 54, 31]
   Tausche:  54 und 31
   Liste nach Tausch:  [6, 22, 11, 42, 12, 31, 54]

Iteration  2: 
   Liste zu Beginn der Iteration:  [6, 22, 11, 42, 12, 31, 54]
   Elemente  6 und 22  müssen nicht getauscht werden
   Tausche:  22 und 11
   Liste nach Tausch:  [6, 11, 22, 42, 12, 31, 54]
   Elemente  22 und 42  müssen nicht getauscht werden
   Tausche:  42 und 12
   Liste nach Tausch:  [6, 11, 22, 12, 42, 31, 54]
   Tausche:  42 und 31
   Liste nach Tausch:  [6, 11, 22, 12, 31, 42, 54]

Iteration  3: 
   Liste zu Beginn der Iteration:  [6, 11, 22, 12, 31, 42, 54]
   Elemente  6 und 11  müssen nicht getauscht werden
   Elemente  11 und 22  müssen nicht getauscht werden
   Tausche:  22 und 12
   Liste nach Tausch:  [6, 11, 12, 22, 31, 42, 54]
   Elemente  22 und 31  müssen nicht getauscht werden

Iteration  4: 
   Liste zu Beginn der Iteration:  [6, 11, 12, 22, 31, 42, 54]
   Elemente  6 und 11  müssen nicht getauscht werden
   Elemente  11 und 12  müssen nicht getauscht werden
   Elemente  12 und 22  müssen nicht getauscht werden
   kein Tausch mehr notwendig, Liste ist nun sortiert
\end{verbatim}

\chapter{Eigenschaften}\label{eigenschaften}

\section{Terminiertheit}\label{terminiertheit}

Terminiertheit bedeutet, dass ein Algorithmus nach endlich vielen
Schritten anhält, oder er bricht kontrolliert ab. Einfache Beispiele:

\begin{itemize}
\tightlist
\item
  Addition zweier Dezimalzahlen
\item
  Summe der ersten N natürlichen Zahlen
\end{itemize}

Allerdings kann die Terminiertheit nicht für alle Algerithmen gezeigt
werden. Das
\href{https://de.wikipedia.org/wiki/Halteproblem}{Halteproblem} besagt,
dass es gibt keinen Verfahren gibt, welches immer zutreffend sagen kann,
ob der Algorithmus für die Eingabe terminiert. Hierzu kann das
\href{https://de.wikipedia.org/wiki/Collatz-Problem}{Collatz-Problem}
als Beispiel herangezogen werden.

Die Zahlenfolge wird wie folgt konstruiert:

\begin{itemize}
\tightlist
\item
  beginne mit irgendeiner natürlichen Zahl \(\mathsf n_0 > 0\)
\item
  ist \(\mathsf n_i\) gerade so ist \(\mathsf n_{i+1} = n_i/2\)
\item
  ist \(\mathsf n_i\) ungerade so ist \(\mathsf n_{i+1} = 3n_i + 1\)
\item
  endet bei \(\mathsf n_i = 1\)
\end{itemize}

Collatz-Vermutung: Jede so konstruierte Zahlenfolge mündet in den Zyklus
4, 2, 1, egal, mit welcher natürlichen Zahl man beginnt. Bisher
unbewiesen.

\section{Determiniertheit}\label{determiniertheit}

Ein deterministischer Algorithmus ist ein Algorithmus, bei dem nur
definierte und reproduzierbare Zustände auftreten. Die Ergebnisse des
Algorithmus sind somit immer reproduzierbar. Beispiele hierfür:

\begin{itemize}
\tightlist
\item
  Addition ganzer Zahlen
\item
  Selectionsort
\item
  Collatz-Sequenz
\end{itemize}

\section{Effizienz}\label{effizienz}

Die Effizienz eines Algorithmus ist nicht strikt definiert und kann
folgende Aspekte umfassen:

\begin{itemize}
\tightlist
\item
  Laufzeit
\item
  Speicheraufwand
\item
  Energieverbrauch
\end{itemize}

Bei bestimmten Anwendungen sind entsprechende Eigenschaften notwendig:

\begin{itemize}
\tightlist
\item
  Speicheraufwand bei \emph{Big Data}, also riesige Datenmengen, z.B. in
  der Bioinformatik
\item
  Laufzeit bei Echtzeitanwendung, z.B. Flugzeugsteuerung,
  Fußgängerdynamik
\end{itemize}

\section{Komplexität}\label{komplexituxe4t}

Bei der Analyse von Algorithmen, gilt es die Komplexiät zu bestimmen,
welche ein Maß für den Aufwand darstellt. Dabei wird nach einer
Aufwandfunktion \(\mathsf f(n)\) gesucht, welche von der Problemgröße
\(\mathsf n\) abhängt. Beispiel für eine Problemgröße:

\begin{itemize}
\tightlist
\item
  Anzahl der Summanden bei einer Summe
\item
  Anzahl der zu sortierenden Zahlen
\end{itemize}

Meist wird dabei die Bestimmung auf eine asymptotische Analyse, d.h.
eine einfache Vergleichsfunktion \(\mathsf g(n)\) mit
\(\mathsf n \rightarrow \infty\), reduziert. Dabei beschränkt
\(\mathsf g(n)\) das Wachstum von \(\mathsf f(n)\).

Die Funktion \(\mathsf g(n)\) wird oft durch ein \(\mathcal{O}\)
gekennzeichnet und gibt so eine möglichst einfache Vergleichsfunktion
an. Beispiele:

\begin{itemize}
\tightlist
\item
  \(\mathsf f_1(n) = n^4 + 5n^2 - 10 \approx \mathcal{O}(n^4) = g_1(n)\)
\item
  \(\mathsf f_2(n) = 2^{n+1} \approx \mathcal{O}(2^n) = g_2(n)\)
\end{itemize}

\begin{figure}[H]

{\centering \includegraphics{index_files/mediabag/books/a-algorithmen/skript/00-bilder/komplexitaet.pdf}

}

\caption{Komplexität eines Algorithmus durch Vergleich einer
Aufwandfunktion mit einer Vergleichsfunktion}

\end{figure}%

Um sich ein besseres Bild zu den Auswirkungen hoher Kompexitäten zu
machen, sei folgendes Beispiel gegeben.

\begin{itemize}
\tightlist
\item
  ein Berechnungsschritt (unabhängig von der Problemgröße \(\mathsf n\))
  sei z.B. 1 s lang
\item
  das \(\mathsf n\) sei beispielsweise 1000
\end{itemize}

Damit ergeben sich folgende (asymptotische) Abschätzungen der Laufzeit:

\begin{itemize}
\tightlist
\item
  \(\mathcal{O}(n)\): 103 s ≈ 1 h
\item
  \(\mathcal{O}(n^2)\): 106 s ≈ 11 d
\item
  \(\mathcal{O}(n^3)\): 109 s ≈ 31 a
\item
  \(\mathcal{O}(2^n)\): 21000 s ≈ \ldots{}
\end{itemize}

\subsection{Komplexität
Selectionsort}\label{komplexituxe4t-selectionsort}

Die Kompexität dieses Verfahrens kann leicht abgeschätzt werden. Bei
jedem Durchlauf wir das Minimum / Maximum gesucht, was anfangs
\(\mathsf n\) Operationen benötigt. Beim nächsten Durchlauf sind es nur
noch \(\mathsf n − 1\) Operationen und so weiter. In der Summe sind es
also

\[ \mathsf f(n) = \sum_{i=0}^n i = \frac{n(n-1)}{2} \approx \mathcal{O}(n^2) \]

Damit hat der Selectionsort eine Komplexität von \(\mathcal{O}(n^2)\).
Die folgende Abbildung verdeutlicht dies nochmals.

\begin{verbatim}
10 42.149
20 181.189
30 417.886
40 757.682
50 1195.949
60 1731.304
70 2368.036
80 3107.657
90 3939.036
100 4882.65
\end{verbatim}

\begin{figure}[H]

{\centering \includegraphics{index_files/mediabag/books/a-algorithmen/skript/00-bilder/sort_selection.pdf}

}

\caption{Abschätzung der Koplexität des Selectionsort-Algorithmus}

\end{figure}%

\subsection{Komplexität Bubblesort}\label{komplexituxe4t-bubblesort}

Die Komplexität des Bubblesort muss unterschieden werden in den
günstigsten Fall (best case), den ungünstigsten Fall (worst case) und
einem durchschnittlichen Fall (average case):

\begin{itemize}
\tightlist
\item
  best case: \(\mathcal{O} (n)\)
\item
  worst case: \(\mathcal{O} (n^2)\)
\item
  average case: \(\mathcal{O} (n^2)\)
\end{itemize}

Der best case ergibt sich zum Beispiel, falls die Eingabeliste bereits
sortiert ist, da der Algorithmus nur einmal durch die Liste gehen muss,
entsprechend n-Mal. Folgende Abbildung verdeutlicht die Anzahl der
durchgeführten Operationen im Falle einer vollständig zufälligen Liste
und einer, bei welcher 95\% der Werte bereits sortiert ist. Dabei wurden
für jedes \(\mathsf n\) jeweils 10000 Listen sortiert. Es ist der
Mittelwert und die minimalen und maximalen Operationen dargestellt.

\begin{verbatim}
10 42.0649
20 180.9895
30 419.2844
40 756.5975
50 1194.2714
60 1731.7116
70 2368.9395
80 3106.3849
90 3941.3358
100 4879.4919
\end{verbatim}

\begin{figure}[H]

{\centering \includegraphics{index_files/mediabag/books/a-algorithmen/skript/00-bilder/sort_bubble_p000.pdf}

}

\caption{Abschätzung der Koplexität des Bubblesort-Algorithmus ohne
Vorsortierung}

\end{figure}%

\begin{verbatim}
10 30.0057
20 127.1876
30 303.4513
40 546.5099
50 904.1873
60 1305.1077
70 1864.5479
80 2432.5939
90 3203.7951
100 3942.713
\end{verbatim}

\begin{figure}[H]

{\centering \includegraphics{index_files/mediabag/books/a-algorithmen/skript/00-bilder/sort_bubble_p095.pdf}

}

\caption{Abschätzung der Koplexität des Bubblesort-Algorithmus mit einer
95\%-igen Vorsortierung}

\end{figure}%

\chapter{Numerische Algorithmen}\label{numerische-algorithmen}

\section{Newton-Raphson-Verfahren}\label{newton-raphson-verfahren}

Eines der einfachsten und auch ältesten Verfahren zur Suche von
Nullstellen von Funktionen ist das
\href{https://de.wikipedia.org/wiki/Newtonverfahren}{Newton-Raphson-Verfahren},
welches bereits im 17-ten Jahrhundert entwickelt und eingestetzt wurde.

\subsection{Anwendungen}\label{anwendungen}

Das Finden von Nullstellen ist die Grundlage für viele Verfahren, welche
z.B. für

\begin{itemize}
\tightlist
\item
  das Lösen von nicht-linearen Gleichungen,
\item
  das Finden von Extremwerten, oder
\item
  Optimierungsverfahren
\end{itemize}

eingesetzt werden kann.

\subsection{Grundidee}\label{grundidee}

Die Grundidee beruht auf einer iterativen Suche der Nullstelle
\(\mathsf x_{ns}\) einer stetig differenzierbaren Funktion
\(\mathsf f(x)\) mit Hilfe der ersten Ableitung \(\mathsf f'(x)\). Durch
das Anlegen von Tangenten an die aktuelle Näherung der Nullstelle
\(\mathsf x_i\) kann die nächste Näherung bestimmt werden.

Bei gegebenen Startwert, \(\mathsf x_0\) für den ersten
Iterationsschritt (\(\mathsf i=0\)), können die folgenden Näherungen
durch

\[\mathsf x_{i+1} = x_i - \frac{f(x_i)}{f'(x_i)} \]

berechnet werden. Dabei bestimmt die Wahl des Startwerts, welche der
ggf. mehreren Nullstellen gefunden wird.

\subsection{Beispiel 1}\label{beispiel-1}

Gegeben ist die Funktion \(\mathsf f(x) = x^2 - 1\). Die Ableitung ist
gegeben durch \(\mathsf f'(x) = 2x\) und die Nullstellen lauten
\(\mathsf x_{ns} = \{-1, 1\}\).

Bei einem Startwert von \(\mathsf x_0 = 0.3\) führt zu folgender
Iteration:

\begin{verbatim}
Startwert x_0 = 0.3000

Iterationsschritt i =  1, x_i = 0.3000
   f(x_i)  = -0.9100
   fp(x_i) = 0.6000
   x_(i+1) = 1.8167

Iterationsschritt i =  2, x_i = 1.8167
   f(x_i)  = 2.3003
   fp(x_i) = 3.6333
   x_(i+1) = 1.1836

Iterationsschritt i =  3, x_i = 1.1836
   f(x_i)  = 0.4008
   fp(x_i) = 2.3671
   x_(i+1) = 1.0142

Iterationsschritt i =  4, x_i = 1.0142
   f(x_i)  = 0.0287
   fp(x_i) = 2.0285
   x_(i+1) = 1.0001


Endergebnis nach 5 Iterationen: x_(ns) = 1.0001
\end{verbatim}

\begin{verbatim}
<Figure size 1650x1050 with 0 Axes>
\end{verbatim}

\begin{figure}[H]

{\centering \includegraphics{index_files/mediabag/books/a-algorithmen/skript/00-bilder/newton_bsp1.pdf}

}

\caption{Newton-Verfahren, Beispiel 1}

\end{figure}%

\section{Schritt 0}

\includegraphics{index_files/mediabag/books/a-algorithmen/skript/00-bilder/newton_bsp1_step_00.pdf}

\section{Schritt 1}

\includegraphics{index_files/mediabag/books/a-algorithmen/skript/00-bilder/newton_bsp1_step_01.pdf}

\section{Schritt 2}

\includegraphics{index_files/mediabag/books/a-algorithmen/skript/00-bilder/newton_bsp1_step_02.pdf}

\section{Schritt 3}

\includegraphics{index_files/mediabag/books/a-algorithmen/skript/00-bilder/newton_bsp1_step_03.pdf}

\section{Schritt 4}

\includegraphics{index_files/mediabag/books/a-algorithmen/skript/00-bilder/newton_bsp1_step_04.pdf}

\subsection{Beispiel 2}\label{beispiel-2}

Gegeben ist die Funktion \(\mathsf f(x) = \sin(x) - 0.5\) mit der
Ableitung \(\mathsf f'(x) = \cos(x)\).

\begin{Shaded}
\begin{Highlighting}[]
\KeywordTok{def}\NormalTok{ f(x):}
    \ControlFlowTok{return}\NormalTok{ np.sin(x) }\OperatorTok{{-}}\FloatTok{0.5}
\KeywordTok{def}\NormalTok{ fp(x):}
    \ControlFlowTok{return}\NormalTok{ np.cos(x)}

\NormalTok{x0 }\OperatorTok{=} \FloatTok{1.3}

\BuiltInTok{print}\NormalTok{(}\StringTok{\textquotesingle{}Startwert x\_0 = }\SpecialCharTok{\{:.4f\}}\StringTok{\textquotesingle{}}\NormalTok{.}\BuiltInTok{format}\NormalTok{(x0))}
\BuiltInTok{print}\NormalTok{()}

\NormalTok{n }\OperatorTok{=} \DecValTok{5}
\NormalTok{xi }\OperatorTok{=}\NormalTok{ [x0]}
\ControlFlowTok{for}\NormalTok{ i }\KeywordTok{in} \BuiltInTok{range}\NormalTok{(}\DecValTok{1}\NormalTok{,n):}
\NormalTok{    xp }\OperatorTok{=}\NormalTok{ xi[i}\OperatorTok{{-}}\DecValTok{1}\NormalTok{]}
\NormalTok{    xn }\OperatorTok{=}\NormalTok{ xp }\OperatorTok{{-}}\NormalTok{ (f(xp)}\OperatorTok{/}\NormalTok{fp(xp))}
    
    \BuiltInTok{print}\NormalTok{(}\StringTok{\textquotesingle{}Iterationsschritt i = }\SpecialCharTok{\{:2d\}}\StringTok{, x\_i = }\SpecialCharTok{\{:.4f\}}\StringTok{\textquotesingle{}}\NormalTok{.}\BuiltInTok{format}\NormalTok{(i, xp))}
    \BuiltInTok{print}\NormalTok{(}\StringTok{\textquotesingle{}   f(x\_i)  = }\SpecialCharTok{\{:.4f\}}\StringTok{\textquotesingle{}}\NormalTok{.}\BuiltInTok{format}\NormalTok{(f(xp)))}
    \BuiltInTok{print}\NormalTok{(}\StringTok{\textquotesingle{}   fp(x\_i) = }\SpecialCharTok{\{:.4f\}}\StringTok{\textquotesingle{}}\NormalTok{.}\BuiltInTok{format}\NormalTok{(fp(xp)))}
    \BuiltInTok{print}\NormalTok{(}\StringTok{\textquotesingle{}   x\_(i+1) = }\SpecialCharTok{\{:.4f\}}\StringTok{\textquotesingle{}}\NormalTok{.}\BuiltInTok{format}\NormalTok{(xn))}
    \BuiltInTok{print}\NormalTok{()}
    
\NormalTok{    xi.append(xn)}
    
\BuiltInTok{print}\NormalTok{()}
\BuiltInTok{print}\NormalTok{(}\StringTok{\textquotesingle{}Endergebnis nach }\SpecialCharTok{\{\}}\StringTok{ Iterationen: x\_(ns) = }\SpecialCharTok{\{:.4f\}}\StringTok{\textquotesingle{}}\NormalTok{.}\BuiltInTok{format}\NormalTok{(n, xi[}\OperatorTok{{-}}\DecValTok{1}\NormalTok{]))}
\end{Highlighting}
\end{Shaded}

\begin{verbatim}
Startwert x_0 = 1.3000

Iterationsschritt i =  1, x_i = 1.3000
   f(x_i)  = 0.4636
   fp(x_i) = 0.2675
   x_(i+1) = -0.4329

Iterationsschritt i =  2, x_i = -0.4329
   f(x_i)  = -0.9195
   fp(x_i) = 0.9077
   x_(i+1) = 0.5801

Iterationsschritt i =  3, x_i = 0.5801
   f(x_i)  = 0.0481
   fp(x_i) = 0.8364
   x_(i+1) = 0.5226

Iterationsschritt i =  4, x_i = 0.5226
   f(x_i)  = -0.0009
   fp(x_i) = 0.8665
   x_(i+1) = 0.5236


Endergebnis nach 5 Iterationen: x_(ns) = 0.5236
\end{verbatim}

\begin{figure}[H]

{\centering \includegraphics{index_files/mediabag/books/a-algorithmen/skript/00-bilder/newton_bsp2.pdf}

}

\caption{Newton-Verfahren, Beispiel 2}

\end{figure}%

\section{Schritt 0}

\includegraphics{index_files/mediabag/books/a-algorithmen/skript/00-bilder/newton_bsp2_step_00.pdf}

\section{Schritt 1}

\includegraphics{index_files/mediabag/books/a-algorithmen/skript/00-bilder/newton_bsp2_step_01.pdf}

\section{Schritt 2}

\includegraphics{index_files/mediabag/books/a-algorithmen/skript/00-bilder/newton_bsp2_step_02.pdf}

\section{Schritt 3}

\includegraphics{index_files/mediabag/books/a-algorithmen/skript/00-bilder/newton_bsp2_step_03.pdf}

\section{Schritt 4}

\includegraphics{index_files/mediabag/books/a-algorithmen/skript/00-bilder/newton_bsp2_step_04.pdf}

\section{Euler-Verfahren}\label{euler-verfahren}

Das explizite
\href{https://de.wikipedia.org/wiki/Explizites_Euler-Verfahren}{Euler-Verfahren}
ist ein einfacher Algorithmus zur Bestimmung von Näherungslösungen von
gewöhnlichen Differentialgleichungen, insbesondere Anfangswertprobleme.
Das Verfahren wird hier anhand einer linearen Differentialgleichung 1.
Ordnung demonstiert, hier ist \(\mathsf y = y(t)\) eine Funktion der
Zeit \(\mathsf t\). Die Differentialgleichung lautet

\[\dot y(t) + a(t)y(t) + b(t) = 0\]

Mit einem vorgegebenen Anfangswert \(\mathsf y_0 = y(t_0)\) kann die
Näherungslösung iterativ bis zur gewünschten Endzeit \(\mathsf t_e\)
bestimmt werden. Dazu muss das betrachtete Zeitintervall
\(\mathsf[t_0, t_e]\) in \(\mathsf n_t\) Teilintervalle aufgeteilt
werden. Die Länge eines Teilintervalls ist

\[\mathsf \Delta t = \frac{t_e - t_0}{n_t}\quad .\]

Das iterative Verfahren beschreibt die Bestimmung der Lösung im nächsten
Zeitinterval \(\mathsf t_{i+1}\)

\[\mathsf  y(t_{i+1}) = y(t_i) - \Delta t \big(a(t_i)y(t_i) + b(t_i)\big)\quad .\]

\subsection{Beispiel 1}\label{beispiel-1-1}

Mit \(\mathsf a(t) = 1\), \(\mathsf b(t) = 0\) und einem Anfangswert von
\(\mathsf y_0 = 1\).

\chapter{Exkurs: Interne Darstellung von Zahlen und
Zeichen}\label{exkurs-interne-darstellung-von-zahlen-und-zeichen}

Die Grundlage für alle modernen Computer ist die
\href{https://de.wikipedia.org/wiki/Digitalisierung}{Digitalisierung}.
Diese ermöglicht es reale Informationen, Kommunikationsformen oder
Anweisungen als eine Folge von zwei Zuständen 1/0 darzustellen.
Computersysteme nutzen diese Reduktion bzw. Vereinfachung auf nur zwei
Zustände zum Speichern, Übertragen und Verarbeiten von Daten.

\section{Analoge und digitale
Signale}\label{analoge-und-digitale-signale}

In der Technik unterscheidet man grundsätzlich zwischen
\textbf{analogen} und \textbf{digitalen Signalen}, wenn Informationen
dargestellt, verarbeitet oder übertragen werden.

\subsection{Analoge Signale}\label{analoge-signale}

Ein analoges Signal ist \textbf{kontinuierlich} in Zeit und Ausprägung.
Es kann unendlich viele Werte innerhalb eines Bereichs annehmen -- wie
z.\,B. die Spannung eines Mikrofonsignals, die mit der Lautstärke
variiert, oder die Temperatur, die sich stetig verändert.

Analoge Signale sind gut geeignet, um natürliche Phänomene direkt
abzubilden, sind aber anfällig für Störungen und schwer exakt zu
speichern oder weiterzuverarbeiten.

\subsection{Digitale Signale}\label{digitale-signale}

Ein digitales Signal besteht aus \textbf{diskreten Werten} -- meist
zwei: 0 und 1. Es wird also in einzelnen Schritten dargestellt und ist
damit für Computer besonders gut geeignet. Die kontinuierlichen Werte
der realen Welt müssen dafür zunächst in digitale Werte
\textbf{umgewandelt} werden (Analog-Digital-Wandlung).

Digitale Signale lassen sich fehlerfrei speichern, übertragen und
beliebig oft kopieren, ohne dass die Information an Qualität verliert.

\subsection{Vergleich: Analoge vs.~digitale
Signale}\label{vergleich-analoge-vs.-digitale-signale}

\begin{longtable}[]{@{}
  >{\raggedright\arraybackslash}p{(\columnwidth - 4\tabcolsep) * \real{0.2250}}
  >{\raggedright\arraybackslash}p{(\columnwidth - 4\tabcolsep) * \real{0.3833}}
  >{\raggedright\arraybackslash}p{(\columnwidth - 4\tabcolsep) * \real{0.3917}}@{}}
\toprule\noalign{}
\begin{minipage}[b]{\linewidth}\raggedright
Merkmal
\end{minipage} & \begin{minipage}[b]{\linewidth}\raggedright
Analoge Signale
\end{minipage} & \begin{minipage}[b]{\linewidth}\raggedright
Digitale Signale
\end{minipage} \\
\midrule\noalign{}
\endhead
\bottomrule\noalign{}
\endlastfoot
Signalverlauf & Stetig, kontinuierlich & Diskret, stufenweise \\
Wertebereich & Unendlich viele Werte innerhalb eines Bereichs & Endliche
Anzahl (z.\,B. 0 und 1) \\
Störanfälligkeit & Hoch -- kleine Störungen wirken sich direkt aus &
Gering -- durch Fehlerkorrektur ausgleichbar \\
Speicherung & Schwierig, da kontinuierlich & Einfach und verlustfrei
möglich \\
Verarbeitung & Aufwendig, da kontinuierlich & Effizient durch digitale
Logik \\
Beispiel & Plattenspieler, Thermometer mit Zeiger & MP3-Datei,
Digitalkamera \\
\end{longtable}

\subsection{Umwandlung analoger zu digitaler
Signale}\label{umwandlung-analoger-zu-digitaler-signale}

Um aus analogen Werten, z.B. aus einem Experiment, digitale Werte für
die Auswertung z u gewinnen, werden Analog-Digital-Wandler (ADC)
genutzt. Einfach gesagt, tastet ein ADC ein Signal mit einer
vorgegebenen (endlichen) Abtastrate ab. Dabei wird der Signalwert einem
der (endlich vielen) vorgegebenen Werteintervalle zugeordnet. Folgende
Abbildung zeigt ein Beispiel für die Umwandlung eines analogen Signals
(blaue Kurve) in ein digitales (orangene Punkte). Hier ist das
Abtastintervall im Zeit- und Wertbereich jeweils Eins, in der Abbildung
durch das graue Gitter veranschaulicht. Damit kann die vom ADC
ermittelte Folge von Werten nur Punkte auf dem Gitter enthalten.

\begin{figure}

\centering{

\includegraphics{books/a-algorithmen/skript/zahlendarstellung_files/figure-pdf/fig-transform_digital_analog-output-1.pdf}

}

\caption{\label{fig-transform_digital_analog}Umwandlung eines digitalen
(blau) zu einem analogen Signal (orange)}

\end{figure}%

\section{Digitale Zahlendarstellung}\label{digitale-zahlendarstellung}

In den vorherigen Abschnitten haben wir gesehen, wie analoge
Informationen in digitale Signale umgewandelt werden können. Damit ein
Computer solche digitalen Informationen verarbeiten kann, müssen sie
intern als \textbf{Zahlen} dargestellt werden - und zwar in einem für
Computer verständlichen Format: dem \textbf{Binärsystem}.

Wir beschäftigen uns in diesem Abschnitt daher mit der Frage, wie Zahlen
\textbf{intern gespeichert und dargestellt} werden. Dabei lernen wir
unter anderem:

\begin{itemize}
\tightlist
\item
  Wie Zahlen in \textbf{verschiedenen Zahlensystemen} dargestellt werden
  können (z.\,B. Binär, Dezimal, Hexadezimal),
\item
  wie man zwischen diesen Systemen \textbf{umrechnet},
\item
  und wie man solche Umrechnungen auch \textbf{algorithmisch}
  beschreiben und in Code umsetzen kann.
\end{itemize}

\subsection{Dualsystem}\label{dualsystem}

Da in der digitalen Elektronik nur mit zwei Zuständen gerechnet wird,
bietet sich das
\href{https://de.wikipedia.org/wiki/Dualsystem}{Dualsystem}, auch
genannt Binärsystem, zur Zahlendarstellung an. Beispiele für
Zahlendarstellungen zur Basis 2, wobei der Index die Basis angibt:

\begin{itemize}
\tightlist
\item
  510 = 1012
\item
  10710 = 11010112
\item
  263510 = 1010010010112
\end{itemize}

Damit lassen sich Zahlen als eine Reihe bzw. Abfolge von
\texttt{0}/\texttt{1}-Zuständen darstellen.

\subsection{Hexadezimalsystem}\label{hexadezimalsystem}

Bei Zahlen zur Basis 16 müssen auch Stellen, welche größer als 9 sind,
abgebildet werden. Hierzu werden Buchstaben eingesetzt, um die Ziffern
`10', dargestellt durch \texttt{A}, bis `15' (\texttt{F}) abzubilden.
Eine oft verwendete Schreibweise für Zahlen im Hexadezimalsystem ist das
Vorstellen von \texttt{0x} vor die Zahl, wie im folgenden Beispielen
gezeigt:

\begin{itemize}
\tightlist
\item
  510 = 516 = 0x5
\item
  10710 = 6B16 = 0x6B
\item
  263510 = A4B16 = 0xA4B
\end{itemize}

\section{Binäre Maßeinheiten}\label{binuxe4re-mauxdfeinheiten}

Da sich in der digitalen Welt alles um Potenzen von 2 dreht, haben sich
aus technischen Gründen folgende Einheiten ergeben:

\begin{itemize}
\tightlist
\item
  1 Bit = eine Ziffer im Binärsystem, Wertebereich: 0 und 1
\item
  1 Byte = acht Ziffern im Binärsystem, Wertebereich: 0 bis 255
\end{itemize}

Um größere Datenmengen praktischer anzugeben, werden folgende Einheiten
genutzt:

\begin{itemize}
\tightlist
\item
  1 KB = 1 kiloByte = 103 Byte
\item
  1 MB = 1 megaByte = 106 Byte
\item
  1 GB = 1 gigaByte = 109 Byte
\item
  1 TB = 1 teraByte = 1012 Byte
\item
  1 PB = 1 petaByte = 1015 Byte
\end{itemize}

\section{Geschwindigkeit der
Datenübertragung}\label{geschwindigkeit-der-datenuxfcbertragung}

Die Geschwindigkeit mit der Daten übertragen werden können wird als
Datenmenge pro Zeit angegeben. Hierbei wird die Zeit meist auf eine
Sekunde bezogen. Beispielhaft sind hier einige Datenübertragungsraten
beim Zugriff auf eine
\href{https://de.wikipedia.org/wiki/Festplattenlaufwerk}{magnetische
Festplatte (HDD)} und auf ein
\href{https://de.wikipedia.org/wiki/Solid-State-Drive}{Halbleiterlaufwerk
(SSD)} aufgeführt.

\begin{itemize}
\tightlist
\item
  Lesen / Schreiben HDD: \textasciitilde200 MB/s
\item
  Lesen / Schreiben SSD: \textasciitilde500 MB/s
\end{itemize}

Als weiteres Beispiel können maximale Übertragunsraten in verschiedenen
Netzwerken genannt werden:

\begin{itemize}
\tightlist
\item
  über das Mobilfunknetz, z.B.
  \href{https://de.wikipedia.org/wiki/Mobilfunkstandard}{3G}: 384 kbit/s
\item
  über ein Netzwerkkabel, z.B.
  \href{https://de.wikipedia.org/wiki/Ethernet}{Fast Ethernet}:
  \textasciitilde100 Mbit/s
\end{itemize}

\section{Darstellung ganzer Zahlen}\label{darstellung-ganzer-zahlen}

Die Grundidee bei der digitalen Darstellung von Zahlen, hier ganze
Zahlen, ist die Verwendung einer festen Anzahl von Bits. Diese bilden
dann eine entsprechende Anazahl von Stellen im Dualsystem ab. Dieser
Idee folgend, kann eine ganze Zahl mit Vorzeichen wie folgt als
8-Bit-Zahl dargestellt werden:

\begin{figure}[H]

{\centering \includegraphics{index_files/mediabag/books/a-algorithmen/skript/00-bilder/zahlendarstellung_integer08.pdf}

}

\caption{Bitzuordnung bei der Darstellung einer ganzen Zahl mit 8 Bit.}

\end{figure}%

Für zwei Zahlen aus dem obigen Beispiel für die Zahldarstellung im
Dualsystem könnte die Bitzuweisung wie folgt aussehen.

\begin{figure}[H]

{\centering \includegraphics{index_files/mediabag/books/a-algorithmen/skript/00-bilder/zahlendarstellung_integer08_5.pdf}

}

\caption{Beispiel der Bitzuordnung für die Zahl 5 mit 8 Bit.}

\end{figure}%%
\begin{figure}[H]

{\centering \includegraphics{index_files/mediabag/books/a-algorithmen/skript/00-bilder/zahlendarstellung_integer08_107.pdf}

}

\caption{Beispiel der Bitzuordnung für die Zahl 107 mit 8 Bit.}

\end{figure}%

Durch die fixe Vorgabe der Stellen im Dualsystem, also hier der Bits,
ergibt sich der Zahlenbereich, welcher mit diesen Bits abgebildet werden
kann. Für die Darstellung von ganzen Zahlen mit 8 Bit, also mit einem
Byte, ergibt sich somit

\begin{itemize}
\tightlist
\item
  kleinste Zahl: 02 = 0
\item
  größte Zahl: 111111112 = 28-1 = 255 .
\end{itemize}

Natürlich können auch länger Bitfolgen für einen größeren Zahlenbereich
genutzt werden. Zusätzlich kann eines der Bits auch genutzt werden, um
das Vorzeichen darzustellen. Folgende Abbildung zeigt die Darstellung
einer vorzeichenbehafteten ganzen Zahl mit 32 Bit.

\begin{figure}[H]

{\centering \includegraphics{index_files/mediabag/books/a-algorithmen/skript/00-bilder/zahlendarstellung_integer32.pdf}

}

\caption{Bitzuordnung bei der Darstellung einer ganzen Zahl samt
Vorzeichen mit 32 Bit.}

\end{figure}%

Der Wertebereich ist in diesem Fall gegeben durch:

\begin{itemize}
\tightlist
\item
  kleinste Zahl = -231 = -2,147,483,648
\item
  größte Zahl = 231 - 1 = 2,147,483,647 .
\end{itemize}

In der Informatik wird solch eine Darstellung von ganzen Zahlen als
\href{https://de.wikipedia.org/wiki/Integer_(Datentyp)}{Integer
Datentyp} bezeichnet. Im Englischen wird dieser als \emph{integer}
bezeichnet.

\section{Umrechnung zwischen Dezimal- und
Binärzahlen}\label{umrechnung-zwischen-dezimal--und-binuxe4rzahlen}

Um zwischen \textbf{Dezimalzahlen (zur Basis 10)} und
\textbf{Binärzahlen (zur Basis 2)} umzuwandeln, gibt es jeweils einfache
Verfahren. Diese lassen sich auch leicht als Algorithmus in
Programmiersprachen umsetzen.

\subsection{Von Binär nach Dezimal}\label{von-binuxe4r-nach-dezimal}

Eine Binärzahl besteht aus einzelnen Stellen (Bits), die jeweils eine
Potenz von 2 repräsentieren. Zur Umrechnung summiert man die Produkte
der Ziffern mit ihrer jeweiligen Stellenwertigkeit:

\[
z = \sum_{i=0}^{n} b_i \cdot 2^i
\]

Dabei ist: - \(b_i \in \{0, 1\}\) das i-te Bit (von rechts gezählt), -
\(2^i\) der Stellenwert der Position, - \(z\) die Dezimalzahl.

\textbf{Beispiel}:\\
Umwandlung von \(1011_2\) nach Dezimal:

\[
\begin{align*}
1011_2 &= 1 \cdot 2^3 + 0 \cdot 2^2 + 1 \cdot 2^1 + 1 \cdot 2^0 \\
       &= 8 + 0 + 2 + 1 = 11_{10}
\end{align*}
\]

\subsection{Von Dezimal nach Binär}\label{von-dezimal-nach-binuxe4r}

Zur Umwandlung einer Dezimalzahl in eine Binärzahl verwendet man die
\textbf{ganzzahlige Division durch 2}. Der jeweilige Rest (0 oder 1)
ergibt die Binärziffer. Man wiederholt diesen Vorgang so lange, bis der
Quotient 0 ist, und liest die Reste \textbf{von unten nach oben}.

\textbf{Beispiel}:\\
Umwandlung von \(11_{10}\) nach Binär:

\[
\begin{array}{rcl}
11 \div 2 &= 5 &\text{Rest } 1 \\
5 \div 2  &= 2 &\text{Rest } 1 \\
2 \div 2  &= 1 &\text{Rest } 0 \\
1 \div 2  &= 0 &\text{Rest } 1 \\
\end{array}
\]

→ Von unten gelesen ergibt das: \(1011_2\)

\begin{tcolorbox}[enhanced jigsaw, left=2mm, leftrule=.75mm, bottomrule=.15mm, title=\textcolor{quarto-callout-note-color}{\faInfo}\hspace{0.5em}{Merke}, colback=white, arc=.35mm, breakable, titlerule=0mm, bottomtitle=1mm, colbacktitle=quarto-callout-note-color!10!white, toprule=.15mm, opacityback=0, coltitle=black, rightrule=.15mm, opacitybacktitle=0.6, toptitle=1mm, colframe=quarto-callout-note-color-frame]

Die Umrechnung funktioniert immer, egal wie groß die Zahl ist -- sie ist
eine systematische Anwendung der Stellenwertsysteme. Computer arbeiten
intern genau auf diese Weise, nur in Hardware.

\end{tcolorbox}

\section{Darstellung reeller Zahlen}\label{darstellung-reeller-zahlen}

Reelle Zahlen können nur angenährt als eine
\href{https://de.wikipedia.org/wiki/Gleitkommazahl}{Gleitkommazahl}
digital dargestellt werden. Dazu wir die zur Verfügung stehende Menge an
Bits auf folgende Zuordnungen aufgeteilt: Vorzeichen \(\mathsf s\),
Exponent \(\mathsf e\) und Mantisse \(\mathsf m\). Jedem dieser Bereiche
wird eine feste Anzahl von Bits zugeordnet wodruch sich der Wertebereich
und Genauigkeit der Darstellung ergibt. Im Allgemeinen kann somit eine
Gleitkommazahl dargestellt werden als

\[ \mathsf z = (-1)^s \cdot m \cdot 2^e .\]

Es existieren mehrere Ansätze für die Abbildung von Gleitkommazahlen.
Insbesondere im \href{https://de.wikipedia.org/wiki/IEEE_754}{IEEE754
Standard} wird folgende Aufteilung definiert: Vorzeichen (1 bit),
Exponent (11 bit) und Mantisse (52 bit):

\begin{figure}[H]

{\centering \includegraphics{index_files/mediabag/books/a-algorithmen/skript/00-bilder/zahlendarstellung_float.pdf}

}

\caption{Bitzuordnung bei der Darstellung einer reellen Zahl mit 64
bit.}

\end{figure}%

Aus der obigen Festlegung der Bitzuweisung, ergeben sich die
Größenordnung für den Wertebereich, welcher durch den Exponenten
vorgegeben ist. Um auch Zahlen kleiner 1 darstellen zu können, kann der
Exponent \(\mathsf e\) auch negative Werte annehmen.

Für den Exponenten \(\mathsf e\) gilt

\begin{itemize}
\tightlist
\item
  kleinster Wert in Etwa: -(210 - 1) = -1023
\item
  größter Wert in Etwa: \textasciitilde{} 210 - 1 = 1023 .
\end{itemize}

Ohne Beachtung der Mantisse und des Vorzeichens, ergibt sich mit den
obigen Werten dieser Bereich für die Größenordnungen:

\begin{itemize}
\tightlist
\item
  kleinste Größenordnung: 2-1023 \textasciitilde{} 10-308
\item
  größte Größenordnung: 21023 \textasciitilde{} 10308
\end{itemize}

Die Genauigkeit, d.h. die kleinste darstellbare Differenz zwischen zwei
Gleitkommazahlen, ergibt sich aus der Mantisse \$ m\$. Eine grobe
Abschätzung der Genauigkeit kann wie folgt durchgeführt werden. Per
Definition deckt die Mantisse einen Zahlenbereich von 0 bis etwa 10 ab.
Dieser Bereich wird in obiger Festlegung mit 52 Bit dargestellt. Hieraus
ergibt sich dann der kleinste Unterschied zu

\begin{itemize}
\tightlist
\item
  kleinster Unterschied zwischen zwei Gleitkommazahlen: 10 / 252
  \textasciitilde{} 2·10-15
\end{itemize}

Betrachtet man nun Dezimalzahlen, so entspricht das etwa der 15-ten
Nachkommastelle.

Der Datentyp, welcher für die Darstellung von Gleitkommazahlen verwendet
wird, wird generell als \emph{float} (engl. \emph{floating point
number}) bezeichnet. Im
\href{https://de.wikipedia.org/wiki/IEEE_754}{IEEE754 Standard} werden
viele verschiedene Darstellungen definiert.

\section{Zeichendarstellung}\label{zeichendarstellung}

Neben Zahlen können auch Zeichen, z.B. für die Darstellung von Text,
abgebildet werden. Die Grundidee ist dabei, dass die Zeichen als
vorzeichenlose ganze Zahlen gespeichert und dann anhand einer Tabelle
interpretiert werden. Ein Beispiel für eine solche Tabelle, welche den
Zahlenwerte Zeichen zuordnet, ist die
\href{https://de.wikipedia.org/wiki/American_Standard_Code_for_Information_Interchange}{ASCII
Tabelle}. In dieser werden 7-Bit-Zahlen, d.h. 128 Zeichen, kodiert. In
der 1963 erstellten --~ und bis heute genutzten -- Tabelle, sind sowohl
nicht-druckbare Zeichen (z.B. Zeilenvorschub, Tabulatorzeichen) als auf
folgende druckbare Zeichen enthalten:

\begin{verbatim}
 !"#$%&'()*+,-./0123456789:;<=>?
@ABCDEFGHIJKLMNOPQRSTUVWXYZ[\]^_
`abcdefghijklmnopqrstuvwxyz{|}~
\end{verbatim}

Wobei das erste Zeichen das Leerzeichen ist.

\chapter{Algorithmische Umrechnung von Zahlen: Dezimal ↔
Binär}\label{algorithmische-umrechnung-von-zahlen-dezimal-binuxe4r}

In diesem Kapitel betrachten wir die Umrechnung zwischen Dezimal- und
Binärzahlen \textbf{algorithmisch}. Dabei analysieren wir den Ablauf der
Rechenschritte, beschreiben sie als \textbf{Pseudocode}, visualisieren
sie mit \textbf{Flussdiagrammen} und setzen sie anschließend
\textbf{manuell in Python um}.

\section{1. Umrechnung: Dezimal →
Binär}\label{umrechnung-dezimal-binuxe4r}

\subsection{Grundidee}\label{grundidee-1}

Wir wiederholen die ganzzahlige Division durch 2 und merken uns den
Rest. Die \textbf{Binärziffern} ergeben sich aus den \textbf{Resten} ---
von unten nach oben gelesen.

\subsection{Flussdiagramm}\label{flussdiagramm}

\begin{figure}[H]

{\centering \includegraphics{index_files/mediabag/books/a-algorithmen/skript/00-bilder/dtb.pdf}

}

\caption{Umrechnug von Dezimal zu Binär}

\end{figure}%

\subsection{Pseudocode}\label{pseudocode}

\begin{verbatim}
Eingabe: Dezimalzahl n
Initialisiere leere Liste ziffern
Solange n > 0:
    rest ← n mod 2
    ziffern an rest anhängen
    n ← n ganzzahlig geteilt durch 2
Ausgabe: ziffern in umgekehrter Reihenfolge
\end{verbatim}

\subsection{Python-Implementierung}\label{python-implementierung}

\begin{Shaded}
\begin{Highlighting}[]
\KeywordTok{def}\NormalTok{ dezimal\_zu\_binaer(n):}
\NormalTok{    ziffern }\OperatorTok{=}\NormalTok{ []}
    \ControlFlowTok{while}\NormalTok{ n }\OperatorTok{\textgreater{}} \DecValTok{0}\NormalTok{:}
\NormalTok{        ziffern.append(n }\OperatorTok{\%} \DecValTok{2}\NormalTok{)}
\NormalTok{        n }\OperatorTok{//=} \DecValTok{2}
    \ControlFlowTok{return}\NormalTok{ ziffern[::}\OperatorTok{{-}}\DecValTok{1}\NormalTok{]}

\NormalTok{dezimal\_zu\_binaer(}\DecValTok{23}\NormalTok{)  }\CommentTok{\# Beispiel: 23 → 10111}
\end{Highlighting}
\end{Shaded}

\begin{verbatim}
[1, 0, 1, 1, 1]
\end{verbatim}

\section{2. Umrechnung: Binär →
Dezimal}\label{umrechnung-binuxe4r-dezimal}

\subsection{Grundidee}\label{grundidee-2}

Jede Stelle repräsentiert eine Zweierpotenz. Wir addieren die Produkte
der Ziffern mit ihrer Potenz.

\subsection{Flussdiagramm}\label{flussdiagramm-1}

\begin{figure}[H]

{\centering \includegraphics{index_files/mediabag/books/a-algorithmen/skript/00-bilder/btd.pdf}

}

\caption{Umrechnug von Binär zu Dezimal}

\end{figure}%

\subsection{Pseudocode}\label{pseudocode-1}

\begin{verbatim}
Eingabe: Liste binärer Ziffern (z. B. [1, 0, 1, 1])
Initialisiere dezimalwert ← 0
Für jede Stelle i von rechts nach links:
    dezimalwert ← dezimalwert + ziffer * 2^position
Ausgabe: dezimalwert
\end{verbatim}

\subsection{Python-Implementierung}\label{python-implementierung-1}

\begin{Shaded}
\begin{Highlighting}[]
\KeywordTok{def}\NormalTok{ binaer\_zu\_dezimal(ziffern):}
\NormalTok{    dezimalwert }\OperatorTok{=} \DecValTok{0}
    \ControlFlowTok{for}\NormalTok{ i }\KeywordTok{in} \BuiltInTok{range}\NormalTok{(}\BuiltInTok{len}\NormalTok{(ziffern)):}
\NormalTok{        potenz }\OperatorTok{=} \BuiltInTok{len}\NormalTok{(ziffern) }\OperatorTok{{-}}\NormalTok{ i }\OperatorTok{{-}} \DecValTok{1}
\NormalTok{        dezimalwert }\OperatorTok{+=}\NormalTok{ ziffern[i] }\OperatorTok{*}\NormalTok{ (}\DecValTok{2} \OperatorTok{**}\NormalTok{ potenz)}
    \ControlFlowTok{return}\NormalTok{ dezimalwert}

\NormalTok{binaer\_zu\_dezimal([}\DecValTok{1}\NormalTok{, }\DecValTok{0}\NormalTok{, }\DecValTok{1}\NormalTok{, }\DecValTok{1}\NormalTok{])  }\CommentTok{\# Ergebnis: 11}
\end{Highlighting}
\end{Shaded}

\begin{verbatim}
11
\end{verbatim}

\section{3. Mathematische
Komplexität}\label{mathematische-komplexituxe4t}

Beide Algorithmen haben eine \textbf{logarithmische Laufzeit} bezogen
auf die Eingabegröße \(n\), denn:

\begin{itemize}
\tightlist
\item
  Die Umrechnung Dezimal → Binär wiederholt die Division durch 2, bis
  \(n = 0\). Das sind \(\log_2(n)\) Schritte.
\item
  Die Umrechnung Binär → Dezimal summiert über \(\log_2(n)\) Stellen.
\end{itemize}

Daher gehören beide zur Klasse der \textbf{logarithmischen Algorithmen}.

\begin{tcolorbox}[enhanced jigsaw, left=2mm, leftrule=.75mm, bottomrule=.15mm, title=\textcolor{quarto-callout-note-color}{\faInfo}\hspace{0.5em}{Hinweis}, colback=white, arc=.35mm, breakable, titlerule=0mm, bottomtitle=1mm, colbacktitle=quarto-callout-note-color!10!white, toprule=.15mm, opacityback=0, coltitle=black, rightrule=.15mm, opacitybacktitle=0.6, toptitle=1mm, colframe=quarto-callout-note-color-frame]

Diese Verfahren sind nicht nur theoretisch interessant -- genau so
arbeiten Computer intern mit Bitfolgen!

\end{tcolorbox}




\end{document}
